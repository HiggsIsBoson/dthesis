\section{Conclusion}
%The completion of the Standard Model (SM) is one of the greatest feat that mankind ever experienced; it suceeded in explaining three out of four forces in the universe; offering a mechanism of particles obtaining the masses (BEH mechanism).
%toghether with the enormous efforts and successes of experimental examinations. Nevertheless there are still quite a room for unrevealed mysteries of universe including unaccounted phenomane such as dark matter or matter anti-matter asymmetry, as well as gliches inside the theory for instance the higgs mass divergence. New physics beyond the SM is hench highly motivated. Supper-symmetry (SUSY) is known as the one of the most favored framwork for the new physics, since it can addresses to a wide range of the problems, extending the SM based on a simple princile of boson-fermion symmetry. \\
%
%The completion of the Standard Model (SM), brought by the discovery of higgs boson, is one of the greatest feat that mankind ever experienced. 
%It suceeded in explaining three out of four forces in the universe upon a fairy straightfoward principle of gauge symmetry,
%and in offering a mechanism of particles obtaining the masses (BEH mechanism), 
%toghether with the enormous efforts and successes of experimental examinations. Nevertheless there are still quite a room for unrevealed mysteries of universe including unaccounted phenomane such as dark matter or matter anti-matter asymmetry, as well as gliches inside the theory for instance the higgs mass divergence. New physics beyond the SM is hench highly motivated. Supper-symmetry (SUSY) is known as the one of the most favored framwork for the new physics, since it can addresses to a wide range of the problems, extending the SM based on a simple princile of boson-fermion symmetry. \\
%As SUSY generally predicts another set of paricle contents with repect to the SM,
%Experimental search
%
This thesis presented the search for gluinos using proton-proton collisions in the Large Hadron Collider (LHC) at the center-of-mass energy of $\sqrt{s}=13\tev$ collected in the ATLAS detector. Focusing on the final state with one leptons, all relevant 45 decay chains for pair produced gluinos are explored, together with various scenarios of the mass spectra, aiming to provide the most general result achiavble in principle. \\

The highlight of the analysis is designing a dedicated data-driven background estimation method, reinforce the confidence on the estimation by reducingthe reliance on simulation which typically less performing in an extreme phase space. \\

Analysis is performed with dataset with 36.1 fb$^{\-1}$ of integrated luminosity. 
In the unblinded signal regions, no significant excess is found.
Constraints are set on each of the 45 models of gluino decay chain.
Exclusion upto $1.7\tev - 2.0 \tev$ in gluino mass, and upto $\sim 1\tev$ in the lightest neutralino mass is widely confirmed 
with typical mass spectra of gluino and EW gauginos, while upto $1.5\tev - 1.9 \tev$ in gluino mass is excluded in case of compressed EW gaugino masses ($\Delta M \sim 20-30\gev$) which is motivated by dark matter relic observations.

