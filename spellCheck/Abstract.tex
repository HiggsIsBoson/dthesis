%\documentclass {article}
\setlength{\topmargin}{-1.5cm}
\setlength{\oddsidemargin}{-0.3cm}
\setlength{\evensidemargin}{-0.3cm}
\setlength{\textwidth}{16.5cm}
\setlength{\textheight}{23cm}
%\usepackage{graphicx}
%\usepackage{float}
%\usepackage{url, braket, setspace}
%\usepackage{amsmath,amsthm,amssymb,bm}
%\usepackage{hyperref}


%%%%%%%%%%%%

%\begin{document}

\section*{Abstract} 
%Despite the enormous success of the Standard Model in particle physics, there are still quite a room for unrevealed mysteries of universe;
Despite the enormous success of the Standard Model in particle physics, there are still a number of problems left to be solved such as the problem of diverging higgs mass and the unaccounted presence of dark matter and so on.
%as well as for the persue toward the ``theoy of everything'' with grand unification and gravity.
It is then strongly motivated to extend the Standard Model, and the Minimal Supersymmetric Standard Model (MSSM) has been one of the most applealing candidates, introducing a boson-fermion symmetry (super-symmetry; SUSY). 
Experimental search of SUSY particles predicted by MSSM has been widely performed over the decade in collider experiments. Though no evidence has been claimed so far, searches in the Large Hadron Collider (LHC) are anticipated since it allows to probe heavier regions with the unprecedented high center-of-mass energy with increased data statistics.
The motivation of gaugino search is increasingly emerging in light of the discovery of higgs boson in its mass of $125\gev$, and gluino search is particularly interesting due to the large production in LHC.   \\

This thesis presents the search for gluinos via proton-proton collisions with the center-of-mass energy of $\sqrt{s}=13\tev$ at LHC, by focusing on the final state with exactly one leptons. 
Using the improved analysis technique and increased data with 36.1 fb$^{\-1}$ of integrated luminosity collected in the ATLAS detector, the sensitivity to heavier gluino is drastically gained.  \\

In this analysis, the main improvement with respect to past searches are two-fold: 
while only a few typical scenario of gluino decays have been studied in the past, 
the new analysis covers all the possible gluino decay chains that can be targeted in 1-lepton final state are explored, 
and the exclusion limits are explicitly provided for the first time; 
A dedicated data-driven strategy of background estimation is introduced, enabling robust estimation in regions where conventional simulation-based method estimation is not very reliable. \\

No significant excess is found in the unblinded dataset, and exclusion limits are set on wide range of gluino decay scenarios. As a general conclusion,
it is confirmed that up to $1.7\tev - 2.0 \tev$ in gluino mass and up to $\sim 1\tev$ in the lightest neutralino mass is excluded for typical mass spectra, 
while the limit extends up to $1.5\tev - 1.9 \tev$ in gluino mass in case of compressed EW gaugino masses ($\Delta M \sim 20-30\gev$) that is motivated by dark matter relic observations.



\clearpage

%{\it Maybe knowledge is as fundamental, or even more fundamental than reality.  \;\;\; }
%\begin{flushright}
%Anton Zeillinger
%\end{flushright}

%{\it The eye sees only what the mind is prepared to comprehend. \;\;\; }
%\begin{flushright}
%Henri Bergson
%\end{flushright}



%\end{document}
