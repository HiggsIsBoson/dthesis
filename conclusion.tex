%\documentclass {article}
\setlength{\topmargin}{-1.5cm}
\setlength{\oddsidemargin}{-0.3cm}
\setlength{\evensidemargin}{-0.3cm}
\setlength{\textwidth}{16.5cm}
\setlength{\textheight}{23cm}
%\usepackage{graphicx}
%\usepackage{float}
%\usepackage{url, braket, setspace}
%\usepackage{amsmath,amsthm,amssymb,bm}
%\usepackage{hyperref}


%%%%%%%%%%%%

%\begin{document}

\chapter{Conclusion} 
Bell inequality is an inequality that lets us learn whether the nature prefers classical physics or quantum ones, as well as signatures of quantum non-locality. While the violation has been extensively verified  in photon experiments, tests with massive particles are still impotent which however have only few examples yet.  

The particle decays in high energy collider experiments have been supposed to be new channels of the testing Bell inequality, using the spin entanglement from decays of particle, and the magical spin measurement scheme with weak decays provided by the nature, which enables the test with various types of systems and interactions as well. \\

On the basis of previous studies, this paper provided (i) a comprehensively review of the theoretical background, (ii) a formal establishment of the method which is free from the locality loophole for the first case in non-optical experiments, (iii) a formulation of a new Bell inequality, (iv) and a discussion on the feasibility of testing the Bell inequality by this method, with today's high specification collider machines, as a new proposition of the experiment. \\

%\end{document}


%\end{document}