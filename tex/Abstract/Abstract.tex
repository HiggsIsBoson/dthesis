%\documentclass {article}
\setlength{\topmargin}{-1.5cm}
\setlength{\oddsidemargin}{-0.3cm}
\setlength{\evensidemargin}{-0.3cm}
\setlength{\textwidth}{16.5cm}
\setlength{\textheight}{23cm}
%\usepackage{graphicx}
%\usepackage{float}
%\usepackage{url, braket, setspace}
%\usepackage{amsmath,amsthm,amssymb,bm}
%\usepackage{hyperref}


%%%%%%%%%%%%

%\begin{document}
\chapter*{Abstract} 
\thispagestyle{empty}
Despite the remarkable success of the Standard Model in particle physics, a number of unresolved problems remain, such as the fine-tuning problem of the Higgs boson mass and the unexplained presence of dark matter.
%as well as the pursuit of a ``theory of everything'' involving grand unification and gravity.
These challenges strongly motivate extensions to the Standard Model, among which the Minimal Supersymmetric Standard Model (MSSM) has been one of the most compelling candidates. In the MSSM, a boson–fermion symmetry, known as supersymmetry (SUSY), is introduced. 
Experimental searches for SUSY particles predicted by the MSSM have been widely carried out over the past decades in collider experiments. 
Although no conclusive evidence has been observed so far, searches at the Large Hadron Collider (LHC), with its unprecedented center-of-mass energy and increased data statistics, enable the exploration of SUSY particles with significantly higher masses.

This thesis presents a search for gluinos in proton–proton collisions at a center-of-mass energy of $\sqrt{s} = 13\ \tev$ at the LHC, 
using a dataset corresponding to an integrated luminosity of 36.1~fb$^{-1}$ collected with the ATLAS detector.
Final states containing exactly one lepton, jets and missing transverse momentum are used.
The analysis is designed to cover a variety of MSSM scenarios, with particular focus on ones that explains the dark matter relic density.
Compared with the previous LHC searches, the sensitivity reach becomes significantly extended with the improved event selection, while adding the robustness at the same by the introduction of data-driven SM background estimation.
%The search for gluinos, the SUSY partners of gluons, has become increasingly motivated following the discovery of the Higgs boson with a mass of $125\ \gev$.   \\
%In this analysis, the main improvements over past searches are twofold: 
%whereas only a few typical scenarios of gluino decays and SUSY mass spectra were previously considered, 
%the new analysis covers a significantly broader range of gluino decay channels and mass spectrum assumptions, including those favored by dark matter relic density observations that had not been explicitly explored;
%
%in addition, a dedicated data-driven background estimation strategy is introduced, 
%allowing robust estimation in regions where conventional simulation-based methods are not necessarily reliable. \\
No statistically significant excess above the Standard Model expectation is observed in the unblinded dataset, and exclusion limits are set on all targeted gluino decay scenarios. 
For typical mass spectra, gluino masses in the range $1.7\ \tev$ to $2.0\ \tev$ and lightest neutralino masses up to approximately $1\ \tev$ are excluded, 
while for dark-matter-motivated mass spectra, gluino masses in the range $1.5\ \tev$ to $1.9\ \tev$ are excluded.

\clearpage


%{\it Maybe knowledge is as fundamental, or even more fundamental than reality.  \;\;\; }
%\begin{flushright}
%Anton Zeillinger
%\end{flushright}

%{\it The eye sees only what the mind is prepared to comprehend. \;\;\; }
%\begin{flushright}
%Henri Bergson
%\end{flushright}



%\end{document}
