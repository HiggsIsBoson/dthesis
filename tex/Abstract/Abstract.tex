%\documentclass {article}
\setlength{\topmargin}{-1.5cm}
\setlength{\oddsidemargin}{-0.3cm}
\setlength{\evensidemargin}{-0.3cm}
\setlength{\textwidth}{16.5cm}
\setlength{\textheight}{23cm}
%\usepackage{graphicx}
%\usepackage{float}
%\usepackage{url, braket, setspace}
%\usepackage{amsmath,amsthm,amssymb,bm}
%\usepackage{hyperref}


%%%%%%%%%%%%

%\begin{document}
\chapter*{Abstract} 
\thispagestyle{empty}
%Despite the enormous success of the Standard Model in particle physics, there are still quite a room for unrevealed mysteries of universe;
Despite the enormous success of the Standard Model in particle physics, there are still a number of problems left to be solved such as the fine tuning problem of the higgs mass, or the unaccounted presence of dark matter and so on.
%as well as for the pursue toward the ``theoy of everything'' with grand unification and gravity.
It is then strongly motivated to extend the Standard Model, and the Minimal Super-symmetric Standard Model (MSSM) has been one of the most appealing candidates, where a boson-fermion symmetry (super-symmetry; SUSY) is introduced. 
Experimental searches of SUSY particles predicted by MSSM has been widely performed over the decade in collider experiments. 
Though no evidence has been claimed so far, searches in the Large Hadron Collider (LHC) are anticipated with the unprecedented high center-of-mass energy and increased data statistics, 
allowing one to probe heavier SUSY particles.
Gluino is one of the SUSY particles of which search is increasingly motivated after the discovery of higgs boson with its mass of $125\gev$.   \\

This thesis presents the updated search for gluinos via proton-proton collisions with the center-of-mass energy of $\sqrt{s}=13\tev$ at LHC, by focusing on the final state with exactly one lepton. 
With respect to the past searches, the sensitivity to heavier gluino is drastically gained using the improved analysis technique and updated data statistics (36.1 fb$^{-1}$ of integrated luminosity) collected in the ATLAS detector. 
%In this analysis, the main improvement with respect to past searches are two-fold: 
%while only a few typical scenarios of gluino decays and SUSY mass spectra have been studied, 
%the new analysis covers a drastically wider range of gluino decays and mass spectra assumptions including ones particularly favored by dark matter relic observation that have been never explicitly explored;
%
%a dedicated data-driven strategy of background estimation is introduced, 
%enabling robust estimation in regions where conventional simulation-based method estimation is not necessarily reliable. \\
No significant data excess is found in the unblinded dataset, and the exclusion limits are set on all the targeted gluino decay scenarios. 
As a general conclusion,
it is confirmed that up to $1.7\tev \sim 2.0 \tev$ in gluino mass and up to $\sim 1\tev$ in the lightest neutralino mass is excluded for typical mass spectra, 
while the limit extends up to $1.5\tev \sim 1.9 \tev$ in gluino mass for the case of the dark matter oriented mass spectra.


\clearpage


%{\it Maybe knowledge is as fundamental, or even more fundamental than reality.  \;\;\; }
%\begin{flushright}
%Anton Zeillinger
%\end{flushright}

%{\it The eye sees only what the mind is prepared to comprehend. \;\;\; }
%\begin{flushright}
%Henri Bergson
%\end{flushright}



%\end{document}
