\chapter{Background Estimation} \label{sec::BGestimation}
Due to the enormously large cross-section of SM processes with respect to the signal,
it is the fate for new physics searches to keep exploring the phase space with tight event selections.
The consequence is the highly untypical kinematics for the remained SM backgrounds, and the modeling is usually challenging since the standard MC simulation is not necessarily accountable as seen in Sec. \ref{sec::BGestimation::dataMC}. \\

This is why the (semi-)data-driven approach is remarkably motivated in search analyses generally. 
The most common practice done over the past is to apply corrections to the MC using the data events around the signal region (``control region''). 
The prediction in the signal regions is then obtained by the corrected MC, assuming the modeling of the phase space between the control region and the signal region is correct. 
We refer this semi-data driven method as \textbf{``kinematical extrapolation method''}. 
The advantage of the kinematical extrapolation is that the prediction does not suffer from severe statistical fluctuation, which often leads to relatively smaller total uncertainty. 
However the drawback is that it has to still rely on the MC for the extrapolation between the control regions to the signal regions, which uncertainty is rather difficult to capture and quantify. \\

On the other hand, statistical error is often the dominatant source of uncertainty in the signal regions.
Therefore it is more preferable to pursue the robustness of the estimation, to avoid the risk of introducing unknown systematic effects despite of larger estimation uncertainty.

A nearly fully data-driven method (\textbf{``object replacement method''}) is introduced following the concept.
This is the method estimating particular background components by extrapolating from the 2-lepton control regions.
In the study, the object replacement method is utilized as much as one can, while the rest of all is covered by the kinematical extrapolation. \\
%
%searchものでは最終的にSRにeventが少なくsystが効くことがほぼないので紙面上の数パーセントのsystを争う意味は皆無である。それよりは考えてなかった, もしくはunknownなeffectによって爆発的な間違いを犯すリスクを下げる方が方針として有能である。
%その後kinematical exp./obj rep.についてそれぞれdetailし、最後にvalidation regionにおけるestimationのvalidationを行う.

This section provides a detailed description about the estimation procedures employed in this analysis. 
After reviewing the breakdown in the signal regions and how they evade the event selection, the both estimation methods are described. 
Finally, the performance is demonstrated using the data in a certain set of regions.

% BG estm はsearch で一番大事
% robustな結果を出すためにかなりhigh levelなdedicated data-driven approachを採用した

%%%%%%%%%%%%%%%%%%%%%%%%%%%%%%%%%%%
\section{Background Breakdown in the Signal Regions} \label{sec::BGcomposition}
The breakdown of physics processes in the signal regions are shown in Figure \ref{fig::BGestimation::BGcomposition_splitLv2}. $\wjets$ and top backgrounds ($\ttbar+Wt$, mostly $\ttbar$) dominate over the $b$-tagged and $b$-vetoed regions respectively.
%, while dibosons and single-top (mostly $Wt$-channel) are sub-dominant. 
%Since the kinematics are similar between $\ttbar$ and single-top, they are merged as ``top backgrounds'' in the rest of the discussion.
The \textbf{3B} towers are completely dominated by $\ttbar$, where 60 $\%$ of them are with heavy flavor jets via radiative gluon splitting ($\ttbar+cc/bb$) while the rest are with one light flavor jet or hadronically decaying $\tau$ faking into $b$-tagged jet ($\ttbar+b_{\mathrm{fake}}$).  \\

\begin{figure}[h]
  \centering
    \subfig{0.85}{figures/BGestimation/BGcomponent/prod-01-825-03/BGcomp_splitLv2_SRBV_norm.pdf}{}
    \subfig{0.85}{figures/BGestimation/BGcomponent/prod-01-825-03/BGcomp_splitLv2_SRBT_norm.eps}{}
    \caption{ Background composition in terms of physics processes in the (a) BV, and (b) BT/3B signal regions. $\ttbar$ and single-top are merged as ``Tops'', and the semi-leptonic and di-leptonic components are respectively labeled as ``1L'' and ``$2L+L\tau_h$''. \label{fig::BGestimation::BGcomposition_splitLv2} }
\end{figure}


\clearpage
\noindent Backgrounds can be also categorized depending on the mechanism they pass the selection,
and different estimation methods are applied based on it.
The categorization is shown in Table \ref{tab::BGestimation::BGclass}.
\tabsmall{|c|c|c|c|c|}{
  \hline
  \multicolumn{2}{|c|}{ Category }    & Origin                                                  &  Main physics process   & Estimation method \\
  \hline
  \multicolumn{2}{|c|}{``Semi-leptonic''}        & On-shell W with diluted $\mt$ & ($W$, $\ttbar$, $VV$) $\ra \ell \nu$ + jets   & Kine. extp. / MC  \\
  \multicolumn{2}{|c|}{}              & / High-mass Drell-Yan                                    &                         &                  \\
  \hline
  ``Di-leptonic'' & $\ell\ell_{\mathrm{mis.}}$ & "Out Acc."       & ($\ttbar$, $Wt$, $WW$) $\ra \ell \nu \ell \nu$ + jets          & Kine. extp.       \\
         &                            & "Mis. Reco."     &                                                                & Obj. rep.         \\
         &                            & "Mis. ID"        &                                                                & Obj. rep.         \\
         &                            & "Mis. OR"        &                                                                & Kine. extp.       \\
  \cline{2-5}
         & $\ell\tau_{\mathrm{h}}$    & 1 real-lepton + $\tau_h$ & $\ttbar$, $Wt$, $WW$ $\ra \ell \nu \tau \nu$ + jets            & Obj. rep.         \\
  \hline
  \multicolumn{2}{|c|}{``Fake''}        & 0 real-lepton + 1 fake-lepton.  & $W\ra\tau\nu$, $Z\ra\nu\nu$          & MC                \\
  \hline
}
{Background classification in terms of the origin.}
{tab::BGestimation::BGclass}
%

The \textbf{``semi-leptonic''} category is defined by events with exactly one real light flavor lepton ($e$ or $\mu$). In the SM, these are uniquely provided by processes with leptonically decaying W-boson, such as from $\wjets$ and $\ttbar$. This is by far the dominant component at 1-lepton pre-selection level, however is drastically suppressed after a tight $\mt$ cut since the they have a sharp cut-off at $\mt \sim m_W (\sim 80~\GeV)$. After the $\mt$ cut, the remnant events are typically either: 1) the Drell-Yan process with virtual heavy intermediate W boson, or 2) events with badly measured MET leading to a prolonged tail in $\mt$. The former contribution is typically larger although the latter becomes more addressing with increasing jet activity as shown in Figure \ref{fig::BGestimation::Wmassline}. In this category, the dominant processes $\wjets$ and $\ttbar+Wt$ are estimated by the kinematical extrapolation, while the other processes are taken from pure MC prediction since they are minor. \\
%%%%%%%%%%
\begin{figure}[h]
  \centering
    \subfig{0.41}{figures/BGestimation/BGcomponent/Wmassline_mt125.pdf}{}
    \subfig{0.57}{figures/BGestimation/BGcomponent/Wjets_onshellFrac_highMT.pdf}{}
    \caption{ (a) Truth invariance mass $m(\ell,\nu)$ of high-$\mt$ $\wjets$ events. Ideally there are only high-mass Drell-Yan type of events, however due to the finite detector resolution, a fraction of on-shell W events with badly measured MET sneak into regions with $\mt>m_W$. (b) The fraction of on-shell events defined by $m(\ell,\nu) \in [60,125]$, as a function of the $\mt$ cut. It is generally below 50 $\%$, however increases with higher jet activity in the event. \label{fig::BGestimation::Wmassline} }
\end{figure}
%%%%%%%%%%

The \textbf{``di-leptonic''} category consists of processes with real two leptons including $\tau$, mainly from di-leptonically decaying $\ttbar$, $Wt$ and $WW$. The presence becomes highly significant with respect to the ``semi-leptonic'' after the $\mt$ cut. This is because the source of MET is multiple for the di-leptonic events which do not have an associated cut-off at $\mt \sim m_{W}$. 
The mechanism of falling into the 1-lepton regions are mainly two-fold, namely 1) ``$\ell\ell_{\mathrm{mis.}}$'' (``missing lepton''): events with two real light flavor leptons ($e/\mu$) and one of them fails the ``baseline'' requirement (See Sec. \ref{sec::objDef::objDef}), and 2) ``$\ell\tau_{\mathrm{h}}$'': events with a real light flavor lepton and a hadronically decaying tau lepton which is not identified as a ``baseline lepton''. 
%
The origin of ``missing lepton'' is further four-fold and symbolized as follow: 
\begin{description}
\item [Out-of-acceptance (``Out Acc.'')] \mbox{} \\
 Leptons traveling outside the acceptance of ``baseline'' requirement i.e. $p_{\mathrm{T}}>7(6)\gev, |\eta|<2.47(2.5)$ for electrons (muons).
\footnote{Defined by the momenta of truth leptons in the MC.}
\item [Mis-reconstructed (``Mis. Reco'')] \mbox{} \\
 Leptons within the $(\pt,\eta)$ acceptance but failing the reconstruction.
\item [Mis-IDed(``Mis. ID'')] \mbox{} \\
 Reconstructed leptons within the $(\pt,\eta)$ acceptance but failing the electron/muon ID.
\item [Killed in the overlap removal procedure (``Mis. OR'')] \mbox{} \\
 Reconstructed leptons within the $(\pt,\eta)$ acceptance passing the ID, but overlapped with light-flavor jets by $\Delta R<0.4$.
\footnote{using the 85$\%$ $b$-tagging efficiency working point, in order to be consistent with the $b$-tagging used in the overlap removal \ref{sec::objDef::OR}. }
which are likely to be killed in the overlap removal (Sec. \ref{sec::objDef::OR}). 
%Technically, here it is defined as an identified lepton with the nearest non-$b$-tagged jet closer than $\Delta<0.4$.
\end{description}

One great thing about this \textbf{``di-leptonic''} component is that 2-lepton regions are available as the control regions. Since no signal regions are set there, exactly the same phase space with respect to SRs can be exploited. Object replacement method is used for the estimation except the ``Out Acc.'' and ``Mis. OR'' events which are estimated by the kinematic extrapolation method together with the  \textbf{``semi-leptonic''} due to some technical challenges. 
%
The third category \textbf{``fake''} involves events with a fake lepton, which is not negligible in regions dealing with soft leptons (\textbf{2J} and \bLowx). The estimation fully relies the MC prediction. The dominant contribution is from $W\ra\tau\nu$ and $Z\ra\nu\nu$ associated with a large MET from (a) neutrino(s). The contribution from the multi-jets process is supposed to be negligible, it is nevertheless dedicatedly cross-checked since the impact could be hazardous due to the huge cross-section. This is done using a series of validation regions referred as VRs-QCD, shown in Appendix \ref{sec::BGestimation::VRQCD}. \\
%(footnote) for definition of ``fake'' please refer Sec. \ref{sec::fakeLepton}.  

The background breakdown based on this categorization is summarized in Figure \ref{fig::BGestimation::BGcomposition_objRep}. \textbf{``semi-leptonic''} is overwhelmingly dominant in the BV signal regions, while \textbf{``di-leptonic''} (particularly ``$\ell\tau_h$'') is shown to be dominant in the BT/3B signal regions respectively.

\clearpage
\begin{figure}[h]
  \centering
    \subfig{0.97}{figures/BGestimation/BGcomponent/prod-01-825-03/BGcomp_objRep_SRBV_norm.pdf}{}
    \subfig{0.97}{figures/BGestimation/BGcomponent/prod-01-825-03/BGcomp_objRep_SRBT_norm.eps}{}
    \caption{ Background breakdown in the (a) BV, and (b) BT/3B signal regions based on the classification in Table \ref{tab::BGestimation::BGclass}. While the BV signal regions are dominated by the ``semi-leptonic'' category, BT/3B signal regions are mainly by ``di-leptonic'', especially the ``$\ltauh$'' component.
 \label{fig::BGestimation::BGcomposition_objRep} }
\end{figure}


%%%%%%%%%%%%%%%%%%%%%%%%%%%%%%%%%%%
\clearpage


%%%%%%%%%%%%%%%%%%%%%%%%%%%%%%%%%%%
\section{The Kinematical Extrapolation Method} \label{sec::BGestimation::kineExtp}
The main consideration in the kinematical extrapolation method is the definition of control regions.
It is basically a procedure of 1) specifying kinematical variables that are well-modeled by the MC which is suitable for extrapolating from the control regions (CRs) to the signal regions (Sec. \ref{sec::BGestimation::dataMC}), and 2) deciding the selection of the CRs (Sec. \ref{sec::BGestimation::CRdef}). 
The MC is then normalized to the data in the CRs. The measured normalization factors and the yields in the CRs are extensively discussed in Sec \ref{sec::BGestimation::kineExtp::result}.
% the most important thing is that the behavior (or the cause) of the MC mis-modeling is more or less understood. 
%What one should avoid is to have multiple sources of mis-modeling that contribute oppositely, leading to underestimation of mis-modeling.
%Therefore,  is simultaneously done in Sec. \ref{sec::BGestimation::dataMC}.  

\subsection{MC vs Data Comparison and the MC mis-modeling} \label{sec::BGestimation::dataMC}
The MC modeling of dominant background processes ($\wjets$ and $\ttbar$) is examined in the pre-selection regions defined in Table \ref{tab::BGestimation::preselectionDef}. 
Each pre-selection region is intended to be dominated by a particular process to be tested. 
%
\tab{c|c|c|c|c|c}{
%\tabsmall{c|c|c|c|c|c}{
  \hline
  Region name            & $\nLepbase$ & $\nLepsignal$ & $\lepOnePt$ [GeV]     & $\nBJet$  & Tested processes   \\
  \hline
  \hline
%  1L   (hardLep/softLep) & 1           & 1             & $>35$ / $[7(6),35]$ for $e$ ($\mu$) & -        & $\wjets$, $\ttbar$ \\
%  1LBV (hardLep/softLep) & 1           & 1             & $>35$ / $[7(6),35]$ for $e$ ($\mu$) & $0$      & $\wjets$           \\
%  1LBT (hardLep/softLep) & 1           & 1             & $>35$ / $[7(6),35]$ for $e$ ($\mu$) & $[1,2]$  & $\ttbar (\ra bqq b\ell\nu)$           \\
  1LBV  & 1           & 1             & $>35$ & $0$      & $\wjets$           \\
  1LBT  & 1           & 1             & $>35$ & $[1,2]$  & $\ttbar/Wt \,\, (\ra bqq b\ell\nu)$           \\
  2LBT                   & 2           & 2             & -                     & $[1,2]$  & $\ttbar/Wt \,\,  (\ra b\ell\nu b\ell\nu)$           \\ 
  1L3B                   & 1           & 1             & $>15$                 & $\geq 3$ & $\ttbar+cc/bb$, $\ttbarFakeB$ \\
  \hline
}
{Definition of pre-selection regions and the corresponding tested physics processes. The MET trigger requirement, event cleaning described Sec. \ref{sec::SRdefinition::eventCleaning}, $\nJetNoGev \geq 2$ and $\met>250$ are applied as the common selection.}
{tab::BGestimation::preselectionDef}
%
%
\paragraph{\underline{$\bf{\wjets:}$}}  \mbox{} \\
Figure \ref{fig::BGestimation::DataMCPreselHardBV1} - \ref{fig::BGestimation::DataMCPreselHardBV2} show the kinematic distributions in the \textbf{1LBV} pre-selection region where $\wjets$ is enriched. While the bulk phase space is well-described by the MC, there is generally striking overestimation in the tail regions. The discrepancy is mainly observed in distributions related to jets, particularly in the high jet multiplicity above 3. Based on that the jets are all from ISRs or FSRs here, and that the multiplicity roughly corresponds to the order of the perturbative QCD, this implies the mis-modeling is due to the higher-order QCD contribution beyond NNLO ignored in the calculation of the MC. \\
\footnote{Note that the MC sample generated by \sherpa 2.2 does not include loop diagrams beyond NLO and neither diagrams with more than 5 partons in the final state.}
%
%Therefore, an order-by-order cross-section correction should be helpful as the first aid. In fact, a simple MC reweighting in terms of jet multiplicity turns to work quite nicely. The reweighting function is derived by fitting linearly the observed data/MC in Figure \ref{fig::BGestimation::DataMCPreselHardBV1} (a):
%\begin{equation}
%w = 1 - 0.1 \times (\nJetNoGev-2) \label{eq::BGestimation::rwgt_nJ},
%\end{equation}
%where $\nJetNoGev$ is the number of jets with transverse momenta greater than $30\gev$. While the jet multiplicity distribution is fully corrected by construction, the other discrepancies can be resolved almost perfectly as well, as shown in \ref{fig::BGestimation::DataMCPreselHardBV_rwgt1}. \\
%
%
%Another observed aspect of the mis-modeling is that it is more striking in terms of soft radiations rather than the hard ones. For example, the average jet transverse momentum distribution (Figure \ref{fig::BGestimation::DataMCPreselHardBV_rwgt1} (c)) is well-modeled above $\sim 200 \gev$, while the slope of data over the MC for jet multiplicity distribution is rather persistent. This in fact backups that reweighting in other mis-modeled variables than jet multiplicity, such as $\meffInc$, actually does not work as successfully, since their tails are basically determined by hard jets. \\
%
%Although this simple linear $\nJetNoGev$ reweighting demonstrated above seems qualitatively reasonable, it is not seriously used as correction due to the technical drawbacks that the optimum coefficients in Eq. \ref{eq::BGestimation::rwgt_nJ} has slight phase space dependence, and that it is difficult to validate them around signal regions since the data statistics is limited. However, it is still a good enough approximation as well as a useful reference expression to understand the behavior and correlation of mis-modeling between variables. The reweighting with Eq. \ref{eq::BGestimation::rwgt_nJ} is then used for emulating the mis-modeling when designing the data-driven background estimation as described in the following sub-sections. \\
%
The variables that do not scale with transverse momenta of outgoing particles (``non-scaling'' variables), such as $\mt$ or $\apl$, keep relatively well-modeled up to the tails. Particularly, $\mt$ is by construction insensitive to most of the kinematics since the tail is mainly determined by the mass-line of off-shell $W$-boson or the MET resolution as discussed above. $\apl$ is also supposed to be robust since it takes a form of ratio of jet momenta. Therefore, these variables are decided to be used for the extrapolation from CRs to SRs. Note that the $\mt$ cut-off region ($\mt \sim m_W$) is slightly mis-modeled typically when tighter selections are applied, presumably 
%due to the simplified treatment of the mass-line in calculating diagrams with many additional partons. 
due to the ISR/FSR mis-modeling mentioned above propagated to the MET resolution.
The effect are particularly visible in the CRs (e.g. Figure \ref{fig::BGestimation::CRpostFit::WRVarx}) or $b$-vetoed SRs.
%The effect becomes visible in validation regions and giving systematical upward pulls by about 1$\sigma$ (Figure \ref{fig::BGestimation::VRPulls}). \\

%\clearpage
%%%%%%%%%%%%%
\begin{figure}[h]
  \centering
    \subfig{0.46}{figures/BGestimation/DataMCComparison/Preselection_hardLepBV/nJet30__Preselection_hardLepBV.pdf}{Jet multiplicity}
    \subfig{0.46}{figures/BGestimation/DataMCComparison/Preselection_hardLepBV/jet1Pt__Preselection_hardLepBV.pdf}{$\pt$ of the leading jet}
    \subfig{0.46}{figures/BGestimation/DataMCComparison/Preselection_hardLepBV/averageJetPt__Preselection_hardLepBV.pdf}{Average $\pt$ of jets with $\pt>30\gev$}
    \subfig{0.46}{figures/BGestimation/DataMCComparison/Preselection_hardLepBV/meffInc30__Preselection_hardLepBV.pdf}{$\meffInc (\meffDef)$}
    \caption{ Kinematical distribution of data (black dots) and MC (colored stack) in the \textbf{1LBV} pre-selection region.  \label{fig::BGestimation::DataMCPreselHardBV1} }
\end{figure}

\begin{figure}[h]
  \centering
    \subfig{0.48}{figures/BGestimation/DataMCComparison/Preselection_hardLepBV/lep1Pt__Preselection_hardLepBV.pdf}{Lepton's $\pt$}
    \subfig{0.48}{figures/BGestimation/DataMCComparison/Preselection_hardLepBV/met__Preselection_hardLepBV.pdf}{$\met$}
    \subfig{0.48}{figures/BGestimation/DataMCComparison/Preselection_hardLepBV/mt__Preselection_hardLepBV.pdf}{$\mt$}
    \subfig{0.48}{figures/BGestimation/DataMCComparison/Preselection_hardLepBV/LepAplanarity__Preselection_hardLepBV.pdf}{$\Apl$}
    \caption{ Kinematical distribution of data (black dots) and MC (colored stack) in the \textbf{1LBV} pre-selection region.  \label{fig::BGestimation::DataMCPreselHardBV2} }
\end{figure}


%%%%%%%%%%%%%%
\begin{figure}[h]
  \centering
    \subfig{0.42}{figures/BGestimation/DataMCComparison/Preselection_hardLepBV/nJet30__Preselection_hardLepBV__rwgt_nJ007_ttPt007.pdf}{Jet multiplicity}
    \subfig{0.42}{figures/BGestimation/DataMCComparison/Preselection_hardLepBV/jet1Pt__Preselection_hardLepBV__rwgt_nJ007_ttPt007.pdf}{$\pt$ of the leading jet}
    \caption{ Kinematical distribution of data (black dots) and MC (colored stack) in the \textbf{1LBV} pre-selection region, with the reweighting $w = 1 - 0.1 \times (\nJetNoGev-2)$ being applied for $\wjets$ MC. 
 \label{fig::BGestimation::DataMCPreselHardBV_rwgt1} }
\end{figure}

\clearpage
\begin{figure}[h]
  \centering
    \subfig{0.41}{figures/BGestimation/DataMCComparison/Preselection_hardLepBV/averageJetPt__Preselection_hardLepBV__rwgt_nJ007_ttPt007.pdf}{Average $\pt$ of jets with $\pt>30\gev$}
    \subfig{0.41}{figures/BGestimation/DataMCComparison/Preselection_hardLepBV/meffInc30__Preselection_hardLepBV__rwgt_nJ007_ttPt007.pdf}{$\meffInc (\meffDef)$}
    \subfig{0.41}{figures/BGestimation/DataMCComparison/Preselection_hardLepBV/lep1Pt__Preselection_hardLepBV__rwgt_nJ007_ttPt007.pdf}{Lepton's $\pt$}
    \subfig{0.41}{figures/BGestimation/DataMCComparison/Preselection_hardLepBV/met__Preselection_hardLepBV__rwgt_nJ007_ttPt007.pdf}{$\met$}
    \subfig{0.41}{figures/BGestimation/DataMCComparison/Preselection_hardLepBV/mt__Preselection_hardLepBV__rwgt_nJ007_ttPt007.pdf}{$\mt$}
    \subfig{0.41}{figures/BGestimation/DataMCComparison/Preselection_hardLepBV/LepAplanarity__Preselection_hardLepBV__rwgt_nJ007_ttPt007.pdf}{$\Apl$}
    \caption{ Kinematical distribution of data (black dots) and MC (colored stack) in the \textbf{1LBV} pre-selection region, with the reweighting $w = 1 - 0.1 \times (\nJetNoGev-2)$ being applied for $\wjets$ MC.  \label{fig::BGestimation::DataMCPreselHardBV_rwgt2} }
\end{figure}


%%%%%%%%%%%%%%
\begin{figure}[h]
  \centering
    \subfigure[]{\includegraphics[width=0.48\textwidth]{figures/BGestimation/DataMCComparison/Preselection_softLepBV/nJet30__Preselection_softLepBV.pdf}}
    \subfigure[]{\includegraphics[width=0.48\textwidth]{figures/BGestimation/DataMCComparison/Preselection_softLepBV/jet1Pt__Preselection_softLepBV.pdf}}
    \subfigure[]{\includegraphics[width=0.48\textwidth]{figures/BGestimation/DataMCComparison/Preselection_softLepBV/averageJetPt__Preselection_softLepBV.pdf}}
    \subfigure[]{\includegraphics[width=0.48\textwidth]{figures/BGestimation/DataMCComparison/Preselection_softLepBV/meffInc30__Preselection_softLepBV.pdf}}
    \caption{ Kinematical distribution of (a) Jet multiplicity ($p_T>30\gev$) (b) leading-jet pt  (c) average jet pt ($p_T>30\gev$)  (d) $\meffInc$ in the soft lepton b-vetoed pre-selection region.  \label{fig::BGestimation::DataMCPreselSoftBV1} }
\end{figure}

\begin{figure}[h]
  \centering
    \subfigure[]{\includegraphics[width=0.48\textwidth]{figures/BGestimation/DataMCComparison/Preselection_softLepBV/lep1Pt__Preselection_softLepBV.pdf}}
    \subfigure[]{\includegraphics[width=0.48\textwidth]{figures/BGestimation/DataMCComparison/Preselection_softLepBV/met__Preselection_softLepBV.pdf}}
    \subfigure[]{\includegraphics[width=0.48\textwidth]{figures/BGestimation/DataMCComparison/Preselection_softLepBV/mt__Preselection_softLepBV.pdf}}
    \subfigure[]{\includegraphics[width=0.48\textwidth]{figures/BGestimation/DataMCComparison/Preselection_softLepBV/LepAplanarity__Preselection_softLepBV.pdf}}
    \caption{ Kinematical distribution of (a) leading-lepton pt (b) $\met$  (c) $\mt$  (d) $\apl$ in the soft lepton b-vetoed pre-selection region.  \label{fig::BGestimation::DataMCPreselSoftBV2} }
\end{figure}

\clearpage
%%%%%%%%%%%%%


%%%%%%%%%%% VRZb
%\begin{figure}[h]
%  \centering
%    \subfigure[]{\includegraphics[width=0.48\textwidth]{figures/BGestimation/VRZb/mll__VRZb_log.pdf}}
%    \subfigure[]{\includegraphics[width=0.48\textwidth]{figures/BGestimation/VRZb/mll__VRZb__Zb130_log.pdf}}
%    \caption{ .  \label{fig::BGestimation::VRZb} }
%\end{figure}




%%%%%%%%%%%%%%%%%%%%%%%%%%%%%%%%%%%%%%%
\clearpage
%\paragraph{\underline{Semi-leptonic/Di-leptonic $\ttbar$:}} \mbox{} \\
\paragraph{\underline{Semi-leptonic $\ttbar$:}} \mbox{} \\
Figure \ref{fig::BGestimation::DataMCPreselHardBT1} - \ref{fig::BGestimation::DataMCPreselHardBT2} are the kinematic distributions in the \textbf{1LBT} pre-selection region dominated by semi-leptonically decaying $\ttbar$. It is seen that MC is overshooting the data with increasing transverse momenta of outgoing particles such as jets, lepton and MET. 

%%%%%%%%%%%%%
\begin{figure}[h]
  \centering
    \subfig{0.46}{figures/BGestimation/DataMCComparison/Preselection_hardLepBT/nJet30__Preselection_hardLepBT.pdf}{Jet multiplicity}
    \subfig{0.46}{figures/BGestimation/DataMCComparison/Preselection_hardLepBT/jet1Pt__Preselection_hardLepBT.pdf}{$\pt$ of the leading jet}
    \subfig{0.46}{figures/BGestimation/DataMCComparison/Preselection_hardLepBT/averageJetPt__Preselection_hardLepBT.pdf}{Average $\pt$ of jets with $\pt>30\gev$}
    \subfig{0.46}{figures/BGestimation/DataMCComparison/Preselection_hardLepBT/meffInc30__Preselection_hardLepBT.pdf}{$\meffInc (\meffDef)$}
    \caption{ Kinematical distribution of data (black dots) and MC (colored stack) in the \textbf{1LBT} pre-selection region.  \label{fig::BGestimation::DataMCPreselHardBT1} }
\end{figure}

\begin{figure}[h]
  \centering
    \subfig{0.48}{figures/BGestimation/DataMCComparison/Preselection_hardLepBT/lep1Pt__Preselection_hardLepBT.pdf}{Lepton's $\pt$}
    \subfig{0.48}{figures/BGestimation/DataMCComparison/Preselection_hardLepBT/met__Preselection_hardLepBT.pdf}{$\met$}
    \subfig{0.48}{figures/BGestimation/DataMCComparison/Preselection_hardLepBT/mt__Preselection_hardLepBT.pdf}{$\mt$}
    \subfig{0.48}{figures/BGestimation/DataMCComparison/Preselection_hardLepBT/LepAplanarity__Preselection_hardLepBT.pdf}{$\Apl$}
    \caption{ Kinematical distribution of data (black dots) and MC (colored stack) in the \textbf{1LBT} pre-selection region.  \label{fig::BGestimation::DataMCPreselHardBT2} }
\end{figure}

%%%%%%%%%%%%%%
\begin{figure}[h]
  \centering
    \subfig{0.47}{figures/BGestimation/DataMCComparison/Preselection_hardLepBT/nJet30__Preselection_hardLepBT__rwgt_nJ007_ttPt007.pdf}{Jet multiplicity}
    \subfig{0.47}{figures/BGestimation/DataMCComparison/Preselection_hardLepBT/jet1Pt__Preselection_hardLepBT__rwgt_nJ007_ttPt007.pdf}{$\pt$ of the leading jet}
    \subfig{0.47}{figures/BGestimation/DataMCComparison/Preselection_hardLepBT/averageJetPt__Preselection_hardLepBT__rwgt_nJ007_ttPt007.pdf}{Average $\pt$ of jets with $\pt>30\gev$}
    \subfig{0.47}{figures/BGestimation/DataMCComparison/Preselection_hardLepBT/meffInc30__Preselection_hardLepBT__rwgt_nJ007_ttPt007.pdf}{$\meffInc (\meffDef)$}
    \caption{ Kinematical distribution of data (black dots) and MC (colored stack) in the \textbf{1LBT} pre-selection region, reweighting $w = 1.05 \times \left[ 1 - 0.061 \,\times p_T(\ttbar) \right]$ being applied for $\ttbar$ MC. 
 \label{fig::BGestimation::DataMCPreselHardBT_rwgt1} }
\end{figure}

\begin{figure}[h]
  \centering
    \subfig{0.48}{figures/BGestimation/DataMCComparison/Preselection_hardLepBT/lep1Pt__Preselection_hardLepBT__rwgt_nJ007_ttPt007.pdf}{Lepton's $\pt$}
    \subfig{0.48}{figures/BGestimation/DataMCComparison/Preselection_hardLepBT/met__Preselection_hardLepBT__rwgt_nJ007_ttPt007.pdf}{$\met$}
    \subfig{0.48}{figures/BGestimation/DataMCComparison/Preselection_hardLepBT/mt__Preselection_hardLepBT__rwgt_nJ007_ttPt007.pdf}{$\mt$}
    \subfig{0.48}{figures/BGestimation/DataMCComparison/Preselection_hardLepBT/LepAplanarity__Preselection_hardLepBT__rwgt_nJ007_ttPt007.pdf}{$\Apl$}
    \caption{ Kinematical distribution of data (black dots) and MC (colored stack) in the \textbf{1LBT} pre-selection region, with the reweighting $w = 1.05 \times \left[ 1 - 0.061 \,\times p_T(\ttbar) \right]$ being applied for $\ttbar$ MC.  \label{fig::BGestimation::DataMCPreselHardBT_rwgt2} }
\end{figure}


%%%%%%%%%%%%%%
\begin{figure}[h]
  \centering
    \subfigure[]{\includegraphics[width=0.48\textwidth]{figures/BGestimation/DataMCComparison/Preselection_softLepBT/nJet30__Preselection_softLepBT.pdf}}
    \subfigure[]{\includegraphics[width=0.48\textwidth]{figures/BGestimation/DataMCComparison/Preselection_softLepBT/jet1Pt__Preselection_softLepBT.pdf}}
    \subfigure[]{\includegraphics[width=0.48\textwidth]{figures/BGestimation/DataMCComparison/Preselection_softLepBT/averageJetPt__Preselection_softLepBT.pdf}}
    \subfigure[]{\includegraphics[width=0.48\textwidth]{figures/BGestimation/DataMCComparison/Preselection_softLepBT/meffInc30__Preselection_softLepBT.pdf}}
    \caption{ Kinematical distribution of (a) Jet multiplicity ($p_T>30\gev$) (b) leading-jet pt  (c) average jet pt ($p_T>30\gev$)  (d) $\meffInc$ in the soft lepton b-tagged pre-selection region.  \label{fig::BGestimation::DataMCPreselSoftBT1} }
\end{figure}

\begin{figure}[h]
  \centering
    \subfigure[]{\includegraphics[width=0.48\textwidth]{figures/BGestimation/DataMCComparison/Preselection_softLepBT/lep1Pt__Preselection_softLepBT.pdf}}
    \subfigure[]{\includegraphics[width=0.48\textwidth]{figures/BGestimation/DataMCComparison/Preselection_softLepBT/met__Preselection_softLepBT.pdf}}
    \subfigure[]{\includegraphics[width=0.48\textwidth]{figures/BGestimation/DataMCComparison/Preselection_softLepBT/mt__Preselection_softLepBT.pdf}}
    \subfigure[]{\includegraphics[width=0.48\textwidth]{figures/BGestimation/DataMCComparison/Preselection_softLepBT/LepAplanarity__Preselection_softLepBT.pdf}}
    \caption{ Kinematical distribution of (a) leading-lepton pt (b) $\met$  (c) $\mt$  (d) $\apl$ in the soft lepton b-tagged pre-selection region.  \label{fig::BGestimation::DataMCPreselSoftBT2} }
\end{figure}
%%%%%%%%%%%%%

\clearpage

\noindent The mis-modeling in $\meffInc$ distribution is particularly concerning, given that the signal regions are designed to exploit its shape. The leading source of the mis-modeling is suspected to be in the description of ISR or FSR radiation. This is because hard jets ($\pt>200 \gev$) become more often non-$\ttbar$ origin in the tail of $\meffInc$, as demonstrated by Figure \ref{fig::BGestimation::ISRFrac_ttbar}, although $\ttbar$ does have 2-4 jets in its tree-level decay. \\

%%%%%%%%
\fig[100]{BGestimation/ISRFrac/ISRFrac_vs_meff.pdf}
{Fraction of ISR and FSR jets in the 4 leading jets with the largest transverse momenta, defined by the ratio of number of events where $i$-th leading that do not match either jets from $\ttbar$ decay by $\Delta R<0.2$ to all the events.}
{fig::BGestimation::ISRFrac_ttbar}
%%%%%%%%

This is in fact also supported by a series of MC reweighting studies shown in Figure \ref{fig::BGestimation::slope_rwgt}  where linear reweighting in various top kinematic variables is attempted to correct the the slope of data/MC in $\meffInc$. It turns that $p_{T}(\ttbar)$ is the variable most sensitive to the mis-modeling, while reweighting in other variables can only change the normalization but the slope. This strongly indicates that the primary problem is in the radiation recoiling the $\ttbar$ rather than in the internal kinematics of the $\ttbar$ system. 
The discrepancies in other variables is also shown to be recovered by the same $\ttPt$ reweighting in Figure \ref{fig::BGestimation::DataMCPreselHardBT_rwgt1}-\ref{fig::BGestimation::DataMCPreselHardBT_rwgt2} in appendix \ref{sec::Appendix::dataMC_rwgt}).
\\
%Therefore modeling of higher-order QCD/EW effects is again suspected as the cause of mis-modeling. \\
%Many explainatory higher-order effects have been proposed to account for the discrepancy, such as QCD-NNLO \cite{ttbar_NNLOQCD} or EW radiative correction \cite{ttbar_NLOEW}. \\
% top reweighiting の歴史
In contrast, the ``non-scaling'' variables such as $\mt$ and aplanarity look relatively well-modeled. Therefore, the same estimation strategy is taken as the case of $\wjets$ i.e. taking these as the extrapolating variables from CRs to SRs. \\

It is still acceptable though, note that the modeling of $\mt$ is not perfect. 
For instance in Figure \ref{fig::BGestimation::DataMCPreselHardBT1}, there is a small bump-like structure in the ratio plot around $\mt = 100\sim200\gev$ corresponding the cut-off of the semi-leptonic $\ttbar$. 
This is suspected to be due to the interference between $\ttbar+Wt\ra WWbb$ and other $WWbb$ diagrams which is not accounted by the generator, which effect is addressing in regions where bulk $\ttbar$ amplitude is suppressed. Corresponding uncertainty is evaluated in Sec. \ref{sec:Uncertainties::normalizedBG} and assigned as theory systematics. \\


\begin{figure}[h]
  \centering
    \subfig{0.48}{figures/BGestimation/slope_rwgt/slopes_rwgt_topPt.pdf}{}
    \subfig{0.48}{figures/BGestimation/slope_rwgt/slopes_rwgt_ttM.pdf}{}
    \subfig{0.48}{figures/BGestimation/slope_rwgt/slopes_rwgt_ttPt.pdf}{}
    \caption{ Response of data/MC in $\meffInc$ against a linear reweighting of $\ttbar$ events in terms of
 (a) average top transverse momentum ($(\pt(t)+\pt(\bar{t}))/2$), 
 (b) invariant mass of $\ttbar$ system ($m_{\ttbar}$) and
 (c) transverse momentum of $\ttbar$ system ($\pt(\ttbar)$). 
$\pt(\ttbar)$ is found to be sensitive to the slope of $\meffInc$ and improve the data/MC discrepancy, while the others are only capable of shifting the normalization. 
\label{fig::BGestimation::slope_rwgt} }
\end{figure}


%\clearpage
%The $p_T(\ttbar)$-based reweighting is found also capable of restoring the discrepancy in other distribution other than $\meffInc$. Applying the reweighting function optimum for correcting the $\meffInc$:
%\begin{align}
%w = 1.05 \times \left[ 1 - 0.061 \,\times p_T(\ttbar) \right] \label{eq::BGestimation::rwgt_ttPt},
%\end{align}
%good data-MC agreement is seen in overall spectra regarding to jets and MET as shown in Figure \ref{fig::BGestimation::DataMCPreselHardBT_rwgt1} - \ref{fig::BGestimation::DataMCPreselHardBT_rwgt2}. \\

%The mis-modleing in lepton transverse momentum distribution seems to have the other origin, seen as the residual discrepancy . \\

%This ends up a small discrepancy in $\mt$ as well, 
%This is thought to be related to the modeling of top polarization which is difficult when higher order diagrams are significantly involved. This \\


\clearpage
\paragraph{\underline{Di-leptonic $\ttbar$:}} \mbox{} \\
Figure \ref{fig::BGestimation::DataMCPresel2LBT1}-\ref{fig::BGestimation::DataMCPresel2LBT2} plot the kinematic distributions in the 2-lepton $b$-tagged pre-selection region (\textbf{2LBT}) where di-leptonically decaying $\ttbar$ dominates. \\

\begin{figure}[h]
  \centering
    \subfigure[]{\includegraphics[width=0.48\textwidth]{figures/BGestimation/DataMCComparison/Preselection_2LBT/nJet30__Preselection_2LBT.pdf}}
    \subfigure[]{\includegraphics[width=0.48\textwidth]{figures/BGestimation/DataMCComparison/Preselection_2LBT/jet1Pt__Preselection_2LBT.pdf}}
    \subfigure[]{\includegraphics[width=0.48\textwidth]{figures/BGestimation/DataMCComparison/Preselection_2LBT/averageJetPt__Preselection_2LBT.pdf}}
    \subfigure[]{\includegraphics[width=0.48\textwidth]{figures/BGestimation/DataMCComparison/Preselection_2LBT/meffInc30__Preselection_2LBT.pdf}}
    \caption{ Kinematical distribution of (a) Jet multiplicity ($p_T>30\gev$) (b) leading-jet pt  (c) average jet pt ($p_T>30\gev$)  (d) $\meffInc$ in the hard lepton b-tagged pre-selection region.  \label{fig::BGestimation::DataMCPresel2LBT1} }
\end{figure}

\begin{figure}[h]
  \centering
    \subfigure[]{\includegraphics[width=0.48\textwidth]{figures/BGestimation/DataMCComparison/Preselection_2LBT/lep1Pt__Preselection_2LBT.pdf}}
    \subfigure[]{\includegraphics[width=0.48\textwidth]{figures/BGestimation/DataMCComparison/Preselection_2LBT/met__Preselection_2LBT.pdf}}
    \subfigure[]{\includegraphics[width=0.48\textwidth]{figures/BGestimation/DataMCComparison/Preselection_2LBT/mt__Preselection_2LBT.pdf}}
    \subfigure[]{\includegraphics[width=0.48\textwidth]{figures/BGestimation/DataMCComparison/Preselection_2LBT/LepAplanarity__Preselection_2LBT.pdf}}
    \caption{ Kinematical distribution of (a) leading-lepton pt (b) $\met$  (c) $\mt$  (d) $\apl$ in the hard lepton b-tagged pre-selection region.  \label{fig::BGestimation::DataMCPresel2LBT2} }
\end{figure}
%%%%%%%%%%%%%

%\begin{figure}[h]
  \centering
    \subfigure[]{\includegraphics[width=0.48\textwidth]{figures/BGestimation/DataMCComparison/Preselection_2LBT/nJet30__Preselection_2LBT__rwgt_nJ007_ttPt007.pdf}}
    \subfigure[]{\includegraphics[width=0.48\textwidth]{figures/BGestimation/DataMCComparison/Preselection_2LBT/jet1Pt__Preselection_2LBT__rwgt_nJ007_ttPt007.pdf}}
    \subfigure[]{\includegraphics[width=0.48\textwidth]{figures/BGestimation/DataMCComparison/Preselection_2LBT/averageJetPt__Preselection_2LBT__rwgt_nJ007_ttPt007.pdf}}
    \subfigure[]{\includegraphics[width=0.48\textwidth]{figures/BGestimation/DataMCComparison/Preselection_2LBT/meffInc30__Preselection_2LBT__rwgt_nJ007_ttPt007.pdf}}
    \caption{ Kinematical distribution of (a) Jet multiplicity ($p_T>30\gev$) (b) leading-jet pt  (c) average jet pt ($p_T>30\gev$)  (d) $\meffInc$ in the hard lepton b-tagged pre-selection region, with the reweighting: $w = 1.05 \times \left[ 1 - 0.061 \,\times p_T(\ttbar) \right]$ (Eq.(\ref{eq::BGestimation::rwgt_ttPt})) being applied for $\ttbar$ MC.  \label{fig::BGestimation::DataMCPresel2LBT_rwgt1} }
\end{figure}

\begin{figure}[h]
  \centering
    \subfigure[]{\includegraphics[width=0.48\textwidth]{figures/BGestimation/DataMCComparison/Preselection_2LBT/lep1Pt__Preselection_2LBT__rwgt_nJ007_ttPt007.pdf}}
    \subfigure[]{\includegraphics[width=0.48\textwidth]{figures/BGestimation/DataMCComparison/Preselection_2LBT/met__Preselection_2LBT__rwgt_nJ007_ttPt007.pdf}}
    \subfigure[]{\includegraphics[width=0.48\textwidth]{figures/BGestimation/DataMCComparison/Preselection_2LBT/mt__Preselection_2LBT__rwgt_nJ007_ttPt007.pdf}}
    \subfigure[]{\includegraphics[width=0.48\textwidth]{figures/BGestimation/DataMCComparison/Preselection_2LBT/LepAplanarity__Preselection_2LBT__rwgt_nJ007_ttPt007.pdf}}
    \caption{ Kinematical distribution of (a) leading-lepton pt (b) $\met$  (c) $\mt$  (d) $\apl$ in the hard lepton b-tagged pre-selection region, with the reweighting: $w = 1.05 \times \left[ 1 - 0.061 \,\times p_T(\ttbar) \right]$ (Eq.(\ref{eq::BGestimation::rwgt_ttPt})) being applied for $\ttbar$ MC.  \label{fig::BGestimation::DataMCPresel2LBT_rwgt2} }
\end{figure}
%%%%%%%%%%%%%

\clearpage

The same trend is observed also in the di-leptonic channel as well; constant slopes in data/MC are seen in variables related to the jet activity (jet transverse momenta, $\meffInc$ distributions or MET etc.); the other non-scaling variables ($mt$, aplanarity etc.) are relatively nicely modeled by MC. 
\noindent It might worth noting that the mis-modeling in jet variables can also be corrected by the same $\ttPt$ reweighting as the semi-leptonic case (Figure \ref{fig::BGestimation::DataMCPresel2LBT_rwgt1}-\ref{fig::BGestimation::DataMCPresel2LBT_rwgt2} in appendix \ref{sec::Appendix::dataMC_rwgt}).
%It might worth noting that the slope in jet transverse momenta and $\meffInc$ can also be corrected by the same reweighting function Eq. \ref{eq::BGestimation::rwgt_ttPt} as the semi-leptonic case. Figure \ref{fig::BGestimation::DataMCPresel2LBT_rwgt1}-\ref{fig::BGestimation::DataMCPresel2LBT_rwgt2} show the distributions with the reweighting applied, where the data-MC discrepancy related to jet kinematics are fairly recovered, and the lepton transverse momentum enjoys much poorer restoration. 
This universality strongly implies that the cause of mis-modeling in $\ttbar$ is likely in the kinematics before the $W$-bosons decay, which is an important underlying assumption for the object replacement method as described later. \\
%One finds the residual mis-modeling after the reweighting on the lepton transverse momentum is larger than the semi-leptonic 
%One thing remarkable about this correction is that the same coefficiencies seems to be also optimal for the di-leptonic $\ttbar$. Di-leptonic pre-selection region is also available for testing the $\ttbar$ modeling. 
The behavior of the ``non-scaling'' variables is also largely similar to the case of semi-leptonic $\ttbar$. The only exception is $\mt$ that the  $\mt$ distribution for di-leptonic $\ttbar$ has no reason to cut-off at $\mt \sim m_W$, therefore it simply scales with lepton transverse momentum and MET. As a result, the $\mt$ distribution of di-leptonic $\ttbar$ is affected by the mis-modeling of jet kinematics. The emerging data/MC discrepancy can be seen in Figure \ref{fig::BGestimation::DataMCPresel2LBT2} (c). To avoid the impact by the mis-modeling in $\mt$, di-leptonic components are designed to be estimated by the other ``object replacement'' method as much as possible, and only small portion (``Out Acc.'' and ``Mis. OR'' in Table \ref{tab::BGestimation::BGclass}) of them is covered by the kinematical extrapolation. \\



%%%%%%%%%%%%%
\paragraph{\underline{$\ttbar$@3B:}} \mbox{} \\
Modeling of $tt+cc/bb$ and $\ttbarFakeB$ are exclusively examined using a preselected region with 3 or more $b$-jets (\textbf{1L3B}). Figure \ref{fig::BGestimation::DataMCPresel3B1} - \ref{fig::BGestimation::DataMCPresel3B2} displays the data-MC comparison in the region. While the shapes seem to be affected by the same type of mis-modeling as observed in inclusive $\ttbar$ selection above, the normalization is also underestimated by about $30\%$ which is thought to be due to the error of $\ttbar+cc/bb$ cross-section. \\
%Reweighting function
%\begin{align}
%w = 1.4 \times \left[ 1 - 0.061 \,\times p_T(\ttbar) \right] \label{eq::BGestimation::rwgt_ttPt3B},
%\end{align}
%is found to well correct the discrepancy, in which only the normalization coefficiency varies from Eq. \ref{eq::BGestimation::rwgt_ttPt} accounting for the cross-section correction for $tt+cc/bb$. Meanwhile, the invariance of the slope coefficiency implies that the source of the shape mis-modeling is common to the bulk $\ttbar$ component as seen above. \\

Despite the $\ttbar$ components in 3B regions suffer from such even more complex mis-modeling than the bulk, the impact on the final result is not dramatic since the majority of them are di-leptonic components in the SRs, therefore they are largely estimated by the object replacement method.

\begin{figure}[h]
  \centering
    \subfigure[]{\includegraphics[width=0.48\textwidth]{figures/BGestimation/DataMCComparison/Preselection_3B/nJet30__Preselection_3B.pdf}}
    \subfigure[]{\includegraphics[width=0.48\textwidth]{figures/BGestimation/DataMCComparison/Preselection_3B/jet1Pt__Preselection_3B.pdf}}
    \subfigure[]{\includegraphics[width=0.48\textwidth]{figures/BGestimation/DataMCComparison/Preselection_3B/averageJetPt__Preselection_3B.pdf}}
    \subfigure[]{\includegraphics[width=0.48\textwidth]{figures/BGestimation/DataMCComparison/Preselection_3B/meffInc30__Preselection_3B.pdf}}
    \caption{ Kinematical distribution of (a) Jet multiplicity ($p_T>30\gev$) (b) leading-jet pt  (c) average jet pt ($p_T>30\gev$)  (d) $\meffInc$ in the 3b-tagged pre-selection region.  \label{fig::BGestimation::DataMCPresel3B1} }
\end{figure}

\begin{figure}[h]
  \centering
    \subfigure[]{\includegraphics[width=0.48\textwidth]{figures/BGestimation/DataMCComparison/Preselection_3B/lep1Pt__Preselection_3B.pdf}}
    \subfigure[]{\includegraphics[width=0.48\textwidth]{figures/BGestimation/DataMCComparison/Preselection_3B/met__Preselection_3B.pdf}}
    \subfigure[]{\includegraphics[width=0.48\textwidth]{figures/BGestimation/DataMCComparison/Preselection_3B/mt__Preselection_3B.pdf}}
    \subfigure[]{\includegraphics[width=0.48\textwidth]{figures/BGestimation/DataMCComparison/Preselection_3B/LepAplanarity__Preselection_3B.pdf}}
    \caption{ Kinematical distribution of (a) leading-lepton pt (b) $\met$  (c) $\mt$  (d) $\apl$ in the  3b-tagged pre-selection region.  \label{fig::BGestimation::DataMCPresel3B2} }
\end{figure}

\begin{figure}[h]
  \centering
    \subfigure[]{\includegraphics[width=0.48\textwidth]{figures/BGestimation/DataMCComparison/Preselection_3B/min_dPhi_4j__Preselection_3B.pdf}}
    \subfigure[]{\includegraphics[width=0.48\textwidth]{figures/BGestimation/DataMCComparison/Preselection_3B/topNess__Preselection_3B.pdf}}
    \caption{ Kinematical distribution of (a) $\mindPhiFourJet$ (b) $\topNess$ in the  3b-tagged pre-selection region.  \label{fig::BGestimation::DataMCPresel3B3} }
\end{figure}


%\begin{figure}[h]
  \centering
    \subfig{0.47}{figures/BGestimation/DataMCComparison/Preselection_3B/nJet30__Preselection_3B__rwgt_ttPt007.pdf}{Jet multiplicity}
    \subfig{0.47}{figures/BGestimation/DataMCComparison/Preselection_3B/jet1Pt__Preselection_3B__rwgt_ttPt007.pdf}{$\pt$ of the leading jet}
    \subfig{0.47}{figures/BGestimation/DataMCComparison/Preselection_3B/averageJetPt__Preselection_3B__rwgt_ttPt007.pdf}{Average $\pt$ of jets with $\pt>30\gev$}
    \subfig{0.47}{figures/BGestimation/DataMCComparison/Preselection_3B/meffInc30__Preselection_3B__rwgt_ttPt007.pdf}{$\meffInc (\meffDef)$}
    \caption{ Kinematical distribution of data (black dots) and MC (colored stack) in the \textbf{1L3B} pre-selection region, with the reweighting $w = 1.4 \times \left[ 1 - 0.061 \,\times p_T(\ttbar) \right]$ being applied for $\ttbar$ MC.  \label{fig::BGestimation::DataMCPresel3B_rwgt1} }
\end{figure}

\begin{figure}[h]
  \centering
    \subfig{0.41}{figures/BGestimation/DataMCComparison/Preselection_3B/lep1Pt__Preselection_3B__rwgt_ttPt007.pdf}{Lepton's $\pt$}
    \subfig{0.41}{figures/BGestimation/DataMCComparison/Preselection_3B/met__Preselection_3B__rwgt_ttPt007.pdf}{$\met$}
    \subfig{0.41}{figures/BGestimation/DataMCComparison/Preselection_3B/mt__Preselection_3B__rwgt_ttPt007.pdf}{$\mt$}
    \subfig{0.41}{figures/BGestimation/DataMCComparison/Preselection_3B/LepAplanarity__Preselection_3B__rwgt_ttPt007.pdf}{$\Apl$}
    \subfig{0.41}{figures/BGestimation/DataMCComparison/Preselection_3B/min_dPhi_4j__Preselection_3B__rwgt_ttPt007.pdf}{$\mindPhiFourJet$}
    \subfig{0.41}{figures/BGestimation/DataMCComparison/Preselection_3B/topNess__Preselection_3B__rwgt_ttPt007.pdf}{Topness}
    \caption{ Kinematical distribution of data (black dots) and MC (colored stack) in the \textbf{1L3B} pre-selection region, with the reweighting: $w = 1.4 \times \left[ 1 - 0.061 \,\times p_T(\ttbar) \right]$ being applied for $\ttbar$ MC.  \label{fig::BGestimation::DataMCPresel3B_rwgt2} }
\end{figure}

\clearpage


%\section{Dibosons}
%\begin{figure}[h]
  \centering
    \subfigure[]{\includegraphics[width=0.48\textwidth]{figures/BGestimation/DataMCComparison/Preselection_2LDB/nJet30__Preselection_2LDB.pdf}}
    \subfigure[]{\includegraphics[width=0.48\textwidth]{figures/BGestimation/DataMCComparison/Preselection_2LDB/jet1Pt__Preselection_2LDB.pdf}}
    \subfigure[]{\includegraphics[width=0.48\textwidth]{figures/BGestimation/DataMCComparison/Preselection_2LDB/averageJetPt__Preselection_2LDB.pdf}}
    \subfigure[]{\includegraphics[width=0.48\textwidth]{figures/BGestimation/DataMCComparison/Preselection_2LDB/meffInc30__Preselection_2LDB.pdf}}
    \caption{ Kinematical distribution of (a) Jet multiplicity ($p_T>30\gev$) (b) leading-jet pt  (c) average jet pt ($p_T>30\gev$)  (d) $\meffInc$ in the hard lepton b-tagged pre-selection region.  \label{fig::BGestimation::DataMCPresel2LDB1} }
\end{figure}

\begin{figure}[h]
  \centering
    \subfigure[]{\includegraphics[width=0.48\textwidth]{figures/BGestimation/DataMCComparison/Preselection_2LDB/lep1Pt__Preselection_2LDB.pdf}}
    \subfigure[]{\includegraphics[width=0.48\textwidth]{figures/BGestimation/DataMCComparison/Preselection_2LDB/met__Preselection_2LDB.pdf}}
    \subfigure[]{\includegraphics[width=0.48\textwidth]{figures/BGestimation/DataMCComparison/Preselection_2LDB/mt__Preselection_2LDB.pdf}}
    \subfigure[]{\includegraphics[width=0.48\textwidth]{figures/BGestimation/DataMCComparison/Preselection_2LDB/LepAplanarity__Preselection_2LDB.pdf}}
    \caption{ Kinematical distribution of (a) leading-lepton pt (b) $\met$  (c) $\mt$  (d) $\apl$ in the hard lepton b-tagged pre-selection region.  \label{fig::BGestimation::DataMCPresel2LDB2} }
\end{figure}
%%%%%%%%%%%%%

%\input{tex/BGestimation/fig_DataMCPresel2LDB_rwgt.tex}


\subsection{Definition of Control Regions and Validation Regions} \label{sec::BGestimation::CRdef}
The key assumption in this method is that the relative modeling of MC between CRs and SRs is correct. 
In other words, CRs and SRs need to suffer from the same extent of MC mis-modeling, 
so that the normalization in CRs can be compatible for SRs. 
Therefore, the most important requirement in CR definition is capturing the similar phase space with respect to the corresponding SR in terms of the mis-modeling. \\

The easiest realization of CR is to revert the SR cuts in kinematical variables well-modeled by MC. 
In this analysis, $\mt$, $\apl$ and topness (and also $\mindPhiFourJet$ for the \textbf{3B} tower) are chosen as such variables. 
A exception is in the \textbf{2J} tower where $\met$ is used instead of $\apl$, since $\apl$ is not used in the signal region definition. \\
%since aplanarity is not used in definition. 
%The modeling of $\met$ is acceptable, since the ISR/FSR contribution in the \textbf{2J} tower is relatively low due to the loose selection in number of jets. \\ \\
%topness, min_dPhi_4j

\noindent A couple of minor modifications follow based on the supplemental requirements below:
\begin{itemize}
\item CR statistics have to be sufficient. \\
Typically, about 10 times more data statistics in CRs with respect to SRs are desired for a stable correction particularly in cases where multiple components are corrected simultaneously (in this analysis, $\wjets$ and the top background i.e. $\ttbar+Wt$). 
For this sake, cuts in variables not fatally sensitive to the mis-modeling is loosened in some of the CRs.
%, even at some cost of being hit by the mis-modeling.
$\met$ is for example always a good candidate to loosen for the large gain in statistics at a relatively low cost of the mis-modeling.
%Although it is affected by the mis-modeling through jet transverse momenta which is known to be the most ill-modeled, 
\footnote{Because the influence of the mis-modeling is diluted through the vectoral summation of them, instead of the scalar sum. }
$\metOverMeff$ is also loosened in \textbf{2J} and $\bHighx$ since it is in a form of ratio which is relatively robust against the mis-modeling.
On the other hand, it is promised that $\nJet$ and $\meffInc$ are never touched since they are critical to the mis-modeling. 
%
\item A lower cut in $\mt$ is needed to reduce the contribution from backgrounds with fake leptons. \\
Low-$\mt$ regions are typically more contaminated by events with fake leptons.
As the MC modeling on the fake rate is generally less reliable, $\mt > 30 \sim 40 \gev$ is applied in CRs. 
\end{itemize}

CRs are defined for each SR bin independently, except that they are shared between $b$-tagged and $b$-vetoed SR bins. Instead, each CR is divided into $b$-tagged (``TR'') an $b$-vetoed bin (``WR'') in which different background components are controled. The correction is performed by normalizing the MC of  $\wjets$ and the top background ($\ttbar$ and single-top) to the data in CRs, while raw MC prediction is quoted for the other minor backgrounds. \\

There are the third type of regions referred as ``validation regions'' designed to validitate the estimation procedure by comparing with the data. They are typically set in between a SR and a CR, where at least one of the cuts in terms of the extrapolation variable is released with respect to the SR. 
Two types of VR is prepared; ``VRa'', in which the $\mt$ cut is relaxed validating the extrapolation in $\mt$; ``VRb'', where the other extrapolation variables (mainly $\apl$ and topness. $\met$ for \textbf{2J}) are validated. A upper cut in $\mt$ is placed in some VRa to suppress the signal contamination. VRs-QCD are defined additionally to examine the contribution from QCD multi-jet processes in SRs which is supposedly negligible, which is detailed in Sec. \ref{sec::BGestimation::VRQCD} in the appendix. \\

The finalized CRs and VRs are summarized together with the corresponding SRs in Table \ref{SRdefinition::regionDef2J} - \ref{SRdefinition::regionDef3B}, which are graphically schematized by Figure \ref{fig::BGestimation::regionsPlot}. While SRs are carefully designed to be orthogonal to CRs and VRs, it is allowed to have overlap between CRs and VRs once the CRs are found to have much larger statistics than that of the VRs so that the overlapped events have no influence to the normalization. For instance, CR and VRa are overlapped in \textbf{3B}. This is intended to boost the CR statistics, while the number of events in VRa is small enough so that they are still nearly statistically independent. \\

\clearpage
%%%%%%%%%%%%%%%%%%%%
\begin{figure}
  \centering
    \subfig{0.39}{figures/BGestimation/regionsPlot/regionsPlot_myAna_2J.pdf}{}
    \subfig{0.39}{figures/BGestimation/regionsPlot/regionsPlot_myAna_6J.pdf}{}
    \subfig{0.39}{figures/BGestimation/regionsPlot/regionsPlot_myAna_Lowx.pdf}{}
    \subfig{0.39}{figures/BGestimation/regionsPlot/regionsPlot_myAna_Highx.pdf}{}
    \subfig{0.39}{figures/BGestimation/regionsPlot/regionsPlot_myAna_3B.pdf}{}
 \caption{ Schematics of CR/VR/SR in each signal region tower. Two major extrapolation variables are chosen as $x$- or $y$-axis to illustrate the difference between the regions. 
Extrapolation in the other variables are explicitly mentioned in the label. 
Note that each CR in the \textbf{3B} tower contains the VRa in it. 
%to boost the CR statistics.
%, which does not disturb the role of validation given that it has 5 times more statistics as that of VRa. 
   \label{fig::BGestimation::regionsPlot} 
 }
\end{figure}

\clearpage

%%%%%%%%%%%%%%%%%%%%%%%%%%%%%%%%%%%%%%%%%%%%%%%%%%%%%%%%%%
%\subsection{Signal contamination}
%\input{tex/BGestimation/fig_sigContami_kineExtp.tex}


%%%%%%%%%%%%%%%%%%%%%%%%%%%%%%%%%%%%%%%%%%%%%%%%%%%%%%%%%%
%\subsection{Evaluation of the Extrapolation Error} \label{sec::BGestimation::nonClosure_kineExtp}
Although the good modeling on the extrapolation variables is confirmed in the pre-selection regions, 
the question is whethere they are still good when the signal region selections are applid.
In fact, there is correlation between the well-modeled variables ($\mt$, $\apl$ etc.) and the ill-modeled ones ($\nJetNoGev$, jet transverse momenta, $\meffInc$ etc.) that are not evident at preselection level, however could be addressing in some particular phase space. 
The extrapolation is also affected by the loosened cuts in variables that are already known to be poorly modeled such as MET, so the associated uncertainty needs be quantified. \\

In this sub-section, the extrapolation error is evaluated by injecting an artificial variation in MC compared to the observed MC mis-modeling, and then measure the yield change in a CR and the corresponding SR. Ideally, they show the same response against the injected variation, so that the normalization in CR can perfectly compensate the effect of mis-modeling in SR. Otherwise, the relative difference in their yield variation directly corresponds to the amount of extrapolation error. \\

Figure \ref{fig::BGestimation::valid_extp_2J} - \ref{fig::BGestimation::valid_extp_3B} present the results where the $\wjets$ and $\ttbar$ MC are varied by reweighting the events with:
\begin{align}
 w & = 1 - x \times (\nJetNoGev-2), \mbox{\phantom{MMMM}}\,\,\,\,\,\, x \in [0,0.18]  \mbox{\phantom{MMMM}} (\wjets) \nn  \\
 w & = 1 - x \,\times p_T(\ttbar)/100\gev, \,\,\,\,\,\,\,\,           x \in [0,0.09]  \mbox{\phantom{MMMM}} (\ttbar),
\label{eq::BGestimation::injected_MCvariation}
\end{align}
respectively. The vertical axis on the top panels show the amount of relative change that CR or SR experience by the injected MC variation as a function of $x$. The relative variation in CR (orange) compares to the normalization factor actually obtained via the fit to data, while that in SR (blue) to the ideal normalization factor need to fully correct the SR. The bottom panel display the ratio, namely the resultant extrapolation error. The realistic $x$ is approximately $x_W=0.1$ and $x_{\ttbar}=0.06$ for $\wjets$ and $\ttbar$ respectively, based on the observation of data/MC in Sec. \ref{sec::BGestimation::dataMC} as Eq. (\ref{eq::BGestimation::rwgt_nJ}), (\ref{eq::BGestimation::rwgt_ttPt}) and (\ref{eq::BGestimation::rwgt_ttPt3B}). B-tagging requirement is removed to maintain sufficient statistics, assuming the kinematics are invariant with it. For the $\ttbar$ process, component estimated by the object replacement method is excluded from the test. The extrapolation error from CRs to corresponding VRs are shown in Appendix Sec.\ref{sec::App::valid_extp_VR}. \\

Observed extrapolation error is generally small, which stays within 10$\%$ (20$\%$) for $\wjets$ ($\ttbar$) at the reference magnitude of mis-modeling ($x_W=0.1, \,\,\, x_{\ttbar}=0.06$). These are quoted as systematics error associated with the method in the fit, which is summarized in Table \ref{tab::Uncertainties::noClosure_kineExtp} of Sec. \ref{sec::Uncertainties::nonClosure}. \\

%
% note that このtestはmt/apl extraplationで最大のIssueであるjetのMis-modelingがどれくらい解消できるかというtestであり、本質的にmt spectraとmis-modelingがどれくらい相関しているかを見ているにすぎない。
%in other preselectionにおけるmtの記述が完璧だとした場合にjet selectionかけたときにどれくらいmt. aplの分布がズレるかというtestである. 
% 
%このtestでevaluateしたerrorは手法に由来するsystematicとしてassignするが、jet以外の要因によるMT spectraのvariationは別個評価する必要がある. (instrumental, theory etc.)
%これらは7章でevaluateする

Note that this check is only quantifying the error in the methodology; how much the MC mis-modeling in terms of jet activities can be corrected if the other MC description is perfect.
In reality, the shape of $\mt$ or $\apl$ distribution can be varies by the other reasons, and the uncertainties have to be assigned additionally. This will be discussed in Sec. \ref{sec::Uncertainties}.

\clearpage
%\begin{figure}[h]
  \centering
    \subfigure[]{\includegraphics[width=0.48\textwidth]{figures/BGestimation/valid_extp/SFTF_Wjets_Preselection_meff1500_extp_mt__nJet30.pdf}}
    \subfigure[]{\includegraphics[width=0.48\textwidth]{figures/BGestimation/valid_extp/SFTF_Wjets_Preselection_meff1500_extp_LepAplanarity__nJet30.pdf}}
    \caption{ Closure error for $\wjets$ process by the extrapolation using (a) $\mt$ (b) $\apl$ as function of the magnitude of injected linear mis-modeling on $\nJet$: $y_{\ttbar} = 1.0 - x \times \nJet)$. All evaluated by MC.   \label{fig::BGestimation::valid_extp1} }
\end{figure}

%%%%%%%%%%%%%%%%
\begin{figure}[h]
  \centering
    \subfigure[]{\includegraphics[width=0.48\textwidth]{figures/BGestimation/valid_extp/SFTF_Wjets_Preselection_meff1500_extp_mt__WPt.pdf}}
    \subfigure[]{\includegraphics[width=0.48\textwidth]{figures/BGestimation/valid_extp/SFTF_Wjets_Preselection_meff1500_extp_LepAplanarity__WPt.pdf}}
    \caption{ Closure error for $\wjets$ process by the extrapolation using (a) $\mt$ (b) $\apl$ as function of the magnitude of injected linear mis-modeling on $\nJet$: $y_{\ttbar} = 1.0 - x \times \nJet)$. All evaluated by MC.   \label{fig::BGestimation::valid_extp2} }
\end{figure}


%%%%%%%%%%%%%%%%
\begin{figure}[h]
  \centering
    \subfigure[]{\includegraphics[width=0.48\textwidth]{figures/BGestimation/valid_extp/SFTF_tt_Preselection_meff1500_extp_mt__ttPt.pdf}}
    \subfigure[]{\includegraphics[width=0.48\textwidth]{figures/BGestimation/valid_extp/SFTF_tt_Preselection_meff1500_extp_LepAplanarity__ttPt.pdf}}
    \caption{ Closure error for $\ttbar$ process by the extrapolation using (a) $\mt$ (b) $\apl$ as function of the magnitude of injected linear mis-modeling on $p_T(\ttbar)$: $y_{\ttbar} = 1.0 - x \times p_T(\ttbar))$. All evaluated by MC.   \label{fig::BGestimation::valid_extp1} }
\end{figure}

%%%%%%%%%%%%%%%%
\begin{figure}[h]
  \centering
    \subfigure[]{\includegraphics[width=0.48\textwidth]{figures/BGestimation/valid_extp/SFTF_tt_Preselection_meff1500_extp_mt__meff.pdf}}
    \subfigure[]{\includegraphics[width=0.48\textwidth]{figures/BGestimation/valid_extp/SFTF_tt_Preselection_meff1500_extp_LepAplanarity__meff.pdf}}
    \caption{ Closure error for $\ttbar$ process by the extrapolation using (a) $\mt$ (b) $\apl$ as function of the magnitude of injected linear mis-modeling on $meffInc$: $y_{\ttbar} = 1.0 - x \times \meffInc)$. All evaluated by MC.   \label{fig::BGestimation::valid_extp1} }
\end{figure}



%%%%%%%%%%%%%%%% SR2J
\begin{figure}[h]
  \centering
    \subfig{0.45}{figures/BGestimation/valid_extp/SFTF_wjets_SR2JMEFF1_extp_var2J__nJet30.pdf}{SR/CR 2J $\meffIncFirst$ for $\wjets$}
    \subfig{0.45}{figures/BGestimation/valid_extp/SFTF_ttNoObjRep_SR2JMEFF1_extp_var2J__ttPt.pdf}{SR/CR 2J $\meffIncFirst$ for $\ttbar$}
    \subfig{0.45}{figures/BGestimation/valid_extp/SFTF_wjets_SR2JMEFF2_extp_var2J__nJet30.pdf}{SR/CR 2J $\meffIncSecond$ for $\wjets$}
    \subfig{0.45}{figures/BGestimation/valid_extp/SFTF_ttNoObjRep_SR2JMEFF2_extp_var2J__ttPt.pdf}{SR/CR 2J $\meffIncSecond$ for $\ttbar$}
    \subfig{0.45}{figures/BGestimation/valid_extp/SFTF_wjets_SR2JMEFF3_extp_var2J__nJet30.pdf}{SR/CR 2J $\meffIncThird$ for $\wjets$}
    \subfig{0.45}{figures/BGestimation/valid_extp/SFTF_ttNoObjRep_SR2JMEFF3_extp_var2J__ttPt.pdf}{SR/CR 2J $\meffIncThird$ for $\ttbar$}
 \caption{Extrapolation error in SR/CR \textbf{2J} evaluated by reweighting the MC with Eq. (\ref{eq::BGestimation::injected_MCvariation}). }
  \label{fig::BGestimation::valid_extp_2J} 
\end{figure}



%%%%%%%%%%%%%%%% SR6J
\begin{figure}[h]
  \centering
    \subfig{0.488}{figures/BGestimation/valid_extp/SFTF_wjets_SR6JMEFF1_extp_var6J__nJet30.pdf}{SR/CR 6J $\meffIncFirst$ for $\wjets$}
    \subfig{0.488}{figures/BGestimation/valid_extp/SFTF_ttNoObjRep_SR6JMEFF1_extp_var6J__ttPt.pdf}{SR/CR 6J $\meffIncFirst$ for $\ttbar$}
    \subfig{0.488}{figures/BGestimation/valid_extp/SFTF_wjets_SR6JMEFF2_extp_var6J__nJet30.pdf}{SR/CR 6J $\meffIncSecond$ for $\wjets$}
    \subfig{0.488}{figures/BGestimation/valid_extp/SFTF_ttNoObjRep_SR6JMEFF2_extp_var6J__ttPt.pdf}{SR/CR 6J $\meffIncSecond$ for $\ttbar$}
    \subfig{0.488}{figures/BGestimation/valid_extp/SFTF_wjets_SR6JMEFF3_extp_var6J__nJet30.pdf}{SR/CR 6J $\meffIncThird$ for $\wjets$}
    \subfig{0.488}{figures/BGestimation/valid_extp/SFTF_ttNoObjRep_SR6JMEFF3_extp_var6J__ttPt.pdf}{SR/CR 6J $\meffIncThird$ for $\ttbar$}
 \caption{Extrapolation error in SR/CR \textbf{6J} evaluated by reweighting the MC with Eq. (\ref{eq::BGestimation::injected_MCvariation}). }
\label{fig::BGestimation::valid_extp_6J} 
\end{figure}



%%%%%%%%%%%%%%%% Lowx/Highx/3B
\begin{figure}[h]
  \centering
    \subfig{0.46}{figures/BGestimation/valid_extp/SFTF_wjets_SRLowx_extp_varLowx__nJet30.pdf}{SR/CR Low-x for $\wjets$}
    \subfig{0.46}{figures/BGestimation/valid_extp/SFTF_ttNoObjRep_SRLowx_extp_varLowx__ttPt.pdf}{SR/CR Low-x for $\ttbar$}
    \subfig{0.46}{figures/BGestimation/valid_extp/SFTF_wjets_SRHighx_extp_varHighx__nJet30.pdf}{SR/CR High-x for $\wjets$}
    \subfig{0.46}{figures/BGestimation/valid_extp/SFTF_ttNoObjRep_SRHighx_extp_varHighx__ttPt.pdf}{SR/CR High-x for $\ttbar$}
    \subfig{0.46}{figures/BGestimation/valid_extp/SFTF_wjets_SR3BMEFF1_extp_var3B__nJet30.pdf}{SR/CR 3B $\meffIncFirst$ for $\wjets$}
    \subfig{0.46}{figures/BGestimation/valid_extp/SFTF_ttNoObjRep_SR3BMEFF1_extp_var3B__ttPt.pdf}{SR/CR 3B $\meffIncFirst$ for $\ttbar$}
    \subfig{0.46}{figures/BGestimation/valid_extp/SFTF_wjets_SR3BMEFF2_extp_var3B__nJet30.pdf}{SR/CR 3B $\meffIncSecond$ for $\wjets$}
    \subfig{0.46}{figures/BGestimation/valid_extp/SFTF_ttNoObjRep_SR3BMEFF2_extp_var3B__ttPt.pdf}{SR/CR 3B $\meffIncSecond$ for $\ttbar$}
 \caption{Extrapolation error in SR/CR (a)(b) \textbf{Low-x}, (c)(d) \textbf{High-x} and (e-h) \textbf{3B} evaluated by reweighting the MC with Eq. (\ref{eq::BGestimation::injected_MCvariation}). }
\label{fig::BGestimation::valid_extp_Varx3B} 
\end{figure}






%\clearpage

%%%%%%%%%%%%%%%%%%%%%%%%%%%%%%%%%%%%%%%%%%%%%%%%%%%%%%%%%%
\subsection{Result of the Normalization} \label{sec::BGestimation::kineExtp::result}
The normalization factors are determined by a simultaneous fit on the WR and TR in which $\wjets$ and $\ttbar$ is dominant respectively. During the fit, all the normalization factors and nuisance parameters characterizing theoretical and experimental systematics are allowed to flow. The detail of the statistical procedure is described in Sec. \ref{sec::Result::statistics}. \\

The data yields in control regions are shown in Table \ref{tab::BGestimation::CRyields_2J} - \ref{tab::BGestimation::CRyields_3B}, with the pre-fit and post-fit prediction by MC. Note that only $\wjets$ and top backgrounds ($\ttbar$ and single-top) are normalized while the yield of the other processes are kept constant during the fit. The effect of signal contamination in control regions is negligible. \\

Fitted normalization factors are summarized in Figure \ref{fig::BGestimation::fittedSFs}. 
Generally small normalization factors are observed in bins with high $\meffInc$, 
reflecting the fact that MC is overpredicting in phase space with hard kinematics. 
%
The normalization factor is about $0.4$ in the worst case, corresponding the an error of $150\%$, 
while the post-fit uncertainty is typically $20\sim40\%$, 
demonstrating a successful correction. 

%\clearpage
%どのregionでもreweightingで修正したものと大体一致
%reweightingがhard-seelection掛けた後のphase spaceでのmis-modelingもcatchできていることの証明
%上でも言ったが一番大事なのはmis modelingがなんのせいかある程度把握してるということ。
%逆に一番やばいのは複数の逆にcontributeしてるmis-modelingがキャンセルすること。
%なのでCR fitの小さいNormalizaiton factorというのは確かにdisasterではあるが、
%このnJet rwgtだけで大体の説明がついてしまうというのはかなり良い兆候で、仮にcalcelする成分が仮にいたとしてもnJetのMis-modelingに比べれば大したことがないということが言える


%---------------------- Fitted normalization factors ------------------- 
\begin{figure}[h]
  \begin{center}
    \includegraphics[width=160mm]{figures/BGestimation/fittedSFs/SFs.pdf}
    \captionof{figure}{Fitted normalization factors for $\wjets$ and the top background ($\ttbar$ plus single-top). The error bars represent the combined systematic and statistical uncertainties. }
    \label{fig::BGestimation::fittedSFs}
  \end{center}
\end{figure}
%-------------------------------

The post-fit distributions for variables used in the extrapolation in each region are shown in Figure \ref{fig::BGestimation::CRpostFit::WR2J}-\ref{fig::BGestimation::CRpostFit::TR3B} in the appendix. 

% data/MCにdescrepancy trendが見られるものがチラホラある
%
% 2JのMET

%
% W+jetsもttbarもmtが若干怪しい
%



\clearpage
\newcommand{\crtablecaption}[1]{%
      Event yields and the background-only fit results in the #1 control regions. 
      Each column corresponds to a bin in $\meffInc$.
      Uncertainties in the MC estimates combine statistical (in the simulated event yields) and systematic uncertainties discussed in Sec \ref{sec::Uncertainties}. The uncertainties in this table are symmetrised for propagation purposes but truncated at zero to remain within the physical boundaries.
}
\newcommand{\crtablecaptionvarx}{%
      Event yields and the background-only fit results in the ``Low-x'' and ``High-x'' control regions. 
      Uncertainties in the MC estimates combine statistical (in the simulated event yields) and systematic uncertainties discussed in Sec \ref{sec::Uncertainties}. The uncertainties in this table are symmetrised for propagation purposes but truncated at zero to remain within the physical boundaries.
}
\begin{table}
  \begin{center}
    \caption{ \label{tab::BGestimation::VRyields_2J} \crtablecaption{``2J''}  }

    \begin{tabular*}{\textwidth}{@{\extracolsep{\fill}}lrrrr}
      \toprule
      \textbf{WR 2J} & $m_{\mathrm{eff.}}\in$[1100,1500] & $m_{\mathrm{eff.}}\in$[1500,1900] & $m_{\mathrm{eff.}}>1900$ \\
      \midrule

Observed data          & $620$              & $127$              & $17$                    \\
\midrule
\midrule
MC total (post-fit)         & $620.06 \pm 24.93$          & $126.89 \pm 11.28$          & $17.01 \pm 4.14$              \\
\midrule
        $W$+jets         & $462.0 \pm 34.1$          & $99.7 \pm 12.6$          & $12.6 \pm 4.4$              \\
        $Z$+jets         & $14.3 \pm 3.9$          & $2.6 \pm 0.7$          & $0.5 \pm 0.1$              \\
        Tops         & $100.9 \pm 17.1$          & $14.9 \pm 3.2$          & $2.6 \pm 0.9$              \\
        Di-boson         & $41.7 \pm 13.7$          & $9.3 \pm 3.6$          & $1.3 \pm 0.4$              \\
        $t\bar{t}+V$         & $1.2 \pm 0.3$          & $0.3 \pm 0.1$          & $0.1 \pm 0.0$              \\
\midrule
\midrule
MC total (pre-fit)              & $859.70 \pm 30.91$          & $177.49 \pm 7.33$          & $32.40 \pm 1.67$              \\
\midrule
        $W$+jets         & $703.35 \pm 19.15$          & $143.40 \pm 4.39$          & $26.19 \pm 1.22$              \\
        $Z$+jets         & $14.26 \pm 3.92$          & $2.58 \pm 0.72$          & $0.45 \pm 0.13$              \\
        Tops         & $99.27 \pm 13.32$          & $21.88 \pm 3.16$          & $4.42 \pm 0.75$              \\
        Di-boson         & $41.63 \pm 13.69$          & $9.32 \pm 3.55$          & $1.26 \pm 0.44$              \\
        $t\bar{t}+V$         & $1.18 \pm 0.25$          & $0.30 \pm 0.07$          & $0.06 \pm 0.02$              \\
    \bottomrule
    %%
    \end{tabular*}


\begin{tabular*}{\textwidth}{@{\extracolsep{\fill}}lrrrr}
  \toprule
  \textbf{TR 2J} & $m_{\mathrm{eff.}}\in$[1100,1500] & $m_{\mathrm{eff.}}\in$[1500,1900] & $m_{\mathrm{eff.}}>1900$ \\
\midrule

Observed data          & $972$              & $150$              & $22$                    \\
\midrule
\midrule
MC total (post-fit)         & $971.82 \pm 31.18$          & $150.01 \pm 12.27$          & $22.00 \pm 4.71$              \\
\midrule
        $W$+jets         & $99.5 \pm 35.0$          & $23.2 \pm 8.4$          & $3.3 \pm 1.7$              \\
        $Z$+jets         & $3.9 \pm 1.0$          & $0.9 \pm 0.2$          & $0.2 \pm 0.1$              \\
        Tops         & $846.1 \pm 48.2$          & $120.2 \pm 15.3$          & $17.4 \pm 5.2$              \\
        Di-boson         & $11.9 \pm 4.4$          & $2.7 \pm 0.9$          & $0.7 \pm 0.3$              \\
        $t\bar{t}+V$         & $10.3 \pm 1.8$          & $3.1 \pm 0.6$          & $0.4 \pm 0.1$              \\
\midrule
\midrule
MC total (pre-fit)              & $1009.13 \pm 52.94$          & $216.02 \pm 11.81$          & $37.88 \pm 2.67$              \\
\midrule
        $W$+jets         & $151.50 \pm 48.57$          & $33.30 \pm 10.57$          & $6.91 \pm 2.25$              \\
        $Z$+jets         & $3.86 \pm 1.05$          & $0.85 \pm 0.23$          & $0.17 \pm 0.05$              \\
        Tops         & $831.49 \pm 14.93$          & $176.11 \pm 4.06$          & $29.70 \pm 1.15$              \\
        Di-boson         & $11.94 \pm 4.35$          & $2.67 \pm 0.92$          & $0.68 \pm 0.27$              \\
        $t\bar{t}+V$         & $10.34 \pm 1.81$          & $3.08 \pm 0.58$          & $0.43 \pm 0.11$              \\
    \bottomrule
    %%
    \end{tabular*}
    
  \end{center}
\end{table}


\begin{table}
  \begin{center}
    \caption{ \label{tab::BGestimation::VRyields_6J} \crtablecaption{``6J''}  }

    \begin{tabular*}{\textwidth}{@{\extracolsep{\fill}}lrrrr}
      \toprule
      \textbf{WR 6J} &  $m_{\mathrm{eff.}}\in$[1100,1600] & $m_{\mathrm{eff.}}\in$[1600,2100] & $m_{\mathrm{eff.}}>2100$ \\
      \midrule

Observed data          & $248$              & $120$              & $53$                    \\
\midrule
\midrule
MC total (post-fit)         & $248.06 \pm 15.84$          & $120.02 \pm 11.21$          & $52.98 \pm 7.30$              \\
\midrule
        $W$+jets         & $147.5 \pm 22.0$          & $83.3 \pm 13.9$          & $30.6 \pm 9.2$              \\
        $Z$+jets         & $2.5 \pm 1.0$          & $1.1 \pm 0.5$          & $0.7 \pm 0.3$              \\
        Tops         & $71.7 \pm 11.5$          & $22.9 \pm 4.4$          & $14.3 \pm 3.1$              \\
        Di-boson         & $25.3 \pm 7.5$          & $12.1 \pm 5.8$          & $7.1 \pm 3.8$              \\
        $t\bar{t}+V$         & $1.1 \pm 0.2$          & $0.6 \pm 0.2$          & $0.3 \pm 0.1$              \\
\midrule
\midrule
MC total (pre-fit)              & $408.20 \pm 19.21$          & $192.94 \pm 10.30$          & $112.45 \pm 7.11$              \\
\midrule
        $W$+jets         & $310.29 \pm 11.30$          & $146.84 \pm 5.42$          & $84.62 \pm 4.04$              \\
        $Z$+jets         & $2.54 \pm 1.03$          & $1.10 \pm 0.46$          & $0.72 \pm 0.33$              \\
        Tops         & $69.12 \pm 8.78$          & $32.38 \pm 4.37$          & $19.72 \pm 2.80$              \\
        Di-boson         & $25.19 \pm 7.45$          & $12.05 \pm 5.78$          & $7.10 \pm 3.80$              \\
        $t\bar{t}+V$         & $1.06 \pm 0.24$          & $0.57 \pm 0.17$          & $0.29 \pm 0.09$              \\
    \bottomrule
    %%
    \end{tabular*}


\begin{tabular*}{\textwidth}{@{\extracolsep{\fill}}lrrrr}
\toprule
\textbf{TR 6J} &  $m_{\mathrm{eff.}}\in$[1100,1600] & $m_{\mathrm{eff.}}\in$[1600,2100] & $m_{\mathrm{eff.}}>2100$ \\
\midrule

Observed data          & $647$              & $232$              & $117$                    \\
\midrule
\midrule
MC total (post-fit)         & $646.88 \pm 25.46$          & $231.79 \pm 15.24$          & $116.91 \pm 10.85$              \\
\midrule
        $W$+jets         & $43.2 \pm 16.5$          & $25.1 \pm 9.7$          & $11.6 \pm 5.5$              \\
        $Z$+jets         & $0.9 \pm 0.4$          & $0.6 \pm 0.2$          & $0.4 \pm 0.2$              \\
        Tops         & $586.2 \pm 31.2$          & $193.1 \pm 18.7$          & $98.8 \pm 12.8$              \\
        Di-boson         & $8.1 \pm 2.5$          & $8.2 \pm 2.7$          & $3.9 \pm 1.7$              \\
        $t\bar{t}+V$         & $8.5 \pm 1.5$          & $4.7 \pm 1.1$          & $2.3 \pm 0.6$              \\
\midrule
\midrule
MC total (pre-fit)              & $672.53 \pm 31.35$          & $329.86 \pm 16.23$          & $174.76 \pm 11.54$              \\
\midrule
        $W$+jets         & $90.62 \pm 28.71$          & $44.24 \pm 14.02$          & $31.99 \pm 10.13$              \\
        $Z$+jets         & $0.88 \pm 0.36$          & $0.58 \pm 0.23$          & $0.41 \pm 0.20$              \\
        Tops         & $564.43 \pm 9.95$          & $272.12 \pm 5.74$          & $136.21 \pm 3.91$              \\
        Di-boson         & $8.11 \pm 2.51$          & $8.21 \pm 2.69$          & $3.84 \pm 1.69$              \\
        $t\bar{t}+V$         & $8.48 \pm 1.53$          & $4.71 \pm 1.14$          & $2.30 \pm 0.63$              \\
    \bottomrule
    %%
    \end{tabular*}
    
  \end{center}
\end{table}


\begin{table}
  \begin{center}
    \caption{ \label{tab::BGestimation::VRyields_Lowx} \crtablecaption{``Low-x''}  }

    \begin{tabular*}{\textwidth}{@{\extracolsep{\fill}}lrr}
      \toprule
      \textbf{CR Low-x} & WR & TR \\
      \midrule


Observed data & $15$ & $25$ \\
\midrule
\midrule
MC total (post-fit) & $15.02 \pm 3.89$ & $24.97 \pm 5.03$ \\
\midrule
$W$+jets & $9.3 \pm 4.2$ & $2.9 \pm 1.8$ \\
$Z$+jets & $0.4 \pm 0.2$ & $0.2 \pm 0.1$ \\
Tops & $2.7 \pm 0.9$ & $20.4 \pm 5.7$ \\
Di-boson & $2.6 \pm 0.8$ & $1.0 \pm 1.0$ \\
$t\bar{t}+V$ & $0.0 \pm 0.0$ & $0.5 \pm 0.1$ \\
\midrule
\midrule
MC total (pre-fit) & $37.17 \pm 2.56$ & $46.38 \pm 3.87$ \\
\midrule
$W$+jets & $29.51 \pm 1.84$ & $9.26 \pm 3.05$ \\
$Z$+jets & $0.38 \pm 0.15$ & $0.17 \pm 0.07$ \\
Tops & $4.62 \pm 0.75$ & $35.47 \pm 1.52$ \\
Di-boson & $2.61 \pm 0.79$ & $0.99 \pm 0.98$ \\
$t\bar{t}+V$ & $0.05 \pm 0.02$ & $0.48 \pm 0.11$ \\
    \bottomrule
    %%
    \end{tabular*}
    
  \end{center}
\end{table}


\begin{table}
  \begin{center}
    \caption{ \label{tab::BGestimation::VRyields_Highx} \crtablecaption{``High-x''}  }


    \begin{tabular*}{\textwidth}{@{\extracolsep{\fill}}lrr}
      \toprule
      \textbf{CR High-x} & WR & TR \\
      \midrule
      

Observed data & $92$ & $73$ \\
\midrule
\midrule
MC total (post-fit) & $91.91 \pm 9.61$ & $72.97 \pm 8.57$ \\
\midrule
$W$+jets & $72.4 \pm 11.1$ & $17.0 \pm 6.5$ \\
$Z$+jets & $1.1 \pm 0.4$ & $0.3 \pm 0.1$ \\
Tops & $10.2 \pm 2.9$ & $52.0 \pm 11.3$ \\
Di-boson & $8.0 \pm 3.5$ & $2.7 \pm 1.4$ \\
$t\bar{t}+V$ & $0.2 \pm 0.1$ & $1.0 \pm 0.4$ \\
\midrule
\midrule
MC total (pre-fit) & $134.04 \pm 6.41$ & $112.69 \pm 9.52$ \\
\midrule
$W$+jets & $108.42 \pm 3.88$ & $25.52 \pm 8.13$ \\
$Z$+jets & $1.13 \pm 0.39$ & $0.29 \pm 0.13$ \\
Tops & $16.32 \pm 2.19$ & $83.19 \pm 3.25$ \\
Di-boson & $7.99 \pm 3.50$ & $2.70 \pm 1.35$ \\
$t\bar{t}+V$ & $0.18 \pm 0.08$ & $0.99 \pm 0.37$ \\
    \bottomrule
    %%
    \end{tabular*}
    
  \end{center}
\end{table}


\begin{table}
  \begin{center}
    \caption{ \label{tab::BGestimation::VRyields_3B} \crtablecaption{``3B''}  }

    \begin{tabular*}{\textwidth}{@{\extracolsep{\fill}}lrrr}
      \toprule
      \textbf{WR 3B} & $m_{\mathrm{eff.}}\in$[1000,1750] & $m_{\mathrm{eff.}}>1750$ \\
      \midrule
Observed data          & $368$              & $107$                    \\
\midrule
\midrule
MC total (post-fit)         & $368.18 \pm 19.69$          & $107.05 \pm 10.56$              \\
\midrule
        $W$+jets         & $146.4 \pm 59.3$          & $58.3 \pm 16.7$              \\
        $Z$+jets         & $5.3 \pm 1.5$          & $2.4 \pm 0.4$              \\
        Tops         & $176.6 \pm 52.2$          & $33.1 \pm 11.5$              \\
        Di-boson         & $37.7 \pm 9.9$          & $12.5 \pm 3.3$              \\
        $t\bar{t}+V$         & $2.2 \pm 0.5$          & $0.8 \pm 0.2$              \\
\midrule
\midrule
MC total (pre-fit)              & $651.86 \pm 28.54$          & $223.90 \pm 10.02$              \\
\midrule
        $W$+jets         & $471.51 \pm 7.38$          & $164.58 \pm 2.94$              \\
        $Z$+jets         & $5.29 \pm 1.45$          & $2.39 \pm 0.38$              \\
        Tops         & $135.10 \pm 21.31$          & $43.59 \pm 7.33$              \\
        Di-boson         & $37.74 \pm 9.93$          & $12.53 \pm 3.26$              \\
        $t\bar{t}+V$         & $2.21 \pm 0.52$          & $0.80 \pm 0.22$              \\
    \bottomrule
    %%
    \end{tabular*}


\begin{tabular*}{\textwidth}{@{\extracolsep{\fill}}lrrr}
  \toprule
  \textbf{TR 3B} & $m_{\mathrm{eff.}}\in$[1000,1750] & $m_{\mathrm{eff.}}>1750$ \\
  \midrule
  
Observed data          & $234$              & $47$                    \\
\midrule
\midrule
MC total (post-fit)         & $233.97 \pm 15.57$          & $46.98 \pm 6.95$              \\
\midrule
        $W$+jets         & $1.4 \pm 1.0$          & $0.9 \pm 0.5$              \\
        $Z$+jets         & $0.1 \pm 0.1$          & $0.1 \pm 0.1$              \\
        Tops         & $227.5 \pm 15.8$          & $44.1 \pm 7.1$              \\
        Di-boson         & $0.2_{-0.2}^{+0.3}$          & $0.2 \pm 0.1$              \\
        $t\bar{t}+V$         & $4.7 \pm 1.2$          & $1.7 \pm 0.4$              \\
\midrule
\midrule
MC total (pre-fit)              & $183.60 \pm 23.01$          & $62.71 \pm 8.28$              \\
\midrule
        $W$+jets         & $4.54 \pm 1.87$          & $2.62 \pm 1.00$              \\
        $Z$+jets         & $0.12 \pm 0.05$          & $0.10 \pm 0.06$              \\
        Tops         & $174.00 \pm 21.42$          & $58.15 \pm 7.43$              \\
        Di-boson         & $0.20_{-0.20}^{+0.27}$          & $0.18 \pm 0.08$              \\
        $t\bar{t}+V$         & $4.75 \pm 1.17$          & $1.66 \pm 0.43$              \\
    \bottomrule
    %%
    \end{tabular*}
    
  \end{center}
\end{table}




%%%%%%%% Post-fit distributions %%%%%%%%%%
\clearpage
%% -------------- CR postFit 2J ---------
\begin{figure}[h]
  \centering
    \subfigure[]{\includegraphics[width=0.41\textwidth]{figures/BGestimation/CRpostFit/WR2JMEFF1/mt__WR2JMEFF1_no_mt_postFit_2SFconfig_noYields.pdf}}
    \subfigure[]{\includegraphics[width=0.41\textwidth]{figures/BGestimation/CRpostFit/WR2JMEFF1/met__WR2JMEFF1_no_met_postFit_2SFconfig_noYields.pdf}}
    \subfigure[]{\includegraphics[width=0.41\textwidth]{figures/BGestimation/CRpostFit/WR2JMEFF2/mt__WR2JMEFF2_no_mt_postFit_2SFconfig_noYields.pdf}}
    \subfigure[]{\includegraphics[width=0.41\textwidth]{figures/BGestimation/CRpostFit/WR2JMEFF2/met__WR2JMEFF2_no_met_postFit_2SFconfig_noYields.pdf}}
    \subfigure[]{\includegraphics[width=0.41\textwidth]{figures/BGestimation/CRpostFit/WR2JMEFF3/mt__WR2JMEFF3_no_mt_postFit_2SFconfig_noYields.pdf}}
    \subfigure[]{\includegraphics[width=0.41\textwidth]{figures/BGestimation/CRpostFit/WR2JMEFF3/met__WR2JMEFF3_no_met_postFit_2SFconfig_noYields.pdf}}
   \caption{
     Post-fit distruibution of (left) $\mt$ (right) $\met$.
     (a,b) WR 2J-$\meffIncFirst$.
     (c,d) WR 2J-$\meffIncSecond$.
     (e,f) WR 2J-$\meffIncThird$. 
     The yellow band in the bottom panel represents only statistical error. The overflow is included in the highest bin.  
   \label{fig::BGestimation::CRpostFit::WR2J}}    
\end{figure}
%----------------------------------
\clearpage
\begin{figure}[h]
  \centering
    \subfigure[]{\includegraphics[width=0.41\textwidth]{figures/BGestimation/CRpostFit/TR2JMEFF1/mt__TR2JMEFF1_no_mt_postFit_2SFconfig_noYields.pdf}}
    \subfigure[]{\includegraphics[width=0.41\textwidth]{figures/BGestimation/CRpostFit/TR2JMEFF1/met__TR2JMEFF1_no_met_postFit_2SFconfig_noYields.pdf}}
    \subfigure[]{\includegraphics[width=0.41\textwidth]{figures/BGestimation/CRpostFit/TR2JMEFF2/mt__TR2JMEFF2_no_mt_postFit_2SFconfig_noYields.pdf}}
    \subfigure[]{\includegraphics[width=0.41\textwidth]{figures/BGestimation/CRpostFit/TR2JMEFF2/met__TR2JMEFF2_no_met_postFit_2SFconfig_noYields.pdf}}
    \subfigure[]{\includegraphics[width=0.41\textwidth]{figures/BGestimation/CRpostFit/TR2JMEFF3/mt__TR2JMEFF3_no_mt_postFit_2SFconfig_noYields.pdf}}
    \subfigure[]{\includegraphics[width=0.41\textwidth]{figures/BGestimation/CRpostFit/TR2JMEFF3/met__TR2JMEFF3_no_met_postFit_2SFconfig_noYields.pdf}}
   \caption{
     Post-fit distruibution of (left) $\mt$ (right) $\met$.
     (a,b) TR 2J-$\meffIncFirst$. 
     (c,d) TR 2J-$\meffIncSecond$.
     (e,f) TR 2J-$\meffIncThird$.
     The yellow band in the bottom panel represents only statistical error. The overflow is included in the highest bin.  
   \label{fig::BGestimation::CRpostFit::TR2J}}    
\end{figure}
%----------------------------------

\clearpage
% -------------- 6J ---------
\begin{figure}[h]
  \centering
    \subfigure[]{\includegraphics[width=0.41\textwidth]{figures/BGestimation/CRpostFit/WR6JMEFF1/mt__WR6JMEFF1_no_mt_postFit_2SFconfig_noYields.pdf}}
    \subfigure[]{\includegraphics[width=0.41\textwidth]{figures/BGestimation/CRpostFit/WR6JMEFF1/LepAplanarity__WR6JMEFF1_no_LepAplanarity_postFit_2SFconfig_noYields.pdf}}
    \subfigure[]{\includegraphics[width=0.41\textwidth]{figures/BGestimation/CRpostFit/WR6JMEFF2/mt__WR6JMEFF2_no_mt_postFit_2SFconfig_noYields.pdf}}
    \subfigure[]{\includegraphics[width=0.41\textwidth]{figures/BGestimation/CRpostFit/WR6JMEFF2/LepAplanarity__WR6JMEFF2_no_LepAplanarity_postFit_2SFconfig_noYields.pdf}}
    \subfigure[]{\includegraphics[width=0.41\textwidth]{figures/BGestimation/CRpostFit/WR6JMEFF3/mt__WR6JMEFF3_no_mt_postFit_2SFconfig_noYields.pdf}}
    \subfigure[]{\includegraphics[width=0.41\textwidth]{figures/BGestimation/CRpostFit/WR6JMEFF3/LepAplanarity__WR6JMEFF3_no_LepAplanarity_postFit_2SFconfig_noYields.pdf}}
   \caption{
     Post-fit distruibution of (left) $\mt$ (right) $\apl$.
     (a,b) WR 6J-$\meffIncFirst$.
     (c,d) WR 6J-$\meffIncSecond$.
     (e,f) WR 6J-$\meffIncThird$.
     The yellow band in the bottom panel represents only statistical error. The overflow is included in the highest bin.  
   \label{fig::BGestimation::CRpostFit::WR6J}}    
\end{figure}
%----------------------------------
\clearpage
\begin{figure}[h]
  \centering
    \subfigure[]{\includegraphics[width=0.41\textwidth]{figures/BGestimation/CRpostFit/TR6JMEFF1/mt__TR6JMEFF1_no_mt_postFit_2SFconfig_noYields.pdf}}
    \subfigure[]{\includegraphics[width=0.41\textwidth]{figures/BGestimation/CRpostFit/TR6JMEFF1/LepAplanarity__TR6JMEFF1_no_LepAplanarity_postFit_2SFconfig_noYields.pdf}}
    \subfigure[]{\includegraphics[width=0.41\textwidth]{figures/BGestimation/CRpostFit/TR6JMEFF2/mt__TR6JMEFF2_no_mt_postFit_2SFconfig_noYields.pdf}}
    \subfigure[]{\includegraphics[width=0.41\textwidth]{figures/BGestimation/CRpostFit/TR6JMEFF2/LepAplanarity__TR6JMEFF2_no_LepAplanarity_postFit_2SFconfig_noYields.pdf}}
    \subfigure[]{\includegraphics[width=0.41\textwidth]{figures/BGestimation/CRpostFit/TR6JMEFF3/mt__TR6JMEFF3_no_mt_postFit_2SFconfig_noYields.pdf}}
    \subfigure[]{\includegraphics[width=0.41\textwidth]{figures/BGestimation/CRpostFit/TR6JMEFF3/LepAplanarity__TR6JMEFF3_no_LepAplanarity_postFit_2SFconfig_noYields.pdf}}
   \caption{
     Post-fit distruibution of (left) $\mt$ (right) $\apl$.
     (a,b) TR 6J-$\meffIncFirst$.
     (c,d) TR 6J-$\meffIncSecond$.
     (e,f) TR 6J-$\meffIncThird$.
     The yellow band in the bottom panel represents only statistical error. The overflow is included in the highest bin.  
   \label{fig::BGestimation::CRpostFit::TR6J}}    
\end{figure}
%----------------------------------



\clearpage
% -------------- varx ---------
\begin{figure}[h]
  \centering
    \subfigure[]{\includegraphics[width=0.41\textwidth]{figures/BGestimation/CRpostFit/WRLowx/mt__WRLowx_no_mt_postFit_2SFconfig_noYields.pdf}}
    \subfigure[]{\includegraphics[width=0.41\textwidth]{figures/BGestimation/CRpostFit/WRLowx/LepAplanarity__WRLowx_no_LepAplanarity_postFit_2SFconfig_noYields.pdf}}
    \subfigure[]{\includegraphics[width=0.41\textwidth]{figures/BGestimation/CRpostFit/WRHighx/mt__WRHighx_no_mt_postFit_2SFconfig_noYields.pdf}}
    \subfigure[]{\includegraphics[width=0.41\textwidth]{figures/BGestimation/CRpostFit/WRHighx/LepAplanarity__WRHighx_no_LepAplanarity_postFit_2SFconfig_noYields.pdf}}
   \caption{   
     Post-fit distruibution of (left) $\mt$ (right) $\apl$.
     (a,b) WR Low-x.
     (c,d) WR High-x.
     The yellow band in the bottom panel represents only statistical error. The overflow is included in the highest bin.  
\label{fig::BGestimation::CRpostFit::WRVarx}}    
\end{figure}
%----------------------------------
\clearpage
\begin{figure}[h]
  \centering
    \subfigure[]{\includegraphics[width=0.41\textwidth]{figures/BGestimation/CRpostFit/TRLowx/mt__TRLowx_no_mt_postFit_2SFconfig_noYields.pdf}}
    \subfigure[]{\includegraphics[width=0.41\textwidth]{figures/BGestimation/CRpostFit/TRLowx/LepAplanarity__TRLowx_no_LepAplanarity_postFit_2SFconfig_noYields.pdf}}
    \subfigure[]{\includegraphics[width=0.41\textwidth]{figures/BGestimation/CRpostFit/TRHighx/mt__TRHighx_no_mt_postFit_2SFconfig_noYields.pdf}}
    \subfigure[]{\includegraphics[width=0.41\textwidth]{figures/BGestimation/CRpostFit/TRHighx/LepAplanarity__TRHighx_no_LepAplanarity_postFit_2SFconfig_noYields.pdf}}
   \caption{ 
     Post-fit distruibution of (left) $\mt$ (right) $\apl$.
     (a,b) TR Low-x.
     (c,d) TR High-x.
     The yellow band in the bottom panel represents only statistical error. The overflow is included in the highest bin.  
     \label{fig::BGestimation::CRpostFit::TRVarx}}    
\end{figure}
%----------------------------------


\clearpage
% -------------- 3B ---------
\begin{figure}[h]
  \centering
    \subfigure[]{\includegraphics[width=0.41\textwidth]{figures/BGestimation/CRpostFit/WR3BMEFF1/mt__WR3BMEFF1_no_mt_postFit_2SFconfig_noYields.pdf}}
    \subfigure[]{\includegraphics[width=0.41\textwidth]{figures/BGestimation/CRpostFit/WR3BMEFF1/topNess__WR3BMEFF1_postFit_2SFconfig_noYields.pdf}}
    \subfigure[]{\includegraphics[width=0.41\textwidth]{figures/BGestimation/CRpostFit/WR3BMEFF2/mt__WR3BMEFF2_no_mt_postFit_2SFconfig_noYields.pdf}}
    \subfigure[]{\includegraphics[width=0.41\textwidth]{figures/BGestimation/CRpostFit/WR3BMEFF2/topNess__WR3BMEFF2_postFit_2SFconfig_noYields.pdf}}
   \caption{   
     Post-fit distruibution of (left) $\mt$ (right) topness.
     (a,b) WR 3B-$\meffIncFirst$.
     (c,d) WR 3B-$\meffIncSecond$.
     The yellow band in the bottom panel represents only statistical error. The overflow is included in the highest bin.  
     \label{fig::BGestimation::CRpostFit::WR3B}}    
\end{figure}
%----------------------------------
\begin{figure}[h]
  \centering
    \subfigure[]{\includegraphics[width=0.41\textwidth]{figures/BGestimation/CRpostFit/TR3BMEFF1/mt__TR3BMEFF1_no_mt_postFit_2SFconfig_noYields.pdf}}
    \subfigure[]{\includegraphics[width=0.41\textwidth]{figures/BGestimation/CRpostFit/TR3BMEFF1/topNess__TR3BMEFF1_postFit_2SFconfig_noYields.pdf}}
    \subfigure[]{\includegraphics[width=0.41\textwidth]{figures/BGestimation/CRpostFit/TR3BMEFF2/mt__TR3BMEFF2_no_mt_postFit_2SFconfig_noYields.pdf}}
    \subfigure[]{\includegraphics[width=0.41\textwidth]{figures/BGestimation/CRpostFit/TR3BMEFF2/topNess__TR3BMEFF2_postFit_2SFconfig_noYields.pdf}}
   \caption{   
     Post-fit distruibution of (left) $\mt$ (right) topness.
     (a,b) TR 3B-$\meffIncFirst$.
     (c,d) TR 3B-$\meffIncSecond$.
     The yellow band in the bottom panel represents only statistical error. The overflow is included in the highest bin.  
     \label{fig::BGestimation::CRpostFit::TR3B}}    
\end{figure}
%----------------------------------




%%%%%%%%%%%%%%%%a%%%%%%%%%%%%%%%%%%%%%%%%%%%%%%%%%%%%%%%%%%%%%%%%%%%%%%
\clearpage
\section{The Object Replacement Method} \label{sec::BGestimation::objRep}
There are several potential concerns over extrapolating in $\mt$ when estimating the ``di-leptonic'' background:
\begin{itemize}
\item The MC modeling itselft is questionable. \\
As seen in Figure \ref{fig::BGestimation::DataMCPresel2LBT2} (c) and discussed in \ref{sec::BGestimation::dataMC} (\underline{Di-leptonic $\ttbar$}),
$\mt$ of ``di-leptonic'' component seems to mis-modeled by MC since it does scale with the lepton momentum and MET for which MC is confirmed to be mis-modeling.


\item Applying the same selection between CRs and SRs in the other varibles no longer helps. \\
The extrapolation in $\mt$ is essentially an extrapolation between the ``semi-leptonic'' components  and the ``di-leptonic'' ones, since CRs are dominated by the ``semi-leptonic'' while SRs (VRs) are by ``di-leptonic''. The outstanding issue for it is that different particles contribute to observables between the ``semi-leptonic'' and ``di-leptonic processes'', and different phase spaces are chosen even when applying the smae selection (see Table \ref{tab::BGestimation::1L2LkineComp}).
For instance, MET is sourced by a single neutrino in the semi-leptonic channel while it is by a vectoral sum of two neutrinos in the di-leptonic one. More seriously, the number of ISR (FSR) jets is different under the same requirement in terms of jet multiplicity, for instance in $\ttbar$, the semi-leptonic channel yields 4 jets by its decay while the di-lepnic channel can only yield 2 (or 3 if hadronic decay product from $\tau$ is tagged as a jet). 
Therefore, applying the same selection between CRs and SRs no longer guarantee that CRs grasp the same phase space as SRs. 


\end{itemize}

%As long as $\ttbar$ is concern, there is no striking difference found in the behavior in terms of data/MC between the semi-leptonic and di-leptonic component, but conceptually this is more conceptual

\tab{c|c|c|c}
{
\hline
                            &  SR                                                            & 1LCR                            & 2LCR \\
\hline
Dominant $\ttbar$ mode   &  $\ttbar \ra b\ell\nu_1 b \tau \nu_2, \tau\ra\tau_h\nu_{\tau}$  & $\ttbar \ra bqq b\ell \nu$       & $\ttbar \ra b\ell \nu_1 b\ell\nu_2$ \\
\hline
$\nJetNoGev$                &  $\sim 2(3) + n_{\mathrm{ISR/FSR}}$                                 & $\sim 4 + n_{\mathrm{ISR/FSR}}$  & $\sim 2 + n_{\mathrm{ISR/FSR}}$     \\
$\met$                      &  $|\bpt(\nu_1)+\bpt(\nu_2)+\bpt(\nu_\tau)|$                      & $|\bpt(\nu)|$                    & $|\bpt(\nu_1)+\bpt(\nu_2)|$          \\
\hline
}
{Comparison of constituents of MET ($\met$) and jet multiplicity ($\nJetNoGev$) between the semi-leptonic $\ttbar$ and di-leptonic $\ttbar$ as example. ``1LCR'' refers to the default control regions used in the kinematical extrapolation method, and ``2LCR'' is the 2-lepton version with the same kinematical selection. Note that the other composite variables using jet and MET (e.g. $\meffInc$ and $\mt$ etc.) are also affected by the difference accordingly.
}
{tab::BGestimation::1L2LkineComp}

The use of 2-lepton control regions (2LCRs) is then naturally motivated
which does not require neither the  extrapolation in $\mt$ nor in the decay modes.
%
Although this does improve the simulation a bit, there is still difference between the kinematics of the ``di-leptonic'' components in SRs and CRs, as shown in Table \ref{tab::BGestimation::1L2LkineComp}.
The problem is found generic to the approach of MC normalization itself, since it can not accomodate the behavior of taus or missing leptons that differ event-by-event . \\

\clearpage
Instead, the approach of event-by-event can deal with the problem. 
The object replacement method is an example of it, which is an integrated method consisting of:
\begin{itemize}
\item "missing lepton replacement" to estimate a part of $\ell\ell_{\mathrm{mis.}}$ events ("Mis. Reco." and "Mis. ID"),
\item "tau replacement" to estimate $\ell\tau_{\mathrm{h}}$,
\end{itemize}
where one of the lepton of data events in 2LCR (seed event) is replaced into a virtual missing lepton or a simulated hadronic tau decay respectively, as schematized in Figure \ref{fig::BGestimation::objRep::schematic1}. 
The detector responses and behavior in object reconstruction of those replaced objetcs are carefully emulated so that the replaced event can directly mimic the events in the signal regions.   \\
%Therefore, there is no extrapolation in kinematics anymore but only in objects, and the modeling of kinematic tail can be fully driven by data.


%IDされる確率で割ってIDされない確率かければいけるっしょ
%前提としてID/mis-Idのsampleはfollow equivalent kinematics (statistical distribution) tatistically equilvelent behavior.というものがある
%
The object replacement method is a nearly full data-driven method where the used of MC is limited in an area of tau decays and the modeling of instrumental effects around the lepton identification.
The MC modeling is highly reliable, ensuring the extrapolation much more robust, compared with the kinematical extrapolation method where the mis-modeling in kinematic tail is always critical. For instance, the discrepancies with data are typically a few percent level and the cause are also mostly well-understood. \\

Note that the whole method relies on the orthogonality between kinematics and object properties:
\begin{align}
  & \frac{d\sigma(\ell\ell)}{d\bm{x}}
  \propto \frac{d\sigma(\ell\ell_{\mathrm{ID}})}{d\bm{x}} 
  \propto \frac{d\sigma(\ell\ell_{\mathrm{mis.}})}{d\bm{x}} 
  \label{eq::BGestimation::objRep::orthKineObj}
\end{align}
and the lepton universality:
\begin{align}
  & \frac{d\sigma(\ell\ell)}{d\bm{x}}
  \propto \frac{d\sigma(\ell\tau)}{d\bm{x}},
  \label{eq::BGestimation::objRep::lepUniv}
\end{align}
where $\ell\ell_{\mathrm{ID}}$ represents events with both two leptons are idendified, and $\ell\ell_{\mathrm{mis.}}$ denotes events with one of them is missed. 
$\bm{x}$ symbolizes kinematical variables. 
Particularly, the kinematics-object orthogonality (Eq. \ref{eq::BGestimation::objRep::orthKineObj}) is of paramount importance, since it allows to extrapolate the object properties measured in a very inclusive region into any arbitrary phase space. 
As long as the lepton reconstruction and identification is concerned, the statement is more or less true
%Though with some minor exceptions, this statement is highly robust when 
because they generally obeys the statistical behavior of detector responses such as fluctuating number of hits or energy deposit, which does not depend on global event kinematics, but rather on the nature of the particle itself (usually only on its momentum) as well as the local material configuration in the detector. Therefore, it is usually enough to parameterize the efficiency of reconstruction or identification simply by the momentum ($\pt, \eta, \phi$) of the particles. This is however not the case when coming to the estimation of lepton being outside the $(\pt,\eta)$ acceptance (``Out-Acc'') or being dropped in the overlap removal (``Mis. OR''), since the probabilities do depend on the momentum of parent particle or the proximity to the nearest jet. Hence, the seed events do not fully represent the kinematics of ``Out-Acc'' and ``Mis. OR'', and this is the reason why these events can not accounted by the object replacement method. \\

% -------------- schematic
\includegraphics[width=160mm]{figures/BGestimation/ObjReplacement/method/schematic_replacement.eps}
\captionof{figure}{Schematic of the object replacement method. One of the leptons in the seed event is replaced into either a virtual missing lepton or a simulated hadronic tau decay. A weight $\kappa$ is assigned to the replaced event (sub-event) to account for the different rate of occurence between the seed events and the sub-event (e.g. probability of a lepton being identified and missed).
More detail is found in Sec. \ref{sec::BGestimation::objRep::perEvtLogic}.  }
\label{fig::BGestimation::objRep::schematic1}
% --------------

%%%%%%% Method %%%%%%%%%%%%%%%%%%%%%%%%%%%%%%%%%%%%
\clearpage
\subsection{The Replacement Procedure and the Per-event Logic} \label{sec::BGestimation::objRep::perEvtLogic}
Figure \ref{fig::BGestimation::objRep::perEvtLogic} is the schematic of the work flow for the replacement procedure for a single seed event, which follows as below: 

\begin{enumerate}
\item Pick up a 2LCR event ("seed event").
\item Replace a lepton of the seed event into a virtual missing lepton or a simulated hadronic decay of tau lepton, if the two leptons satisfy a certain criteria. This replaced event is called "sub-event".
\item (For tau replacement) Apply the calibration for the hadronic tau.
\item Re-calculate the event-level kinematics such as $\met$ or $\meffInc$ etc.
\item Assign a weight $\kappa$ for each sub-event as the transfer factor from 2LCR to 1-lepton regions.
\item (For tau replacement) Repeat the step 2-5 by $N=50$ times and take the average, in order to accommodate the statistical nature of tau decay. Note that the number of iteration $N$ only dictates the level of ``smoothing'' thus has no essential impact on the final result. The average is taken by scaling the $\kappa$ by 1/N.
\item Change the roles (tagged/replaced) between the two leptons and repeat 2-6. 
\item Apply the analysis level selection (e.g. signal region selection) and post-selection to reject singal contamination for the generated sub-events. 
\item Collect the accepted sub-events and fill them into an ``event-level histogram''. 100$\%$ of statistical uncertainty is assigned for each bin of the event-level histogram, accounting for all the sub-events are generated from the common seed event.
\item Loop over all seed events and sum up all the event-level histograms with ordinary statistical treatment where the uncertainty is quadratically summed for each bin of the histogram. 
\end{enumerate}

% -------------- perEvtLogic
\figNoH[175]{BGestimation/ObjReplacement/method/flow_replacement.eps}
{Work flow of the replacement procedures for a single seed event.
Sub-events are generated through both the mis-lepton replacement 
}
{fig::BGestimation::objRep::perEvtLogic}
%-------------------------------

More detail and caveats about each step is provided below:  \\

\noindent \textbf{Seed event selection and trigger} \\
For seed event selection, looser kinematical selection is generally preferred, to collect the necessary seed events as completely as possible. In particular, as MET and $\mt$ change their values the most during the replacement, those cuts have to be drastically relaxed with respective to regions that one wants to estimate. 
For instance, Figure \ref{fig::BGestimation::seedMET} shows the MET distribution of seed events when the MET after the replacement above  $250\gev$ is required. About $40\%$ of seeds are with seed MET below $250\gev$, indicating that it will be underestimated by about $40\%$ if naively requiring the same MET for seed events as the estimated regions. \\ 

While MET trigger is available for collecting the bulk events above its off-line threshold $\met>250\gev$, the single-lepton trigger (SLT) is introduced to complement the seeds events with $\met<250\gev$.
%In order to account for the different triggers efficiency with respect to the MET trigger that is used in the signal regions definition, the inverse trigger efficiency (Figure \ref{fig::BGestimation::objRep::SLTeff}) is applied for events triggered by SLT, assuming the MET trigger is $100\%$ efficient. 
In spite of its relatively low efficiency ($70\%-90\%$) and the tight off-line threshold of $\pt > 28 (26) \gev$ for the single-electron (single-muon) trigger, SLT is still fully efficient for the seed events since there are two leptons being the candidate to fire the trigger. Eventually, as shown in Figure \ref{fig::BGestimation::seedMET}, more than $95\%$ of the overall trigger efficiency can be maintained. \\

%%%%%%%%%%
\figNoH[110]{BGestimation/ObjReplacement/method/seed_collection/met_seed_trig.pdf}
{Seed MET distribution (gray) for the $\llmis$ and $\ltauh$ events from $\ttbar$ resulting in $\met>250\gev$. The seed MET is defined by the vectoral sum of neutrinos from top decays: $\left|\bpt(\nu_1)+\bpt(\nu_2)\right|$, which is roughly equivalent to the MET in corresponding seed events ($\ttbar \ra b\ell\nu_1 \bar{b}\ell\nu_2$). Over $95\%$ of the seed events are shown to be accepted by the combined trigger strategy defined in Table \ref{tab::BGestimation::objRep::def2LCR} (pink). }
{fig::BGestimation::seedMET}
%%%%%%%%%%

Although the enhanced backgrounds due to the lowered MET selection for 2LCR does not impact as much on the final result since most of them are skimmed out at the analysis-level selection applied after the replacement, the decent cut $\met>100$ is required to suppress the bulk background components in 2LCR (Z+jets, 1L+fake lepton etc.) and make sure avoiding the large uncertainty from MC subtraction. The seed event loss due to the selection $\met>100$ is negligible when estimating SRs/VRs. Table \ref{tab::BGestimation::objRep::def2LCR} shows the definition of common 2LCR. \\

%, and Figure \ref{fig::BGestimation::objRep::2LCR} are the kinematic distributions of MC overlaid with data in the region. 

% ------------- 
\begin{table}[h]
  \begin{center}
    \caption{Definition of 2-lepton control region (2LCR).}
    \begin{tabular}{ c }
      \hline
       $n_{\ell, \mathrm{baseline}}=2$, $n_{\ell, \mathrm{signal}}\ge 1$ \\
      \hline
       MET trigger,  $\met>250\gev$   \\
       or \\
       At least 1 signal lepton with $\pt>28 \gev$ firing the single-lepton trigger, $\met>100\gev$   \\
      \hline
    \end{tabular}  \label{tab::BGestimation::objRep::def2LCR}
  \end{center}
\end{table}
% ------------- 

%\clearpage
% -------------- 2LCR
%\begin{figure}[h]
%  \centering
%    \subfig{0.48}{figures/BGestimation/ObjReplacement/2LCR/met__2LCR_common_log.pdf}{}
%    \subfig{0.48}{figures/BGestimation/ObjReplacement/2LCR/nJet30__2LCR_common_log.pdf}{}
%    \subfig{0.48}{figures/BGestimation/ObjReplacement/2LCR/lep2Pt__2LCR_common_log.pdf}{}
%    \subfig{0.48}{figures/BGestimation/ObjReplacement/2LCR/mll__2LCR_common_log.pdf}{} 
%   \caption{ MC and data distributions of (a) $\met$, (b) jet multiplicity, (c) sub-leading lepton transverse momentum (d) the di-lepton invariant mass in the common 2LCR defined in Table \ref{tab::BGestimation::objRep::def2LCR}.  \label{fig::BGestimation::objRep::2LCR}}    
%\end{figure}
%----------------------------------
%\clearpage


\noindent \textbf{Requirement on seed leptons for the replacement} \\
A seed event with lepton $\ell_1, \ell_2$ have two choices for the replacement namely 1) keeping $\ell_1$ and replacing $\ell_2$ or 2) keeping $\ell_2$ and replacing $\ell_1$. The replacement is proceeded only if the lepton to-be-replaced (``replaced lepton'') and the lepton to-be-kept (``tag lepton'') satisfy a certain condition shown in Table \ref{tab::BGestimation::objRep::seedLepReq}. As the tag lepton eventually corresponds to the single lepton used in the analysis in 1-lepton regions, it has to be in a consistent object definition in defining the signal regions (Table \ref{tab::objDef::summary}). On the other hand, generally looser requirement is preferred for the replaced lepton, instead from the CR statistics point of view. Therefore, it is loosened to ``baseline'',  except when estimating $b$-vetoed regions where it is kept to ``signal'' since  the impact of fake lepton background in 2LCR is relatively large otherwise. \\

Note that the replacement can happen twice from the identical seed event if the both combinations (tag, rep.)=($\ell_1, \ell_2$), ($\ell_2 \ell_1$) are eligible. 

\tab{ c|cc }
{
\hline
            & $b$-tagged, $b$-inclusive  &  $b$-vetoed \\ 
\hline
\hline
Tag lepton  & signal                 & signal \\
Replaced lepton  & baseline                 & signal \\
\hline
}
{Lepton definition used for tag and replaced lepton versus the type of regions to be estimated.}
{tab::BGestimation::objRep::seedLepReq}


\noindent \textbf{Treatment of virtual missing lepton} \\
As mentioned in Sec. \ref{sec::objDef::OR}, electrons are usually also identified as jets.
The doubly-counted object, either an electron or a jet, is discarded during the overlap removal.
Therefore, electrons failing the reconstruction or identification will simply recognized as jets without experiencing the overlap removal. 
To emulate this effect, in case of replacing an electron in a seed event, the record that the electron is reconstructed as a jet candidate is retrieved, 
and the 4-vector of electron is replaced into the that of the jet candidate. 
As the jet candidate is fully calibrated in the hadronic scale, no correction is needed additionally. 
In an occasion where electrons do not have corresponding jet candidates, which typically happens when the transverse momentum is too low, the electron is replaced into a missing particle with the 4-momentum of the original electron. \\ 

Muons failing the reconstruction or identification are almost never identified as any other objects. Instead, they are included in the track soft term in the MET calculation, and in principle this needs to be emulated in the missing muon replacement. This is technically possible, however the bottleneck is that the muon track quality is totally different between well-identified muons and unidentified ones, and particularly it is difficult to reproduce the resolution of bad muon track from good one with a meaningful correction. As it turns that simply including the 4-momentum replaced muon into the soft term even leads to worse performance than not (as demonstrated in Figure \ref{fig::BGestimation::objRep::mcClosure::metSoftTerm_mu}), replaced muons are decided to be simply treated as a virtual missing particle in the same momentum, and added into MET.
Although this rough treatment causes a non-zero error in the estimation as one will see in Sec. \ref{sec::BGestimation::objRep::mcClosure}, fortunately the impact on final estimation is marginal because the rate of missing muon events are generally very low, compared with the other components (missing electron events or $\ltauh$) due to the very high efficiency of muon reconstruction and identification. \\


\noindent \textbf{Simulation of tau decays and the $\tau_h$-to-jet calibration } \\
Tau decays are simulated by TAUOLA \cite{TAUOLA1,TAUOLA2,TAUOLA3} assuming the taus are unpolarized. This assumption is incorrect given the parent $W$-bosons are left-handed, however the impact on the final result is found to be marginal. This is discussed in Sec. \ref{sec::BGestimation::objRep::mcClosure}. Branching for leptonic decay is set to zero to reduce the number of loops. \\

Given that the analysis is free from explicit tau selections, hadronic taus within the $p_{\mathrm{T}}$-$\eta$ acceptance most likely undergo the jet clustering, as well as $b$-tagging and the jet calibration once they pass the JVT cut (Sec. \ref{sec::objDef::jets::JVT}). 
On the other hand, the output of TAUOLA is merely a 4-vector of truth level hadronic tau. 
Therefore, following pseudo-calibration is applied for the truth-level $\tau_h$, to emulate the effect either of the detector response, jet calibration, and the $b$-tagging.\\

\begin{enumerate}
\item Scale the transverse momentum of truth $\tau_h$. \\
The scale of a truth $\tau_h$ to an anti-$k_{\mathrm{T}}$ jets is derived using the $t\bar{t}$ MC samples, 
by comparing the transverse momenta of truth hadronic taus and that of $\Delta R$-matched reconstructed jet by $\Delta R<0.2$
. It is defined by the mean value of the residual distribution (Figure \ref{fig::BGestimation::objRep::tau_ptResidual}) and parameterized in terms of $p_{\mathrm{T}}$ and $\eta$ of truth hadronic taus (Figure \ref{fig::BGestimation::objRep::tau_scale}). The scale is always positive and rises significantly in the low-$\pt$ limit, due to the fact that the anti-$k_{\mathrm{T}}$ jet contains extra underlying tracks inside that become the pedestal. The difference in the calibration between light jets and $b$-tagged jets are ignored. \\


\item Smear the $p_{\mathrm{T}}$ of the hadronic tau. \\
After applying the scale above, smearing is subsequently adopted to account for the detector resolution.
The resolution is taken from the Gaussian-fitted RMS of the residual distribution on which the scale above is defined as well (Figure \ref{fig::BGestimation::objRep::tau_ptResidual}), and parameterized as function of $p_{\mathrm{T}}$ and $\eta$ of truth hadronic taus (Figure \ref{fig::BGestimation::objRep::tau_resol}). The smearing is applied based on the Gaussian profile with RMS being the resolution. \\


\item Emulation of the JVT cut and $b$-tagging.  \\
After the $\pt$-scaling and smearing, hadronic taus with $p_{\mathrm{T}}>30\gev$, $|\eta|<2.8$ are identified as jet candidates.
Some of them are then ramdomly dropped to emulate the effect of the JVT cut, based on the calculated efficiency of JVT cut using the simulated $\ttbar$ sample (Figure \ref{fig::BGestimation::objRep::effJVT}).
% tau to fake-bは結構あることをここでいう
A random $b$-tagging is further performed on the remained jets, by assigning a random $b$-tagging score (MV2c10) following according to the profile obtained from the $\Delta R$-matched reconstructed jets in the  $\ttbar$ MC (Figure \ref{fig::BGestimation::objRep::tau_bTagScore}). 
%
While the JVT efficiency is mapped as a function of $p_{\mathrm{T}}$ and $\eta$ of the jet candidates, 
the $b$-tagging score profile is only separated in terms of tau decay modes (1-prong and 3-prong).
\end{enumerate}

% ----------------- tauRF
\clearpage
\begin{figure}[htbp]
  \begin{center}
      \includegraphics[width=100mm]{figures/BGestimation/ObjReplacement/method/tauRF/ptResidual.eps} 
      \caption{The residual of tau momentum measurement: $[\pt(\mathrm{reco. } \tau\mathrm{-jet})-\pt(\mathrm{tr. } \tau_{h})]$/$\pt(\mathrm{tr.} \tau_{h})$ calculated using the simulated $\ttbar$ sample. $\pt(\mathrm{tr.} \tau_{h})$ is the transverse momentum of truth-level hadronic tau defined as $|\bm{p}(\tau)-\bm{p}(\nu_{\tau})|$ and $\pt(\mathrm{reco.} \tau-\mathrm{jet})$ is the corresponding reconstructed anti-$k_{\mathrm{T}}$ jet matched by $\Delta R<0.2$. }
      \label{fig::BGestimation::objRep::tau_ptResidual}
    %
    \begin{minipage}[t]{.45\textwidth}
      \centering        
      \includegraphics[width=85mm]{figures/BGestimation/ObjReplacement/method/tauRF/tauJet_scale.eps} 
      \caption{Scale of anti-$k_{\mathrm{T}}$ jets for truth hadronic taus, defined as the mean of the residual distribution Figure \ref{fig::BGestimation::objRep::tau_ptResidual}.}
      \label{fig::BGestimation::objRep::tau_scale}
    \end{minipage}
    \hfill
    \begin{minipage}[t]{.45\textwidth}
      \centering
      \includegraphics[width=85mm]{figures/BGestimation/ObjReplacement/method/tauRF/tauJet_resol.eps}
      \caption{Resolution of hadronic tau, defined by the Gaussian-fitted RMS of the residual distribution Figure \ref{fig::BGestimation::objRep::tau_ptResidual}.}
      \label{fig::BGestimation::objRep::tau_resol}
    \end{minipage}
    %              
  \end{center}  
\end{figure}
\clearpage
\begin{figure}[htbp]
  \begin{center}
    \begin{minipage}[t]{.45\textwidth}
      \centering
      \includegraphics[width=85mm]{figures/BGestimation/ObjReplacement/method/tauRF/heff_vs_recoJetPt_recoJetEta.eps}
      \caption{The efficiency map for the JVT cut for a reconstructed hadronic tau jet as function of its $p_{\mathrm{T}}$ and $\eta$, calculated using the $\ttbar$ MC sample. The efficiency is defined by the fraction of jets passing the JVT cuts and matched to truth hadronic taus by $\Delta R<0.2$ that pass $p_{\mathrm{T}}>30\gev$, $|\eta|<2.8$. }
      \label{fig::BGestimation::objRep::effJVT}
    \end{minipage}
    \hfill
    \begin{minipage}[t]{.45\textwidth}
      \centering
      \includegraphics[width=75mm]{figures/BGestimation/ObjReplacement/method/tauRF/hadTau_bTagScore.eps}
      \caption{Profile of $b$-tagging score (MV2c10) for signal jets originated from hadronic tau decays. Only the dependency on the decay modes (1-prong or 3-prong) is taken into account. The threshold for the $b$-tagging is at 0.44. 
The 3-prong events result in a higher fake rate into $b$-tagged jets, reflecting the fact that the secondary vertex structure more resembles to that of $b$-hadrons. The simulated $\ttbar$ MC sample is used to derive the profiles. }
      \label{fig::BGestimation::objRep::tau_bTagScore}
    \end{minipage}
    %
  \end{center}
\end{figure}
%% ------------------------------------------------------------------------

\noindent \textbf{Transfer factor} \\
A weight $\kappa$ is assigned to each sub-event, to account for the different probability of occurrence between the seed event and the replaced sub-event. For instance, in the missing lepton replacement, this corresponds to the difference between probability of a lepton being identified and being failed i.e. the inefficiency over the efficiency:
\begin{align}
\kappa = \frac{
  1-\epsilon_{\mathrm{baseline}}  (\ell_{\mathrm{rep.}})
}{
  \epsilon_{\mathrm{rep.}} (\ell_{\mathrm{rep.}})
}. 
\end{align}
Note that the efficiencies used in the enumerator and the denominator is different; that in the enumerator for the working point used in second lepton veto (namely ``baseline''); that in the denominator is for the working point that the replaced lepton is required, which can be either ``baseline'' and ``signal'' depending on cases (see Table \ref{tab::BGestimation::objRep::seedLepReq}). \\

As for the tau replacement, the transfer factor is 
\begin{align}
\kappa = \frac{
  \mathrm{Br}(\tau\ra\tau_{\mathrm{h}}\nu)
}{
  2N\epsilon_{\mathrm{rep.}}(\ell_{\mathrm{rep.}})
},
\end{align}
where $N$ is number of iterations per replacement (set to 50 in this study), and $\epsilon_{\mathrm{rep.}}$ the efficiency for working point used for requiring replaced lepton. The factor 2 in the dnominator originates from the fact that two channels ($e\ell$ and $\mu\ell$) are available as seeds for estimating a single channel $\tau\ell$ (see Table \ref{tab::BGestimation::objRep::seedRepFlavor}). \\

\tab{ c|| c|c|c}
{
 \hline
 Seed   & Replaced lepton & sub-evt. by mis. lep. rep.  &  sub-evt. by tau rep.    \\
 \hline
 \hline
 $e^+e^-$     & $e^-$   & $e^+e^-_{\mathrm{mis}}$       & $e^+\tau^-$   \\
              & $e^+$   & $e^+_{\mathrm{mis}}e^-$       & $\tau^+e^-$   \\
 \hline
 $e^+\mu^-$   & $\mu_-$ & $e^+\mu^-_{\mathrm{mis}}$     & $e^+\tau^-$   \\
              & $e^+$   & $e^+_{\mathrm{mis}}\mu^-$     & $\tau^+\mu^-$   \\
 \hline
 $\mu^+e^-$   & $e^+$   & $\mu^+e^-_{\mathrm{mis}}$     & $\mu^+\tau^-$   \\
              & $\mu^+$ & $\mu^+_{\mathrm{mis}}e^-$     & $\tau^+e^-$   \\
 \hline
 $\mu^+\mu^-$ & $\mu^-$ & $\mu^+\mu^-_{\mathrm{mis}}$   & $\mu^+\tau^-$   \\
              & $\mu^+$ & $\mu^+_{\mathrm{mis}}\mu^-$   & $\tau^+\mu^-$   \\
 \hline
}
{Correspondence between the seed events and the generated sub-events, in terms of charge and lepton flavor. One finds that the sub-events generated by tau replacement need to weighted by $1/2$ otherwise will be double-counted.
}
{tab::BGestimation::objRep::seedRepFlavor}

By its definition, $\alpha := 1/N_{\mathrm{acc.}}\kappa$ roughly gives the ratio of expected effective statistics in CR with respect to the SR, where $N_{\mathrm{acc.}}$ is average number of accepted sub-events after the kinematical cuts. $\alpha$ is typically $3\sim5$ for the missing lepton replacement, and about 1.4 ($\sim 1/\mathrm{Br}(\tau\ra\tau_{\mathrm{h}}\nu)$) for the tau replacement at the pre-selection level. It is typically enhanced by about factor of 2 when $\mt(\ell_{\mathrm{tag.}},\met)>m_W$ is required (corresponding to $\mt>m_W$ in 1-lepton regions, which is always the case for VRa and SR). This is due to the fact that most of the di-leptonic SM processes follow 
\footnote{This holds when the event contains exactly two semi-leptonically $W$-bosons, and the two leptons and MET are only supplied from their decays. More detail found in \cite{MT2_Lester_Summers}.} :
\begin{align}
\min \left[ \mt(\ell_1,\met),\mt(\ell_2,\met) \right] < m_W, \label{eq::MT2condition}
\end{align}
in other words, either of the $\mt$ must be below $m_W$. 
Given that both leptons in the seed event have the chance to be replaced,  
nearly half of the generated sub-events will be discarded by requiring $\mt(\ell_{\mathrm{tag.}},\met)>m_W$.
Accordingly, together with the fact that the contribution from the tau replacement is dominant, the effective CR statistics for the object replacement method is constantly about 3 times more than that in SR. This factor of 3 gain in statistics is in fact subtle; given that the expected yields in SRs are typically a few events, it immediately leads to $20\%-50\%$ statistical uncertainty by itself. Therefore CR statistic is always the biggest source of uncertainty in this method. \\


%%----------------------------------------------------
\noindent \textbf{Lepton efficiency} \\
The lepton efficiency used in the transfer factor calculation is calculated using $t\bar{t}$ MC sample. The efficiency of ID/baseline/signal lepton requirement is respectively defined as the fraction of truth leptons that are matched with reconstructed leptons passing the ID / identified / signal lepton requirement by $\Delta R<0.2$. Leptons overlapped with jets (if the nearest jet closer than $\Delta<0.4$) are excluded since their efficiency is biased. The efficiencies are parameterized as a function of lepton flavor ($e/\mu$), $p_{\mathrm{T}}$ and $\eta$ of truth leptons. The data/MC scale factor measured using the $Z\ra ee/\mu\mu$ events are applied. The resultant efficiency maps are shown in Figure \ref{fig::BGestimation::objRep::lep_efficiency}. \\

%% ------------------------------------------------------------------------
\begin{figure}
  \begin{center}
    %
    \begin{minipage}[t]{.49\textwidth}
      \centering
      \includegraphics[width=75mm]{figures/BGestimation/ObjReplacement/method/lepeff/el_trPtEta_truthToID.eps}
      \hspace{10mm} (a) Efficiency of truth electrons passing the reconstruction and ID requirement.
      \label{fig::BGestimation::objRep::heff_mc_el_truthToID}
    \end{minipage}
    \begin{minipage}[t]{.49\textwidth}
      \centering
      \includegraphics[width=75mm]{figures/BGestimation/ObjReplacement/method/lepeff/el_trPtEta_truthToSig.eps}
      \hspace{10mm} (b) Efficiency of truth electrons passing the signal lepton requirement
      \label{fig::BGestimation::objRep::heff_mc_el_truthToSig}
    \end{minipage}
    %
    \begin{minipage}[t]{.49\textwidth}
      \centering
      \includegraphics[width=75mm]{figures/BGestimation/ObjReplacement/method/lepeff/mu_trPtEta_truthToID.eps}
      \hspace{10mm} (c) Efficiency of truth muons passing the reconstruction and ID requirement.
      \label{fig::BGestimation::objRep::heff_mc_mu_truthToID}
    \end{minipage}
    \begin{minipage}[t]{.49\textwidth}
      \centering
      \includegraphics[width=75mm]{figures/BGestimation/ObjReplacement/method/lepeff/mu_trPtEta_truthToSig.eps}
      \hspace{10mm} (d) Efficiency of truth muons passing the signal lepton requirement
      \label{fig::BGestimation::objRep::heff_mc_mu_truthToSig}
    \end{minipage}
    %
    \caption{Off-line lepton selection efficiency used in the transfer factor calulation.
      The efficiency is defined by the fraction of truth leptons to match with reconstructed leptons passing the ID / identified / signal lepton requirement by $\Delta R<0.2$. 
}
    \label{fig::BGestimation::objRep::lep_efficiency}
  \end{center}
\end{figure}


% -------------------------------------
\noindent \textbf{Post-selection for rejecting signal contamination} \\
Signal contamination is generally not negligible when estimating SR-like regions in this method,
since there are a class of benchmark models that result in $3 \sim 4$ $W$-bosons giving comparable di-leptonic branching as the semi-leptonic one. 
The contamination is generally disfavored since it will elevate the expected background level, causing the deterioration of either discover and exclusion sensitivity. \\

A post-selection in terms of $\mtRep$ shown in Table \ref{fig::BGestimation::objRep::mtRep} is applied for sub-events passing the kinematical selections, to get rid such the signal contamination. 
The key observation is that only SM processes follow the condition Eq. (\ref{eq::MT2condition}), therefore have a sharp cut-off in $\mtRep \sim m_W$ when $\mt(\ell_{\mathrm{tag}},\met)>m_W$, as shown in Figure \ref{fig::BGestimation::objRep::mtRep}. \\

The cut is designed to maintain the efficiency greater than $90\%$ in SRs. The efficiency drop will be eventually compensated. On the other hand, signal contamination is largely suppressed typically to $y_{\mathrm{contami.}}/y_{\mathrm{S}} = 0.05\sim0.15$ in SRs after the post selection, where $y_{\mathrm{contami.}}$ ($y_{\mathrm{S}}$) is the expected increase of expected background due to the contamination (expected signal yield) in the region. \\

% 
% The impact of signal contamination is generally not negligible since there are a class of target models yielding 3-4 W-boson and comparable 2L branching as 1L.
% BGと同じように1Lに入ったdi-leptonicなsignal eventsの量が推定される. まず断っておきたいのは, このcontamiによるexp. BGのincreaseは1Lのsignalの量に比べれば大した量にはならない(3-10倍いる)ので発見への影響はほとんどない。ただsignal regionとかの推定をしたい場合、SRでのexp. BGは数発とかいうレベルになるのでBGに比べたらnon-negligibleになることがしばしばあり、exclに関しては実際injected signal hypothesisに大してきちんとresponseが計算できるなら問題ではないが、signalのmodeling uncertaintyなどの影響も入ってくるので一般的に抑えられるなら抑えた方がいい。

%BGだけ残してsignalが普遍的に消えるような有効なcleaning cutはあるか?そこでここでの求めたいstandard model componentはほとんどWが二つ出るようなイベントで、WW\ra lvlvのようなevent topologyでは2つのMTが両方mW以上になることはほとんどないことに注目する。つまりVR/SRといったmt(tagLep,MET)>mWを要求している場合では、mt(repLep,MET)はmWであることがほとんどである。一方でSUSY signal全般としての性質としてsemi-leptonic decaying W以外にもMET sourceがあるためmTがmWでtruncateすることはない。なのでmtRepのupper cutは普遍的にeffectiveである。signal regionにおけるBG expに対するcontaminationの影響が30\%以下になるようupper cutを選んだ結果表のようになった。
%このcleaning cutで削れた分はMCで見積もったinverse eff.をかけてconpensateすることにした。
%MCによるとSR BVで最大でexp. BGが20%減るようなところもあるが、そういったところそもそもobj Repで求める数が少ないので問題ない。
%resultant signal contaminationは図~の通り。依然として場所によっては30-50%くらいいってしまうところもあるが、contamiが多いところは基本的にsignal regionに入る量も圧倒的に多いので、5 sigmal->4sigmaとかになるようなことしか起こらない。一方で微妙な感度を持ってるようなsignal pointはcontamiも少ないため感度に影響はない。なのでこれでokということにする。

%Post-selection imposed on replaced sub-events in estimating regions in each tower. Isolation requirement for the BV regions is intended to further reject the backgrounds in 2LCR which is mainly $\wjets/\ttbar$ accompanied with 1 fake lepton. The upper cut in $\mtRep$ are meant to prevent the impact of signal contamination. Upper cut in MET is additionally needed for the 3B tower to cope with higher expected signal contamination, providing the targeting signatures generally involve more top quarks with higher lepton multiplicity which leads to higher acceptance in 2-lepton regions. 
%are summarized in \ref{tab::BGestimation::objRep::postSelection}.


%%%%%

\tab{ c|c|c }{
  \hline
  Region                                &  $\mtRep$ [GeV]  &  SM efficiency        \\
  \hline
  \hline
  SR                                    & $<250$           &  $0.9 \sim 0.98$            \\
  \hline
  \hline
  VRa                                   &  $<300$          &     $>0.97$    \\
  VRb                                   &  $-$             &     1    \\
  \hline
}
{Post-selection applied for signal contamination rejection. Inclusive efficiency of sub-events from SM background seeds are calculated using MC, shown in the rightest column. The efficiency drop is corrected based on the inverse efficiency.}
{tab::BGestimation::objRep::postSelection}


%%%%%%%%%%%%%%%%%%%%
\begin{figure}[h]
  \centering
    \subfig{0.48}{figures/BGestimation/ObjReplacement/method/dist_sigContami/dist_sigContami_SR6JMEFFIncl_mt_repLep.pdf}{}
    \subfig{0.48}{figures/BGestimation/ObjReplacement/method/dist_sigContami/dist_sigContami_SR3BMEFFIncl_mt_repLep.pdf}{}
 \caption{ 
   $\mtRep$ distribution of sub-events passing the (a) SR 6J $\meffInc$-inclusive selection, and (b) SR 3B $\meffInc$-inclusive selection. 
   \label{fig::BGestimation::objRep::mtRep} 
 }
\end{figure}

%%%%%%%%%%%%%%%%%%%%
\noindent \textbf{Event-level histogram and the statistical treatment} \\
Multiple sub-events are generated by both the missing lepton replacement and the tau replacement from a single seed event. Those passing the analysis selections are collected and filled into a common histogram, referred as an ``event-level histogram'' 
%(this corresponds to a one-bin histogram when one only wants to estimate the yield in a particular region)
. To account for their full statistical correlation between the filled sub-events, $100\%$ error is then assigned to each bin of the event-level histogram. 
The summed event-level histograms over all seed events will be the desired distribution. While the statistical error on each bin is simply the quadratic sum of those over the all event-level histograms, there is generally also the inter-bin correlation since the bins of event-level histograms are not statistically independent between each other. This correlated uncertainty in fact needs to be modeled when performing the combined fit with multiple signal bins, which is examined and summarized in Sec. \ref{sec::BGestimation::objRep::binCor}.


% ---------- Evt level histogram
\begin{center}
\includegraphics[width=80mm]{figures/BGestimation/ObjReplacement/method/evtLevel_histogram.eps}
\captionof{figure}{An example of event-level histogram. 100$\%$ uncertainty is assigned for each bin to account for the fact that all the entries are from the same seed. The final estimation is given by the sum of the event-level histograms over all seed events.}
\label{fig::BGestimation::objRep::evtLevelHist}
\end{center}
% ------------------------------------------


%%%%%%%%%%%%%%%%%%%%%%%%%% MC Closure test %%%%%%%%%%%%%%%%%%%%%%%%%%%%%%%%
\clearpage
\subsection{Closure Test using $t\bar{t}$ MC Samples} \label{sec::BGestimation::objRep::mcClosure} 
The methodologies are tested by comparing yields in regions with exactly one baseline lepton, between the estimation using the seed events in 2LCR and the actual $\llmis$ or ($\ltauh$) events. 
The test is referred as ``closure test'' where the level of disagreement (non-closure) indicates the generic accuracy about the method. The evaluated non-closure is assigned as systematics uncertainty. 
In the MC closure test, simulated $\ttbar$ sample is used in both seed events and the tested $\llmis$ ($\ltauh$) events; 
the sample of seed events is provided by $\ttbar \ra b\ell\nu b\ell\nu$ and that of the tested events are by $\ttbar \ra b\ell\nu b \ell_{\mathrm{mis.}} \nu$ and $\ttbar \ra b\ell\nu b \tau \nu, \tau\ra\tau_h\nu_{\tau}$. All irrelevant processes are excluded thus no subtraction is needed.
The common 2LCR selection as defined in Table \ref{tab::BGestimation::objRep::def2LCR} is applied for seed events selection, except that the MET cut is removed in order to boost the statistics. \\

Figure \ref{fig::ObjReplace::mcClosure_MisLep_el}-\ref{fig::ObjReplace::mcClosure_TauRep_emu} show the result with $\pt>35\gev$ is required for the tag lepton. The test result for the case with a soft lepton ($\pt\in[6,35]\gev$) is displayed in the Appendix \ref{sec::App::objRep_closure_softLep}. 
%and fig.\ref{fig::BGestimation::objRep::mcClosure_softLep_MisLep_el} $\sim$ \ref{fig::BGestimation::objRep::mcClosure_softLep_TauRep_emu} are for the soft lepton regions. 

Good closure is seen in overall kinematics. Non-closure generally stay within $10\%$ ($5\%$), and never exceeds $30\%$ ($10\%$) significantly for the missing lepton replacement (the tau replacement).
Although the closure of missing lepton replacement is worse than that of tau replacement, it is not worrisome since the contribution of $\ell\ell_{\mathrm{mis.}}$ is typically $\sim5$ times smaller than $\ell\tau_{\mathrm{h}}$. \\

Closure tests are also performed in phase space close to signal regions. 
Figure \ref{fig::BGestimation::objRep::mcClosure::MisLepTauRep_regionYieldsBT}-\ref{fig::BGestimation::objRep::mcClosure::MisLepTauRep_regionYieldsBV} 
are the btag/bveto-splitted closure in various regions requiring high MET, $m_{T}$, $m_{\mathrm{eff.}}$ etc. The non-closure stay within 10$\%$ with respect to the overall estimation.\\

% ---------- mcClosure::plot::hardLep
%%%%% METHNAME %%%%%%%%%%%%%%%%%%%%%%%%%%
\begin{figure}[h]
  \centering
    \subfig{0.32}{figures/BGestimation/ObjReplacement/mcClosure/MisLep_el/MisLep_el_jet1Pt__trMode4_NoSys.pdf}{$\pt$ of leading jet}
    \subfig{0.32}{figures/BGestimation/ObjReplacement/mcClosure/MisLep_el/MisLep_el_meffInc30__trMode4_NoSys.pdf}{$\meffInc$}
    \subfig{0.32}{figures/BGestimation/ObjReplacement/mcClosure/MisLep_el/MisLep_el_met__trMode4_NoSys.pdf}{$\met$}
    \subfig{0.32}{figures/BGestimation/ObjReplacement/mcClosure/MisLep_el/MisLep_el_mt__trMode4_NoSys.pdf}{$\mt$}
    \subfig{0.32}{figures/BGestimation/ObjReplacement/mcClosure/MisLep_el/MisLep_el_tagLepPt__trMode4_NoSys.pdf}{$\pt$ of tag lepton}
    \subfig{0.32}{figures/BGestimation/ObjReplacement/mcClosure/MisLep_el/MisLep_el_metOverMeff__trMode4_NoSys.pdf}{$\met/\meffInc$}
    \subfig{0.32}{figures/BGestimation/ObjReplacement/mcClosure/MisLep_el/MisLep_el_LepAplanarity__trMode4_NoSys.pdf}{Aplanarity}
    \subfig{0.32}{figures/BGestimation/ObjReplacement/mcClosure/MisLep_el/MisLep_el_nJet30__trMode4_NoSys.pdf}{Jet multiplicity}
    \subfig{0.32}{figures/BGestimation/ObjReplacement/mcClosure/MisLep_el/MisLep_el_nBJet30__trMode4_NoSys.pdf}{$b$-jet multiplicity}
    \caption{ MC closure test for \textbf{missing lepton replacement} using $t\bar{t}$ MC sample. Seed events are collected by the single-lepton trigger. $p_T>35\gev$ for the leading lepton is required. \textbf{Only electrons in the seed events are replaced}. Red points in the bottom plots show the ratio of integrated yields for the two histograms above the x-position that the point indicates. \label{fig::ObjReplace::mcClosure_MisLep_el} }
\end{figure}
 %%%%%%%%%%%%%%%%%%%%%%%%%%

%%%%% METHNAME %%%%%%%%%%%%%%%%%%%%%%%%%%
\begin{figure}[h]
  \centering
    \subfig{0.32}{figures/BGestimation/ObjReplacement/mcClosure/MisLep_mu/MisLep_mu_jet1Pt__trMode4_NoSys.pdf}{$\pt$ of leading jet}
    \subfig{0.32}{figures/BGestimation/ObjReplacement/mcClosure/MisLep_mu/MisLep_mu_meffInc30__trMode4_NoSys.pdf}{$\meffInc$}
    \subfig{0.32}{figures/BGestimation/ObjReplacement/mcClosure/MisLep_mu/MisLep_mu_met__trMode4_NoSys.pdf}{$\met$}
    \subfig{0.32}{figures/BGestimation/ObjReplacement/mcClosure/MisLep_mu/MisLep_mu_mt__trMode4_NoSys.pdf}{$\mt$}
    \subfig{0.32}{figures/BGestimation/ObjReplacement/mcClosure/MisLep_mu/MisLep_mu_tagLepPt__trMode4_NoSys.pdf}{$\pt$ of tag lepton}
    \subfig{0.32}{figures/BGestimation/ObjReplacement/mcClosure/MisLep_mu/MisLep_mu_metOverMeff__trMode4_NoSys.pdf}{$\met/\meffInc$}
    \subfig{0.32}{figures/BGestimation/ObjReplacement/mcClosure/MisLep_mu/MisLep_mu_LepAplanarity__trMode4_NoSys.pdf}{Aplanarity}
    \subfig{0.32}{figures/BGestimation/ObjReplacement/mcClosure/MisLep_mu/MisLep_mu_nJet30__trMode4_NoSys.pdf}{Jet multiplicity}
    \subfig{0.32}{figures/BGestimation/ObjReplacement/mcClosure/MisLep_mu/MisLep_mu_nBJet30__trMode4_NoSys.pdf}{$b$-jet multiplicity}
    \caption{ MC closure test for \textbf{missing lepton replacement} using $t\bar{t}$ MC sample. Seed events are collected by the single-lepton trigger. $p_T>35\gev$ for the leading lepton is required. \textbf{Only muon in the seed events are replaced}. Red points in the bottom plots show the ratio of integrated yields for the two histograms above the x-position that the point indicates. \label{fig::ObjReplace::mcClosure_MisLep_mu} }
\end{figure}
 %%%%%%%%%%%%%%%%%%%%%%%%%%

%%%%% METHNAME %%%%%%%%%%%%%%%%%%%%%%%%%%
\begin{figure}[h]
    \subfig{0.32}{figures/BGestimation/ObjReplacement/mcClosure/TauRep_emu/TauRep_emu_jet1Pt__trMode4_NoSys.pdf}{$\pt$ of leading jet}
    \subfig{0.32}{figures/BGestimation/ObjReplacement/mcClosure/TauRep_emu/TauRep_emu_meffInc30__trMode4_NoSys.pdf}{$\meffInc$}
    \subfig{0.32}{figures/BGestimation/ObjReplacement/mcClosure/TauRep_emu/TauRep_emu_met__trMode4_NoSys.pdf}{$\met$}
    \subfig{0.32}{figures/BGestimation/ObjReplacement/mcClosure/TauRep_emu/TauRep_emu_mt__trMode4_NoSys.pdf}{$\mt$}
    \subfig{0.32}{figures/BGestimation/ObjReplacement/mcClosure/TauRep_emu/TauRep_emu_tagLepPt__trMode4_NoSys.pdf}{$\pt$ of tag lepton}
    \subfig{0.32}{figures/BGestimation/ObjReplacement/mcClosure/TauRep_emu/TauRep_emu_metOverMeff__trMode4_NoSys.pdf}{$\met/\meffInc$}
    \subfig{0.32}{figures/BGestimation/ObjReplacement/mcClosure/TauRep_emu/TauRep_emu_LepAplanarity__trMode4_NoSys.pdf}{Aplanarity}
    \subfig{0.32}{figures/BGestimation/ObjReplacement/mcClosure/TauRep_emu/TauRep_emu_nJet30__trMode4_NoSys.pdf}{Jet multiplicity}
    \subfig{0.32}{figures/BGestimation/ObjReplacement/mcClosure/TauRep_emu/TauRep_emu_nBJet30__trMode4_NoSys.pdf}{$b$-jet multiplicity}
    \caption{ MC closure test for \textbf{tau replacement} using $t\bar{t}$ MC sample. Seed events are collected by the single-lepton trigger. $p_T>35\gev$ for the leading lepton is required. \textbf{Both electrons and muons in the seed events are replaced}. Red points in the bottom plots show the ratio of integrated yields for the two histograms above the x-position that the point indicates. \label{fig::ObjReplace::mcClosure_TauRep_emu} }
\end{figure}
 %%%%%%%%%%%%%%%%%%%%%%%%%%



%%%
\if 0
% -------------- mcClosure::MisLep_regionYields
\begin{figure}[htbp]
  \begin{center}
    \includegraphics[width=160mm]{figures/BGestimation/ObjReplacement/mcClosure_ATLASNote/MisLep_emu/MisLep_emu_regionYieldsBT__trMode4_NoSys.eps}
    \captionof{figure}{MC closure test for \textbf{missing lepton replacement} in SR-like \textbf{b-tagged} regions. Pre-selection $p_{T}(\ell_{1})>35\gev$ is applied on top of the cuts noted by the labels.}
    \label{fig::BGestimation::objRep::mcClosure::MisLep_regionYieldsBT}
%%
    \includegraphics[width=160mm]{figures/BGestimation/ObjReplacement/mcClosure_ATLASNote/MisLep_emu/MisLep_emu_regionYieldsBV__trMode4_NoSys.eps}
    \captionof{figure}{MC closure test for \textbf{missing lepton replacement} in SR-like \textbf{b-vetoed} regions. Pre-selection $p_{T}(\ell_{1})>35\gev$ is applied on top of the cuts noted by the labels.}
    \label{fig::BGestimation::objRep::mcClosure::MisLep_regionYieldsBV}
  \end{center}
\end{figure}
%-------------------------------

% -------------- mcClosure::TauRep_regionYields
\begin{figure}[htbp]
  \begin{center}
    \includegraphics[width=160mm]{figures/BGestimation/ObjReplacement/mcClosure_ATLASNote/TauRep_emu/TauRep_emu_regionYieldsBT__trMode4_NoSys.eps}
    \captionof{figure}{MC closure test for \textbf{tau replacement} in SR-like \textbf{b-tagged} regions. Pre-selection $p_{T}(\ell_{1})>35\gev$ is applied on top of the cuts noted by the labels.}
    \label{fig::BGestimation::objRep::mcClosure::TauRep_regionYieldsBT}
%%
    \includegraphics[width=160mm]{figures/BGestimation/ObjReplacement/mcClosure_ATLASNote/TauRep_emu/TauRep_emu_regionYieldsBV__trMode4_NoSys.eps}
    \captionof{figure}{MC closure test for \textbf{tau replacement} in SR-like \textbf{b-vetoed} regions. Pre-selection $p_{T}(\ell_{1})>35\gev$ is applied on top of the cuts noted by the labels.}
    \label{fig::BGestimation::objRep::mcClosure::TauRep_regionYieldsBV}
  \end{center}
\end{figure}
%-------------------------------
\fi

% -------------- mcClosure::MisLep+TauReo_regionYields
\begin{figure}[htbp]
  \begin{center}
    \includegraphics[width=160mm]{figures/BGestimation/ObjReplacement/mcClosure_ATLASNote/All_emu/All_emu_regionYieldsBT__trMode4_NoSys.eps}
    \captionof{figure}{MC closure test for \textbf{combined estimation of missing-lep. rep. and tau replacement} in SR-like \textbf{b-tagged} regions. Pre-selection $p_{T}(\ell_{1})>35\gev$ is applied on top of the cuts noted by the labels.}
    \label{fig::BGestimation::objRep::mcClosure::MisLepTauRep_regionYieldsBT}
%%
    \includegraphics[width=160mm]{figures/BGestimation/ObjReplacement/mcClosure_ATLASNote/All_emu/All_emu_regionYieldsBV__trMode4_NoSys.eps}
    \captionof{figure}{MC closure test for \textbf{combined estimation of missing-lep. rep. and tau replacement} in SR-like \textbf{b-vetoed} regions. Pre-selection $p_{T}(\ell_{1})>35\gev$ is applied on top of the cuts noted by the labels.}
    \label{fig::BGestimation::objRep::mcClosure::MisLepTauRep_regionYieldsBV}
  \end{center}
\end{figure}
%-------------------------------




\clearpage
%\paragraph{Source of non-closure} 
Visible non-closures are found in some distributions such as MET and jet transverse momentum, and the cause is nailed down as following:

\begin{description}
\item [Kinematical bias triggered by the two-lepton requirement in seed event selection] \mbox{} \\
Though the orthogonality between kinematics and object properties (Eq. \ref{eq::BGestimation::objRep::orthKineObj}) generally hold as a good approximation, there is still some exception. The most notable example is when the parent particles of the two leptons in a seed event are heavily boosted, and the leptons get collimated and overlapped each other. This leads to deteriorated reconstruction/ID efficiency, therefore selecting events with exactly two leptons will discard part of such type of seed events automatically. 
The estimated spectra is biased and generally become softer. Electrons address more severe effect because the efficiency drop in the boosted environment is more striking than the case of muons.
%%%とはいえそれはとてつもないtailでの話で、我々が相手にしてるようなphase spaceではまだそんなことは起こらないのでclosure testで見えてる傾向を説明が、別に解析にimpactを与える訳ではない


% -------------- mcClosure::polCorrection
\item [Wrong assumption on tau polarization (for tau replacement)] \mbox{} \\
For technical simplicity, tau decays are performed based on the unpolarized tau assumption, which is not true given that tau leptons here are mostly generated through weak decays of $W$-bosons. 
This can be seen in the non-closure observed in the visible tau fraction $x_{\tau} := E(\tau_{h})/E(\tau)$ as shown in Figure \ref{fig::BGestimation::objRep::mcClosure::rwgt_x} (a), and the non-closure found in the tails of the MET or $\mt$ distribution is mainly ascribed to the propagated effect, given that those non-closure can be recovered by the ad hoc reweighting in terms of $x_{\tau}$ (see Figure \ref{fig::BGestimation::objRep::mcClosure::rwgt_x_mt}).
%Figure \ref{fig::BGestimation::objRep::mcClosure::rwgt_x} (a) shows the visible tau fraction $x_{\tau} := E(\tau_{h})/E(\tau)$, a variable sensitive to tau polarization, for taus in the $\ttbar$ process in a blue line, and for the case of unpolarized hypothesis in a red line. This discrepancy is known to eventually propagated to the non-closures in the tail of MET and $\mtFull$ such as the left plots in Figure \ref{fig::BGestimation::objRep::mcClosure::rwgt_x} and fig. \ref{fig::BGestimation::objRep::mcClosure::rwgt_x_mt}. On the other hand, these non-closure can be cured by a simple reweighting in terms of $x_{\tau}$, as they are purely caused by the issue of polarization modeling. Obtaining the reweighting function by fitting the non-closure in $x_{\tau}$ with a third polynomial as shown in Figure \ref{fig::BGestimation::objRep::mcClosure::rwgt_x} (c), nicely recovered closures in MET and $\mtFull$ are confirmed as in the right plots in Figure \ref{fig::BGestimation::objRep::mcClosure::rwgt_x} and fig. \ref{fig::BGestimation::objRep::mcClosure::rwgt_x_mt} respectively.  \\
\footnote{This $x_{\tau}$-reweighting correction is not brought into practice in the estimation, because the $x_{\tau}$-profile varies by the physics processes (e.g. $\ttbar$, $Wt$ or $WW$ etc.) and the information of their relative breakdown needs to be provided from MC which uncertainty is not easy to evaluate.}
%Although the fitting function is found to be almost phase space independent as shown in Figure \ref{fig::BGestimation::objRep::mcClosure::rwgt_x} (c)
Since the impact of this non-closure is marginal in estimating VRs and SRs ($<5\%$) fortunately, it is decided to be left as it is.

\begin{figure}[h]
  \centering
    \subfig{0.39}{figures/BGestimation/ObjReplacement/mcClosure_ATLASNote/source_nonClosure/taupol/TauRep_emu_tr_hadTau_x_noATLASLabel.pdf}{}
    \subfig{0.455}{figures/BGestimation/ObjReplacement/mcClosure_ATLASNote/source_nonClosure/taupol/fit_hadtaux.eps}{}
    \caption{ (a) Closure test in terms $x_{\tau} := E(\tau_{h})/E(\tau)$. (b) Non-closure in terms of $x_{\tau}$ in various phase spaces. The non-closure cab be well-modeled by a common third polynomial function, largely irrespective to phase space.
  \label{fig::BGestimation::objRep::mcClosure::rwgt_x} }
\end{figure}

\begin{figure}[h]
  \centering
    \subfig{0.23}{figures/BGestimation/ObjReplacement/mcClosure_ATLASNote/source_nonClosure/taupol/TauRep_emu_mt_noATLASLabel.pdf}{}
    \subfig{0.23}{figures/BGestimation/ObjReplacement/mcClosure_ATLASNote/source_nonClosure/taupol/rwgt_x_TauRep_emu_mt_noATLASLabel.pdf}{}
    \subfig{0.23}{figures/BGestimation/ObjReplacement/mcClosure_ATLASNote/source_nonClosure/taupol/TauRep_emu_met_noATLASLabel.pdf}{}
    \subfig{0.23}{figures/BGestimation/ObjReplacement/mcClosure_ATLASNote/source_nonClosure/taupol/rwgt_x_TauRep_emu_met_noATLASLabel.pdf}{}
    \caption{ (a) $m_{\mathrm{T}}$ and (c) MET distribution before the reweighting in $x_{\tau}$, and (b)(d) after the reweighting. The reweighting fucntion is given by Figure \ref{fig::BGestimation::objRep::mcClosure::rwgt_x} (b)  \label{fig::BGestimation::objRep::mcClosure::rwgt_x_mt} }
\end{figure}

%\begin{figure}[h]
%  \centering

%\end{figure}


% -------------- mcClosure::metSoftTerm
\clearpage
\item [Treatment of missing muon (for missing muon replacement)]  \mbox{} \\
While the emulated missing muons are completely regarded as invisible particles in the replacement algorithm, the momenta of real unidentified muons do contribute to MET since their tracks are often included in the track soft term. This imperfect emulation leads to a non-closure around MET-related variables in the missing muon replacement. Naively thinking, this can be improved by simply stopping adding the missing muons momenta into MET. However, this is unfortunately not the case, as shown in Figure \ref{fig::BGestimation::objRep::mcClosure::metSoftTerm_mu} where the improvement is limited in bulk region of the MET spectrum and the closure in the tail gets even worse. This is mainly because the poor momentum resolution of high pt unidentified muons is not emulated in the replacement. As the implementation of the full emulation is too costly compared with the small portion of missing muons backgrounds in the estimated regions, it is decided to keep the original treatment. Instead, the $30\%$ of non-closure error is assigned to the estimation of the missing muon background. \\

\begin{figure}[h]
  \centering
    \subfig{0.4}{figures/BGestimation/ObjReplacement/mcClosure_ATLASNote/source_nonClosure/metSoftTerm/MisLep_mu_met_noATLASLabel.pdf}{}
    \subfig{0.4}{figures/BGestimation/ObjReplacement/mcClosure_ATLASNote/source_nonClosure/metSoftTerm/InclSoftTerm_MisLep_mu_met_noATLASLabel.pdf}{}
    \caption{The MC closure of MET distribution for the missing muon sub-events, with missing muons are fully regarded (a) as invisible particles (default treatment), or (b) as visible particles.
      \label{fig::BGestimation::objRep::mcClosure::metSoftTerm_mu} }
\end{figure}

\end{description}
\clearpage



%%%%%%%%%%%%%%%%%%%%%%%%%%%%%%%%%%%%%%%%%%%%%%%%%%
\clearpage
\subsection{Subtraction of Bogus Sub-events} \label{sec::BGestimation::objRep::subtraction}
One of the merits about the object replacement method is that 
the ``di-leptonic'' background can be inclusively estimated with a complete coverage regardless of the physics processes.
\footnote{The main contriubtion is from $\ttbar$, $Wt$ and $WW$, while minor contribution is found from $\ttbar+W/Z$ and $WZ$ etc.}
The replacement of seed leptons from $W$-decays are totally valid, since the replaced sub-events do represent part of existing events in 1-lepton regions. 
On the other hand, the generated sub-events are not always sensible depending on the origin of the seed leptons:
\begin{itemize}
\item Seed leptons from $Z$-boson decays requests attention i.e. the sub-events of tau replacement will lead to a bogus topology of $Z\ra \tau_h \ell \,\, (\ell=e,\mu)$ which never happens, thus these sub-events (bogus sub-events) are need to be subtracted.  
%
\item Likewise, seed leptons from leptonic tau decays ($\tau \ra \ell \nu\bar{\nu}$) have the same issue that tau replacement leads to bogus sub-event where tau decays into tau again. 
%
\item Replacing a fake lepton can only end up in bogus sub-events. While the sub-traction takes place on sub-event basis, if can be only done statistically i.e. evaluate total contribution from bogus sub-events and subtract once.
\end{itemize}

\noindent The summary of legal and illegal replacement is given in Table \ref{tab::BGestimation::objRep::relProc} where bogus sub-events are label as $``\times''$.
Note that the decision is made at the sub-event level (not at the seed event level), therefore even $W(\ra \ell \nu)+\ell_{\mathrm{fake}}$ can be seed events as long as one replaces $\ell$ rather than $\ell_{\mathrm{fake}}$.  \\

The largest source of bogus sub-events are seed events with $\tau \ra \ell \nu\bar{\nu}$ (denoted as $\tau_{\ell}$).
The contribution is quite large, accounting for $10\%\sim20\%$ of the estimated yields by the tau replacement. Therefore, a naive MC subtraction could introduce culprits from the MC mis-modeling, for example on $\ttbar$ as overviewed in Sec. \ref{sec::BGestimation::dataMC}. Instead, to avoid the impact, the subtraction is done in a form of ratio, such as:
\begin{align}
  y^{\mathrm{Data}}_{\ell} 
  & = y^{\mathrm{Data}}_{\ell+\tau_{\ell}} \times \frac{y^{\mathrm{MC}}_{\ell}}{y^{\mathrm{MC}}_{\ell}+y^{\mathrm{MC}}_{\tau_{\ell}}}  
%  & = y^{\mathrm{Data}}_{\ell\ell+\ell\tau_{\ell}} - \frac{y^{\mathrm{Data}}_{\ell\ell}}{y^{\mathrm{MC}}_{\ell\ell}} \times y^{\mathrm{MC}}_{\ell\tau_{\ell}}
\end{align}
where $y^{\mathrm{Data}}_{\ell}$ ($y^{\mathrm{Data}}_{\ell+\tau_{\ell}}$) denote the total yield estimated by tau replacement using data before (after) the subtraction, and $y^{\mathrm{MC}}_{\ell}$ ($y^{\mathrm{MC}}_{\tau_{\mathrm{\ell}}}$) the contribution from legal (bogus) sub-events of tau replcement estimated by MC. \\
%need some more explanation

The subtraction of the $\ell\ell_{\mathrm{fake}}$ is a little sensitive as MC modeling on fake leptons is less reliable in general. Therefore, relatively more aggressive suppression is applied at the stage of seed selection (Table \ref{tab::BGestimation::objRep::seedLepReq}) by requiring tighter isolation, in case that it could be addressing. \\


\begin{table}[h]
  \begin{center}
    \caption{Correspondence between origin of seed lepton and estimated components by the missing lepton replacement or the tau replacement. $X$ represents any arbitrary particles. $''\times''$ indicates that the generated sub-events represent non-existing processes (``bogus sub-events'') that requires the subtraction. The subscripts ${\mathrm{mis.}}$ denote missing leptons (leptons categorized in ``Mis. Reco'' and ``Mis. ID'' defined in Table \ref{tab::BGestimation::BGclass}).
}

    \begin{tabular}{  c | c | c  }
      \hline 
      Parent of seed lepton &   sub-events of mis. lep. rep.  &  sub-events of tau rep.  \\
%
      \hline 
      \hline      
      $W (\ra\ell\nu)$ &       $W+X, \, W\ra \ell_{\mathrm{mis.}}\nu$       &      $W+X, \, W\ra\tau_{\mathrm{h}} \nu$ \\
      \hline
      $Z (\ra\ell\ell)$ &       $Z+X, \, Z\ra \ell_{\mathrm{mis.}}\ell$     &      $\times$ \\
      \hline
      $\tau (\ra \tau_\ell \nu)$ &              $\tau_{\ell,{\mathrm{mis.}}}+X$             &      $\times$ \\
      \hline
      Fake &       $\times$       &      $\times$ \\
      \hline
    \end{tabular}  \label{tab::BGestimation::objRep::relProc}
  \end{center}
\end{table}








%%%%%%%%%%%%%%%%%%%%%%%%%% Data Closure test %%%%%%%%%%%%%%%%%%%%%%%%%%%%%%%%
\clearpage
%\subsection{Closure Test using Data in the Loose Validation Regions.} \label{sec::BGestimation::objRep::dataClosure}
\subsection{Closure Test using Data} \label{sec::BGestimation::objRep::dataClosure}
In order to demonstrate the procedures beyond the ideal MC closure tests done in Sec. \ref{sec::BGestimation::objRep::mcClosure} such as the bogus sub-event subtraction (Sec. \ref{sec::BGestimation::objRep::subtraction}),
%where only di-leptonic $\ttbar$ is taken into account, 
another validation study is done using the data events.

Since the nominal VRs (Table \ref{SRdefinition::regionDef2J} - \ref{SRdefinition::regionDef3B}) tends to have too tight selections resulting in small data statistics, 
a set of high-$\mt$ regions ``VRs-objRep'' with relatively loose selections are deliberately defined, in which the object replacement estimation and data is compared. Nine complementary bins are defined as shown in Table \ref{tab::BGestimation::objRep::VRobjRep}. \\

It is populated by $\llmis + \ltauh$ events with the purity of $\sim 30\sim60\%$, and the rest of backgrounds that are not covered by the object replacement (namely the ``semi-leptonic'', ``2L-Out. Acc'' and ``2L-Mis. OR'' components) are estimated by a kinematics extrapolation where the MC of $\wjets$ and the top background is normalized to data in the corresponding control region bins (``CRs-objRep'') defined in Table \ref{tab::BGestimation::objRep::VRobjRep} which are only different in the $\mt$ cut with respect to VRs-objRep. An upper cut in aplanarity is placed in both VRs-objRep and CRs-objRep so that the signal contamination is reasonably subdued. Statistical uncertainty from the control region statistics, and a flat $5\%$ non-closure error is assigned for the object replacement estimation in all the VR bins. \\

The result is presented in Figure \ref{fig::BGestimation::objRep::dataClosure::result}. 
Nice agreement is observed where data is consistent with expectations within the uncertainty. 

\tab{c|ccccc}{
  \hline
               &  $\nJet$  &  $\met$ [GeV]   &   $\mt$ [GeV] (CR-objRep)  &   $\meffInc$ [GeV]   &  Aplanarity \\ 
  \hline
  \hline
  bin-1        & $\geq 4$  &  $>200$   &   $>125$  ($\in[60,125]$)     &   $>1500$      &  $<0.03$    \\        
  bin-2        & $\geq 4$  &  $>200$   &   $>125$  ($\in[60,125]$)     &   $>2000$      &  $<0.03$    \\        
  bin-3        & $\geq 4$  &  $>200$   &   $>175$  ($\in[60,125]$)     &   $>1000$      &  $<0.03$    \\        
  bin-4        & $\geq 4$  &  $>200$   &   $>400$  ($\in[60,125]$)     &   $-$          &  $<0.03$    \\        
  bin-5        & $\geq 4$  &  $>200$   &   $>400$  ($\in[60,125]$)     &   $>1000$      &  $<0.03$    \\        
  bin-6        & $\geq 4$  &  $>300$   &   $>175$  ($\in[60,125]$)     &   $-$          &  $<0.03$    \\        
  bin-7        & $\geq 4$  &  $>400$   &   $>175$  ($\in[60,125]$)     &   $>1000$      &  $<0.03$    \\        
  bin-8        & $\geq 6$  &  $>400$   &   $>400$  ($\in[60,125]$)     &   $-$          &  $<0.03$    \\        
  bin-9        & $\geq 6$  &  $>200$   &   $>125$  ($\in[60,125]$)     &   $>1500$      &  $<0.03$    \\        
  \hline
}
{Definition of VRs(CRs)-objRep. MC of $\wjets$ and the top background are normalized in corresponding CR-objRep.}
{tab::BGestimation::objRep::VRobjRep}

\clearpage
% -------------- dataClosure::result
\includegraphics[width=180mm]{figures/BGestimation/ObjReplacement/dataClosure/hist_regionYields_myVRsOnly_wide_metTrig0_NoSys_mode0_noATLASLabel.pdf}
\captionof{figure}{
Closure test in VR-objRep bins. The white component is the yield of the $\llmis + \ltauh$ events that are estimated by the object replacement, 
while the colored stack represents the ``semi-leptonic'' (purple) and rest of the ``di-leptonic'' component (``2L-Out. Acc / Mis. OR'', orange) respectively. 
The bottom row plots the ratio between the estimated yield and the actual number of data. 
The gray dashed band shows the uncertainty in the estimation, including statistical error due to the CR statistics and a flat $5\%$ non-closure for the object replacement.}
\label{fig::BGestimation::objRep::dataClosure::result}
%-------------------------------


%%%%%%%%%%%%%%%%%%%%%%%%%%%%%%%%%%%%%%%%%%%%%%%%%%%%%%%%%%
%\subsection{Signal contamination}
%\input{tex/BGestimation/fig_sigContami_objRep.tex}


%%%%%%%%%%%%%%%%%%%%%%%%%%%%%% VR plots %%%%%%%%%%%%%%%%%%%%%%%%%%
\clearpage
\section{Unblinded Validation Regions} \label{sec::BGestimation::unblindedVRs}
The background estimation is inclusively tested in validation regions VRa and VRb defined in Table \ref{SRdefinition::regionDef2J} - \ref{SRdefinition::regionDef3B},
where the phase space are close enough to the signal regions, giving the sensible demonstration of the estimation. \\

Table \ref{tab::BGestimation::VRyields_2J} - \ref{tab::BGestimation::VRyields_3B} show the data yields compared with the expected backgrounds together with the breakdown. 
The components estimated by the object replacement are merged and denoted as ``Di-leptonic'' in the tables, 
while the yields for the other components provided by the kinematical extrapolation are exlusively listed for each physic process. 
The attached errors are all post-fit uncertainty obtained by profiling the nuisance parameters (detail can be found in Sec. \ref{sec::Result::statistics}). \\

The visualized comparison between data and background expectation is illustrated in \ref{fig::BGestimation::VRPulls}, together with the pulls defined by the number of gaussian-equivalent deviation. 
The tension with respect to data never exceeds $2\sigma$, which can still be ascribed to the effects that the systematic uncertainties assignd mainly in the kinematical extrapolation. For instance, the trend of underestion on $\wjets$ in some of the VRb (in particular 2J) is expected by to the correlation between extrapolating cariables and the ill-modeled variables, which will be discussed in detail in Sec. \ref{sec::BGestimation::kineExtp}. 
15$\%$ of uncertainty is in fact assigned for this effect (based on Figure \ref{fig::BGestimation::valid_extp_VRb2J}, with the the mis-modeling parameter $w$ to be at $\sim0.1$). \\

Another source of systematical underestimation is understood by the potential MC mis-modeling in the $\mt$ shape as mentioned in Sec. \ref{sec::BGestimation::dataMC}; for $\wjets$, 
the cut-off at $\mt\sim m_W$ in MC is sharper than that in the data.
%supposedly due to the narrow mass width approximation in the matrix-element calculation in \sherpa. 
No theoretical uncertainties are dedicatedly assigned for this effect, however it could still be explained by other theoretical uncertainties given the $\sim 1 \sigma$ discrepancy; 
for $\ttbar+Wt$, lack of full description of interference between the non-top $WWbb$ diagrams is the potential reason for the underestimation for which $5\% \sim 30\%$ of systematics is assigned for the portion that the kinematical extrapolation estimates. 
All in all, underestimation upto $1\sigma$ is expected therefore we don't regard this as an issue.  \\
%Similar uncorrected systematic effects are also known in the object replacement where the non-closures discussed above always lead to underestimation, which give rises a small systematical underestimation in phase space with extremely hard kinematics.

The post-fit kinematical distributions in VRs are presented in Figure \ref{fig::BGestimation::SRVRpostFit::VR2J}-\ref{fig::BGestimation::SRVRpostFit::VR3B} in the appendix.

\clearpage

\newcommand{\vrtablecaption}[1]{%
      Event yields and the background-only fit results in the #1 control regions. 
      Each column corresponds to a bin in $\meffInc$.
      Uncertainties in the MC estimates combine statistical (in the simulated event yields) and systematic uncertainties discussed in Sec \ref{sec::Uncertainties}. The uncertainties in this table are symmetrised for propagation purposes but truncated at zero to remain within the physical boundaries.
}
\newcommand{\vrtablecaptionvarx}{%
      Event yields and the background-only fit results in the ``Low-x'' and ``High-x'' control regions. 
      Uncertainties in the MC estimates combine statistical (in the simulated event yields) and systematic uncertainties discussed in Sec \ref{sec::Uncertainties}. The uncertainties in this table are symmetrised for propagation purposes but truncated at zero to remain within the physical boundaries.
}
\begin{table}
  \begin{center}
    \caption{ \label{tab::BGestimation::VRyields_2J} \vrtablecaption{``2J''}  }
    
    \begin{tabular*}{\textwidth}{@{\extracolsep{\fill}}lrrrr}
      \toprule
      \textbf{VRa 2J} & $m_{\mathrm{eff.}}\in$[1100,1500] & $m_{\mathrm{eff.}}\in$[1500,1900] & $m_{\mathrm{eff.}}>1900$ \\
      \midrule
      
Observed data          & $222$              & $46$              & $23$                    \\
\midrule
Expected background         & $209.88 \pm 21.80$          & $38.67 \pm 5.51$          & $17.28 \pm 3.25$              \\
\midrule
        Di-leptonic         & $91.5 \pm 11.1$          & $11.7 \pm 2.6$          & $5.7 \pm 1.6$              \\
        $W$+jets         & $20.9 \pm 3.8$          & $8.0 \pm 2.6$          & $3.2 \pm 1.3$              \\
        $Z$+jets         & $2.6 \pm 0.7$          & $0.8 \pm 0.2$          & $0.3 \pm 0.1$              \\
        Tops         & $85.6 \pm 18.0$          & $15.6 \pm 3.8$          & $6.9 \pm 2.8$              \\
        Di-boson         & $6.0 \pm 2.2$          & $1.5 \pm 0.5$          & $0.8 \pm 0.3$              \\
        $t\bar{t}+V$         & $3.2 \pm 0.5$          & $0.9 \pm 0.2$          & $0.4 \pm 0.1$              \\
\toprule
\textbf{VRb 2J} & $m_{\mathrm{eff.}}\in$[1100,1500] & $m_{\mathrm{eff.}}\in$[1500,1900] & $m_{\mathrm{eff.}}>1900$ \\
\midrule
Observed data          & $390$              & $113$              & $52$                    \\
\midrule
Expected background         & $314.33 \pm 36.92$          & $104.33 \pm 13.80$          & $41.34 \pm 8.95$              \\
\midrule
        Di-leptonic         & $10.5 \pm 2.4$          & $3.0 \pm 1.1$          & $3.7 \pm 1.3$              \\
        $W$+jets         & $219.5 \pm 34.9$          & $76.8 \pm 13.0$          & $24.9 \pm 9.2$              \\
        $Z$+jets         & $5.1 \pm 1.3$          & $2.0 \pm 0.6$          & $0.8 \pm 0.2$              \\
        Tops         & $56.7 \pm 14.1$          & $15.7 \pm 4.5$          & $8.1 \pm 3.2$              \\
        Di-boson         & $21.3 \pm 7.4$          & $6.3 \pm 4.5$          & $3.5 \pm 1.1$              \\
        $t\bar{t}+V$         & $1.2 \pm 0.2$          & $0.5 \pm 0.1$          & $0.4 \pm 0.1$              \\
        \bottomrule
        %%      
        \end{tabular*}

  \end{center}
\end{table}
%%%%%%%%%%%%%%%%%%%%%%%%%%%%%%%%%%%%%%%%%%%%%%%%%%%%%%%%%%%%




\begin{table}
  \begin{center}
    \caption{ \label{tab::BGestimation::VRyields_6J} \vrtablecaption{``6J''}  }

    \begin{tabular*}{\textwidth}{@{\extracolsep{\fill}}lrrrr}
      \toprule
      \textbf{VRa 6J} & $m_{\mathrm{eff.}}\in$[1100,1600] & $m_{\mathrm{eff.}}\in$[1600,2100] & $m_{\mathrm{eff.}}>2100$ \\
      \midrule
      
Observed data          & $130$              & $60$              & $31$                    \\
\midrule
Expected background         & $134.22 \pm 18.17$          & $48.27 \pm 7.79$          & $28.71 \pm 4.57$              \\
\midrule
        Di-leptonic         & $71.9 \pm 15.2$          & $24.7 \pm 6.8$          & $11.5 \pm 3.5$              \\
        $W$+jets         & $7.6 \pm 1.8$          & $4.0 \pm 1.0$          & $2.5 \pm 0.9$              \\
        $Z$+jets         & $0.6 \pm 0.2$          & $0.3 \pm 0.1$          & $0.2 \pm 0.1$              \\
        Tops         & $45.7 \pm 9.8$          & $16.0 \pm 3.7$          & $12.0 \pm 2.9$              \\
        Di-boson         & $4.6 \pm 1.5$          & $2.1 \pm 0.7$          & $1.6 \pm 0.7$              \\
        $t\bar{t}+V$         & $3.8 \pm 0.7$          & $1.2 \pm 0.3$          & $0.9 \pm 0.2$              \\
\toprule
\textbf{VRb 6J} &  $m_{\mathrm{eff.}}\in$[1100,1600] & $m_{\mathrm{eff.}}\in$[1600,2100] & $m_{\mathrm{eff.}}>2100$ \\
\midrule
Observed data          & $99$              & $53$              & $26$                    \\
\midrule
Expected background         & $84.21 \pm 10.42$          & $43.22 \pm 5.50$          & $25.15 \pm 3.89$              \\
\midrule
        Di-leptonic         & $0.9 \pm 0.4$          & $1.0 \pm 0.8$          & $0.5 \pm 0.4$              \\
        $W$+jets         & $32.9 \pm 6.5$          & $21.8 \pm 4.7$          & $8.6 \pm 2.9$              \\
        $Z$+jets         & $0.4 \pm 0.2$          & $0.3 \pm 0.1$          & $0.2 \pm 0.1$              \\
        Tops         & $43.1 \pm 9.0$          & $16.3 \pm 3.8$          & $13.0 \pm 3.4$              \\
        Di-boson         & $5.6 \pm 2.6$          & $2.9 \pm 1.8$          & $2.3 \pm 1.1$              \\
        $t\bar{t}+V$         & $1.3 \pm 0.3$          & $0.9 \pm 0.2$          & $0.5 \pm 0.2$              \\
        \bottomrule
        %%      
        \end{tabular*}

  \end{center}
\end{table}
%%%%%%%%%%%%%%%%%%%%%%%%%%%%%%%%%%%%%%%%%%%%%%%%%%%%%%%%%%%%



\begin{table}
  \begin{center}
    \caption{ \label{tab::BGestimation::VRyields_Lowx} \crtablecaption{``Low-x''}  }

    \begin{tabular*}{\textwidth}{@{\extracolsep{\fill}}lrr}
      \toprule
      \textbf{VR Low-x} & VRa & VRb \\
      \midrule

Observed data & $20$ & $23$ \\
\midrule
Expected background & $14.91 \pm 2.09$ & $15.77 \pm 3.31$ \\
\midrule
Di-leptonic & $6.5 \pm 1.2$ & $0.6 \pm 0.3$ \\
$W$+jets & $1.5 \pm 0.8$ & $6.9 \pm 3.3$ \\
$Z$+jets & $0.5 \pm 0.2$ & $0.5 \pm 0.2$ \\
Tops & $5.0 \pm 1.7$ & $6.1 \pm 2.1$ \\
Di-boson & $1.0 \pm 0.3$ & $1.4 \pm 0.4$ \\
$t\bar{t}+V$ & $0.4 \pm 0.1$ & $0.4 \pm 0.1$ \\
        \bottomrule
        %%      
        \end{tabular*}

  \end{center}
\end{table}
%%%%%%%%%%%%%%%%%%%%%%%%%%%%%%%%%%%%%%%%%%%%%%%%%%%%%%%%%%%%



\begin{table}
  \begin{center}
    \caption{ \label{tab::BGestimation::VRyields_Highx} \crtablecaption{``High-x''}  }

    \begin{tabular*}{\textwidth}{@{\extracolsep{\fill}}lrr}
      \toprule
      \textbf{VR High-x}  & VRa & VRb  \\
      \midrule

Observed data & $66$ & $119$ \\
\midrule
Expected background & $49.33 \pm 8.80$ & $102.12 \pm 13.40$ \\
\midrule
Di-leptonic & $18.9 \pm 5.4$ & $0.0 \pm 0.0$ \\
$W$+jets & $8.8 \pm 1.8$ & $70.6 \pm 13.3$ \\
$Z$+jets & $0.4 \pm 0.1$ & $0.7 \pm 0.3$ \\
Tops & $16.0 \pm 6.8$ & $21.5 \pm 7.5$ \\
Di-boson & $4.2 \pm 1.6$ & $8.5 \pm 3.2$ \\
$t\bar{t}+V$ & $1.1 \pm 0.4$ & $0.8 \pm 0.3$ \\
        \bottomrule
        %%      
        \end{tabular*}

  \end{center}
\end{table}
%%%%%%%%%%%%%%%%%%%%%%%%%%%%%%%%%%%%%%%%%%%%%%%%%%%%%%%%%%%%



\begin{table}
  \begin{center}
    \caption{ \label{tab::BGestimation::VRyields_3B} \vrtablecaption{``3B''}  }

    \begin{tabular*}{\textwidth}{@{\extracolsep{\fill}}lrrr}
      \toprule
      \textbf{VRa 3B} & $m_{\mathrm{eff.}}\in$[1000,1750] & $m_{\mathrm{eff.}}>1750$ \\
      \midrule
Observed data          & $11$              & $8$                    \\
\midrule
Expected background         & $12.46 \pm 5.81$          & $5.31 \pm 1.58$              \\
\midrule
        Di-leptonic         & $7.3 \pm 5.5$          & $2.7 \pm 1.3$              \\
        $W$+jets         & $0.0_{-0.0}^{+0.0}$          & $0.0 \pm 0.0$              \\
        $Z$+jets         & $0.0 \pm 0.0$          & $0.0 \pm 0.0$              \\
        Tops         & $4.8 \pm 1.7$          & $2.4 \pm 0.9$              \\
        Di-boson         & $0.1_{-0.1}^{+0.1}$          & $0.0 \pm 0.0$              \\
        $t\bar{t}+V$         & $0.2 \pm 0.1$          & $0.2 \pm 0.0$              \\
\toprule
\textbf{VRb 3B} & $m_{\mathrm{eff.}}\in$[1000,1750] & $m_{\mathrm{eff.}}>1750$ \\
\midrule
Observed data          & $69$              & $12$                    \\
\midrule
Expected background         & $60.09 \pm 15.83$          & $9.55 \pm 2.77$              \\
\midrule
        Di-leptonic         & $3.3 \pm 1.4$          & $0.8 \pm 0.6$              \\
        $W$+jets         & $0.8 \pm 0.5$          & $0.4 \pm 0.2$              \\
        $Z$+jets         & $0.1 \pm 0.0$          & $0.0 \pm 0.0$              \\
        Tops         & $54.1 \pm 15.7$          & $7.8 \pm 2.7$              \\
        Di-boson         & $0.1 \pm 0.1$          & $0.2 \pm 0.1$              \\
        $t\bar{t}+V$         & $1.7 \pm 0.4$          & $0.4 \pm 0.1$              \\
        \bottomrule
        %%      
        \end{tabular*}

  \end{center}
\end{table}
%%%%%%%%%%%%%%%%%%%%%%%%%%%%%%%%%%%%%%%%%%%%%%%%%%%%%%%%%%%%





% -------------- VRpulls
\figNoH[170]{BGestimation/PullVRsSRs/histpull_VRs.pdf}
{(Top) Number of observed data (black dots) and the estimated background yields (histogram) in the nominal validation regions (VRa/VRb). 
The white component is the backgrounds estimated by the object replacement method, while the colored ones are by the kinematical extrapolation method. 
The dashed band represents the combined statistical and systematic uncertainties on the total estimated backgrounds. (Bottom) Pull between the data and the estimation. Pulls in regions dominated by $\wjets$ and tops are painted by pink and blue respectively.}
{fig::BGestimation::VRPulls}
%-------------------------------                                                                     

\clearpage
%% -------------- VR postFit 2J ---------
\begin{figure}[h]
  \centering
    \subfigure[]{\includegraphics[width=0.41\textwidth]{figures/BGestimation/SRVRpostFit/mt__VRa2JMEFF1_no_mt_postFit_2SFconfig_noYields_objRep.pdf}}
    \subfigure[]{\includegraphics[width=0.41\textwidth]{figures/BGestimation/SRVRpostFit/met__VRb2JMEFF1_no_met_postFit_2SFconfig_noYields_objRep.pdf}}
    \subfigure[]{\includegraphics[width=0.41\textwidth]{figures/BGestimation/SRVRpostFit/mt__VRa2JMEFF2_no_mt_postFit_2SFconfig_noYields_objRep.pdf}}
    \subfigure[]{\includegraphics[width=0.41\textwidth]{figures/BGestimation/SRVRpostFit/met__VRb2JMEFF2_no_met_postFit_2SFconfig_noYields_objRep.pdf}}
    \subfigure[]{\includegraphics[width=0.41\textwidth]{figures/BGestimation/SRVRpostFit/mt__VRa2JMEFF3_no_mt_postFit_2SFconfig_noYields_objRep.pdf}}
    \subfigure[]{\includegraphics[width=0.41\textwidth]{figures/BGestimation/SRVRpostFit/met__VRb2JMEFF3_no_met_postFit_2SFconfig_noYields_objRep.pdf}}
   \caption{
     Post-fit distruibution of (left) $\mt$ in VRa, and (right) $\met$ in VRb.
     (a,b) VR 2J-$\meffIncFirst$.
     (c,d) VR 2J-$\meffIncSecond$.
     (e,f) VR 2J-$\meffIncThird$. 
     The yellow band in the bottom panel represents statistical error. The overflow is included in the highest bin.  
     \label{fig::BGestimation::SRVRpostFit::VR2J}
   }
\end{figure}

\clearpage
% -------------- 6J ---------
\begin{figure}[h]
  \centering
    \subfigure[]{\includegraphics[width=0.41\textwidth]{figures/BGestimation/SRVRpostFit/mt__VRa6JMEFF1_no_mt_postFit_2SFconfig_noYields_objRep.pdf}}
    \subfigure[]{\includegraphics[width=0.41\textwidth]{figures/BGestimation/SRVRpostFit/LepAplanarity__VRb6JMEFF1_no_LepAplanarity_postFit_2SFconfig_noYields_objRep.pdf}}
    \subfigure[]{\includegraphics[width=0.41\textwidth]{figures/BGestimation/SRVRpostFit/mt__VRa6JMEFF2_no_mt_postFit_2SFconfig_noYields_objRep.pdf}}
    \subfigure[]{\includegraphics[width=0.41\textwidth]{figures/BGestimation/SRVRpostFit/LepAplanarity__VRb6JMEFF2_no_LepAplanarity_postFit_2SFconfig_noYields_objRep.pdf}}
    \subfigure[]{\includegraphics[width=0.41\textwidth]{figures/BGestimation/SRVRpostFit/mt__VRa6JMEFF3_no_mt_postFit_2SFconfig_noYields_objRep.pdf}}
    \subfigure[]{\includegraphics[width=0.41\textwidth]{figures/BGestimation/SRVRpostFit/LepAplanarity__VRb6JMEFF3_no_LepAplanarity_postFit_2SFconfig_noYields_objRep.pdf}}
    \caption{
      Post-fit distruibution of (left) $\mt$ in VRa, and (right) $\apl$ in VRb.
      (a,b) VR 6J-$\meffIncFirst$.
      (c,d) VR 6J-$\meffIncSecond$.
      (e,f) VR 6J-$\meffIncThird$.
      The yellow band in the bottom panel represents only statistical error. The overflow is included in the highest bin.  
      \label{fig::BGestimation::SRVRpostFit::VR6J}
    }
\end{figure}
%----------------------------------


\clearpage
% -------------- varx ---------
\begin{figure}[h]
  \centering
    \subfigure[]{\includegraphics[width=0.41\textwidth]{figures/BGestimation/SRVRpostFit/mt__VRaLowx_no_mt_postFit_2SFconfig_noYields_objRep.pdf}}
    \subfigure[]{\includegraphics[width=0.41\textwidth]{figures/BGestimation/SRVRpostFit/LepAplanarity__VRbLowx_no_LepAplanarity_postFit_2SFconfig_noYields_objRep.pdf}}
    \subfigure[]{\includegraphics[width=0.41\textwidth]{figures/BGestimation/SRVRpostFit/mt__VRaHighx_no_mt_postFit_2SFconfig_noYields_objRep.pdf}}
    \subfigure[]{\includegraphics[width=0.41\textwidth]{figures/BGestimation/SRVRpostFit/LepAplanarity__VRbHighx_no_LepAplanarity_postFit_2SFconfig_noYields_objRep.pdf}}
   \caption{   
     Post-fit distruibution of (left) $\mt$ in VRa, and (right) $\apl$ in VRb.
     (a,b) VR Low-x.
     (c,d) VR High-x.
     The yellow band in the bottom panel represents only statistical error. The overflow is included in the highest bin.  
     \label{fig::BGestimation::SRVRpostFit::VRVarx}
   }
\end{figure}
% ----------------------------------------------

\clearpage
% -------------- 3B ---------
\begin{figure}[h]
  \centering
    \subfigure[]{\includegraphics[width=0.41\textwidth]{figures/BGestimation/SRVRpostFit/mt__VRa3BMEFF1_no_mt_postFit_2SFconfig_noYields_objRep.pdf}}
    \subfigure[]{\includegraphics[width=0.41\textwidth]{figures/BGestimation/SRVRpostFit/topNess__VRb3BMEFF1_no_topNess_postFit_2SFconfig_noYields_objRep.pdf}}
    \subfigure[]{\includegraphics[width=0.41\textwidth]{figures/BGestimation/SRVRpostFit/mt__VRa3BMEFF2_no_mt_postFit_2SFconfig_noYields_objRep.pdf}}
    \subfigure[]{\includegraphics[width=0.41\textwidth]{figures/BGestimation/SRVRpostFit/topNess__VRb3BMEFF2_no_topNess_postFit_2SFconfig_noYields_objRep.pdf}}
    \caption{   
      Post-fit distruibution of (left) $\mt$ in VRa, and (right) topness in VRb.
      (a,b) VR 3B-$\meffIncFirst$.
      (c,d) VR 3B-$\meffIncSecond$.
      The yellow band in the bottom panel represents only statistical error. The overflow is included in the highest bin.  
      \label{fig::BGestimation::SRVRpostFit::VR3B}
    }
\end{figure}




%%%%%%% Summary of post fit uncertainties %%%%%%%%%%%%%%%%                                                                                                                                                                                                                
%\begin{figure}
%  \begin{center}
%    \includegraphics[width=170mm]{figures/Result/systSummary/syst_summary_VRs.pdf}
%    \captionof{figure}{
%    Post-fit systematic uncertainty with respective to the expected yield in VRa/VRb.
%    Total systematics uncertainty is shown by the filled orange histogram, and the breakdowns are by dashed lines.
%    While the systematics in $b$-tagged bins are purely dominated by control region statistics, it is comparable to the other sources in the $b$-veto bins. 
%    }
%    \label{fig::Result::systSummary}
%  \end{center}
%\end{figure}
%-------------------------------    

