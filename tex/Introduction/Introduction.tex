%\chapter{Introduction} \label{sec::Introduction}
\chapter{Theoretical Backgrounds and Search Strategy} \label{sec::Introduction}
This chapter provides the backgrounds necessary to motivate the study in the thesis.
It starts with an introduction of the theoretical framework and the status-quo of the SUSY searches.
In the latter part of the chapter, a class of most motivated SUSY scenarios are analyzed and proposed.
Lastly the search strategy towards it is discussed.
%

\section{The Standard Model of Elementary Particles}
Before jumping into SUSY, the Standard Model (SM) of particle physics, current our best validated knowledge about the universe, is quickly reviewed (widely referred from the discussion in Peskin $\&$ Schr{\"{o}}der \cite{PeskinSchroeder}, and Halzen $\&$ Martin \cite{HalzenMartin}). \\

The particle content of the SM is shown in Table \ref{tab::Introduction::SMFermions} and Table \ref{tab::Introduction::SMBosons}. 
There are three types of particles: fermions with the spin of 1/2 that consist matters: 
gauge bosons with the spin of 1 mediating the interaction acting between particles: 
and the spin-0 Higgs boson feeding their masses through the Brout-Englert-Higgs (or BEH) mechanism \cite{SSBBroutEnglert,SSBHiggs}.  \\

\tab{c | c c c | c c c c c}
{
\hline
        &  \multicolumn{3}{c|}{Generation} &   $Q$ &  $T$  & $T^3$ &  $Y$ & $N_C$\\
\cline{2-4} 
        &  1st   &  2nd    &  3rd         &     &     &    &   \\
\hline
\hline
Quarks  &  $\colv{u \\ d}_{\mL}$   &  $\colv{c \\ s}_{\mL}$    &  $\colv{t \\ b}_{\mL}$ &   $\colv{2/3 \\ -1/3}$ &  1/2  & $\colv{1/2 \\ -1/2}$ &  $1/3$ & 3\\
        &  $u_{\mR}$   &  $c_{\mR}$    &  $t_{\mR}$ &   $2/3$  &  0  & 0 &  $4/3$  & 3\\
        &  $d_{\mR}$   &  $s_{\mR}$    &  $b_{\mR}$ &   $-1/3$ &  0  & 0 &  $-2/3$ & 3\\
\hline
Leptons  &  $\colv{\nu_e \\ e^-}_{\mL}$   &  $\colv{\nu_\mu \\ \mu^-}_{\mL}$    &  $\colv{\nu_\tau \\ \tau^-}_{\mL}$ &   $\colv{0 \\ -1}$ &  1/2  & $\colv{1/2 \\ -1/2}$ &  $-1$ & 0 \\
        &  $e_{\mR}$   &  $\mu_{\mR}$    &  $\tau_{\mR}$ &   $-1$  &  0  & 0 &  $-2$ & 0 \\
\hline
}
{Fermion contents in the SM. The quantum numbers $Q$, $T$, $T^3$ and $Y$ are respectively electric charge, weak iso-spin number, the third component of weak iso-spin and weak hyper charge. $N_C$ represents the number of color states. The subscripts L, R indicate the chirality (left- or right-handed respectively), and the pharentheses denote the $SU(2)_L$ doublet.
}
{tab::Introduction::SMFermions}


\tab{c c | c c c c c}
{
\hline
             &          &  $Q$ &  $T$  & $T^3$ &  $Y$ & $N_C$ \\
\hline
\hline
gluon        & $g$      &   0  &  0    &  0    &  0   &  8    \\
\hline
weak bosons  & $W^\pm$  &$\pm1$&  1    &$\pm1$ &  0   &  0    \\
             & $Z$      &   0  &  0    &  0    &  0   &  0    \\
\hline
photon       & $\gamma$ &   0  &  0    &  0    &  0   &  0    \\
\hline
\hline
higgs       & $h$       &   0  &  1/2  &  -1/2 &  1   &  0    \\
\hline
}
{Gauge bosons and higgs in the SM. The notation for the quantum numbers are the same with Table \ref{tab::Introduction::SMFermions}.}
{tab::Introduction::SMBosons}

The three types of gauge bosons; gluon ($g$); weak bosons ($W^{\pm},Z$) and photon ($\gamma$) characterize strong interaction, weak interaction and electromagnetic interaction respectively. 
Fermions have two families; quarks which sense all the three gauge interactions; leptons which couple only via weak and electromagnetic interaction. Both families have up- and down-type. There are also two more duplications of them (``2nd / 3rd generation'') with exactly the same properties except the masses. Each fermions furthermore have the charge conjugated partner called anti-fermions. \\


\subsection{The Gauge Principle and Particle Interaction} \label{sec::Introduction::gaugePrinciple}
A successful theory for elementary particles must be quantum and relativistic.
The theory of SM is constructed in a relativistic framework of field theory, fully exploiting the virtue that time ($t$) and position ($\bm{x}$) are treated equivalently in that both are coordinates rather than observables. It is characterized by a Lorentz-invariant Lagrangian in which particles are described by a function in terms of $x^\mu$ (``fields'') following the Lorentz transformation law of corresponding spin expression. 
The free Lagrangian for a fermion are given by:
\begin{align}
\mathcal{L} = i \bar{\psi} \gamma^\mu \partial_\mu \psi - m \bar{\psi}\psi + \mathrm{h.c.}
\label{eq::SMfreeLag}
\end{align}
where $\psi$ is a spinor field with the mass of $m$, and $\gamma^\mu$ is the 4-dimensional gamma matrices. 
The first term corresponds to the kinetic terms and the second is to the mass term of the fermion.
%\begin{align}
%\colv{\bm{1} & \\ \bm{1} & }
%\mathrm{(In Weyl's representation)}
%\end{align}
%Note that $\sigma_i$ are Pauli's matrices

Interaction between particles are ruled by a local symmetries referred as ``gauge symmetry''. 
The interaction terms are obtained by imposing on the free Lagrangian an invariant nature against the gauge transformation.
In case of electromagnetic interaction, for instance, the gauge transformation is given by:
\begin{align}
\psi \ra e^{i\theta(x) Q} \psi = e^{i\theta(x) q} \psi
\end{align}
where $Q$ is the generator of the $U(1)$ transformation, $q$ is charge that the fermion $f$ has, and $\theta(x)$ is an arbitrary time-space dependent phase.
The free Lagrangian in Eq. (\ref{eq::SMfreeLag}) is not invariant under this transformation, however can be fix by a small hack in the differential in the free Lagrangian ($\partial_\mu$) such as:
\begin{align}
\partial_\mu \ra D_\mu := \partial_\mu - ieA_\mu (x)
\end{align}
where $e$ is the elementary charge and $A(x)$ is a vector field transformed by the gauge transformation:
\begin{align}
A_\mu \ra A_\mu + \frac{1}{e} \partial_\mu \theta(x).
\end{align}
%The replaced $D_\mu$ is referred to ``covariant differential''.

The interaction term then emerges as the extra terms in the Lagrangian:
\begin{align}
\mathcal{L}_{\mathrm{int.}} = e\bar{\psi} \gamma^\mu \psi A_\mu.
\label{eq::SMfreeLag}
\end{align}
From the consistency with classical Maxwell equation, this describes the electromagnetic force acting on the fermion, and $A_\mu$ corresponds to the electromagnetic potential in the classical electromagnetism or the particle field for photon. \\



\subsection{Perturbation and Renormalization}
The effect of interaction is often characterized via transition amplitude from an initial state ($i$) to a final state ($f$):
\begin{align}
\bra{f} e^{-i\mathcal{H}_{\mathrm{int.}}t} \ket{i},
\end{align}
where $\mathcal{H}_{\mathrm{int.}}$ is the interaction Hamiltonian obtained by a Legendre transformation of interaction Lagrangian. The amplitude is often a basic building block of phenomenological predictions such as interaction cross-section or decay branch, however it is in most of the cases not analytically calculable. It is therefore done through a perturbation expansion in terms of the coupling constant of the interaction, for which $\alpha := e^2/4\pi$ is conventionally used for electromagnetic interaction.  \\

The small coupling constant of electromagnetic interaction ($\alpha \sim 1/137$) may sound to guarantee a good convergence behavior of the expansion in which the impact from the truncated orders in the series is small enough. It is however found that the higher-order contribution immediately leads to divergence quite everywhere in cross-section calculation (infrared / ultraviolet divergences), causing the theory unpredictable. This problem was solved by a procedure called ``renormalization'' 
%\cite{Tomonaga,Feynman}, 
where theory parameters (i.e. the masses and coupling constants) are redefined to absorb the infinities, maintaining a finite cross-section calculation. 
%\begin{align}
%\mathcal{O} (\mu) & = \mathcal{O}_{\mathrm{bare}} + \Delta \mathcal{O} (\mu), \nn \\
%\end{align}
%%%%
Historically, this formulation firstly succeeded in QED, and then understood by that the gauge symmetry played an important role in calcelling the divergence \cite{Ward1950,Takahashi1957}. From this moment, gauge symmetry started establishing the status as a guidance principle in constructing theories, beyond merely a prescription. It is also shown with considerable generality that well-behaving theory (``renormalizable theory'') must respect gauge symmetry \cite{tHooft1972}. \\

The consequence of renormalization also provided a critical insight that the magnitude of theory parameters effectively vary depending on the energy scale with which the interaction happen. The evolution is characterized by the renormalization group equation (RGE), for example, as for the coupling constant ($\alpha$): 
\begin{align}
\dfrac{1}{\alpha(Q)^2}-\dfrac{1}{\alpha(Q_0)^2} = -\dfrac{\beta(\alpha)}{2\pi} \log{\left(\dfrac{Q}{Q_0}\right)},
\end{align}
where $Q$ is the scale defined by the typical momentum transfer of the interaction process, and $\beta(\alpha)$ is the beta function, proportional to $\alpha^2$ at 1-loop level. This evolution is known as the ``running'' effect, which is an useful proxy for exploring the behavior of theory over the scale. \\


\subsection{QED, QCD, and the Electroweak Theory}
The Lagrangian for Quantum Electromagnetic Theory (QED) is given by adding the kinetic terms of photon ($-\frac{1}{4}F_{\mu\nu} F^{\mu\nu}$) to one obtained in Sec. \ref{sec::Introduction::gaugePrinciple}:
\begin{align}
\mathcal{L} = -\frac{1}{4}F_{\mu\nu} F^{\mu\nu} + i \bar{\psi} \gamma^\mu \partial_\mu \psi - m \bar{\psi}\psi \, + \mathrm{h.c.}
\end{align}
with $F_{\mu\nu}$ being the field strength:
\begin{align}
F_{\mu\nu} := \partial_\mu A_\nu - \partial_\nu A_\mu.
\end{align}


Similar to what is done in QED with the gauge group of $U(1)$, the Lagrangian for strong and weak interaction can be generated by considering gauge groups of $SU(2)_L$ and $SU(3)$:
\begin{align}
\psi \ra e^{i \theta_a(x) \lambda^a}, \psi \,\,\,\,\,\,\,\, a=1,2,\dots (N^2-1) \,\,\,\,\,\, \mbox{\phantom{MM}  (for $SU(N)$)}  \nn 
\end{align}
with $\lambda^a$ being the generators of the gauge group. 
The choice of the gauge groups are motivated by:
\begin{itemize}
\item ($SU(2)$ for weak interaction) the observation of approximate iso-spin symmetry in theories of nucleus decay, 
\item ($SU(3)$ for strong interaction) the factor of 3 enhancement in cross-section of the Drell-Yan process for quark-antiquark production with respect to muon pair production: $\sigma(ee\ra q\bar{q})/\sigma(ee\ra\mu\mu) = 3N_q$ where $N_q$ is number quark species (see Figure \ref{fig::Introduction::eeqqRfactor}). 
\end{itemize}


\fig[160]{Introduction/eeqq_Rfactor.pdf}
{Measurement of the $R$-factor ($:= [d\sigma(ee\ra q\bar{q})/dQ]/[d\sigma(ee\ra\mu\mu)/dQ]$ ) versus the center-of-energy of the $ee$-collision ($Q$). 
the factor of 3 enhancement by the color factor is needed in addition to number of opened channels of quark-antiquark production to account for the observation.} 
{fig::Introduction::eeqqRfactor}



\paragraph{Strong Interaction}  \mbox{} \\
The Lagrangian for strong interaction is:
\begin{align}
\mathcal{L}_{\mathrm{QCD}} & = -\frac{1}{4} \bm{\hat{G}}_{\mu\nu} \bm{\hat{G}}^{\mu\nu} + \,\, \bar{q}(i \gamma^\mu D_\mu - m) q + \mathrm{h.c.}, \nn \\
D_\mu & := \partial_\mu + ig_s \sum_{a=1}^8 G^a_{\mu}\frac{\lambda_a}{2} \nn \\
\bm{\hat{G}}_{\mu\nu} & := \partial_\mu \bm{G}_\nu - \partial_\nu \bm{G}_\mu - g_s \, \bm{G}_\mu \times \bm{G}_\nu, \nn \\
\bm{G}_\mu & := \{G^a_\mu; a=1,2,\dots,8\}
\label{eq::QCDLag}
\end{align}
where $G^a_\mu$ and $q$ represent the fields for gluons and quarks respectively.
$g_s$ is related to the strong coupling constant $\alpha_s$ by $\alpha_s = g_s^2/4\pi$. 
The charge of strong interaction is called ``color'', and the theoretical framework is referred to Quantum Chromo Dynamics (QCD). 
Quarks are in the triplet and gluons are in the octet expression with 3 and 8 degenerated states respectively.
In addition, due to the non-Abelian nature of $SU(3)$, gluon has self-interaction with coupling to itself. One distinct consequence of this is the negative running coupling:
\begin{align}
\alpha_s(Q) = \frac{4\pi\alpha_s(\mu_R)}{4\pi + \beta_0 \alpha_s(\mu_R)\log{(Q^2/\Lambda^2_{\mathrm{QCD}})} }
\end{align}
where $\beta = 11 -2n_f/3$ ($n_f$ is number of quarks with the mass above $Q$), $\mu_R$ the renormalization scale (a reference scale of renormalization, different from the physical energy scale $Q$), and $\Lambda_{\mathrm{QCD}}$ the QCD cut-off scale at $\sim 200\mev$. The indication of $\beta<0$ is decreasing coupling constant with increased energy scale $Q$.
Despite of the generally larger coupling than that of electromagnetic interaction, in the energy scale interested in LHC ($Q>100\gev$), $\alpha_s$ typically about 0.1, which is small enough to recover the perturbative picture (``asymptotic freedom'' 
%\cite{asymptoticFreedom}
). 
On the other hand, the coupling becomes increasingly strong as approaching to $\Lambda_{\mathrm{QCD}}$, leading to an immediate catastrophe of the perturbation picture. 
As a result of this strong coupling, colored particles are forced to combine each other to form a color singlet state (``confinement''),


\paragraph{Electro-weak interaction} \mbox{} \\
Weak interaction is described by a larger gauge group $SU(2)_L \times U(1)_Y$, in a manner where weak and electromagnetic interaction reside altogether \cite{WeakGlashow,WeakWeinberg,WeakSalam}. 
The basic idea is that they share the common origin at high energy scale and branch into separate interactions at some point through a spontaneous symmetry breaking $SU(2)_L \times U(1)_Y \ra U(1)_Q$.
The regime of unified interaction is commonly referred as electroweak (EW) interaction. \\

The gauge transformation distinguishes chirality of fermions, in that $SU(2)_L$ selectively acts to the left-handed component, 
accounting for the observed parity violating nature of weak interaction \cite{PVLeeYang,PVWu}:
\begin{align}
& \psi_{\mL} \ra e^{i\theta T_3+i\Theta Y} \psi_{\mL}   \\
& \psi_{\mR} \ra e^{i\Theta Y} \psi_{\mR}.
\end{align}
The Lagrangian arrives at:
\begin{align}
\mathcal{L}_{\mathrm{EW}} & = -\frac{1}{4} \bm{\hat{W}}_{\mu\nu} \bm{\hat{W}}^{\mu\nu}-\frac{1}{4}B_{\mu\nu} B^{\mu\nu} + \,\, \bar{\psi}(i \gamma^\mu D_\mu - m) \psi + \mathrm{h.c.}, \nn \\
%
D_\mu & := \partial_\mu + ig \sum_{a=1}^3 W^a_{\mu} \tau_a  + ig^{'} \frac{Y}{2} B_\mu \nn \\
\bm{\hat{W}}_{\mu\nu} & = \partial_\mu \bm{W}_\nu - \partial_\nu \bm{W}_\mu - g \bm{W}_\mu \times \bm{W}_\nu \nn \\
\bm{W}_\mu & := \{W^a_\mu; a=1,2,3\} \nn \\
B_{\mu\nu} & = \partial_\mu B_\nu - \partial_\nu B_\mu
\label{eq::EWLag1}
\end{align}
where $W^a_\mu$ and $B\_mu$ are the fields of EW gauge bosons, and $g, g^{'}$ are the coupling respectively for $SU(2)_L$ and $U(1)_Y$. $\bm{\tau} (= \bm{\sigma}/2)$ are generators of $SU(2)$. \\

The Lagrangian can be also re-written by introducing weak currents $J_\mu$:
\begin{align}
\mathcal{L}_{\mathrm{EW}} & = -\frac{1}{4} \, \sum_{a=1}^3 W^a_{\mu\nu} W^{a\mu\nu}-\frac{1}{4}B_{\mu\nu} B^{\mu\nu} \nn \\
& -\frac{g}{2} (J_\mu^+ W^{-\mu} + J_\mu^- W^{+\mu}) -g J_\mu^3 W^{3\mu} - \frac{g^{'}}{2}J_\mu^YB^\mu  + \mathrm{h.c.}  \nn \\
 J^{\pm}_\mu & := \frac{1}{\sqrt{2}} (W_\mu^1\mp iW_\mu^2) \nn \\
 J^a_\mu & := \bar{\psi}_L \gamma^\mu \tau_q W_\mu^a \psi_L  \,\,\, (a=1,2,3) \nn \\
 J^Y_\mu & := Y \bar{\psi}_L \gamma^\mu  \psi_L.
\label{eq::EWLag2}
\end{align}
$J_\mu^\pm$ represent currents changing $T_3$, while $J_\mu^0$ and $J_\mu^Y$ neutral current conserving either $T_3$ and $Y$. \\

The EW symmetry breaking is expressed by mixing the fields $(W_\mu^3, B_\mu)$ into $(Z_\mu, A_\mu)$:
\begin{align}
\colv{Z_\mu \\ A_\mu} := 
     \left(
   \begin{array}{cc}
     \cos{\theta_W} &  -\sin{\theta_W} \\ 
     \sin{\theta_W} &  \cos{\theta_W} 
   \end{array}
     \right)
 \colv{W_\mu^3 \\ B_\mu}
\end{align}
with a mixing angle (Weinberg angle $\theta_W$) of:
\begin{align}
\tan{\theta_W} := \frac{g^{'}}{g}.
\end{align}
%
The current terms in the Lagrangian Eq. (\ref{eq::EWLag2}) then becomes:
\begin{align}
 & -\frac{g}{2} (J_\mu^+ W^{-\mu} + J_\mu^- W^{+\mu})  \nn \\
 & + \frac{g}{\cos{\theta_W}} \left( -\cos^2{\theta_W} J_\mu^3 + \frac{\sin^2{\theta_W}}{2} J_\mu^Y  \right) \, Z^{\mu}  \nn \\
 & - g\sin{\theta_W} \left( J_\mu^3 + \frac{1}{2} J_\mu^Y \right) A^\mu 
\end{align}
By choosing $Y := 2(Q-T^3)$, $A_\mu$ becomes associated with the gauge field of electromagnetic interaction, 
and the electric charge is found to be related to the weak coupling constant by the Weinberg angle: $e=g\sin{\theta_W}$. \\


\subsection{Electroweak Symmetry Breaking and the Higgs boson}
One outstanding problem in the EW Lagrangian is the prohibition of mass terms, for both gauge bosons and fermions, since they explicitly violates the gauge invariance. 
%since left-handed and right-handed fields obey different $SU(2)_L$ gauge transformation:
%\begin{align}
%  m\bar{\psi}\psi = m \bar{\psi}_{\mL} \psi_{\mR}.
%\end{align}
The BEH mechanism \cite{SSBBroutEnglert,SSBHiggs} is then employed to solve the problem, by assuming a $SU(2)$ doublet $\phi$ ($Y=-1, T=1/2$) with scalar fields $\phi=(\phi_1,\phi_2)=(\phi^+,\phi^0)$ and a potential $V(\phi)$ added in the Lagrangian:
\begin{align}
\mathcal{L}_{\mathrm{Higgs}} & := \left(D_\mu \phi \right)^\dg \left(D^\mu \phi \right) - V(\phi) \nn \\
                     V(\phi) & := \mu^2 \phi^\dg \phi + \lambda (\phi^\dg \phi)^2.
\label{eq::HiggsPotential}
\end{align}
While the minimum value of the potential is always found in $\phi=(0,0)$ in the $\phi_1-\phi_2$ plane when $\mu^2>0$, 
negative $\mu^2$ leads to non-trivial minima in $v := |\phi|^2 = -\mu^2/2\lambda$. 
This causes a shift of the vacuum expectation value: $\bra{0} \phi \ket{0} = 0 \ra v$ (spontaneous symmetry breaking). \\

Redefining the field $\phi$ by the variation around the new vaccum (0,$v$) $h(x)$:
\begin{align}
\phi = \colv{0 \\ v+h(x)}
\label{eq::SSB}
\end{align}
and applying the $\partial_\mu \ra D_\mu$ prescription to Eq. (\ref{eq::HiggsPotential}),
one finds the mass terms for $W,Z$ to be:
\begin{align}
m_W & = gv/2 \nn \\
m_Z & = \sqrt{g^2+g^{' 2}} v/2. \nn
\end{align}

The mass of the scalar field $h$ is also found to be:
\begin{align}
m_h & = \sqrt{-2\mu^2}.  \nn
\end{align}
This $h$ also behavoes as a physical mode, referred as higgs particle. \\

The fermion masses are fed by adding following Gauge invariant terms to the Lagrangian:
\begin{align}
\mathcal{L}_\mathrm{Yukawa} := 
& -\bar{\psi}_{i,L} y_\mup^{ij} \phi   \psi_{j,R} -\bar{\psi}_{i,R} y_\mup^{ij} \phi^\dg     \psi_{j,L} \nn \\
& -\bar{\psi}_{i,L} y_\mdn^{ij} \phi^c \psi_{j,R} -\bar{\psi}_{i,R} y_\mdn^{ij} \phi^{c \dg} \psi_{j,L} \nn \\
& -\bar{\psi}_{i,L} y_e^{ij} \phi   \psi_{j,R} -\bar{\psi}_{i,R} y_e^{ij} \phi^\dg     \psi_{j,L} 
\label{SMLagYukawa}
\end{align}
where $i, j = 1, 2, 3$ index the generation of the fermions.
$y_\mup^{ij}$, $y_\mdn^{ij}$, and $y_e^{ij}$ are the components of Yukawa matrices respectively for up-, down-type quarks and down-type leptons. 
The Yukawa matrices are $3\times 3$ matrices spanning over the family space, in which Yukawa couplings for each fermion are accommodated.
The off-diagonal components are also responsible for the mixing between generations, 
which are set all zero for down-type leptons, 
while they are non-zero in case of quarks characterized by the CKM matrix \cite{CPVKM}. \\ 

Inserting Eq. (\ref{eq::SSB}), $\mathcal{L}_{\mathrm{Yukawa}}$ is finally reduced to:
\begin{align}
\mathcal{L}_{\mathrm{Yukawa}} & = \sum_f y_f v \bar{\psi} \psi + y_f \bar{\psi}{\psi} h \nn \\
& = \sum_f m_f \bar{\psi} \psi + y_f \bar{\psi}{\psi} h,
\end{align}
where $f$ is the index of fermions, with $y_f$ ($\phi_f$) being the mass eigenvalues (eigenstates) of the Yukawa matrices. \\

Higgs boson was discovered in LHC in 2012 \cite{HiggsDiscATLAS,HiggsDiscCMS}, bringing the last piece of the Standard Model in human knowledge. Measurements on its properties including the mass, spin \cite{HiggsSpinATLAS,HiggsSpinCMS} and couplings \cite{ATLASHiggsCouplingRun1,CMSHiggsCouplingRun1} are underway, which is all consistent with the SM so far. Figure \ref{fig::Introduction::higgsCoupling} shows the coupling measurement by ATLAS and CMS in LHC Run1. Further Precision measurement is planned in the later stages in LHC as well as the future linear collider projects such as ILC (International Linear Collider).
%\begin{align}
%\end{align}

%%%%%%%%%%
\begin{figure}[h]
  \centering
    \subfig{0.45}{figures/Introduction/ATLAS_HiggsCoupling_Run1.pdf}{}
    \subfig{0.45}{figures/Introduction/CMS_HiggsCoupling_Run1.pdf}{}
    \caption{Higgs coupling measurement in LHC Run1 carried out by (a) ATLAS \cite{ATLASHiggsCouplingRun1} and (b) CMS \cite{CMSHiggsCouplingRun1}. }
    \label{fig::Introduction::higgsCoupling}
\end{figure}

%\section{Standard model processes in LHC}

%\section{Summary of the SM Lagrangian}
%The complete list of the SM Lagrangian is given by:
%\begin{align}
%\mathcal{L}_{\mathrm{SM}}
% & = -\frac{1}{4} F_{\mu\nu} F^{\mu\nu} \\
% & = i \bar{\psi}\gamma_{\mu}D^\mu \psi + \mathrm{h.c.} \\
% & = \psi_i y_{ij} \psi_j + \mathrm{h.c.} \\
% & = |D_\mu \phi|^2 - V(\phi).
%\end{align}


%
%
\clearpage
\section{Remained Problems for the Standard Model and the SUSY Solution} \label{sec::Introduction::homeworkSM}
Despite the enormous sucess of the SM, there are still couple of problems left to be solved,
from phenomenological ones such as the inaccountability to dark matter, 
to conceptial ones towards the ultimate theory (e.g. the too many parameters, the naturalness as a theory etc.).
This section will overview several most notable ones, enough motivating the beyond-the-SM (BSM) theoriees including SUSY.
A particular emphasis is put on that TeV-scale SUSY is preferred in order to being the solution of the
 problems.
Also, an example is shown to  illustrate why SUSY is particularly important among the candidates of the solutions. \\

\subsection{The Fine-tuning Problem in Higgs Mass} \label{sec::Introduction::homeworkSM::fineTuning}
Though divergences appearing in higher-order calculations in SM are universaly cured in renormalization by the cenceling with the counter terms, 
it has been pointed that the magnitude of the cancelling terms are unnaturally large in case of the radiation correction on the higgs mass 
\cite{HMassHeiararchyWeinberg1,HMassHeiararchyWeinberg2,HMassHeiararchyGildener,HMassHeiararchySusskind}. 
%Since the SM higgs has no partner in the same multiplet represented by a certain symmetry, there is no counter terms possible to cancel the leading quadratic divergence happening in the self-energy correction, forcing the higgs mass to explicitly contain the dependence on the cut-off scale $\Lambda$ upto which the loop momentum is integrated.
For instance, the 1-loop correction given by a top-quark loop (Figure \ref{fig::Introduction::higgsMass_loop} (a)) before renormalization is:
\begin{align}
\Delta m_h ^2 = - \frac{3|\lambda|^2}{8\pi^2} \Lambda^2 + O\,(\log\Lambda),
\label{eq::naturalness1}
\end{align}
%
which is related by the renormalized mass ($m_{h,\mathrm{obs.}}$) and the bare mass ($m_{h,\mathrm{bare}}$) with:
\begin{align}
m^2_{h,\mathrm{obs.}} = m^2_{h,\mathrm{bare}} + \Delta m_h ^2.
\label{eq::naturalnessBare}
\end{align}
The magnitude of the correction term $\Delta m_h ^2$ can be order of $10^{38}(\mathrm{GeV})^2$ assuming SM is valid upto the Planck scale: $\Lambda \sim 10^{19} (\mathrm{GeV})^2$, while the observed mass is $125\gev$.
Naively thinking this implies that the bare mass $m_{h,\mathrm{bare}}$ and the correction $\Delta m_h$ has to cancel in a precision of $10^{-17}$ (``fine tuning problem'' or ``naturalness problem''). It is highly unnatural for a theory to contain such extraordinary scale hierarchy in it, therefore it is preferred to conceive the underlying mechanism behind it.  \\

%\footnote{Such as the fine tuned mass splitting of proton between neutron, which is naturally driven by the approximate symmetry in iso-spin.}
%\footnote{Fish does try to invent a Schrodinger eq. that naturally predicts a world with only water, because they know it!}
The simplest solution is to add a partner particle yielding the opposite loop contribution to cancel it (Figure \ref{fig::Introduction::higgsMass_loop} (b)).
In SUSY, this is done by introducing scalar-top (bosonic partner of top-quark ``stop'') with the mass of $m_S$ and the same couplings as tops. The quadratic terms cancel out as:
%SUSYはまさにこのタイプにあたり、higgsと同じcouplingでint.するboson版のpartnerを回すことによってloop correctionはこうなる。
%for the particles in the loop of higgs self-energy, with the negative yet identical magnitude 
\begin{align}
\tilde{\Delta} m_h ^2 & =  2 \times \frac{3|\lambda|^2}{16\pi^2} \Lambda^2 + O\,(\log\Lambda) \nn \\
\Delta m^2_{h, \mathrm{stop}} 
&  = \Delta m_h ^2 + \tilde{\Delta} m_h ^2  = O\,(\log\Lambda)
%&  = - \frac{\lambda}{8\pi^2} m_S^2 \log{\left(\frac{\Lambda}{m_S}\right)},
\label{eq::naturalness}
\end{align}
where the $10^{-34}$ order of fine-tuning is no longer needed. \\
%他にもhiggsにさらなるinternal structureがある考えてloopの積分を途中でtrancateするというやり方である。(composite higgs) technicolorなどのmodelがよくstudyされている

\begin{figure}[h]
  \centering
    \subfig{0.35}{figures/Introduction/higgsMass_topLoop.pdf}{Top loop.}
    \subfig{0.3}{figures/Introduction/higgsMass_stopLoop.pdf}{Scalar-top (stop) loop.}
    \caption{Feynman diagrams of 1-loop processes contributing to higgs mass by (a) top and (b) scalar-top. }
    \label{fig::Introduction::higgsMass_loop}
\end{figure}


\subsection{Grand Unification}
It is the ultimate desire for physicists to explain all phenomena in the universe by a single principle. 
While in the SM, the EW symmetry breaking $SU(2)_L \times U(1) \ra \slashed{SU(2)} \times U(1)_Q$ implies a common origin of electromagnetic and weak interaction, this encourages physicists to conceive another unification together with strong interaction at a higher scale (Grand Unification Theory; GUT). \\

Running coupling constants are useful proxies to analyze the possibility of such unification.
The evolution of coupling constants along scale is given by the RGE:
\begin{align}
\dfrac{1}{\alpha_i(Q)^2}-\dfrac{1}{\alpha_i(Q_0)^2} = -\dfrac{\beta_i}{2\pi} \log{\left(\dfrac{Q}{Q_0}\right)},
\end{align}
with the indices $i=1,2,3$ denote strong, weak and electro-magnetic interaction respectively.
$\beta_i$ are the beta functions. In the SM at 1-loop level, these are:
\begin{align}
\colv{b_1 \\ b_2 \\ b_3} = \colv{ 1/10 \\ -43/6 \\ -11} + n_{\mathrm{gen}} \colv{4/3 \\ 4/3 \\ 4/3},
\end{align}
where $n_{\mathrm{gen}}$ is the number of generation of fermions, which is equal to $3$ for $Q>m_t$.
One naively expects a convergence of the three couplings at a certain scale ($\mu_{\mGUT}$) in case of the grand unification.
Unfortunately, this does not happen in the SM, as illustrated in Figure \ref{fig::Introduction::GUT} (a).
However, it can be relatively easily realized in the SUSY regime, where more fermion particles can participate in the game changing the slope of the running. For instance, the beta function for MSSM is:
\begin{align}
\colv{b_1 \\ b_2 \\ b_3} = \colv{3/5 \\ 1 \\ -3},
\end{align}
and the coupling unification is achieved at $\mu_{\mGUT} \sim 10^{16}\gev$, as shown in Figure \ref{fig::Introduction::GUT} (b). 
This is superizing given that the convergence can be easily violated with even a little different particle content, and this is one of the reasons that SUSY is particularly special among the BSM frameworks.
\\


\fig[100]{Introduction/GUT.pdf}
{Two-loop renormalization group evolution of the inverse gauge coupling $1/\alpha_i$ in case of SM (dashed lines), and a scenario in MSSM (solid lines) where the masses of SUSY partners are set between $500\gev$ and $1.5\tev$ \cite{SUSYPrimer}.}
{fig::Introduction::GUT}



\subsection{Dark Matter}
%Historically, the argument of dark matter (DM) originated from astronomical observations implying excessive masses in nebulae or galaxy center beyond the expectation from spectroscopy \cite{DMNebulae,DMGallaxyRotation}. 
Historically, the argument of dark matter (DM) originated from observations on velocity of galaxy rotation, implying excessive masses in galaxy center beyond the expectation from spectroscopy \cite{DMGallaxyRotation1,DMGallaxyRotation2}. 
The non-baryonic dark matter hypothesis has been strongly supported by the a number of observatory facts that comes up later such as the mass tomography on The Bullet Cluster using the gravitational lensing effect and so on. 
Currently the most commonly considered framework of dark matter is the $\Lambda$-CDM model (Cold Dark Matter) in which DM is assumed to:
%\footnote{Less common though, frameworks involving warm DM \cite{} or Strongly Interacting Massive Particle (SIMP) \cite{SIMPMurayama} have been also proposed.}
\begin{itemize}
\item only sense very weak interaction such as gravity (Weakly Interacting Massive Particles; WIMPs) 
\footnote{This almost requires electrically neutral, but completely forbidden \cite{chargedDM}.}
\item be non-relativistic, given that DM is relatively spatially localized such as in galaxy center.
\end{itemize}
The density abundance is dedicatedly measured via cosmic microwave background (CMB) by WMAP \cite{WMAP2013} and Planck \cite{Planck2015} under the $\Lambda$-CDM regime:
\begin{align}
\Omega_{\mathrm{CDM}} h^2 = 
\begin{cases}
  0.1138 \pm 0.0045 \mbox{\phantom{MMMMM} (WMAP)} \\
%  0.1153 \pm 0.0019 \mbox{\phantom{MMMMM} (WMAP, WMAP+eCMB+BAO+$H_0$)} \\
  0.1186 \pm 0.0020 \mbox{\phantom{MMMMM} (Planck, TT+lowP+lensing)},
\label{eq::obsDMRelic}
\end{cases}
\end{align}
%where $\Omega$ is the energy density normalized by the critical energy density, and $h$ the Hubble constant.
While the SM has no candidates for DM, SUSY provides several attractive candidates when assuming the R-parity conservation (Sec \ref{sec::MSSMLag}) in which the lightest SUSY particle (LSP) becomes stable. It is worth noting that the LSP mass will be constrained by an upper bound about $3\tev$, when trying to explain the whole abandunce by SUSY. 
This makes SUSY as a phenomenologically important model rather than a purely theoretical framework,
this is a strong motivation of considering TeV-scale SUSY. \\

%A number of new physics models are motivated by DM 他にもDM candidatesを含むNPはいっぱいある。Axionとか

%There are also a number of experiments for direct detection, in which the SUSY DM scenario can be tested. \\



%\section{Others}
%\paragraph{Strong-CP problem}
%かなりの精度で現在CPは保たれているのにSMではStrong CPVを禁止するschemeがなく、しかもunstableである。
%なのでSMにそういうschemeを入れるというideaが色々あり、その帰結としてaxionが出る。


%\paragraph{Bariogenethesis and CP violation}
%CKMの定式化とCroninの実験を通じて今やCPが破れているのはこの世の常識となりつつあるが
%matter-antimatter asymmetryを説明するほどのmagnitudeはない。よってextra sourceが必要である。
%SM-likeのframeworkの範疇ではhiggs sectorやneutrino sectorでのCPVが期待できる。特にneutrinoは破れていることが割ともう決定されそうである
%new physicsに頼る場合はSUSYとかaxionかなあ

%もっとどうにもならなそうな難しい問題としては
% parameter大杉, quark mass hieararchyなどがある
%\footnote{これは理論parameterにhieararchyがあるだけで、理論の中にfine tuneの構造があるわけではないという点でhiggs massよりは深刻ではない}
% 
% Gravity, 
% Dark Energy
% 

%\paragraph{}
%Essentially no clue.
%修正重力?
%一応そういうものがあった場合のtestは多少proposeされていて、LHCでttbarつかって見えるものもある
%, ???SUSY????????#parameters????, gravity, EW hierarchy problem, ??CP??}


%%%%%%%%%%%%%%%%%%
\section{Super-Symmetry and the MSSM}
%The key concept of the super-symmetry (SUSY) is to assign an invariace on theories against boson-fermion transformation:
%\begin{align}
%& \hat{Q} \, \ket{ \mathrm{Boson}}   = \xi \ket{\mathrm{Fermion}}  \nn \\ 
%& \hat{Q} \, \ket{ \mathrm{Fermion}} = \lambda \ket{\mathrm{Boson}}
%\end{align}
%このthesisではMSSMだけを基本的には扱う
Minimal Super-Symmetric Standard Model (MSSM) is a SUSY framework where minimum matter contents and degrees of freedom are newly introduced with respect to the SM such as:
\begin{itemize}
\item Only one set of SUSY partners is employed ($\mathcal{N}_{\mathrm{SUSY}}=1$),
\item SUSY partners of SM fermions have the spin of 0, while the partners for boson in SM (gauge boson and higgs) are spin-1/2
\item Use only two higgs doublets to construct the higgs sector.
\footnote{Introducing multiple VEV is the simplest solution against the quantum anomalies that newly arise when extending to SUSY.}
\end{itemize}

Though it is called ``minimal'', MSSM is a framework general enough to expressing the typical natures of SUSY at phenomenology level, therefore this thesis will confine the scope within MSSM. An overview on MSSM is given in the rest of the section, widely based on the reference \cite{SUSYPrimer}. \\


%%%%%%%%%%%%%%%%%%%%%%%%%%%%%%%%%%%%%%%%%%%%
\subsection{Particle Contents in MSSM}
\renewcommand{\arraystretch}{1.5}
\tab{ c c | c c c c c }
{
\hline
\multicolumn{2}{c|}{Super-multiplet}   & SM sect.              & SUSY partner                          & $n[SU(3)_C]$ &  $n[SU(2)_L]$ & $Y$ \\
\hline
\hline
gluon/gluino       & $G$             & $g$                     &  $\tg$                                &  8  &  1  &  0  \\ 
\hline
EW gauge boson /   & $W$             & $W^{\pm},W^0$         &  $\tW^{\pm},\tW^0$                    &  1  &  3  &  0  \\
EW gaugino         & $B$             & $B^0$         &  $\tB^0$                    &  1  &  1  &  0  \\
\hline
lepton / slepton   & $L$             & $(\nu_e,e)_\mL$       &  $(\tilde{\nu}_e,\tilde{e})_L$        &  1  &  2  &  -1  \\ 
                   & $\spe$             & $\te_\mR$             &  $e_\mR$                              &  1  &  1  &  -2  \\ 
\hline
quark / sqaurk     & $Q$             & $(u_\mL,d_\mL)$       &  $\left(\tu_\mL, \td_\mL \right)$     &  3  &  2  &  1/3  \\
                   & $\spu$             & $u_\mR$               &  $\tu_\mR$                            &  3  &  1  &  4/3 \\
                   & $\spd$             & $d_\mR$               &  $\td_\mR$                            &  3  &  1  & -2/3  \\
\hline
Higgs boson /      & $H_\mup$      & $(H_\mup^+,H_\mup^0)$ & $(\tH_\mup^+,\tH_\mup^0)$             &  1  &  2  &  1  \\
higgsino           & $H_\mdn$      & $(H_\mdn^0,H_\mdn^-)$ & $(\tH_\mdn^0,\tH_\mdn^-)$             &  1  &  2  & -1  \\
\hline
}
{
Matter content of MSSM. The left column defines the naming convention for SUSY particles. $n[SU(3)_C]$($n[SU(2)_L]$) represents the degree of freedom of the $SU(3)_C$($SU(2)_L$) multiplet that the field(s) belongs to. All of them belongs to the single of $U(1)_Y$, thus the $U(1)$ charge $Y$ is shown instead. There are also two set of replications for the 2nd and 3rd generation of (s)quarks/(s)leptons, which are not shown here.
}
{tab::Introduction::particleContMSSM}
\renewcommand{\arraystretch}{1.}

The particle contents are summarized in Table \ref{tab::Introduction::particleContMSSM}. Note that scalar-fermions (sfermions) have two modes indexed by $L,R$ indicating that they are the SUSY partners of left-handed or right-handed SM fermions respectively. On the other hand, gauginos are all Majorana, in order to match the degree of freedom with either the patner gauge bosons and higgs bosons. \\

MSSM higgs sector has two higgs doublets ($\bm{H}_\mup := (H_\mup^+,H_\mup^0)$, $\bm{H}_\mdn := (H_\mdn^-,H_\mdn^0)$) with their own vacuum expectation values (VEV):
\begin{align}
 v_\mup :=  \left< H^0_\mup \right>, \,\,\,\,  v_\mdn :=  \left< H^0_\mdn \right>, \nn
\end{align}
where each provides the masses for up- or down-type fermions respectively. Their splitting is commonly parametrized using a mixing angle $\beta$ as:
\begin{align}
\tan\beta := v_\mup/v_\mdn.
\end{align}
The consistency with SM is ensured by relating the VEVs as: 
\begin{align}
v^2_{\mathrm{SM}} = v^2_{\mathrm{u}} + v^2_{\mathrm{d}}.
\end{align}

Note that if gravity is quantized in the picture of QFT, there should be also the corresponding gauge boson "graviton" and its SUSY partner "gravitino" along a natural extension. In some SUSY scenarios, gravitino do act a important role such as in GMSB (Gauge Mediated SUSY Breaking), however we do not assume them in the study of this thesis. \\


%%%%%%%%%%%
\subsection{The MSSM Lagrangian} \label{sec::MSSMLag}
Construction of a super-symmetric Lagrangian is commonly done by the method of super-potential or super-space. Though the derivation is skipped here, it may worth noting that it is not as simple as just adding terms accounting for the extra particle contents to the SM Lagrangian, but a procedure with delicate consideration over the SUSY breaking and the nature of quadratic divergence (Eq. \ref{eq::naturalness}). The resultant MSSM Lagrangian consists of following two parts:
\begin{align}
\mathcal{L}^{\mathrm{MSSM}}
& = \mathcal{L}^{\mathrm{MSSM}}_{\mathrm{SUSY}} + \mathcal{L}^{\mathrm{MSSM}}_{\mathrm{soft}}. \label{eq::MSSMLag}
\end{align}
%
\noindent The first term is the SUSY invariant part of the Lagrangian which is given by:
\begin{align}
\mathcal{L}^{\mathrm{MSSM}}_{\mathrm{SUSY}} 
& = \frac{1}{4} F_{a\mu\nu} F^{a\mu\nu} + D^\mu\phi^* D_\mu \phi + \psi^\dg \bar{\sigma}^\mu D_\mu \psi + i \lambda^{\dg a} \bar{\sigma} D_\mu \lambda_a 
\mbox{\phantom{MkMMMM}(Kinetic terms)  } \nn \\
%
&  -\frac{1}{2} W^{ij} \psi_i \psi_j + h.c.
\mbox{\phantom{MMMMMMMMMMMMMMMM}(Yukawa interaction terms)  } \nn \\
%
&  - \sqrt{2} g (\phi^{*}T^a\psi) \lambda_a + h.c.
\mbox{\phantom{lMMMMMMMMMMMMM}(Gaugino interaction terms)  } \nn \\
%
&  - \sum_i\left| \frac{\delta W}{\delta\phi_i} \right|^2 +  \frac{1}{2} (g_a \phi^* T^a \phi)^2  \nn
\mbox{\phantom{MMMMMMM}(Residual terms from the aux. fields)  } 
\end{align}
where $\psi$ is SM fermions are $\phi$ is the corresponding spin-0 SUSY partners, while $\lambda$ are gauginos. $W_{ij}$ is the second derivative of super-potential $W$, with $W$ being defined by:
\begin{align}
W_{ij} & := \frac{\delta^2 W}{\delta \phi_i \delta \phi_j}, \nn \\
W & := \spu \bm{y}_\mup Q H_\mup - \spd \bm{y}_\mdn Q H_\mdn - \spe \bm{y}_e L H_\mdn + \mu H_\mdn H_\mup.
\end{align}
$\bm{y}_\mup$, $\bm{y}_\mdn$ and $\bm{y}_e$ are the same Yukawa matrices in Eq. (\ref{SMLagYukawa}). Note that no theory parameters are newly introduced compared with SM in $\mathcal{L}^{\mathrm{MSSM}}_{\mathrm{SUSY}}$.
The soft SUSY breaking term $\mathcal{L}^{\mathrm{MSSM}}_{\mathrm{soft}}$ is SUSY variant part of the Lagrangian. Further caveats are provided as below:   \\
%
%%%%%%%%%%%%%
\paragraph{\underline{SUSY breaking}}  \mbox{}
While an exact super-symmetry requires the SUSY partners being in the identical masses with respect to the SM particles, it is not the case at least in the energy scale of current our universe since no SUSY particles have been discovered so far. Therefore, a realistic SUSY model as an effective theory at the EW scale, must contain a scheme of SUSY breaking in its Lagrangian ($\mathcal{L}^{\mathrm{MSSM}}_{\mathrm{soft}}$). On the other hand, we don't want to ruin the desired features in SUSY at the cost of it, particularly as the solution of the higgs mass fine-tuning problem (Sec. \ref{sec::Introduction::homeworkSM::fineTuning}).
Therefore, it is common to restrict the SUSY breaking in a form of ``soft breaking'' where the cancelation of the quadratic divergence in the higgs mass loop correction Eq. (\ref{eq::naturalness}) is maintained. \\
%This is done by restricting the terms in $\mathcal{L}^{\mathrm{MSSM}}_{\mathrm{soft}}$ with:
%\begin{align}
%M_a\lambda^a \lambda^a, a^{ijk} \phi_i \phi_j \phi_k,  b^{ij}\phi_i \phi_j, t^i\phi
%\end{align}

The most general form of the soft breaking terms is given by:
%
\begin{align}
\mathcal{L}^{\mathrm{MSSM}}_{\mathrm{soft}} 
& = \frac{1}{2} \left( M_3 \, \tg\tg + M_2 \tW\tW + M_1 \tB\tB + \mathrm{c.c.}  \right)  \mbox{\phantom{MMMMMMM} (gaugino mass terms)}  \label{eq::MSSM_gauge} \\
%
& - Q^{\dg} \, \bm{m}_{Q}^2    \, Q 
  - L^{\dg} \, \bm{m}_{L}^2    \, L 
  - \spu    \, \bm{m}_{\spu}^2 \, \spu^{\dg} 
  - \spd    \, \bm{m}_{\spd}^2 \, \spd^{\dg} 
  - \spe    \, \bm{m}_{\spe}^2 \, \spe^{\dg}  \mbox{\phantom{M}(sfermion mass terms)  }  \label{eq::MSSM_sfermions} \\
%
& -  \left( 
  \spu \bm{a}_\mup Q H_\mup 
- \spd \bm{a}_\mdn Q H_\mdn 
- \spe \bm{a}_e    L H_\mdn 
+ \mathrm{c.c.}  \right)  \mbox{\phantom{MMMMMM} (trilinear coupling)} \label{eq::MSSM_trilinear} \\
%
& - m^2_{H_\mup} H_{\mup}^{\dg} H_\mup  
  - m^2_{H_\mdn} H_{\mdn}^{\dg} H_\mdn  
  - (b H_\mup H_\mdn + \mathrm{c.c.})   \mbox{\phantom{MMMMMM} (Higgs potential)} \label{eq::MSSM_higgs}
%
%\label{eq::MSSMLag}
\end{align}
The notation of the particle fields ($\tg$,$\tW$,$\tB$) and super-multiplet ($Q$,$L$,$\spu$,$\spd$,$\spe$,$H_\mup$,$H_\mdn$) follow the definition in Table \ref{tab::Introduction::particleContMSSM}.
The first line (Eq. (\ref{eq::MSSM_gauge})) show the mass terms for gauginos, with $M_1$, $M_2$ and $M_3$ are respectively bino, wino and gluino mass. Eq. (\ref{eq::MSSM_sfermions}) involves the Yukawa terms for SUSY particles where the former are the standard sfermion mass terms, and Eq. (\ref{eq::MSSM_trilinear}) the trilinear terms describing the Yukawa interaction coupling left-handed and right-handed sfermions, emerged as the cross terms of super-multiplet. The mass matrices ($\bm{m}_Q$, $\bm{m}_L$, $\bm{m}_{\spu}$, $\bm{m}_{\spd}$, $\bm{m}_{\spe}$), and the A terms ($\bm{a}_\mup$, $\bm{a}_\mdn$ and $\bm{a}_e$) are $3\times3$ matrices spanned in family space, equivalent to the CKM matrix in the SM sector multiplied by sparticles masses. The last terms \label{eq::MSSM_higgs} are the MSSM higgs potential, controlling the EW symmetry breaking.  \\
% trilinear = helicity flipping term by higgs interaction in SM. In MSSM, left/right is differentiated so this can not be understand as mass terms of generic interaction term

Though not specifically targeted in the thesis, there are a number of models in the market offering explicit mechanisms of the soft SUSY breaking. The most minimal models are known as GMSB (Gauge-Mediated SUSY Breaking \cite{GMSB}), AMSB (Anomaly-Mediated SUSY Breaking \cite{AMSB1,AMSB2}) or mSUGRA (minimal SUper Gravity \cite{SUGRA}). \\
%and the next-to-minimal ones including their generalization: NUMH (Non-Universal Higgs Mass model), GGM (General Gauge Mediation model). 


\paragraph{\underline{R-parity}}  \mbox{}
A quantum number $R$ associated with the number of ``SUSY partner'' (analogous to the lepton number or baryon number etc.) can be defined by the spin, baryon number and lepton number as:
\begin{align}
R := (-1)^{3(B-L)+2S}.
\end{align}
The corresponding symmetry is referred to R-parity, which conservation law will prohibit single production of SUSY particles, as well as SM particles annihilating into a resonance of a SUSY particle. 
This leads a set of spectacular phenomenological advantages: 
\begin{itemize}
\item The lightest SUSY particles (LSP) become the DM candidates if they are electric neutral, in particular the lightest neutralino is the most commonly assumed. 
\item Proton decays via diagrams in Figure \ref{fig::Introduction::protonDecay} are prohibited, naturally reconciling with the constraints set by experiments \cite{protonDecaySuperK}.
\end{itemize}
In the framework of MSSM, the R-parity conservation (RPC) is explicitly assumed, which is equivalent to discard following terms in the most general soft breaking Lagrangian:
\begin{align}
W_{\Delta L=1} & = \frac{1}{2} \lambda^{ijk} L_i L_j \bar{e}_\mathrm{k} + \lambda^{'\, ijk} L_i Q_j \bar{d}_k + \mu^{'i} L_i H_\mup \nn \\
W_{\Delta B=1} & = \frac{1}{2} \lambda^{''\, ijk} \bar{u}_i  \bar{d}_j  \bar{d}_k.
\label{eq::RPVterms}
\end{align}

%%%
\fig[100]{Introduction/protonDecay}
{An example process of a proton decay triggered by intermediate SUSY particles (scalar-strenge quark here). $\lambda''_{112}$ and $\lambda'_{112}$ are couplings for corresponding interaction vertices which violate $R$-parity.}
{fig::Introduction::protonDecay}
%%%

%It is however also true that RPC is not a kind of requirement that 
%R-parity violating scenarioも実験では探索されている. 



\subsection{Mass Spectra}
The masses of SUSY particles are derived by specifying the coefficient associated with mass terms (e.g. $m$ in $m\phi\phi$), after a full expansion of the Lagrangian in Eq. (\ref{eq::MSSMLag}). This is effectively done by extracting relevant terms and performing the diagonalization on the mass matrices, accounting for the mixing between eigenstates of interactions. \\

\paragraph{Squarks and sleptons}
Sfermion masses are fed solely from the soft Lagrangian. Generally, they are allowed to mix between different generations via the off-diagonal components either in the mass matrices or the A terms. These are however known to lead to a significant rate of flavor changing natural current which are experimentally highly disfavored thus usually set to zero:
\begin{align}
&  \bm{m}^2_{Q}    = m^2_{Q} \,    \bm{1}, \,\,\,\,\,\,
  \bm{m}_{L}^2    = m_{L}^2 \,    \bm{1}, \,\,\,\,\,\,
  \bm{m}_{\ubar}^2 = m_{\ubar}^2 \, \bm{1}, \,\,\,\,\,\,
  \bm{m}_{\dbar}^2 = m_{\dbar}^2 \, \bm{1}, \,\,\,\,\,\,
  \bm{m}_{\ebar}^2 = m_{\ebar}^2 \, \bm{1}, \,\,\,\,\,\, \nn \\
%
& \bm{a}_\mup = A_\mup \, \bm{1}, \,\,\,\,\,\,
  \bm{a}_\mdn = A_\mdn \, \bm{1}, \,\,\,\,\,\,
  \bm{a}_e    = A_e \, \bm{1}
\end{align}
In addition, it is also allowed to mix left-handed sfermion and right-handed sfermion since they share the same gauge quantum numbers. Ignoring the off-diagonal components of the Yukawa matrix, the mass matrix for sfermion $\tilde{f}$ reduces to:
\begin{align}
& \left(  
  \begin{array}{cc}
    m^2_{\tilde{f}_{\mL}} + m_Z^2 \,(T_{3,f}-Q_f\sin{\theta_W}^2) \cos{2\beta} + m^2_f    &  v_f ( A_f - \mu y_f)         \\
    v_f ( A_f - \mu y_f)                &     m^2_{\tilde{f}_{\mR}} + m_Z^2 \, Q_f\sin{\theta_W}^2\cos{2\beta} + m^2_f    
  \end{array} 
\right), \nn  \\
%
& v_f = \begin{cases}
  v_{\mup} \mbox{\phantom{MMM}} (\tilde{f}=\tilde{u}, \tilde{c}, \tilde{t}) \\
  v_{\mdn} \mbox{\phantom{MMM}} (\tilde{f}=\tilde{d}, \tilde{s}, \tilde{b}) 
     \end{cases}
%
\end{align}
where $T_{3,f}$ and $Q_f$ are the iso-spin and electric charge of $\tilde{f}$. As the magnitude off-diagonal component scales with the Yukawa coupling, the effect of the mixing can be only sizable in case of third generation sfermions (stop, sbottom and stau).
%The splitted two mass eigenstates are often referred to $f_1$ and $f_2$ with $m_2>m_1$.
This is why the third generation sfermions are particularly phenomenologically important,   
since the masses of lighter eigenstates can be significantly lowered, enhancing the chance of being within experimental reach.  \\



\paragraph{Gauginos}
The mass terms of EW gauginos and higgsinos are sourced by $\mathcal{L}^{\mathrm{MSSM}}_{\mathrm{SUSY}}$. The eigenstate of charged EW gauginos (charginos; $\tW^\pm, \tH^+_\mup, \tH^-_\mdn$) in the same signs will mix each other.
The mass matrices are common and described as:
\begin{align}
\left(  
  \begin{array}{cc}
    M_2                    &  \sqrt{2} m_W \sin{\beta} \\
    \sqrt{2} m_W \cos{\beta} &  \mu                    \\
  \end{array}
\right).
\end{align}
%
\noindent The diagonalized mass eigenstates are then:
\begin{align}
m_{\tilde{\chi}_{1,2}^{\pm}}^2 = \dfrac{1}{2} \left[ (M_2^2+\mu^2+2m_W^2) \mp \sqrt{(M_2^2+\mu^2+2m_W^2)^2 - 4(\mu M_2-m_W^2\sin2\beta)^2} \right].
\end{align}
%
\noindent The mass matrix for neutral EW gauginos (neutralinos; $\tB, \tW^0, \tH^0_\mup, \tH^0_\mdn$) are given as:
\begin{align}
\left(  
  \begin{array}{cccc}
    M_1                       &  0                         & -\cos\beta\sin\theta_W m_Z &  \sin\beta\sin\theta_W m_Z   \\
    0                         &  M_2                       &  \cos\beta\cos\theta_W m_Z & -\sin\beta\cos\theta_W m_Z   \\
   -\cos\beta\sin\theta_W m_Z &  \cos\beta\cos\theta_W m_Z &  0                         & -\mu                         \\
    \sin\beta\sin\theta_W m_Z & -\sin\beta\cos\theta_W m_Z & -\mu                       &  0                           \\
  \end{array}
\right). \nn \\ \nn
\end{align}
The eigenformula is quartic and thus it is not easy to spell out the exact expression of the eigenvalues, 
however the asymmptotic expression below is commonly found when $|\mu| \gg |M_2| > |M_1| $, corresponsing to the wino-NLSP/bino-LSP scenario:
\begin{align}
m_1 & = M_1 + \frac{m_Z^2\sin^2{\theta_W}}{M_1^2-\mu^2} (M_1 + \mu\sin2\beta)   \nn \\
m_2 & = M_2 + \frac{m_Z^2\cos^2{\theta_W}}{M_2^2-\mu^2} (M_2 + \mu\sin2\beta)   \nn \\
m_3 & = \mu + \frac{m_Z^2(1+\sin{2\beta})}{2(\mu-M_1)(\mu-M_2)} (\mu-\cos{\theta_W}M_1-\sin{\theta_W}M_2)  \nn \\
m_4 & = \mu + \frac{m_Z^2(1-\sin{2\beta})}{2(\mu+M_1)(\mu+M_2)} (\mu+\cos{\theta_W}M_1+\sin{\theta_W}M_2).
\end{align}
The conventional notation for neutralino masses $m_{\tilde{\chi}_{1-4}^0}$ 
are defined by sorting these eigenvalues as $m_{\tilde{\chi}_{1}^0}<m_{\tilde{\chi}_{2}^0}<m_{\tilde{\chi}_{3}^0}<m_{\tilde{\chi}_{4}^0}$. \\
%The matrix diagonalizing the mass matrix is found to be:
%\begin{align}
%\left(  
%  \begin{array}{cccc}
%    1                       &  0                         & -\cos\beta\sin\theta_W m_Z &  \sin\beta\sin\theta_W m_Z   \\
%    0                         &  M_2                       &  \cos\beta\cos\theta_W m_Z & -\sin\beta\cos\theta_W m_Z   \\
%   -\cos\beta\sin\theta_W m_Z &  \cos\beta\cos\theta_W m_Z &  0                         & -\mu                         \\
%    \sin\beta\sin\theta_W m_Z & -\sin\beta\cos\theta_W m_Z & -\mu                       &  0                           \\
%  \end{array}
%\right). \nn \\ \nn
%\end{align}
%which represents the composition of bino, wino, and higgsinos in the mass eigenstates.

\noindent Finally, gluinos are color-octet fermions which do not mixed to any other sfermions. The mass is simply defined by:
\begin{align}
\mG = M_3. \label{eq::gluinoMass}
\end{align}



\paragraph{The MSSM Higgs sector}
% higgs sector
Due to the two higgs doublets with 4 real and 4 imaginary parts, 
there are in total five degree of freedoms as physical particles after the gauge fixing.
The MSSM higgs potential is given by:
\begin{align}
V 
& = \left( |\mu|^2 + m^2_{H_\mup} \right)  \left( |H^0_\mup|^2+|H^+_\mup|^2 \right) \nn \\
& + \left( |\mu|^2 + m^2_{H_\mdn} \right)  \left( |H^0_\mdn|^2+|H^-_\mdn|^2 \right) \nn \\
& + \left[ b (H^+_\mup H^-_\mdn-H^0_\mup H^0_\mdn) + \mathrm{c.c.} \right] \nn \\
& + \frac{1}{8} \left(g^2+g'^2 \right) \left(   |H^0_\mup|^2+|H^+_\mup|^2-|H^0_\mdn|^2-|H^-_\mdn|^2   \right) ^2  \nn \\
& + \frac{1}{2} |H^+_\mup H^{0*}_\mup + H^+_\mdn H^{-*}_\mdn|. 
\label{eq::MSSMhiggsP}
\end{align}
Similarly to the case in SM, implementing the spontaneous symmetry breaking by
plugging $H_{\mup,\mdn} \ra v_{\mup,\mdn} + \eta_{\mup,\mdn}$ into Eq. (\ref{eq::MSSMhiggsP}),
and requiring $dV/dv_\mup = dV/dv_\mdn = 0$, one arrives:
\begin{align}
\sin{2\beta} & = \dfrac{2b}{m^2_{H_\mup} + m^2_{H_\mdn} + 2|\mu|^2 } \label{eq::MSSMSSB1} \\ 
\frac{1}{2} m_Z^2 & = -|\mu|^2 + \dfrac{m^2_{H_\mup}-m^2_{H_\mdn}\tan^2{\beta}}{\tan^2{\beta}-1} \label{eq::MSSMSSB2}
\end{align}
The higgs masses are found by the masses terms with inserting Eq. (\ref{eq::MSSMSSB1})-(\ref{eq::MSSMSSB2}) back to Eq. (\ref{eq::MSSMhiggsP}):
\begin{align}
m^2_{A} & = 2 |\mu|^2 + m^2_{H_\mup} + m^2_{H_\mdn}, \nn \\
m^2_{H^{\pm}} & = m^2_{A^0} + m^2_W \nn \\
m^2_{h,H} & = \dfrac{1}{2} \left( m^2_{A^0} + m_Z^2 \mp \sqrt{(m^2_{A^0} + m^2_Z)^2-4m_Z^2 m^2_{A^0}\cos^2{2\beta}}   \right),
\label{eq::MSSMhiggsMass}
\end{align}
where $H^{\pm}$ is the charged, $A$ the CP-odd higgs respectively. 
$H$ and $h$ are the mass eigenstates of CP-even neutral higgs, where the lighter one $h$ is often associated with the SM higgs. Given that no observation of $H$ has been claimed upto $400\gev-1\tev$, it is generally preferred to have large mass splitting between $h$ and $H$, which implies a large $\tan{\beta}$. \\





%%%%%%%%%%%%%%%%%%%%%%%%%%%%%%%%%%%%%%%%%%%%
\subsection{Running Masses and GUT} \label{sec::Introduction::SUSYMassGUT}
Though the SUSY masses are mostly free parameters in MSSM, 
an useful insight can be obtained from an quick analysis under the GUT regime in which the coupling constants are unify at the GUT scale: $\mu_{\mathrm{GUT}}\sim 10^{16-17}\gev$. 
%
In the SUSY context, the mass unification is often in addition considered, typically under the regime where:
\begin{itemize}
\item all sfermions masses converge to $m_{1/2}$
\item all gaugino masses converge to $m_0$
\item all higgs boson ($H_u$, $H_d$) masses converge to $(\mu^2+m_0^2)^{1/2}$.
\end{itemize}
This configuration is particular advantageous in that it naturally causes EW symmetry breaking at the EW scale, 
%and adopted in many minimal models such as SUGRA, NUMH and CMSSM (Constrained MSSM). \\
and adopted in many minimal models including SUGRA and so on. \\
%
Starting with gaugino masses, using the general condition satisfied in the 1-loop renormalization:
$$
\frac{d(M_i/\alpha_i)}{d\mu} = 0, \,\,\,\, (i=1,2,3),
$$
it turns that $(M_i/\alpha_i)$ is constant in arbitrary scales. Therefore, one obtains:
\begin{align}
\frac{M_i}{\alpha_i}|_{\mu=\mu_{\mathrm{EW}}}  = \frac{M_i}{\alpha_i}|_{\mu=\mu_{\mathrm{GUT}}} = \frac{m_{1/2}}{\alpha_{\mathrm{GUT}}},
\end{align}
resulting in an univeral ratio in gaugino masses valid in any scale:
\begin{align}
M_1 : M_2 : M_3 \sim 6:2:1. \label{eq::gauginoMass_621}
\end{align}
This is the reason this mass hierarchy between gluino, wino and bino are especially motivated and commonly assumed in SUSY phenomenology, though it is true that the assumption of mass unification may be too strong. \\

As for sfermions, the running masses also provide some idea about the mass spectra at the EW scale.
The running masses are calculated unambiguously using the renormalization group equations:
\begin{align}
m_{\tilde{d}_{\mathrm{L}}}^2   & = m_0^2 + K_3                    + K_2                    + \frac{1}{36} K_1 + \Delta_{\tilde{d}_{\mathrm{L}}}   \nn  \\ 
m_{\tilde{u}_{\mathrm{L}}}^2   & = m_0^2 + K_3                    + K_2                    + \frac{1}{36} K_1 + \Delta_{\tilde{u}_{\mathrm{L}}}   \nn  \\ 
m_{\tilde{d}_{\mathrm{R}}}^2   & = m_0^2 + K_3                    + \mbox{\phantom{$K_2$ +}} \frac{1}{9} \,\,\,  K_1 +  \Delta_{\tilde{d}_{\mathrm{R}}}   \nn  \\ 
m_{\tilde{u}_{\mathrm{R}}}^2   & = m_0^2 + K_3                    + \mbox{\phantom{$K_2$ +}} \frac{4}{9} \,\,\,  K_1 +  \Delta_{\tilde{u}_{\mathrm{R}}}   \nn  \\ 
m_{\tilde{e}_{\mathrm{L}}}^2   & = m_0^2 + \mbox{\phantom{$K_3$ +}} K_2                    + \frac{1}{4} \,\,\,  K_1 +  \Delta_{\tilde{e}_{\mathrm{L}}}   \nn  \\ 
m_{\tilde{\nu}_{\mathrm{L}}}^2 & = m_0^2 + \mbox{\phantom{$K_3$ +}} K_2                    + \frac{1}{4} \,\,\,  K_1 +  \Delta_{\tilde{\nu}_{\mathrm{L}}} \nn  \\ 
m_{\tilde{e}_{\mathrm{R}}}^2   & = m_0^2 + \mbox{\phantom{$K_3$ +}} \mbox{\phantom{$K_2$ +}} \mbox{\phantom{kk}}\, K_1 + \Delta_{\tilde{e}_{\mathrm{R}}}   \nn  
\end{align}
where $K_1$, $K_2$ and $K_3$ respectively denotes the contribution from the interaction of $U(1)_Y$, $SU(2)_L$ and $SU(3)_C$, which are approximately:
\begin{align}
K_1 \sim 0.15 \, m_{1/2}^2, \,\,\, K_2 \sim 0.5 \, m_{1/2}^2, \,\,\, K_3 \sim 6 \, m_{1/2}^2,
\end{align}
and the correction factors $\Delta_{\tilde{f}}$ are given by:
\begin{align}
\Delta_{\tilde{f}_\mathrm{L}} & = (T_3-Q\sin^2{\theta_W}) \, m_Z^2 \cos{2\beta} + m_f^2   \nn \\
\Delta_{\tilde{f}_\mathrm{R}} & = Q\sin^2{\theta_W} \, m_Z^2 \cos{2\beta} + m_f^2.   \nn 
\end{align}
Since the effect of running masses are always larger for squarks than sleptons due to the $SU(3)_C$ interaction, 
it generally implies lighter masses for sleptons.
%Also, it is shown that there is no strong flavor dependence in the running, given that $m_0 \sim O(TeV)$.
The typical running mass spectra is shown in Figure \ref{fig::Introduction::runningMass}. \\

%%%%%
\fig[100]{Introduction/runningMass.pdf}
{Evolution of scalar and gaugino mass parameters in the MSSM with mSUGRA boundary conditions \cite{SUSYPrimer}. The parameters are $m_0=200 \gev$, $m_{1/2}=600 \gev$, $A^0=-600\gev$, $\tan{\beta}=10$ and sign($\mu$)$>0$.}
{fig::Introduction::runningMass}
%%%%


%\clearpage
%%%%%%%%%%%%%%%%%%%%%%%%%%%%%%%%%%%%%%%%%%%%
\section{Experimental Constraints on SUSY so far}  \label{sec::Introduction::ExpConst}
\subsection{Constraints from Observed Standard Model Higgs Mass}
It is a striking fact that in MSSM the mass of 125 GeV higgs ($h$) is bounded by:
\begin{align}
m_{h} < m_Z \cos{2\beta} < m_Z = 91.2 \gev,
\end{align}
according to Eq. (\ref{eq::MSSMhiggsMass}).
Therefore, a sizable radiation correction is needed to achieve 125 GeV.
The 1-loop correction is dominantly given by the remnant of cancellation of top and stop loop in Eq. (\ref{eq::naturalness}):
\begin{align}
\Delta m_h^2 := \frac{3}{4} \frac{m_t^4}{v_{\mSM}^2} \left[ \log{\frac{m_{\ttop}^2}{m_t^2}} + \frac{X_t^2}{m_{\ttop}^2} \left(1-\frac{X_t^2}{12m_{\ttop}^2} \right)  \right],
\end{align}
which has to accord with 
\begin{align}
\sqrt{(125\gev)^2 - m_Z^2} \sim 85\gev.
\end{align}
This is a tremendously powerful constraint that forces either of following two ambivalent choices:
\begin{enumerate}
\item without assuming anything on stop mixing (e.g. $X_t$ is free) and $O(10\tev)$ of stop mass, with relatively large fine tuning ($\Delta_{m_{h}}>1000$), as shown in Figure \ref{fig::Introduction::higgsMass_mixing}. 
\item maximal stop mixing ($X_t \sim \sqrt{6}m_{\ttop}$), and $500\gev-1\tev$ of stop mass, with mild fine tuning ($\Delta_{m_{h}} \sim 100$).
\end{enumerate}
The consequent implication from the former choice is that all the squrks and sleptons are heavy, and only gauginos could be explored in LHC,
while the latter leads to light stop (or sbottom) accessible by the LHC energy while the others are not necessarily so. \\

\begin{figure}[h]
  \centering
    \subfig{0.55}{figures/Introduction/higgsMass_mixing.pdf}{}
    \subfig{0.42}{figures/Introduction/higgsMass_mixing2.pdf}{}
    \caption{Relation of mass of SM-like higgs and stop mass in MSSM \cite{Higgs125SUSYHall}. (a) The SM-like higgs mass as a function of lightest stop mass ($m_{\ttop}$), with the no ($X_t=0$) or maximal stop mixing ($X_t \sim \sqrt{6}m_{\ttop}$). Red/blue solid lines correspond the computation using Suspect/FeynHiggs.  (b) A 2D-constraint on the stop mass and stop mixing $X_t/m_{\ttop}$ by observed SM-like higgs mass, with $m_{\tQ} = m_{u_3} = m_{\ttop}$ and $\tan{\beta}=20$. The dashed contour shows the gauge of fine tuning $\Delta_{m_{h}}$ defined by Eq. (\ref{eq::fineTuingDef2}). }
    \label{fig::Introduction::higgsMass_mixing}
\end{figure}

The higgs mass fine tuning argument in MSSM is rather subtle, since the observed $m_{h}$ is no longer as straightforwardly associated with its own mass parameter $H_\mup$ as in the case in SM (Sec. \ref{sec::Introduction::homeworkSM::fineTuning}), but also involved by the other MSSM parameters as seen in Eq. (\ref{eq::MSSMhiggsMass}).
The magnitude of fine tuning is usually quoted by the linear response of any arbitrary MSSM parameters $p_i$ \cite{Higgs125SUSYHall}:
\begin{align}
\Delta_{m_{h}} := \max_i \left|  \frac{\partial \log [m^2_h (\mbox{1-loop})] }{\partial \log{p_i}} \right|.
\label{eq::fineTuingDef2}
\end{align}
In scenario 1. above, the resultant fine tuning is typically $1/\Delta_{m_{h}} \sim O(10^{-3})$, while $\sim 1\%$ is achievable in the scnerio 2 in the most optimistic case with $\sim 500\gev$ stop.  \\
%
As a level of $\sim O(10^{-3})$ of the fine tuning is not as fatal as that in the SM ($10^{-34}$),
in the thesis, we pursue the former scenario, and probing gluinos in the experiment assuming all the squarks are all decoupled. \\
%


%%%%%%%%%%%%%%%%%%%%%%%%%%%%%%%%%%%%%%
\clearpage
\subsection{Constraint from Dark Mater Relic Density} \label{sec::Introduction::DMconstraint}
%\paragraph{Relic density}
The main stream of current DM theory is based on the ``cold matter'' regime in which DM used to be in a thermal equilibrium at the beginning of the universe, and cooled down according to the cosmic expansion later on, and being decoupled at a certain scale, fixing the abundance upto now. The relics is strongly related by the annihilation cross-section, which can be calculated within the MSSM framework. \\

Phenomenologically there are a couple of major classes of DM scenarios depending on the component of LSP.
The case of pure bino-LSP can be almost immediately excluded, 
in a limit where all the squarks are decoupled, 
since it has to then rely on the annihilation channel via sleptons \cite{SUSYDM_WTN}, 
where 
%\footnote{A small exception is when the LSP is in the mass of $m_Z/2, m_h/2$ decaying into $Z$ or $h$ with large cross-section due to the threshold enhancement, or close to the mass of stau where co-annihilation cross-section is enhance by the similar mechanism.}
$m_{\tlep}<110\gev$ is needed to achieve the observed relic abundance (Eq. (\ref{eq::obsDMRelic})) which is actually already excluded by LEP2. 
%
On the other hand, 
the annihilation cross-section tends to be too large in case of pure-wino or pure-higgsino LSP, where roughly $\sim 3\tev$ of wino mass or $\sim 1\tev$ of higgsino mass is needed to match with the observed relic Eq. (\ref{eq::obsDMRelic}), which is unfortunately beyond the LHC reach. 
%
What if the mixed case? It is particular interesting to consider doping a bit of wino or higgsino component into bino-dominated LSP, where moderated annihilation cross-section and experimental accessible LSP mass can be achieved simultaneously. This type of LSP is called ``well-tempered'' neutralino LSP \cite{SUSYDM_WTN}, typically predicting a moderately small mass splitting between the next-to-the-lightest SUSY particle (NLSP) and the LSP with $20-60\gev$ \cite{SUSYDM_BWCA,Bramante2016}. \\

\noindent Note that a number of caveat remarks are to be added on the discussion:
\begin{itemize}
\item The observed relics is always based on $\Lambda$-CDM within the cold DM regime. The constraint on SUSY could therefore drastically different if DM is ``warm'' produced non-thermally. 
\item The DM annihilation cross-section calculation so far is dominantly done at the lowest-order (LO) in the perturbation. The contribution of higher order terms will generally increase annihilation cross-section. 
\item Non-perturbative effects (continuous interaction) in a collision of non-relativistic particles often lead to a sizable increase in annihilation cross-section (``Sommerfeld enhancement''). 
\item Is is a bit awkward though, it is possible for other new physics to supply the DM relics when SUSY is not capable of explaining the entire relic.
\end{itemize}
Given these too many uncertainties, it is sensible to regard the relic constraint as soft constraint. However, generally it is more fatal to have excessive relics than the opposite case, here we promise to respect the observed relic more as upper bound. \\

\figNoH[120]{Introduction/WTN_dM_NLSP_LSP.pdf}
{Mass spliting between NLSP (next-to-the-lightest SUSY particle) and LSP, as function of $M_1$, $M_2$ and $\mu$ when assigning the DM relic constraint \cite{Bramante2016}. The effect of Sommerfeld enhancement is taken into the calculation. Within the reach by the LHC energy ($\min{(M_1,M_2)} <1\tev$), the resultant NLSP-LSP mass splitting is about $20\gev\sim30\gev$. Black points correspond to parameter space excluded by LEP.}
{fig::Introduction::WTN_dM_NLSP_LSP}

			

%%%%%%%%%%%%%%%%%%%%%%%%%%%%%%%%%%%%%%
\clearpage
\subsection{Constraint from Direct Search at Collider Experiments}  \label{sec::Introduction::ExpConstColliders}
The direct search of SUSY had been widely performed in collider experiments including LEP, Tevatron and LHC covering over a number of signatures and scenarios. 
Unfortuanately no evidence has been claimed, it is interpreted into constraints either on specific full models (mainly SUGRA-type models, GMSB and cMSSM), or on particular production and decay chains (``simplified model'' as discussed in Sec \ref{sec::Introduction::strategy}).
%simplified modelによるdecay chainごとへのupper limit(後述), またpMSSMといたfull model orientedなmodelに対してinterpretationがなされている。
This sub-section overviews the status of constraints placed on simplified models.

\paragraph{Gluinos}
The best job is done by hadron collider experiments due to its outstandingly high production cross-section.
It is particularly the case in LHC Run2, dominating the sensitivity in most of the scenarios in terms of the mass spectra and gluino decays. \\

The exclusion limits on the most typical gluino decays set by ATLAS and CMS are shown in Figure \ref{fig::Introduction::LHCLimitGG}, namely (a) the direct decay where gluino directly fall into LSP with emitting two quarks, or (b) the 1-step decay via NLSP chargino. Upto $\sim 2\tev$ in gluino mass is excluded for case with large mass splitting between gluino and LSP, and $1.2 \sim \tev$ for the most pessimistic case where gluino and LSP are highly compressed. Note that the listed limits are all up-to-date published results as of July 2017. While most of them is with full 2016 dataset (integrated luminosity of $\mathcal{L} \sim 36\,\,\ifb$), the ATLAS 1-lepton analysis (ATLAS-CONF-2016-054) is with smaller dataset ($\mathcal{L}=14.8\,\,\ifb$). \textbf{This study is meant for the update of it with the up-to-date dataset ($\mathcal{L}=36.1\,\,\ifb$) as well as the improved analysis method}.  \\

Gluino decaying with top quarks addresses particular importance since it can be enhanced by the light stop which is motivated by naturalness. 
They are exclusively searched with dedicated signal regions, and the resultant limit in given in Figure \ref{fig::Introduction::LHCLimitGtt}. 
\textbf{This type of models are also the scope of the thesis, for which an improved result will be provided with respect to the existing ones.}

%%%%%
\clearpage
\begin{figure}
  \centering
    \subfig{0.48}{figures/Introduction/LHC_SUSY_GG_direct_36.pdf}{Constraints on $pp\ra \tg\tg, \tg \ra q \bar{q} \tLSP$.}
    \subfig{0.48}{figures/Introduction/LHC_SUSY_GG_1step_36.pdf}{Constraints on $pp\ra \tg\tg, \tg \ra q\bar{q} \tchic$.}
    \caption{Up-to-date constraints set by ATLAS and CMS on (a) direct gluino decay: $\tg \ra q \bar{q} \tLSP$, and (b) the 1-step chargino-mediated gluino decay: $\tg \ra q\bar{q} \tchic$ with the mass being in the middle between gluino and the LSP. The article numbers for corresponding references are labeled on the plots. ``0L'' and ``1L'' respectively denote searches with 0-lepton and 1-lepton final state.}
    \label{fig::Introduction::LHCLimitGG}
\end{figure}
%
\begin{figure}
  \centering
    \subfig{0.48}{figures/Introduction/ATLAS_SUSY_Gtt_36.pdf}{Constrains set by ATLAS Run2.}
    \subfig{0.48}{figures/Introduction/CMS_gg_tt_36.pdf}{Constrains set by CMS Run2.}
    \caption{Up-to-date constraints on pair produced gluinos directly decaying with top quarks ($\tg \ra t\bar{t} \tLSP$) set by (a) ATLAS and (b) CMS.
    The summary plots are referred from \cite{ATLAS_SUSY_PublicResult} (ATLAS) and \cite{CMS_SUSY_PubResult} (CMS). 
    }
    \label{fig::Introduction::LHCLimitGtt}
\end{figure}

%%%%%
\clearpage
\paragraph{Squarks}
A class of analyses are dedicated for direct stop production with numerious stop decay scenarios and mass configuratons.
The strongest limits are provided by LHC, and upto about $400\gev \sim 1\tev$ in stop mass is generally excluded. Figure \ref{fig::Introduction::LHCLimitStop} presents the example limits on the direct stop decay scenario: $\ttop \ra t \tilde{\chi}_1^0$ provided by ATLAS and CMS.

\begin{figure}[h]
  \centering
    \subfig{0.5}{figures/Introduction/ATLAS_SUSY_Stop_tLSP.pdf}{Constraints set ATLAS by Run2.}
    \subfig{0.46}{figures/Introduction/CMS_stop_36.pdf}{Constraints set by CMS Run2.}
    \caption{Up-to-date constraints on stop pair production with direct decay $\ttop \ra t \tilde{\chi}_1^0$ set by (a) ATLAS and (b) CMS.
      The summary plots are referred from \cite{ATLAS_SUSY_PublicResult} (ATLAS) and \cite{CMS_SUSY_PubResult} (CMS).
    }
    \label{fig::Introduction::LHCLimitStop}
\end{figure}


%%%%%
\clearpage
\paragraph{Electroweak Gauginos}
A number of searches for direct EW gaugino prodicution have been performed in LEP, Tevatron and LHC, and LHC provides the majority of current storngest limits. 
The targeted signature is mostly pair produced NLSPs ($\tchic$ or $\tchin$) decaying to LSP, where decoupled squarks are often assumed. 
\footnote{Under the decoupled squark scenario, bino production is strongly suppressed.} \\

Bino-LSP/wino-NLSP is the most commonly assumed configuration since it is easily explored; 
the signal typically leaves multiple leptons and large missing $E_{T}$ in the final states.
The exclusion limits set by ATLAS and CMS are shonw in Figure \ref{fig::Introduction::LHCLimitEWKino}.
About upto $600\gev$ of NLSP mass is excluded for cases with large NLSP-LSP mass splitting, and $150-250\gev$ for small splitting.  

\begin{figure}[h]
  \centering
    \subfig{0.504}{figures/Introduction/ATLAS_EWgaugino_WZh_36.pdf}{Constraints set by ATLAS Run2.}
    \subfig{0.456}{figures/Introduction/CMS_EWgaugino_WZh_36.pdf}{Constraints set by CMS Run2.}
    \caption{Up-to-date constraints on direct EW gaugino production with decays via $W/Z/h$ set by (a) ATLAS \cite{ATLAS_SUSY_EW2L3L} and (b) CMS \cite{CMS_SUSY_EWK2L3L}.
      The summary plots are referred from \cite{ATLAS_SUSY_PublicResult} (ATLAS) and \cite{CMS_SUSY_PubResult} (CMS).
    }
    \label{fig::Introduction::LHCLimitEWKino}
\end{figure}
The wino-LSP scenario is explored using a strikingly different approach. Since the mass splitting between NLSP wino-chargino and the wino-LSP is extremely compressed ($150 \sim 160\mev$), wino-chargino retains $O(\mathrm{ns})$ of moderately long lifetime, resulting in a characterstic disappearing track signature where a traveling chargino track stops halfway in the tracker due to the decay into a soft pion. The results from ATLAS (Run2) and CMS (Run1) are given in Figure \ref{fig::Introduction::LHCLimitWino}.
The exlusion runs upto $300-500\gev$ in wino mass at the lifetime (or the NLSP-LSP mass splitting) predicted by MSSM. \\

\begin{figure}
  \centering
    \subfig{0.5}{figures/Introduction/ATLAS_SUSY_LLDT_36.pdf}{Constraints set ATLAS by Run2.}
    \subfig{0.46}{figures/Introduction/CMS_SUSY_LLDT_Run1.pdf}{Constraints set CMS by Run2.}
    \caption{Constraints on the wino-LSP scenario set by (a) ATLAS \cite{ATLAS_SUSY_LLDT} and (b) CMS \cite{CMS_SUSY_LLDT_Run1}.}
    \label{fig::Introduction::LHCLimitWino}
\end{figure}
Although motivated by in light of naturalness, 
almost no constraint is set for direct higgsino production so far by LHC, 
due to the marginal production cross-section ($\sim 1/4$ of that of the wino production) as well as the experimentally challenging small NLSP-LSP splitting generally predicted in case of higgsino LSP.
%\footnote{This does not mean LHC gives no constraints for the higgsino LSP scenario, since it is possible to have NNLSP production decaying into LSP, which should be constrained by the }
The strongest limit on direct higgsino production is still held by LEP2.
The limit is shown in Figure \ref{fig::Introduction::LEP2_chargino_comb}, where upto $\sim 90\gev$ of LSP mass is excluded.

\fig[80]{Introduction/LEP2_chargino_comb.pdf}
{Exclusion limit on direct production of higgsino pairs set by LEP2. Combined result from all the four experiments is shown \cite{LEP2LowDMChargino}.} 
{fig::Introduction::LEP2_chargino_comb}


%\section{Constraint from Indirect Search Experiments}
%\paragraph{Flavor}
% ud occilation?suppression???light squark??mixing?? or light squark?????????

%\paragraph{Proton Decay}
% R-parity conservation?imply

%\paragraph{Limit on long-lived gluino from cosmology}
% Super Long-lived gluino?????			




%%%%%%%%%%%%%%%%%%%%%%%%%%%%%%%%%%%%%%%%%%%%
\clearpage
\section{Target SUSY Scenario and the Search Strategy} 
\subsection{Target SUSY Scenario}
To summarize the discussion above, 
thesis focuses on the MSSM scenarios where:
\begin{itemize}
 \item Squarks are all heavy $(>3\tev)$.
 \item Allow the higgs mass fine tuning at order of $10^{-3}$.
 \item LSP is neutralino.
 \item Loosely respect the observed DM relic (Eq. (\ref{eq::obsDMRelic})).
\end{itemize}

The targeted experimental signature is the pair production of gluinos (Figure \ref{fig::Introduction::gluinoPairProd}) with the mass of $800\gev-2\tev$. Although the seach is inclusively carried out with no particular assumption on the mass spectra, a special attention will be made for the case of $\dmcn = 20\gev \sim 30\gev$ motivated by the well-tempered neutralino DM scenario. \\

%\footnote{Bino DMのco-annihilation scenarioとしてgluinoとbino LSPのcompressed modelを考えるものもあるが、この場合はgluinoがlong-livedになるためdisplaced vertex signatureで見ることになり, このthesisの範疇にはない.}

%%%
\fig[120]{Introduction/gluinoPairProd}
{Feynmann diagrams for tree-level gluino pair production in LHC \cite{gluinoSquarkLHC}.}
{fig::Introduction::gluinoPairProd}
%%%

\clearpage
\subsection{The Strategy of Decay Chain Based Search} \label{sec::Introduction::strategy}
Though minimality is still one of our guiding philosophy, 
given that the most straightforward (and a bit too much constrained) models has already been largely excluded so far,
it is sensible to move towards inclusiveness nowadays.
%依然として最もgeneralでminimalなscenarioを探す.
%ただしRun1で最も"都合のいい''SUSY model/scenarioが死んだことを踏まえて, less model orientedなgeneral searchをやるのがいいだろう
The most ambitious extreme would be interpreting into MSSM which accommodates $>100$ parameters in the most general form.
This sounds way too unrealistic to constrain, however it is also true that most of the MSSM parameters only affect the spins or decay branchings of SUSY particles, rather than kinematics i.e. they do not change the signal acceptance. On the other hands, kinematics of SUSY signatures are dominantly determined by SUSY mass spectra. Therefore once a full decay chain is specified, it is very often the case where only a few masses are the relevant parameters to constrain. In other words, setting the cross-section upper limit on each decay chain and mass spectra is no less general than considering the full parameter space of the MSSM
\footnote{This is equivalent to admit our search has no sensitivity in determining model parameters other than masses. }
.\\

Furthermore, placing an upper limit on particular a decay chain $A\ra B$ is essentially equivalent to setting a mass limit on following model called ``simplified model'' where:
\begin{itemize}
\item Br$(A \ra B)$ is 100$\%$.
\item Masses of the particles appearing in the decay chain are treated as free, while the other MSSM parameters are fixed to an arbitrary values. For instance, in LHC analysis, the EW gaugino mixing is usually set so that NLSP and LSP become wino- and bino-dominant respectively. 
\end{itemize}
This simplified modelinterpretation has already been widely employed in the past searches particularly in LHC, however the coverage over decay chains and mass spectra has been far from complete. For instance in the gluino searches, only a few the most simple decay chains mentioned above are published, which often makes one hard to figure out actual gluino limit. This thesis is however going to attempt a radical patch to this problem by extending the interpretatoin to all the decently different simplified models in terms of gluino decay chains and gluino/EW gaugino masses. Following sub-sections are dedicated to illustrate how to find a minimal set of such models. \\

% よって、終状態ごとにupperlimitをつけるのは以下で定義するsimplified modelに対してlimitをつけるのとほぼ同義になる:
% 一応squark/EW gaugino mixingはhelicity angleを通じてkinamaticsに影響を与えるが, gluino decayの場合かなり限定的。
% (footnote) stop/EW gauginoでは大きく影響する場合がある. なので適当なstop mixing, wino NLSP - bino LSPにfixして話を進めるのはダメなときがある
%MSSMの100 dimension parameter space を2-3D times 110にreduce.

%%%%%%%%%%
%\clearpage
\subsection{Target Gluino Decay Chains} \label{sec::Introduction::targetModels}
%Gluinos are generated in pair under the assumption of R-parity conservation.
Under the decoupled squarks scenario, gluino always decays 3-body; 2 SM quarks and a EW gaugino via heavy virtual squarks:
\begin{align}
\tg \ra 
  \begin{cases}
    (u\bar{d}, c\bar{s}, t\bar{b}) \times (\tilde{\chi}_{1,2}^{-}) \nn \\
    (d\bar{u}, s\bar{c}, b\bar{t}) \times (\tilde{\chi}_{1,2}^{+})  \nn \\
    (u\bar{u},d\bar{d},s\bar{s},c\bar{c},b\bar{b},t\bar{t}) \times (\tilde{\chi}_{1-4}^{0}). \nn 
  \end{cases}
\label{eq::gluinoDecays}
\end{align}
Including the subsequent EW gaugino decays, it leads to an enormous number of final states.
However kinematically some of them are approximately equivalent which can be merged or trimmed. 
For instance, since the experimental acceptance is nearly equal between light quark flavors ($u,d,s,c$), 
they can be merged into a single simplified model where gluino has equal decay branches into $u,d,s,c$. \\

In addition, the four higgsino (or two wino) states can be regarded as a single state since their masses are highly compressed each other 
\footnote{The splitting will be rarely greater than $50\gev$ even when all $M_1$, $M_2$ and $\mu$ are at the same mass leading to the maximum mixing.} 
leading the same kinematics. 
The mass spectra can be eventually reduced into the three scenario schematized in Figure \ref{fig::Introduction::signal_massConfig}: 
\begin{itemize}
\item ``direct'' decay in which gluino directly de-excites into LSP,
\item ``1-step'' decay with one intermediate EW gaugino state,
\item ``2-step'' decay in which gluino decays via two resolved intermediate EW gauginos mass states.
\end{itemize}

%%%%%%%%%%
\begin{figure}[h]
  \centering
    \subfig{0.21}{figures/Introduction/massConfig_direct.pdf}{}
    \subfig{0.32}{figures/Introduction/massConfig_no_higgsino.pdf}{}
    \subfig{0.39}{figures/Introduction/massConfig_with_higgsino.pdf}{}
    \caption{ 
Illustration of possible gluino decay paths under various scenario of the mass spectra. 
(a) All the EW gauginos are heavier than gluino except the LSP (gluino decay: direct).
(b) One of the EW gauginos (bino, wino, higgsino) is heavier than gluino 
%\footnote{Most likely higgsino is decoupled, since bino and wino are often considered lighter than gluino based on the gaugino mass unification regime (Eq. (\ref{eq::gauginoMass_621}))}
while the other EW gauginos are lighter (gluino decay: direct or 1-step).
(c) All the EW gauginos are below gluino mass (gluino decay: direct or 1-step or 2-step).
\label{fig::Introduction::signal_massConfig} }
\end{figure}
%%%%%%%%%%

As for the scenario (c) in Figure \ref{fig::Introduction::signal_massConfig}, MSSM parameter scans demonstrate that the probability of 2-step decays are generally much lower than that of direct or 1-step decays, except for some of the cases where each of the intermediate masses are aligned with relatively equal distance. Given the much more complexity of 2-step decays on top of that, we confine our scope within the direct and 1-step decays in this study. \\

For subsequent EW gaugino decays, charginos are always assumed to emit on-shell or off-shell $W$-boson, while there are two options for neutralino decays i.e. via $Z$ or $h$. The decays into slepton is ignored here, majorly for convenience sake of restricting the number of final states, however with a few justifications; 1) under the regime of sfermion mass unification (Sec. \ref{sec::Introduction::SUSYMassGUT}), slepton masses are in the same order of squark masses which are assumed to be decoupled here ; 2) when respecting the observed DM relic abundance, the mass splitting between next-lightest neutralino and the LSP becomes generally small (typically $<50\gev$) which slepton mass can hardly happen to be just in between. \\
\footnote{There are some exceptional cases where such mass heirarchy is actually motivated e.g. scenarios oriented to explain the g-2 anomaly.}

With all these consideration, the targeted gluino decay chains are reduced into Table \ref{tab::Introduction::gluinoDecay2}
with corresponding Feymann diagrams shown in Figure \ref{fig::Introduction::gluinoDecay_summary}. 
\tab{c|l}{
\hline
Direct decay (3) & $\tg \ra (q\bar{q},b\bar{b},t\bar{t}) \tilde{\chi}_{1}^{0}$  \\
\hline
\hline
1-step decay (8) & $\tg \ra (q\bar{q}', t\bar{b}(b\bar{t})) \,\,\, \tilde{\chi}_{1}^\mp, \,\,\,\,\,\,\, \tilde{\chi}_{1}^{\mp} \ra W^\mp \tilde{\chi}_{1}^{0} $ \\
                 & $\tg \ra (q\bar{q}, b\bar{b},t\bar{t}) \,\,\,  \tilde{\chi}_{2}^{0}, \,\,\,\,\, \tilde{\chi}_{2}^{0} \ra Z\tilde{\chi}_{1}^{0}$ \\
                 & $\tg \ra (q\bar{q}, b\bar{b},t\bar{t})  \,\,\, \tilde{\chi}_{2}^{0}, \,\,\,\,\, \tilde{\chi}_{2}^{0} \ra h\tilde{\chi}_{1}^{0}$ \\ 
\hline
}
{Summary of targeted gluino decay chains. The number in the pharenthese indicates the numbers of chains in the categoty.}
{tab::Introduction::gluinoDecay2}


\fig[150]{Samples/signalModels/gluinoDecay_summary.pdf}
{Target gluino decay chains.}
{fig::Introduction::gluinoDecay_summary}


The full decay chains of pair produced gluinos become increasingly complicated: 11 symmetric decays (two gluinos experience the same decay chains), 55 symmetric decays (two gluinos experience different decay chains).
In total, 66 decay chains are identified as the candidate for the targets. \\

%%%%%%%%%%
\subsection{Target Signal Models for 1-lepton Final State}
In LHC, analyses are conventionally divided based on number of hard leptons in the final state, since either signal kinematics and the background strategy are drastically different. In gluino decays, when ignoring the decays into sleptons, leptons are always generated via decays of W/Z/H bosons. Therefore, giving their small leptonical branching ratio, 0-lepton or 1-lepton final state are the most promising channels for inclusive search, while 2/3-leptons final states are more specialized in specific types of scenarios such as long-chain multi-step gluino decays involving a large number of W/Z/H bosons. \\

This thesis focuses on the final state with exactly one lepton. 
After excluding the decay chains with marginal branching ratio into final state with exactly 1-lepton,
45 decay chains are selected as the benchmark models for the thesis.
The full list are shown in Table \ref{tab::Introduction::modelsBV} - Table \ref{tab::Introduction::models3B}, with the naming convention for each decay chain defined as:
\begin{align}
  \mbox{Model name} & := [aaXX][bbYY]  \nn \\
  aa,bb & = \mbox{``QQ'',``BB'',``TT'',``BT''}  \nn \\
  XX,YY & = \mbox{``N1'',``C1'',``N2Z'',``N2H''}
%  \label{eq::gridNameConvention}
\end{align}
where each sub-block ($[aaXX]$,$[bbYY]$) denotes the full chain of one gluino decay, corresponding to either of the topology shown in Figure \ref{fig::Introduction::gluinoDecay_summary}. \\

Since the signal regions will be segmented based on the numberof $b$-tagged jets,
the benchmark models are further categorized (BV/BT/3B) based on the number of expected $b$-quarks in the final state. The reference models for each $b$-categories are respectively chosen as \textbf{QQC1QQC1},\textbf{QQC1BTC1} and \textbf{TTN1TTN1} for BV, BT and 3B (Figure \ref{fig:Introduction::refModels}), which will be used as the reference in designing signal regions and other various studies. The Feynman deagrams for the reference models are illustrated in Figure \ref{fig:Introduction::refModels}.\\

Note that simplified models with asymmetric gluino decays are not realistic due to the assumption of $100\%$ branching ratio, since there is always branching to symmetric decays when asymmetric decays happen. However, this is in fact a more user friendly presentation since it provides the upper limit on the acceptance for each decay chain so that the compatibility between observation and models can be easily tested using it, which is not the case in case of an interpretation with a realistic models where many sorts of decays are mixed.

%Although the 0-lepton final state often is more advantageous in terms of coverage of models and signal acceptance, 1-lepton addresses some unique merits over the 0-lepton channel as following:
%- 0Lよりclean. QCDとかnon-collisionみたいな変なBGの心配をしなくていい
%- 2Lよりはsignalが多い. 2LをCRにできる.  (0Lで1LをCRとすると結構contamiが無視できない?)  あとCRがpure
%- 0Lとのsynergyを取ることでSUSYとのconsistencyについて議論できる

% compressedとかだとそこに書いてあるよりmultiplicity減るっていうことにも一応言及

\begin{figure}[h]
  \centering
    \subfig{0.32}{figures/diagrams/GG_QQC1QQC1.pdf}{\textbf{QQC1QQC1}}
    \subfig{0.32}{figures/diagrams/GG_QQC1BTC1.pdf}{\textbf{QQC1BTC1}}
    \subfig{0.32}{figures/diagrams/GG_TTN1TTN1.pdf}{\textbf{TTN1TTN1}}
    \caption{ Feymann diagrams for the reference models. }
    \label{fig:Introduction::refModels}
\end{figure}
%%%%%%%%%%%%%%%%%%



%%%%%%%%%%%%%%%%%%%%%%%
\clearpage


\tab{c c c c c c}{
  \hline
  \textbf{1-step decay}  & $n_{\mathrm{J}}$ &  $n_{\mathrm{B}}$ & Br(1L)/Br(0L) & Br(1L)/Br(2L)  & det sim.?  \\ \hline
  \hline
  QQN1QQC1 & 5.5 & 0.0 & 0.33 & - & \\ \hline
  \textcolor{red}{\textbf{QQC1QQC1}} & 7.0 & 0.0 & 0.67 & 6 & \checkmark   \\ \hline
  QQC1QQN2Z & 7.3 & 0.3 & 0.35 & 3.86 & \checkmark   \\ \hline
  \hline
}
{Target models with no b-jets at tree level (BV models). The average jet multiplicity ($n_{\mathrm{J}}$) and b-jet multiplicity ($n_{\mathrm{B}}$) are calculated based on number of quarks and b-quarks appearing in the final state. The PDG values \cite{PDG2016} are referred for branching ratio of top, W/Z/h bosons. $``\checkmark''$ specifies the models with the final result derived using the samples with the fast detector simulation (ATLFast 2 ~\cite{atlfast}), while the others are with emulated truth samples.}
{tab::Introduction::modelsBV}



\tab{c c c c c c}{
  \hline
  \textbf{Direct decay}  & $n_{\mathrm{J}}$ &  $n_{\mathrm{B}}$ & Br(1L)/Br(0L) & Br(1L)/Br(2L)  & det sim.?  \\ \hline
  \hline
  QQN1TTN1 & 7.0 & 2.0 & 0.67 & 6 & \\ \hline
  \hline
  \textbf{1-step decay}  & $n_{\mathrm{J}}$ &  $n_{\mathrm{B}}$ & Br(1L)/Br(0L) & Br(1L)/Br(2L)  & det sim.?  \\ \hline
  \hline
  QQC1QQN2H & 7.4 & 1.1 & 0.46 & 7.07 & \checkmark   \\ \hline
  QQN1BTC1 & 7.0 & 2.0 & 0.67 & 6 & \\ \hline
  QQN1TTN2Z & 8.8 & 2.3 & 0.68 & 3.30 & \\ \hline
  \textcolor{red}{\textbf{QQC1BTC1}} & 8.5 & 2.0 & 1.0 & 3 & \checkmark   \\ \hline
  QQC1BBN2Z & 7.3 & 2.3 & 0.35 & 3.86 & \\ \hline
  QQC1TTN2Z & 10.3 & 2.3 & 1.02 & 2.34 & \\ \hline
  QQN2ZTTN2Z & 10.7 & 2.6 & 0.7 & 2.31 & \\ \hline
  BBN1QQC1 & 5.5 & 2.0 & 0.33 & - & \\ \hline
  BTC1QQN2Z & 8.8 & 2.3 & 0.68 & 3.30 & \\ \hline
  TTN1QQC1 & 8.5 & 2.0 & 1.0 & 3 & \\ \hline
  TTN1QQN2Z & 8.8 & 2.3 & 0.68 & 3.30 & \\ \hline
  \hline
}
{Target models with 1 or 2 b-jets at tree level (BT models). 
Definition of $n_{\mathrm{B,J}}$, branching and $``\checkmark''$ are the same as Table \ref{tab::Introduction::modelsBV}.}
{tab::Introduction::modelsBT}



\tab{c c c c c c}{
  \hline
  \textbf{Direct decay}  & $n_{\mathrm{J}}$ &  $n_{\mathrm{B}}$ & Br(1L)/Br(0L) & Br(1L)/Br(2L)  & det sim.?  \\ \hline
  \hline
  BBN1TTN1 & 7.0 & 4.0 & 0.67 & 6 & \\ \hline
  \textcolor{red}{\textbf{TTN1TTN1}} & 10 & 3.9 & 1.33 & 2 & \checkmark   \\ \hline
  \hline
  \textbf{1-step decay}  & $n_{\mathrm{J}}$ &  $n_{\mathrm{B}}$ & Br(1L)/Br(0L) & Br(1L)/Br(2L)  & det sim.?  \\ \hline
  \hline
  QQN1TTN2H & 8.9 & 3.1 & 0.79 & 3.64 & \\ \hline
  QQC1BBN2H & 7.4 & 3.1 & 0.46 & 7.07 & \\ \hline
  QQC1TTN2H & 10.4 & 3.1 & 1.12 & 2.34 & \\ \hline
  QQN2ZTTN2H & 10.8 & 3.4 & 0.8 & 2.56 & \\ \hline
  QQN2HTTN2H & 10.8 & 4.3 & 0.91 & 2.70 & \\ \hline
  BBN1BTC1 & 7.0 & 4.0 & 0.67 & 6 & \\ \hline
  BBN1TTN2Z & 8.8 & 4.3 & 0.68 & 3.30 & \\ \hline
  BBN1TTN2H & 8.9 & 5.1 & 0.79 & 3.64 & \\ \hline
  BBN2ZTTN2Z & 10.7 & 4.6 & 0.7 & 2.31 & \\ \hline
  BBN2ZTTN2H & 10.8 & 5.4 & 0.8 & 2.56 & \\ \hline
  BBN2HTTN2H & 10.8 & 6.3 & 0.91 & 2.70 & \\ \hline
  BTC1QQN2H & 8.9 & 3.1 & 0.79 & 3.64 & \\ \hline
  BTC1BTC1 & 10 & 4.0 & 1.33 & 2 & \\ \hline
  BTC1BBN2Z & 8.8 & 4.3 & 0.68 & 3.30 & \\ \hline
  BTC1BBN2H & 8.9 & 5.1 & 0.79 & 3.64 & \\ \hline
  BTC1TTN2Z & 11.8 & 4.3 & 1.35 & 1.75 & \\ \hline
  BTC1TTN2H & 11.9 & 5.1 & 1.46 & 1.70 & \\ \hline
  TTN1QQN2H & 8.9 & 3.1 & 0.79 & 3.64 & \\ \hline
  TTN1BTC1 & 10 & 4.0 & 1.33 & 2 & \\ \hline
  TTN1BBN2Z & 8.8 & 4.3 & 0.68 & 3.30 & \\ \hline
  TTN1BBN2H & 8.9 & 5.1 & 0.79 & 3.64 & \\ \hline
  TTN1TTN2Z & 11.8 & 4.2 & 1.35 & 1.75 & \\ \hline
  TTN1TTN2H & 11.9 & 5.1 & 1.46 & 1.70 & \\ \hline
  TTN2ZQQN2H & 10.8 & 3.4 & 0.8 & 2.56 & \\ \hline        
  TTN2ZBBN2H & 10.8 & 5.4 & 0.8 & 2.56 & \\ \hline
  TTN2ZTTN2Z & 13.7 & 4.5 & 1.36 & 1.55 & \\ \hline
  TTN2ZTTN2H & 13.8 & 5.4 & 1.47 & 1.53 & \\ \hline
  TTN2HTTN2H & 13.8 & 6.2 & 1.58 & 1.49 & \\ \hline
  \hline
}
{Target models with 3 or more b-jets at tree level (3B models).
Definition of $n_{\mathrm{B,J}}$, branching and $``\checkmark''$ are the same as Table \ref{tab::Introduction::modelsBV} and \ref{tab::Introduction::modelsBT}.}
{tab::Introduction::models3B}

%%%%%%%%%%%%%%%%%%%%



%%%%%%%%%%%%%%%%%%%%%%%%%%%% 



\if 0
\clearpage
\section{Structure of the thesis}
This dissertation is organized as follows:
%\textbf{\underline{Introductory Overviews}}
\begin{itemize}
%\item Chapter \label{sec::Introduction} firstly provides the motivation and theoretical background about gluino search, and outlines the goal of the study.
\item Chapter \ref{sec::Detector} overviews the experiment apparatus used in the study; the LHC and the ATLAS detector. 
\item Chapter \ref{sec::objDef} describes the off-line algorithms utilized for reconstruction and identification of particles and hadronic jets. 
\item Chapter \ref{sec::Samples} describes the setup of the MC simulation employed in the analysis. 
%\end{itemize}
%
%\textbf{\underline{Main Sections}}
%\begin{itemize}
\item Chapter \ref{sec::SRdefinition} describes the pre-selection and designed signal regions.
\item Chapter \ref{sec::BGestimation} includes comprehensive discussion on the background estimation method and its validation.
\item Chapter \ref{sec::Uncertainties} overviews the evaluated systematic uncertainties associated with background estimation and signal modeling. 
\item Chapter \ref{sec::Result} summerizes the results and resultant limits. 
\item Chapter \ref{sec::Discussion} discusses the impact of the obtained result.
\item Chapter \ref{sec::Conclusion} close the thesis with concluding remarks.
\end{itemize}

\fi

