\chapter{Introduction} \label{sec::preface}
\fig[60]{Introduction/cat.jpg}
{ 
What is \textit{matter} made of? 
The answer is quarks and leptons \cite{cat}.
}
{fig::cat}
The origin of \textit{things} has been an ultimate question of human beings since civilization.
The pursue by modern science dates from the establishment of the atomic theory, 
and the quest towards the fundamental element has been continuously proceeded,
together with the development of theory of forces acting between the them.
A great milestone is set in recent years by the theoretical and experimental establishment of the Standard Model (SM).
With this, today we understand a kitten (Figure \ref{fig::cat}) is made of $O(10^{27})$ atoms (bounded by the electric force);
each of which accommodates a nucleus and electrons inside;
and the a nucleus consists of a bunch of protons and neutrons (bounded by the strong force);
which are made of a couple of quarks and numerous virtual partons.
No any further elements are needed.
With the quarks and leptons that form matter, gauge bosons mediating forces and a Higgs boson feeding the masses,
the SM succeeds in explaining the origin of \textit{matter} that we are familiar with and most of the phenomena at microscopic level. \\
%
%
%Note that this does not means
%さてここで終わりということでもよかったかもしれないが、我々は貪欲だったため身の回りのmatter・現象についても気になり始めた。
%例えばSMでは説明つかないunfamiliar matterが発見されたことによって, SMを拡張する必要が生じた
%またSMはいい理論だが究極の理論ではない。e.g. gravity. こっちのためにも新しい理論を求めて人類はまだ頑張り続けてる
%
%またSMでは重力はカバーされていない。GRという素晴らしい重力の理論は存在するが、我々は物事ごとに違う説明する方法は好まないでここまでやってきたことを考えると
%究極的には統一されることが一番望ましい。
%

Though it could be the end of the story, we are greedy enough to move forward.
The SM is a nice theory, however still not perfect either in terms of the unaccountability for some observational facts such as dark matter or neutrino masses, 
or its hurdles towards an ultimate theory such as the unification of the three gauge forces (``grand unification'') and inclusion of gravity. 
The SM is awaiting for next breakthrough, 
just as the paradigm shift that the quantum mechanics and the relativistic theory brought about on classical mechanics, 
and both theoretical and experimental approaches towards beyond-the-SM (BSM) has been actively going on.
This work is done in context of the pursue, which is an experimental search of new particles predicted by super-symmetry (SUSY) theory,
 using the proton-proton collisions in the Large Hadron Collider (LHC) at a center-of-mass energy of $\sqrt{s}=13\tev$.
SUSY is know as one of the most motivating BSM frameworks, in which a set of partner particles of the SM are introduced, 
providing good dark candidates as well as prospect towards grand unification and further.
Since it works particularly well when the SUSY particles are at a few TeV in their masses, the direct production in LHC is feasible in a number of scenarios.
This thesis describes the search for unprobed heavy gluino with the mass around $1\tev\sim2\tev$, using the updated LHC dataset and advanced analysis technique.

\paragraph{Organization of the dissertation} \mbox{}  \\
The first part involves the introductory chapters describing the theoretical background, experimental apparatus, and toolkits used in the analysis:
\begin{itemize}
\item Chapter \ref{sec::Introduction} provides the backgrounds necessary to motivate the study in the thesis and the status-quo of gluino search, as well as an analysis of motivated SUSY scenarios today.
\item Chapter \ref{sec::Detector} overviews the experiment apparatus used in the study; the LHC and the ATLAS detector. 
\item Chapter \ref{sec::objDef} describes the reconstruction algorithms used for particle identification and jet clustering. 
\item Chapter \ref{sec::Samples} outlines the method of event simulation and the setup employed in the analysis. 
\end{itemize}
%
\noindent The main body of the study is given by following chapters:
\begin{itemize}
\item Chapter \ref{sec::SRdefinition} describes the event selection of signal regions in which gluino is enhanced and background suppressed in an optimal manner.
\item Chapter \ref{sec::BGestimation} comprehensively discusses the background estimation procedure and its validation.
\item Chapter \ref{sec::Uncertainties} overviews and evaluates the systematic uncertainties associated with background estimation and signal modeling. 
\end{itemize}
%
\noindent The result are presented and analyzed in following chapters:
\begin{itemize}
\item Chapter \ref{sec::Result} summarizes the results and resultant limits. 
\item Chapter \ref{sec::Discussion} discusses the impact of the obtained result.
\item Chapter \ref{sec::Conclusion} close the thesis with concluding remarks.
\end{itemize}


%\footnote{It has been addressed since the old Greek; some said it is water (Thales \cite{}); fire (); air (Anaximenes); soil (Xenophanes); fire (Heracraytus); or atoms (Democritus \cite{democtritusAtom}).}
