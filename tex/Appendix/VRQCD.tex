%%%%%%%%%%%%%%%%%%%%%%%%%%%%%% QCD %%%%%%%%%%%%%%%%%%%%%%%%%%
\clearpage
\subsection{Validation of Fake-Lepton Background Esimation} \label{sec::BGestimation::VRQCD}
The ``fake'' background (defined in Table \ref{tab::BGestimation::BGclass}) involves two types of component:
\begin{itemize}
\item Multi-jets background \\
This includes the QCD di-jet events and full-hadronic decays of $\vjets$, which is ignored in the estimation since it is supposed to be negligible after requiring one signal lepton and $\met>250$ in the events, based on the MC study and the past Run2 ATLAS 1-lepton analyses \cite{strong1L_ICHEP2016_CONF}\cite{strong1L_3p2fb_paper}.
However, it always needs caution since the impact could be fatal once it turns to contribute because of its huge cross-section. 

\item Leptonical $V$+jets \\
The dominating component in the ``fake'' events is $W\ra\tau\nu$ and $Z\ra\nu\nu$. 
These are included in the $V$+jets MC thus accounted in the estimation via the kinematical extrapolation method.
However, the modeling on the fake rate of lepton candidates is very complicated and the MC description is known to be sometimes unreliable (often under-estimating).
\end{itemize}
Therefore, a data-driven validation is motivated to make sure the estiamtion is not really under-estimating,
using a set of specific validation regions (VRs-QCD) listed in Table \ref{SRdefinition::regionDef2J} - \ref{SRdefinition::regionDef3B}.  
VR-QCDs are defined by inverting the isolation requirement on the final state lepton with respect to the SRs. 
The abundance of ``fake'' components is enhanced by about a factor of $5\sim10$ with respect to the SRs, due to the high ``fake'' rejection power of the isolation requirement. 
Figure \ref{fig::BGestimation::VRsQCD} presents the result. 
Note that the normalization factors (Figure \ref{fig::BGestimation::fittedSFs}) are applied for $\wjets$ and the top background.
No particular data excess is found, implying a reasonable MC modeling on fake lepton. 

\fig[130]{BGestimation/VRQCD/VRsQCD.pdf}
{Observed yields (black dots) and background expected by the kinemtatical extrapolation (colored stack) in VRs-QCD. 
 The error bands include both statistical and systematic (the same theory systematics is quoted as the SRs) uncertainty.
 }
{fig::BGestimation::VRsQCD}

