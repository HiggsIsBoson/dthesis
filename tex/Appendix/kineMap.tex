\subsection{Kinematics vs SUSY Masses} \label{sec::App::kineMap}
The trend of the kinematical variables over the mass grids are shown in Figure \ref{fig::SRdefinition::kineMap_QQC1QQC1_x12}-\ref{fig::SRdefinition::kineMap_TTN1TTN1}. The color scale (z-axis) indicates the mean of the distribution of the variable that the the signal events in the mass point result in. The examples of three \textbf{QQC1QQC1} grids ($\xhalf$, $\varx$, $\DMth$) and one \textbf{TTN1TTN1} grid are shown. 
While the variables related to transverse momenta of outgoing particles such as $\meffInc$, $\lepPt$ and $\met$ simply scale with the mass splitting, the other variables (e.g. aplanarity, $\metOverMeff$ etc.) typically vary depending on the relative mass spilitting, the cuts in which are therefore helpful for segmenting the signal region towers.
These are used in deciding the initial cuts for the signal regions optimization. \\


%\clearpage
\noindent \underline{\textbf{QQC1QQC1 ($\xhalf$ grid)}} \\
The initial selection for tower \textbf{2J} and \textbf{6J} are decided based on the profile below.
Lepton $\pt$ and $\met/\meffInc$ are found to be helpful for \textbf{2J} targeting the diagonal region,
while requiring hard kinematics as well as high $\apl$ is useful for scenarios with high-mass gluino which \textbf{6J} is targeting. 
%
\begin{figure}[h]
  \centering
    \subfig{0.32}{figures/SRdefinition/kineMap/GG_symQQC1_x12_meff.pdf}{$\meffInc$}
    \subfig{0.32}{figures/SRdefinition/kineMap/GG_symQQC1_x12_met.pdf}{$\met$}
    \subfig{0.32}{figures/SRdefinition/kineMap/GG_symQQC1_x12_lep1Pt.pdf}{Lepton's $\pt$}
    \subfig{0.32}{figures/SRdefinition/kineMap/GG_symQQC1_x12_mt.pdf}{$\mt$}
    \subfig{0.32}{figures/SRdefinition/kineMap/GG_symQQC1_x12_metOverMeff.pdf}{$\met/\meffInc$}
    \subfig{0.32}{figures/SRdefinition/kineMap/GG_symQQC1_x12_LepAplanarity.pdf}{$\Apl$}
    \caption{ 
      Mean value of kinematical variables after the 1-lepton pre-selection (Table \ref{tab::SRdefinition::Preselection}) as function of $\mG$ and $\mLSP$ for the reference model \textbf{QQC1QQC1} in the $\xhalf$ grid.
    }
    \label{fig::SRdefinition::kineMap_QQC1QQC1_x12} 
\end{figure}
 
\clearpage
\noindent \underline{\textbf{QQC1QQC1 ($\varx$ grid)}} \\
The initial selection for tower \textbf{High-x} is designed based on the profile below.
Note that the low-$x$ region of this grid is not dedicatedly targeted by the analysis since the assumption of nearly massless LSP is supposed to be obsolete, and instead the \textbf{Low-x} tower targets the DM-oriented grids in which LSP is generally massive.
High leptonic activity including $\mt$ and relatively low jet activity is favored for the mass region that is targeted by \textbf{High-x}. Moderated $\apl$ cut ($\sim 0.02$) is also found helpful.
%
\begin{figure}[h]
  \centering
    \subfig{0.37}{figures/SRdefinition/kineMap/GG_symQQC1_varx_meff.pdf}{$\meffInc$}
    \subfig{0.37}{figures/SRdefinition/kineMap/GG_symQQC1_varx_met.pdf}{$\met$}
    \subfig{0.37}{figures/SRdefinition/kineMap/GG_symQQC1_varx_lep1Pt.pdf}{Lepton's $\pt$}
    \subfig{0.37}{figures/SRdefinition/kineMap/GG_symQQC1_varx_mt.pdf}{$\mt$}
    \subfig{0.37}{figures/SRdefinition/kineMap/GG_symQQC1_varx_metOverMeff.pdf}{$\met/\meffInc$}
    \subfig{0.37}{figures/SRdefinition/kineMap/GG_symQQC1_varx_LepAplanarity.pdf}{$\Apl$}
    \caption{
      Mean value of kinematical variables after the 1-lepton pre-selection (Table \ref{tab::SRdefinition::Preselection}) as function of $\mG$ and $x \, (:= \dmc/\dmg)$ for the reference model \textbf{QQC1QQC1} in the $\varx$ grid.
    }
    \label{fig::SRdefinition::kineMap_QQC1QQC1_varx} 
\end{figure}

\clearpage
\noindent \underline{\textbf{QQC1QQC1 ($\DMth$ grid)}} \\ 
The initial selection for tower \textbf{Low-x} is designed based on the profile below.
Note that only scenarios with massive LSP and with non-degenerated gluino and LSP are dedicatedly followed.
A combination of soft lepton and high jet activity is generally favored.
\begin{figure}[h]
  \centering
    \subfig{0.38}{figures/SRdefinition/kineMap/GG_symQQC1_dM30_meff.pdf}{$\meffInc$}
    \subfig{0.38}{figures/SRdefinition/kineMap/GG_symQQC1_dM30_met.pdf}{$\met$}
    \subfig{0.38}{figures/SRdefinition/kineMap/GG_symQQC1_dM30_lep1Pt.pdf}{Lepton's $\pt$}
    \subfig{0.38}{figures/SRdefinition/kineMap/GG_symQQC1_dM30_mt.pdf}{$\mt$}
    \subfig{0.38}{figures/SRdefinition/kineMap/GG_symQQC1_dM30_metOverMeff.pdf}{$\met/\meffInc$}
    \subfig{0.38}{figures/SRdefinition/kineMap/GG_symQQC1_dM30_LepAplanarity.pdf}{$\Apl$}
    \caption{ 
      Mean value of kinematical variables after the 1-lepton pre-selection (Table \ref{tab::SRdefinition::Preselection}) as function of $\mG$ and $\mLSP$ for the reference model \textbf{QQC1QQC1} in the $\DMth$ grid.
    }
    \label{fig::SRdefinition::kineMap_QQC1QQC1_dM30} 
\end{figure}

%%%%%%%%%%%
\clearpage
\noindent \underline{\textbf{TTN1TTN1}} \\ 
The initial selection for tower \textbf{3B} is decided based on the profile below.
Note that the diagonal region where the off-shell tops result in soft $b$-jets that are not tagged in the analysis is not covered by \textbf{3B} but by \textbf{2J}. While almost all variables display nice separtion power for scenarios with large mass splitting,
$\apl$, $\mindPhiFourJet$ and topness are found to be the key variables for moderated  mass splitting cases.
\begin{figure}[h]
  \centering
    \subfig{0.32}{figures/SRdefinition/kineMap/GG_symTTN1_x12_nJet30.pdf}{Jet multiplicity ($\pt>30\gev$)}
    \subfig{0.32}{figures/SRdefinition/kineMap/GG_symTTN1_x12_nBJet30.pdf}{$b$-jet multiplicity ($\pt>30\gev$)}
    \subfig{0.32}{figures/SRdefinition/kineMap/GG_symTTN1_x12_meff.pdf}{$\meffInc$}
    \subfig{0.32}{figures/SRdefinition/kineMap/GG_symTTN1_x12_met.pdf}{$\met$}
    \subfig{0.32}{figures/SRdefinition/kineMap/GG_symTTN1_x12_lep1Pt.pdf}{Lepton's $\pt$}
    \subfig{0.32}{figures/SRdefinition/kineMap/GG_symTTN1_x12_mt.pdf}{$\mt$}
    \subfig{0.32}{figures/SRdefinition/kineMap/GG_symTTN1_x12_LepAplanarity.pdf}{$\Apl$}
    \subfig{0.32}{figures/SRdefinition/kineMap/GG_symTTN1_x12_min_dPhi_4j.pdf}{$\mindPhiFourJet$}
    \subfig{0.32}{figures/SRdefinition/kineMap/GG_symTTN1_x12_topNess.pdf}{Topness}
    \caption{ 
      Mean value of kinematical variables after the 1-lepton pre-selection (Table \ref{tab::SRdefinition::Preselection}) as function of $\mG$ and $\mLSP$ for the reference model \textbf{TTN1TTN1}. 
    }
    \label{fig::SRdefinition::kineMap_TTN1TTN1} 
\end{figure}
 

