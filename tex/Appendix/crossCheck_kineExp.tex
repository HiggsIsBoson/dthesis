%\clearpage
\subsection{Comparison between Kinematical Extrapolation and Object Replacement}  \label{sec::crossCheckBGestm}
For the cross-check of ``di-leptonic'' backgorund estimation, 
it is also estimated by the kinematical extrapolation method and compared with the nominal prediction  provided by the object replacement.
Figure \ref{fig::crossCheck::pull_KineExtp_ObjRep} show the direct comparison in the predicted yields in SRs (VRs) between the two methods.
Note that the same normalization factors are obtained in Sec. \ref{sec::BGestimation::kineExtp::result} are used in case of the kinematical extrapolation. 
While they are found to be consistent in most of the regions, object replacement tends to predict a bit higher yields in high-$\mt$ regions where kinematical extrapolation is known to under-estimate due to the potential MC mis-modeling $\mt$ cut-off as mentioned in Sec. \ref{sec::BGestimation::unblindedVRs}. \\

Figure \ref{fig::crossCheck::SRPulls} show the comparison of total background expectation and the pulls in VRs (SRs) between the two methods, where fairly consistent results are obtained.
On the other hand, the source of uncertainty on the estimation is in contrast between the two estimation methods.
Figure \ref{fig::crossCheck::systSummary} summarizes the total uncertainty and the breakdown for the two methods. 
Though the total uncertainty is comparable with each other, the leading source is theory uncertainty for the kinemtical extrapolation while the CR statistics is the dominant source for the object replacement.


\clearpage
%%%%%%%%%%%%%%%%% KineExtp vs objRep pulls %%%%%%%%%%%%%%%%%%%%%%%%%%%%%%%%%%%%%%%
\begin{figure}[h]
  \centering
    \subfig{0.84}{figures/Result/comp_kine_objRep/objRep_VRs.pdf}{Comparison of predicted ``di-leptonic'' yields in \textbf{VRs} between the two methods.}
    \subfig{0.84}{figures/Result/comp_kine_objRep/objRep_SRs.pdf}{Comparison of predicted ``di-leptonic'' yields in \textbf{SRs} between the two methods.}
    \caption{ 
    (Top pannels) Estimated yields of the di-leptonic components (``Mis-Reco'', ``Mis-ID'' and $``\ell\tau_h''$) in (a) VRs or in (b) SRs by the nominal method (pink) and the kinematical extrapolation method (blue). Error bars included both statistical and systematic uncertainty. 
    (Bottom pannels) Pull between the two estimations.
    }
    \label{fig::crossCheck::pull_KineExtp_ObjRep}
\end{figure}


\clearpage
%%%%%%% Pulls %%%%%%%%%%%%%%%%
\begin{figure}[h]
  \centering
    \subfig{0.85}{figures/BGestimation/PullVRsSRs/histpull_VRs.pdf}{Estimated yields and pulls in \textbf{VRs} with the \textbf{nominal} estimation (same as Figure \ref{fig::BGestimation::VRPulls}).}
    \hfill
    \subfig{0.85}{figures/BGestimation/PullVRsSRs_kineExpOnly/histpull_VRs.pdf}{Estimated yields and pulls in \textbf{VRs} with the \textbf{alternative} estimation in which all the background are estimated by the kinematical extrapolation.}
    \caption{ 
      Total background expectationand the observed pulls in VRs. The dashed band represents uncertainty on the background estimation including both statistical and systematical uncertainty.
    }
    \label{fig::crossCheck::SRPulls}
\end{figure}


\clearpage
%%%%%%% Pulls %%%%%%%%%%%%%%%%
\begin{figure}[h]
  \centering
    \subfig{0.85}{figures/BGestimation/PullVRsSRs/histpull_SRs.pdf}{Estimated yields and pulls in \textbf{SRs} with the \textbf{nominal} estimation (same as Figure \ref{fig::Result::SRPulls}).}
    \hfill
    \subfig{0.85}{figures/BGestimation/PullVRsSRs_kineExpOnly/histpull_SRs.pdf}{Estimated yields and pulls in \textbf{SRs} with the \textbf{alternative} estimation in which all the background are estimated by the kinematical extrapolation.}
    \caption{       
      Total background expectationand the observed pulls in SRs. The dashed band represents uncertainty on the background estimation including both statistical and systematical uncertainty.
    }
    \label{fig::crossCheck::SRPulls}
\end{figure}



\clearpage
%%%%%%% Summary of post fit uncertainties %%%%%%%%%%%%%%%%
\begin{figure}[h]
  \centering
    \subfig{0.84}{figures/Result/systSummary/syst_summary_SRs.pdf}{Uncertaintiy associated with the \textbf{nominal} estimation (same as Figure \ref{fig::Result::systSummary})}
    \subfig{0.84}{figures/Result/systSummary_kineExpOnly/syst_summary_SRs.pdf}{Uncertainty associated with the \textbf{alternative} estimation in which all the backgroud is estimated by the kinematical extrapolation method.}
    \caption{ 
      Post-fit systematic uncertainty with respective to the expected yield in the signal regions. 
      Total systematics uncertainty is shown by the filled orange histogram, and the breakdowns are by dashed lines.
    }
    \label{fig::crossCheck::systSummary}
\end{figure}


