\chapter{Experiment Apparatus:  The ATLAS Detector at the LHC} \label{sec::Detector}

\section{The Large Hadron Collider} 
The Large Hadron Collider (LHC) \cite{LHC} is a 27 km long circular proton accelerator embedded underground of the Geneva area.
%and French villages overviewing the Mont Blanc \cite{MontBlanc}.
It is designed to collide protons at a center-of-mass energy of $\sqrt{s} = 14 \tev$, at the four detector cites (ATLAS\cite{ATLAS_exp}, CMS\cite{CMS}, ALICE\cite{ALICE} and LHC$b$\cite{LHCb}) built on the accelerator ring. ATLAS and CMS are general purpose detectors designed to study a vast range of physics programs, while LHC$b$ and ALICE are specialized in studying $b$-hadrons and heavy-ion collisions respectively. 
The operation started in 2010, offering proton-proton (pp) collisions at a center-of-mass energies of $7\tev$ and $8\tev$ with $4.7\ifb$ and $20.3\ifb$ of integrated luminosity until 2012 (Run1). The center-of-mass energies has been almost doubled to $13\tev$
in the runs starting from 2015 (Run2). The LHC has also delivered lead-ion (Pb-Pb) collisions with a center-of-mass energy of $\sqrt{s_{NN}} = 2.76\tev$ and proton-lead (p-Pb) collisions with $\sqrt{s_{NN}} = 5.02\tev$. \\
% In a chamber, 
% 

% The protons are then accelerated (段階的に) sequently through the booster accelerator (linac, PS, SPS) and finally injected into LHC. 
% In LHC

% number of pileup
The acceleration of protons with various steps: 
Protons are firstly seeded from hydrogen gas, by blowing the electrons off the hydrogen atoms using electric field.
They are injected in the linear accelerator LINAC2 accelerated upto $50\mev$, and sent to the Proton Synchrotron Booster (PSB) with being accelerated up to an energy of $1.4\gev$. 
The subsequent accelerator is the Proton Synchrotron (PS) elevating the energy of the protons to $25\gev$, and injecting them into the Super Proton Synchrotron (SPS). After being accelerated to $450\gev$ in SPS, the protons finally enter the two LHC pipes running the beam oppositely each other. The whole acceleration chain is illustrated in Figure \ref{fig::Detector::LHC}. \\

\figNoH[140]{Detector/LHC_schematic.pdf}
{The LHC and associated booster accelerator system. \cite{LHCPhoto}}
{fig::Detector::LHC}

The LHC accelrator consists of octant-shaped 2.45 km arcs with 1232 superconducting magnets located at the curves, providing 8.33T of magnetic field to bend the proton trajectory.
%They are operated at 1.9K cooled by liquid-helium, 
In total, 39 bunch-trains can be filled simultaneously at the design condition, and 2808 bunches per beam are brought to collision in the LHC. Each bunch contains about $10^{11}$ protons. 
The beam bunches are collided with a crossing angle of 285 mrad. 
The peak luminosity amounts upto $L = 0.7-1.4 \times 10^{34} \mathrm{cm}^{2} \mathrm{s}^{-1}$ in the 2015-2016 runs, as shown in Figure \ref{fig::Detector::DAQ} (a). \\

Due to the high frequency of collisions and the dense proton bunches, multiple proton collisions can take place within the same bunch crossing, referred as ``pile-up''. 
% to maximizing the chance of interesting reaction 
The average pile-up $\mu$, defined as the mean number of interactions per bunch crossing,  
has been evolved according to the peak luminosity increase. 
The $\mu$ profile in Run2 is shown in Figure \ref{fig::Detector::DAQ} (b) where $\mu = 20\sim40$ is typically achieved. 

%%%%%%%%%
\begin{figure}[h]
  \centering
    \subfig{0.485}{figures/Detector/peakLumi2016.png}{}
    \subfig{0.485}{figures/Detector/mu_2015_2016.png}{}
    \caption{  (a) Peak luminosity evolution in 2016 runs \cite{DAQ2016}, and (b) the pile-up profile obtained in 2015-2016 runs  \cite{lumiPubResult}.
      \label{fig::Detector::DAQ}
    }
\end{figure}
%%%%%%%%%%

%and are categorized as in-time or out-of-time pile-up. In-time pile-up events are caused by additional interactions of protons in the same bunch collision. The out-of-time pile-up occurs when traces from an event in a different bunch-bunch are recorded. 

\clearpage



%%%%%%%%%%%%%%%%%%%%%%%%%%%%%%%%%%%%%%%%%%%%%%%%%%%%%%%%%
\section{The ATLAS Detector}
\subsection{Overview}
ATLAS (A Toroidal LHC ApparatuS) is a general purpose detector, 
aiming to a wide range of physics programs from precision measurements to the energy frontier experiments, 
through a dedicated measurement of particles produced in the $pp$ collisions. 
The detector extends over 44m in width and 25m in height weighing 7000 tons in total, covering the interaction point (IP) by a cylindrical barrel and two end-caps, 
achieving a nearly full solid angle coverage. 
%
The cut-away image is shown in Figure \ref{fig::Detector::ATLAS_whole}. \\

\noindent The purposes of the detector are mainly two-fold:
\begin{itemize}
\item Identification of particle species
\item Determination of particle's energy and momentum.
\end{itemize}
based on the two complementary concepts of measurement:
\begin{itemize}
\item Fast measurement to provide triggers
\item Precision measurement for physics analyses.
\end{itemize} 
To achieve those functionalities, following sub-detectors are arranged in a designed order from the inner to the outer with respect to the IP:

\begin{itemize}
\item \textbf{An inner detector to identify and measure electrically charged particles.} \\
Charged particle can easily interact with materials by ionizing the molecules inside. 
The path of flight can be ``imaged'' as a track by measuring the position of ionization. 
In ATLAS, a complex of discrete layers of silicon sensors and a continuously volumed gas chambers are placed in the innermost part. 
Applying a magnetic field in it, particle momenta can be determined by quantifying the curvature of the bent trajectories. 

\item \textbf{Calorimeters to measure the energy of electron, photon and hadrons.} \\
Electrons and photons traveling inside materials above critical energy lose their energy through electromagnetic showering; photons create $e^+e^-$ pairs; electron/positron spew bremsstrahlung photons; the daughters are multiplicated by coninuting the recursive splitting ending up in a particle shower. 
Most of the energy are absorbed after traversing about 20 radiation lengths ($X_0$) of material. 
%
Hadrons also cause similar cascade reactions. 
The shower branch evolves by interacting with nucleus in the material via strong interaction, meanwhile produced $\pi_0$s promptly decay into two photons which shower electromagnetically. 
The resultant shower is a combination of a long hadronic shower and small local EM clusters in it. 
Electromagnetic and hadronic calorimeters are set as the outer layers of the trackers.\\

\item \textbf{A muon spectrometer to measure the muons penetrating the detector.} \\
Among all the particles that interact with material, muons are only exception that do not seriously deposit the energy in the calorimeter. 
This is due to the fact that muons happen to in the mass giving the minimum EM interaction with material (Minimum Ionizing Particle; MIP) at $O(10\mev) \sim O(1\tev)$ scale.
This is actually a lovely coincident for human being (or poor particle physicists), 
since they can be easily identified i.e. particles punching through the calorimeter are automatically muons. 
The muon spectrometer located the outermost serves for identifying such muons as well as measuring the tracks together with the information from the inner tracker.
\end{itemize}

The presence of non-interacting particles such as neutrinos and hypothetical new particles can be indirectly detected through the transeverse momentum imbalance, due to the momentum conservation in the transverse direction in each collision.
This is referred to missing $\ET$ ($\met$), 
\footnote{The ``$\ET$'' in the name is due to a historical reason; it used to be calculated only using calorimeter deposits, which is now actually outdated
%; also, momentum and energy effectively give no differences for most of detected particles in LHC, since their energy typically overwhelms the masses.
}
defined by the negative of the vectoral sum of transverse momentum of all detected particles.


In the following sections, each of the sub-detector system is overviewed, comprehensively based on references \cite{ATLAS_exp} and \cite{ATLAS_TDR}. \\

\figNoH[110]{Detector/figures_AtlasDetectorLabelled.png}
{Full-body view of the ATLAS detector \cite{ATLAScosmicPerf}. The geometry is completely forward-back symmeric.}
{fig::Detector::ATLAS_whole}


%%%%%%%%%%%%%%%%%%%%%%%%%%%%%%%%%%%%%%%%%%%%%%%%%%%%%%%%%
\subsection{Coordinate System}
For referencing the position of the detector as well as the orientation of particles, a right-handed Cartesian coordinate system is defined where the interaction point is the origin; the $x$-axis pointing to the center of the LHC ring; the $y$-axis and $z$-axis are accordingly the direction of sky or the beam direction respectively. Polar angle $\theta$ and azimuthal angle $\phi$ are defined by the cylindrical representation $(\theta,\phi,z)$: $\theta$ ranges from 0 to $\pi$ with respect to the $z$-axis, and $\phi$ runs from $-\pi$ to $\pi$ from the $x$-axis. The two end-caps in the ATLAS detector are referred as ``A-side'' and ``C-side'', 
\footnote{Reportedly named after the direction towards (Geneva) Airport and the Charie's Pub in St. Genis-Pouilly from the ATLAS respectively.}
corresponding to the position of positive and negative coordinate in the $z$-axis. \\

It is the unfortunate fate for hadron colliders that particles generated by collisions are usually highly boosted along $z$-axis, 
since the energy of the initial interacting partons inside the hardons are asymmetric. 
From this point of view, a set of variables with Lorentz-invariant nature are introduced for describing the momentum or position for such particles. 
In particular, it is useful to define the transverse component of variables, such as transverse momentum $\pt := p\sin{\theta}$ or transverse energy $E := E\sin{\theta}$. 
The advantage over the use of $p$ or $E$ is obvious that they do express the intrinsic hardness of the particles in the center-of-mass frame of the reaction, 
and also that the vectoral sum of all particles conserves before and after the collision. \\

Similarly, pseudo-rapidity $\eta$ is defined below, serving as the coordinate of polar angle:
\begin{align}
\eta := -\ln \left( \tan{\frac{\theta}{2}} \right).
\end{align}
It has two practical advantages over $\theta$; the difference in pseudo-rapidity between particles $\Delta\eta$ are invariant against the boost towards $z$-direction. 
\footnote{
This is true when the particles are massless, which is approximately valid given that the boos along $z$-axis is sourced by the momentum of order of the beam energy.}
; $\eta$ has an effectively wider dynamic range upto a very forward region thanks to the finer measure, where $\theta$ suffers from the degeneracy in $\cos{\theta}\sim 1$, 
thus more convenient in expressing the orientation of forward particles.

Angular distance between two particles are commonly expressed by $R$, defined as: 
\begin{align}
\Delta R := \sqrt{(\Delta\eta)^2+(\Delta\phi)^2}.
\end{align}

%\figNoH[110]{Detector/ACside.pdf}{.}{fig::Detector::ACside}


%%%%%%%%%%%%%%%%%%%%%%%%%%%%%%%%%%%%%%%%%%%%%%%%%%%%%%%%%
\subsection{Inner Detectors}
The inner detector (ID) is placed the inner-most of the ATLAS detector, designed to measure the tracks of charged particles, 
as well as precisely determining the position of vertices of the hardest scattering in interest.
It consists of a silicon tracker (the pixel detector and the semiconductor tracker ;SCT) at the inner radii,
and the Transition Radiation Tracker (TRT) for continuous tracking at the outer radii. 
The detector arrangement is illustrated in Figure \ref{fig::Detector::innerDetector_xsec} and Figure \ref{fig::Detector::innerDetector}.
The outer radius is surrounded by the central solenoid, providing a magnetic field of 2T along the $z$-axis,
to bend the tracks traveling inside the ID volume.
%
As a general requirement, ID has to contain material as less as possible, to avoid disturbing the measurement downstream by the energy loss. 
Figure \ref{fig::Detector::IDmaterial} shows the total material profile of the ID as function of $|\eta|$. 
The material volume is suppressed below $2.5$ radiation length and 1 nucleus interaction length, which is low enough compared with energy dropped in the calorimeter.


%%%%%%%%%
\figNoH[110]{Detector/innerDetector_xsec.pdf}
{Cross-section of the ATLAS inner detectors \cite{ATLAS_exp}.}
{fig::Detector::innerDetector_xsec}


\figNoH[110]{Detector/ATLAS_innerDetector.jpg}
{Cut-away view of the ATLAS inner-detector \cite{ATLAS_exp}.}
{fig::Detector::innerDetector}

%%%%%%%%%
\begin{figure}
  \centering
    \subfig{0.48}{figures/Detector/radLength_ID.pdf}{}
    \subfig{0.48}{figures/Detector/intLength_ID.pdf}{}
    \caption{ Simulated material profile of whole ID in unit of (a) electro-magnetic radiation length and (b) nucleus interaction length \cite{ATLAS_exp}.
      The peak in $|\eta|\sim 1.5$ corresponds to the barrel-end-cap transition area through which service cables travel.
      \label{fig::Detector::IDmaterial} }
\end{figure}
%%%%%%%%%%



%%%%%%%%%%%%%%%%%%%%%%%%%%%%%%%%%%
\paragraph{The silicon trackers: Pixel and SCT} 
The detection principle of silicon detector is based on the electron-hole pair creation induced by a traverse of a charged particle.
%Thanks to the narrow band gap ($1.7 \mathrm{eV}$). 
Those electron-hole pairs are then inhaled by the bias voltage applied on the sensor, and transferred into an electric signal. 
%Silicon is particularly advantageous in 
The choice of silicon is largely due to its radiation hardness sufficient to endure the enormously high radiation around the IP. 
On the other hand, the performance (e.g. noise level, gain) is relatively sensitive to temperature, therefore they are kept in low temperature ($-5 \sim 0\cdeg$) during the operation.  \\

The pixel detector is the unit of layers of pixelated silicon sensors located closest to the IP of all the detector component. 
%The two-dimensional segmentation of the sensors gives space points without any ambiguities. 
Oxygen enriched $n$-in-$n$ silicon semiconductor is used for the sensors.
Four cylindrical layers are placed in the barrel at the radial distance of 31 mm $\sim$ 122.5 mm with respect to the IP, 
and 3 disk layers cover each side of the end-cap, providing an acceptance of $|\eta|<2.5$. 
The innermost layer in the barrel provides the highest precision referred as the ``insertable b-layer'' (IBL) installed during the long shutdown between Run1 and Run2.
%playing a prominent role in identifying the secondary vertex of late decaying particles ($\tau$, $b$-hadrons etc.). 
The pixels are in the 50 $\times$ 250 $\um$ granularity in the IBL, and 50 $\times$ 400 $\um$ in the other layers. 
The resolution is purely determined by the pixel size. 
A spatial resolution of $4\um$ and $115\um$ is achieved along the radial and beam $z$-direction respectively, 
by combining the hit information from the four layers. \\

The SCT is located outside of the pixel detector. 
The sensors are made by single-sided $p$-on-$n$ silicon semiconductors.
The strips of barrel SCT aligning along the $z$-axis with $80 \um$ pitch, giving a precision position in the $r-\phi$ plane. 
A slight angle stereo (40 mrad) alternated by layers is applied to the arrangement, 
providing decent $z$-position determination in addition. 
The barrel region is surrounded by four layers, while nine discs are placed in each end-cap.
The intrinsic resolution is $17\um (580 \um)$ in $r-\phi (z)$ direction respectively.
%Each sensor has 768 strips readout individually.
The strips in the end-cap SCT are aligned in a mesh in terms of $x-y$, capable of 3D position determination together with the $z$-coordinate of the disks.

%The total area coverage of silicon amounts up to $61m^2$, with 6.2 million readout channels.
% track 200um離れてれば分離可能
% thickness 285um

%%%%%%%%%%%%%


%基本的な流れ
%-前後とのつながり
%-一番小さいunitと検出動作原理, 読み出し, operation volrageとtemp, 素材への要求とか
%-それらがどう配置されてるか, geometryの特徴, hitの数とか
%-performance


%%%%%%%%%%%%%%%%%%%%%%%%%%%%%%%%%%%%%%%%%
%The pixel detector [1-9] is designed to provide a very high-granularity, high-precision set of measurements as close to the interaction point as possible. The system provides three precision measurements over the full acceptance, and mostly determines the impact parameter resolution and the ability of the Inner Detector to find short-lived particles such as B hadrons and τ lep- tons. The two-dimensional segmentation of the sensors gives space points without any of the ambiguities associated with crossed strip geometries, but requires the use of advanced electron- ic techniques and interconnections for the readout. The readout chips are of large area, with in- dividual circuits for each pixel element, including buffering to store the data while awaiting the level-1 trigger decision. Each chip must be bump-bonded to the detector substrate in order to achieve the required density of connections. In addition, the chips must be radiation hardened to withstand over 300 kGy of ionizing radiation and over 5×1014 neutrons per cm2 over ten years of operation. The system contains a total of 140 million detector elements, each 50 μm in the Rφ direction and 300 μm in z, which are invaluable for the task of pattern recognition in the crowded environment of the LHC.
%The system consists of three barrels at average radii of ~4 cm, 10 cm, and 13 cm, and five disks on each side, between radii of 11 and 20 cm, which complete the angular coverage. The system is designed to be highly modular, containing approximately 1 500 barrel modules and 700 disk modules, and uses only one type of support structure in the barrel and two types in the disks.
%The pixel modules are designed to be identical in the barrel and the disks. Each module is 62.4 mm long and 21.4 mm wide, with 61 440 pixel elements read out by 16 chips, each serving an array of 24 by 160 pixels. The output signals are routed on the sensor surface to a hybrid on top of the chips, and from there to a separate clock-and-control integrated circuit. The modules are overlapped on the support structure in order to give hermetic coverage. The thickness of each layer is expected to be about 1.7% of a radiation length at normal incidence.
%
%%%%%%%%%%%%%%%%%%%%%%%%%%%%%%%%%%



%%%
%Each module consists of four . On each side of the module, two detectors are wire-bonded together to form 12.8 cm long strips. Two such detector pairs are then glued together back-to-back at a 40 mrad angle, separated by a heat transport plate, and the electronics is mounted above the detectors on a hybrid. The readout chain consists of a front-end amplifier and discriminator, followed by a binary pipeline which stores the hits above threshold until the level-1 trigger decision. The end-cap modules are very similar in construction but use tapered strips, with one set aligned radially. To obtain optimal η-coverage across all end-cap wheels, end-cap modules consist of strips of either ~12 cm length (at the outer radii) or 6-7 cm length (at the innermost radi- us).
%%%
%end-cap diskは3種類 (inner/midle/outer)のdimension
%pn type  bias voltage


%%%%%%%%%%%%%%%%%%%%%%%%%%%%%%%%%%
\clearpage
\paragraph{Trasition radiation tracker (TRT)}
% gas mixture
% TRTの原理もうちょい
% radiation at the boundary between two di-electric material s
TRT is a gaseous detector designed for tracking particles as well as identifying the species using the characteristic transition radiation.
The detector is filled with 4mm-diameter straw tubes in which xenon-based active gas is confined.
Ionized secondary electrons are collected by the 30 $\um$-diameter gold-plated tungsten-Rhenium anode wire in the center of each straws.
73 layers of aligned straw tubes are arranged in the barrel, and 160 layers in the end-cap sectors. 
The tube length is 144 cm (37 cm) in the barrel (end-cap) region. 
The barrel tubes are arranged in parallel along the beam pipe, with 7 mm of interval between layers.
The intrinsic position resolution per straw is about 130 $\um$.
A traverse of charged particle fires 36 straws on average. \\

Transition material is inserted between the straws.
19 $\um$-diameter polypropylene fibers are used in barrel, and 15 $\um$-thick polypropylene radiator foils isolated by a polypropylene net are set for the end-caps.
Transition radiation can address unique sensitivity in particle identification, particularly to  $e/\pi$ separation, 
since the intensity is sensitive to incident particle's velocity (proportional to $\gamma=E/m$) rather than the energy or momentum. 
Given that the signal of transition radiation typically yield more amplitude than the nominal gas ionization, two different thresholds are set in the TRT ; 
the lower threshold to collect the signal of ionization caused by a particle traverse; the high threshold defining the signal of transition radiation. 
The high threshold is carefully designed so that only electrons in the typical range of energy ($0.5\gev - 150 \gev$) can fire while pions are inert to it. \\

Figure \ref{fig::Detector::TRTPerf} shows the $\gamma$-dependence of high threshold rate, demonstrating a good separation of particles in the electron-like momentum and pion-like momentum. \\

%%%%%%%%%%%%%%
\figNoH[170]{Detector/TRTPerf.pdf}
{TRT high threshold rate as function of Lorentz factor ($\gamma=E/m$) of incident particles \cite{TRTPub}.
The $\gamma$ scale of typical pions and electrons are labeled aside. The left (right) plot corresponds to the rate in barrel (end-caps) respectively.}
{fig::Detector::TRTPerf}


%%%%%%%%%%%%%%%%%%%%%%%%%%%%%%%%%%
\paragraph{Combined Tracking Performance}
The combined tracking performance has been validated via the measurement of cosmic muons \cite{ATLAScosmicPerf}. 
The resolution for a single muon track is obtained as function of muon transverse momentum:  
% Typically, three pixel lay- ers and eight strip layers (four space points) are crossed by each track. A large number of track- ing points (typically 36 per track) is provided by the straw tube tracker (TRT) [1-8],
%The combination of two complentary technologies provides very robust pattern recognition and high precision in
\begin{align}
\frac{\sigma_{\pt}}{\pt} = 1.6\% \oplus \frac{0.053\%}{\gev}\times \pt.  \label{eq::Detector::perf_ID}
\end{align}
%ATLAS_ID_performance_cosmic

\mbox{} \\


%%%%%%%%%%%%%%%%%%%%%%%%%%%%%%%%%%%%%%%%%%%%%%%%%%%%%%%%%
\subsection{Calorimetery}
The ATLAS calorimetery located outside the ID is composed of the electromagnetic calorimeter (EM calorimeter), the hadronic calorimeter (HC), and the forward calorimeter. 
The whole view is given by Figure \ref{fig::Detector::calo}.
%The idea is ... to absorb the incident particles and  inside and calculate the energy deposit . intenseな物質とのhard interactionでcascade状シャワー作って
The calorimeters employ two detector thechnologies:
\begin{itemize}
\item Liquid-Argon sampling calorimeter (LAr) with alternately sandwiching the lead absorber layers and the sensor layer filled with liquid-argon.
\item ``Tile calorimeter'' consisting of the sensor layers with scintillator tiles and steel absorber.
\end{itemize}
The detector technology and the spatial segmentation in each pseudo-rapidity coverage are summarized in Table \ref{fig::Detector::caloSpec}. 
Thanks to the fast response of the readout, calorimeter can provide the function of trigger, based on the fast processing of particle identification and the energy measurement using the information of individual showers, as detailed in Sec. \ref{sed::Detector::TDAQ}. \\

\fig[120]{Detector/ATLAS_calorimeter.jpg}
{Cut-away view of the ATLAS calorimetery \cite{ATLAS_exp}.}
{fig::Detector::calo}

\clearpage
\figNoH[140]{Detector/caloSpec.pdf}
{Summary of partition and geometry of the ATLAS calorimetery \cite{ATLAS_TDR}.}
{fig::Detector::caloSpec}
\clearpage


%%%%%%%%%%%%%%%%%%%%%%%%%%%%%%%%%%
\paragraph{Electromagnetic calorimeter}
%構造
%
% Basicなrequirementとしては, 
% Desnse material is preferred for absorber in general 
%e-piのPIDの観点からEGはECAL内で終わってほしいし、hadronはHCALでなるべくシャワーをスタートさせてほしい。そういうわけでradiation lengthの差は大きければ大きいほどよい。
The basic unit of LAr calorimeter consists of a gap filled with liquid argon (gap width: 1.1-2.2mm) generating the ionized electrons, a copper-kapton electrodes to collect the ionized charge, 
and a steel-claded lead absorber layer to develop the EM shower (layer width: 1.13-1.53mm). A bias voltage of 2000V between the electrodes and the absorbers is applied, achieving the drift time of 450ns. The readout signal is amplified by a pre-amplifier, and shaped into a 13 ns widthsignal pulse by a bi-polar shaper managing the 25 ns width bunch crossings.
The detector is maintained at a constant temperature of $88K$ by cryostats surrounding the barrel EM calorimeter. \\
%The readout current is tranfered into pulse signal ... 読み出しの話

%%%%%%%%%%
\figNoH[100]{Detector/LAr_cell.pdf}
{Geometry of barrel LAr sampling layers. 
Position resolution is addressed by the innermost sampling layer by the highest $\eta\times\phi$ granularity of $0.0031\times0.098$,
and the energy measurement is mainly provided by the second layer with the largest volume.
The third layer standing behind in the plot is the tail catcher providing information of the shower profile.
    \cite{ATLAS_TDR}.}
{fig::Detector::LArcell}
%%%%%%%%%%

The geometry and cell segmentation varies between barrel and end-cap depending on the desired function.
Figure \ref{fig::Detector::LArcell} illustrates the segmentation in the barrel ECM. 3 sampling blocks are placed along shower with different $\eta\times\phi$ segmentation.
The first sampling layer has the finest $\eta\times\phi$ granularity ($0.0031\times0.098$) identifying the precise angular position of the incident particle. The second sampling addresses the largest volume ($16X_0$) containing the most of shower in which the energy is mainly measured. The third sampling layer is intended to measure the tail of EM showers, providing information about the longitudinal profile together with the other layers. The layer units are arranged in an accordion geometry, which is the characteristic to the barrel ECM, designed to be fully hermitic in terms of angular acceptance. 
%end-cap EM calorimeterの特徴
%Only the latter two sampling layers are 
In order to compensate the upstream energy loss, a presampling layer is additionally placed in front of the first layer of the EM calorimeter for both barrel and the end-caps.
%with the thickness of 11 mm (5 mm) for barrel (end-cap).
The total thickness amounts to $>22X_0$ in the barrel and  $>24X_0$ in the end-cap, which can fully accommodate the EM showers of photons or electrons in an energy of upto a few TeV.
The transition region between the barrel and end-caps ($1.37 < |\eta| <1.52$) is dedicated to detector services and therefore not fully instrumented. \\

%needed for EG reconstruction -> fine segmentation , METの計算
%ATLASは特にpi0->gamgam veto(?)とconverted photonのIDに力入れてるので特にangular resolution に優れたECALになってる。
%3D showers are reconstructed by units of topo clustering

%sampled -> measured bipolar pulse shape transfer?
%pileup contribution should be subtracted all the time->negative biasing

The designed resolution is given in Eq. \ref{eq::Detector::perf_calo} \cite{ATLAS_LAr_TDR}:
\begin{align}
\frac{\sigma_E}{E} = \frac{10\%}{\sqrt{E}} \oplus \frac{17\%}{E} \oplus 0.7\%.
\label{eq::Detector::perf_calo}
\end{align}
\noindent The energy resolution for the off-line objects can be further improved through the dedicated calibration exploiting the full detail of the shower and information from the other detector. 
%This high pointing resolution is the characteristic advantage of the ATLAS EM calorimeter, which vastly benefits the particle identification (photon, high energy tau etc.) as well as MET reconstruction. \\



%%%%%%%%%%%%%%%%%%%%%%%%%%%%%%%%%%
\paragraph{Hadronic Calorimeter}
The ATLAS hadronic calorimeter consists of the barrel Tile HC ($|\eta|<1.7$) and end-cap LAr HC.
%The Tile HC is the sampling calorimeter composed by a periodic units of plastic scintillators tiles and steel absorber.
Barrel Tile HC is segmented into three sections, the central barrel section ($|\eta|<1.0$) and the two extended barrel sections ($1.0<|\eta|<1.7$), using different channel dimensions. 
There are three sampling layers along the shower development with the thickness of 1.5$\lambda$, 4.1$\lambda$ and 1.8$\lambda$ for barrel, and 1.5$\lambda$, 2.6$\lambda$ and 3.3$\lambda$ for extended barrel respectively. Figure \ref{fig::Detector::caloCell} (a) schematizes one module in the Tile HC. Generated scintillation photons are read out by the photo-multiplier tubes equipped at the ends of the module via wavelength shifting fibers. 
The end-cap HC is the sampling calorimeter with liquid-argon sensor layers and copper absorber. The choice of material is dominantly based on the durability against the extremely high radiation flux in the forward region. \\
 
The intrinsic resolution of barrel Tile HC and end-cap LAr HC for an individual hadron jet is given by Eq. \ref{eq::Detector::perf_calo} \cite{ATLAS_Tile_TDR}: \\
% simulated ? measured?
%The gap between the sections is 4cm, in which the service cables for ID travel through.
\begin{align}
& \frac{\sigma_E}{E} = \frac{50\%}{\sqrt{E}} \oplus 3\%, \,\,\,\,\,\,\,\,\,\,\,\,  \mbox{(Tile HC)} \\
& \frac{\sigma_E}{E} = \frac{100\%}{\sqrt{E}} \oplus 10\%,  \,\,\,\,\, \mbox{(End-Cap LAr HC)}
\label{eq::Detector::perf_calo}
\end{align}


%%%%%%%%%%%%%%%%%%%%%%%%%%%%%%%%%%
\clearpage
\paragraph{Forward Calorimeter}
A set of LAr calorimeter layers are arranged in a very forward region close to the beam axis covering $3.1<|\eta|<4.9$, 
designed to capture the full content of jets or particles from hard scattering particles from extremely boosted center-of-mass. The location with respect to the adjacent calorimeter systems are illustrated as Figure \ref{fig::Detector::caloCell} (b).
Forward calorimeter is made by three sampling layers in which both functions of EM calorimeter and hadronic calorimeter are integrated; The first layer is with copper absorber working as EM calorimeter, and the later two layers are with tungsten functioning as EM calorimeter. The overlap region with respect to the end-cap HC is deliberated to realize smooth transition. \\


%%%%%%%%%%
\begin{figure}
  \centering
    \subfig{0.375}{figures/Detector/tile_cell.pdf}{}
    \subfig{0.525}{figures/Detector/caloForward.pdf}{}
    \caption{ (a) Illustration of a Tile HC module. (b)  Alignment of each detectors in an end-cap; end-cap LAr EM calorimeter (EMEC); end-cap LAr Hadronic calorimeter (HEC); and the Forward calorimeter (FCal)   ) \cite{ATLAS_exp}.
      \label{fig::Detector::caloCell} }
\end{figure}
%%%%%%%%%%




%%%%%%%%%%%%%%%%%%%%%%%%%%%%%%%%%%%%%%%%%%%%%%%%%%%%%%%%
\clearpage
\subsection{Muon Spectrometer}
Muon spectrometers are located outermost in the ATLAS, consisting of four sub-detectors; Monitored Drift Tube (MDT); Cathode Strip Chamber (CSC); Resistive Plate Chamber (RPC); and the Thin-Gap Chamber (TGC). 
The former two are dedicated to precision measurement of muon tracks and the latter two are to triggering. 
The spectrometer covers the pseudo-rapidity range $ |\eta|< 2.7$ and allows identification of muons with momenta above 3 GeV and precise determination of $\pt$ up to about 1 TeV with $10\%$ momentum resolution.  \\

\fig[100]{Detector/ATLAS_muonDetector.pdf}
{Global view of the ATLAS muon spectrometers \cite{ATLAS_exp}.}
{fig::Detector::muon}


\fig[120]{Detector/muon_xsec2.pdf}
{Cross-section of the ATLAS Muon spectrometer \cite{ATLAScosmicPerf}.}
{fig::Detector::muon_xesc2}

\clearpage
The magnetic field for tracking is sourced by the three pieces of toroidal superconducting magnets i.e. two end-cap toroids and a barrel toroid embedded in the space inside the muon spectrometers. $3.9T$ and $4.1T$ B-field is provided in the barrel and end-cap region respectively. The internal volume of toroidal coils are vacant (``air-core''), in order to reduce the material with which muons experience the multiple scattering. The integrated B-filed profile at the position of MDT is shown in Figure \ref{fig::Detector::BField_MDT}, while the global schematic of the magnet system is given in Figure \ref{fig::Detector::magnet}. \\

\fig[100]{Detector/magnet_captioned.pdf}
{Schematic of the ATLAS magnet system with one central solenoid and 3 toroidals (barrel+2 end-caps) \cite{ATLAS_exp}.}
{fig::Detector::magnet}

%\figNoH[110]{Detector/magnet_photo.pdf}
%{Phono of ATLAS cross-seciotn.}
%{fig::Detector::magnet_photo}

\fig[100]{Detector/Bfield_MDT.pdf}
{ Simulated magnetic field integral provided by a single troid octant, from the innermost MDT layer to the outermost. \cite{ATLAS_exp}.}
{fig::Detector::BField_MDT}
\clearpage



%%%%
%momentum measurement from 3GeV to 1TeV (?), with 10% resolution at 1TeV
% |eta|<1.7 for barrel and 1.6<|eta1<2.7 for end-cap
%toroidal, 8 coil per toroid
%∫B_T d_T: 1.5Tm-5.5Tm for barrel, 1Tm - 7.5Tm for end-cap
%%%%

\paragraph{Monitor Drift Tubes (MDT)}
MDT is a gaseous drift chamber filled with the basic detection elements of 30 mm-diameter aluminum tubes that are covered by a 400 $\um$-thick wall. 
Drifting electrons are absorbed by a 50 $\um$-diameter tungsten-Rhenium wire in the center of a tube with a bias voltage of 3080 V is applied, and read out by a low-impedance current sensitive preamplifier.
The gas mixture is with Ar ($93\%$) and $\mathrm{CO_2}$ ($7\%$), maintaining the maximum drift time of 700 ns. The position resolution by a single wire is about 80 $\um$.
%In order to boost the number of hits per particle passage, the tubes are assembled in a layer, and the layers are stacked along the path of parcitcles flight, forming a muti-layer, as schematized in Figure \ref{}. 
There are three layers of MDT chambers located both in barrel and end-cap, covering a pseudo-rapidity range of $|\eta|<2.0$.
The limitation in the $\eta$-coverage is determined by its maximum durable rate ($150 cm^{-1}s^{-1}$). CSC takes over the role in such forward region.
%% performance


\paragraph{Cathode Strip Chamber (CSC)}
%%読み出し
The CSCs are multi-wire proportional chambers covering the forward region ($|\eta|>2.0$) in the end-caps, providing 2D position of incident particles.
It is operated with a gas mixture of Ar $(80\%)$ and CO2 ($20\%$) and with a bias voltage of 1900 V applied.
The cells are symmetric in terms of the pitch of readout cathodes and the anode-cathode spacing, which is equally set to 2.54 mm.
Since the spatial resolution of the CSCs is sensitive to the inclination of tracks and the Lorentz angle, the chamber is fixed at tilted posture so that tracks originating from the IP become approximately orthogonal to the chamber surface.
%% performance


\paragraph{Resistiv Plate Chamber (RPC)}
The RPCs are digital gaseous detectors specialized in fast timing response for triggering.
They are mechanically mounted on the surface in the barrel MDT, covering the pseudo-rapidity range of $|\eta|>1.05$.
The elementary detection unit is a gas gap filled with non-flammable gas mixture (94.7$\%$ $\mathrm{C_2 H_2 F_4}$, 5$\%$ Iso-$\mathrm{C_4 H_10}$, 0.3$\%$ $\mathrm{SF_6}$). An uniform high electric field ($\sim$ 4900 V/mm) is applied so that the ionized electrons amplitude by themselves via the avalanches. Signals are read out by a metal strip attached on both ends of the gaps, arranged with a pitch of 30 mm $\sim$ 39.5 mm.
The typical spatial and timing resolution achieved by a RPC chamber are 1 cm and 2 ns respectively.

%2 plates for measuring eta, phi position


\paragraph{Thin-Gap Chamber (TGC)}
The TGCs are a special type of multi-wire proportional chambers characterized by the notably small distance between the anode wires and the read out cathode strips (1.4mm).
A quick drain of secondary electrons is achieved by the quenching gas mixture of $\mathrm{CO_{2}}$ ($55\%$) and n-pentan ($45\%$), yielding the timing response of 5 ns. TGCs also contribute to the momentum determination by supplementing the measurement in $\phi$ by MDT.
%A high voltage of 2900V is applied during the operation. 
Three modules are placed per end-cap, covering $1.05<|\eta|<2.7$ by the innermost one and $1.05<|\eta|<2.4$ by the two behind. Trigger is generated using tracks in $1.05<|\eta|<2.4$, while all tracks are subjected to the momentum measurement.







%%%%%%%%%%%%%%%%%%%%%%%%%%%%%%%%%%%%%%%%%%%%%%%%%%%%%%%%
\clearpage
\subsection{Luminosity Detectors}
Luminosity determination is particular important since it provides the reference of normalizing simulated dataset which enables the comparison to data. 
The instantaneous luminosity is calculated by the formula below:
\begin{align}
\mathcal{L} = \frac{\mu n_b f_b}{\sigma},
\end{align}
where $n_b$ is the number of colliding bunches and $f_b$ the frequency of the beam circulation.
$\sigma$ is total fiducial cross-section of $pp$-interaction including both elastic and inelastic scattering, and $\mu$ is the average number of such interaction per bunch crossing. While $\sigma$ is provided by a dedicated calibration (van der Meer scan \cite{VdMScan}) measuring the lateral beam profile using overlapping two beams, $\mu$ is obtained directly by exploiting the rate information from luminosity detectors located in the very forward region nearby the beam pipe. Dedicated calibration and luminosity determination algorithm studied in \cite{LumiMeasurement}.
%Three luminosity detectors contribute to the luminosity measurement:
Two luminosity detectors mainly contribute to the luminosity measurement:
\begin{description}
%\item[BCM] (Beam Conditions Monitor) \\

\item[LUCID] (LUminosity measurements using Cherenkov Integrating Detector) \\
LUCIDs are located at the both ends of the ATLAS detector at a distance of 17m from the IP, covering the pseudo-rapidity range of $5.6<|\eta|<6.0$.
The LUCID detector consists of 16 aluminum tubes filled with $\mathrm{C}_4 \mathrm{F}_{10}$ gas filled inside, 
designed to count the Cherenkov photons kicked out by charged particles flying along the beam axis which are mainly generated by proton-proton inelastic scattering in the IP.
%The measurement is translated into luminosity, which is used for online monitoring of the instantaneous luminosity as well as the beam condition. 
%collcted by the photomultiplier set in the back-end

%array of 20 Cerenkov tubes (?) 
%pp非弾性散乱から来る破片のcharged particleの数を数える


\item[ALFA] (Absolute Luminosity For ATLAS) \\
ALFA is located beyond the ATLAS envelope at $z=\pm 240$ m, sandwiching the beam pipe from top and bottom.
The detectors are composed of 8 scintillating fibers, designed to measure the elastic scattering component of the $pp$-interaction. \\

%小角のpp弾性散乱で飛んで来たpを測る. optical theoremでtotal cross-sectionわかるので、
\end{description}



%%%%%%%%%%%%%%%%%%%%%%%%%%%%%%%%%%%%%%%%%%%%%%%%%%%%%%%%%
\subsection{Trigger and Data Acquisition System}\label{sed::Detector::TDAQ}
While ATLAS enjoys incredibly high collision rate of about 100 MHz (40 MHz beam bunch crossing together with pile-up),
these data cannot entirely read out due to the limitation from data transmission as well as the computation resource.
Luckily or unluckily, most of them are junk QCD reactions resulting in cheap low $\pt$ jets, 
the rate can be drastically suppressed by requiring hard jets, leptons or $\met$ in the events. \\

The ATLAS Trigger and Data Acquisition System (TDAQ) \cite{ATLASTrigger2015} is the data acquisition system handling the trigger and readout. 
The schematic of the readout streams are shown in Figure \ref{fig::Detector::triggerFlow}.
It consists of a two-staged trigger pipeline served by the hardware-based Level-1 Trigger (L1) and the software-based High-Level Trigger (HLT).
The idea is to reject the major trivial QCD events in L1, based on a fast particle reconstruction with coarse resolution, and perform further filtering in HLT using more sophisticated reconstruction and energy measurent benefited by the timing latency that L1 earns. The benchmark of rate suppression is 100 kHz at the end of L1 and down to 1 kHz after the HLT on average. \\
The L1 consists of two independent sub-trigger systems; L1Calo identifying the EM or hadronic clusters in calorimeter and reconstruct primitive jets, electrons, photons and taus (L1 objects) with calibrated energy in EM scale; L1Muon identifying and measuring the tracks in the muon spectrometer designed to accept events with muons. 
The object reconstruction is based on the coarsely segmented blocks of combined detector channel called ``trigger tower'' with $\eta\times\phi$ granularity of $0.1 \times 0.1$. 
%RoIの話も?
$\met$ is also calculated at the L1 stage by the vectoral sum of the calorimeter deposits, referred as L1XE. Trigger accept is issued by the Central Trigger Processors (CTP) when the L1 objects meet certain criteria in terms of $\pt$ threshold and number of objects.  \\

In the HLT, offline-like algorithms are employed to refine the energy of L1 objects, or recover the mis-identified objects (low-$\pt$ muons most typically) by scanning over whole detector. 
This is performed by a set of custom farmwares with a processing time of 0.2s on an average. 
The event triggered by the HLT is subsequently sent to event storage infrastructures outside the ATLAS. 
Figure \ref{fig::Detector::trigger_ratePhys} illustrates the rate of HLT acceptance in 2016 operation. 
The performance of triggers relevant to the analysis is dedicatedly overviewed in Sec. \ref{sec::SRdefinition::trigger}.

%Relevant trigger chains to the thesis is shown as Table \ref{}.
%The $\met$ trigger is used as the main trigger.
%The single-lepton triggers supplement the 

% HLTはofflineにかなり近いparticle reconstruction/identification/calibrationを使う
%
\figNoH[110]{Detector/triggerFlow.pdf}
{The logic of ATLAS trigger system \cite{ATLASTrigger2015}. Trigger detectors have separated readout line for trigger, sending input information for trigger decision to CTP. 
The CTP reconstructs L1 objects and issue a global accept signal relieving the buffered data, once the trigger criteria are satisfied. 
The $(\eta,\phi)$ position of identified trigger object is sent to downstream HLT, in which offline-like software-based triggers run to filter events further.
L1 topological trigger (L1 Toplo) and Fast Tracker (FTK) have been in commissioning since 2015.  \\
%L1Topo is designed to offer triggers exploiting the event topology information (e.g. azimuthal angle difference between two particles) using the combined output from L1Calo and L1Muon. FTK is the hardware-based fast tracking module working in the L1 time designed to feeding the reconstructed tracks HLT.
}
{fig::Detector::triggerFlow}


\clearpage
\figNoH[110]{Detector/trigger2016_physicsRate.png}
{Rate of HLT streams for physics analyses during the 2016 data-taking \cite{trigPubResult}. Horizontal axis is in unit of lumi-clock, the smallest unit of data-taking in the same configuration.}
{fig::Detector::trigger_ratePhys}

%%%%%%%%%%%%%%%%%%%%%%%%%%%%%%%%%%%%%%%

\section{Recorded Data by ATLAS}
The $pp$-collision data analyzed in this study has been collected by ATLAS during 2015 and 2016. 
%The evolution of the peak luminosity and average pileup in the 2016 runs are shown in Figure \ref{fig::Detector::DAQ}.
Quality requirements are applied for the recorded data base on each lumi-block which is the smallest unit of data-taking defined as a period in the same run configuration and conditions of beam and detector. 
Rejected data is typically at the periods with more than a certain of fraction of modules in the sub-detectors being disabled or in a wrong operation configuration (e.g. voltage or temperature etc.).
%Integrated luminosity delivered by LHC and recorded in ATLAS during the 2016 operation is shown in Figure \ref{fig::Detector::dataInt}. 
After the quality requirement, the total integrated luminosity available for the analysis is 36.1 $\ifb$ with the measurement error of $3.2\%$. \\

%\figNoH[110]{Detector/intLumi2016.png}
%{Evolution of integrated luminosity delivered by LHC (green) and recorded in ATLAS (yellow) in the 2016 operation \cite{DAQ2016}.}
%{fig::Detector::dataInt}

