\chapter{Systematic Uncertainties}  \label{sec::Uncertainties}
Uncertainties associated with background estimations and the signal modeling is dedicatedly discussed in this section.
They are largely three-fold: instrumental uncertainties, theoretical uncertainties and the generic uncertainties for the background estimation methods. 
This chapter overviews the sources and discusses the evaluation. \\

%%%%%%%%%%%%%%%%%%%%%%%%% Instrumental %%%%%%%%%%%%%%%%%%%%%%%%%%%%%%
\section{Instrumental Uncertainty}
Instrumental uncertainties are the systematic uncertainty regarding to the experiment, including the imperfection of calibration and mis-modeling of detector response and so on.
%The explicit list of implemented uncertainties are summarized in the Appendix Sec. \ref{sec::Uncertainties::fullListSyst}.  \\
%They are implemented by generating the corresponding MC variations in which the scale, resolution or the efficiency for objects are tuned event-by-event.
%The expected yields in the CRs and SRs (VRs) by the varied samples are then input into the fit, and the difference with respect to those of nominal sample is taken as the $1\sigma$ deviation by the systematics. \\
  
\paragraph{Jets} \mbox{} \\
Despite the dedicated calibration procedures as described in Sec. \ref{sec::objDef::jets}, 
the residual uncertainty on the jet energy scale (JES) is often the largest source of instrumental uncertainty, 
since a slight shift in jet energy can cause a drastically change in the tail of the distributions.
87 independent uncertainties are modeled from each step in the calibration, including the MC uncertainty and observed discrepancy between MC and data.
In the analysis, those with similar behavior are statistically combined, reducing into 8 independent uncertainties. \\

The sub-leading jet uncertainty is on the energy resolution (JER). 
JER measurement is done by the same dataset used in the in-situ JES calibration (see Sec. \ref{sec::objDef::jets::calib}), using the balanced well-measured objects in di-jet or $Z/\gamma^*$+jets events \cite{144_JESmeas_2015data}. The uncertainty is quoted from the data/MC discrepancy, as well as the magnitude of the noise term reflecting our imperfect understanding of the origin. Figure \ref{fig::objDef::JERUnct2015_pt_144} show the measured total uncertainty on JES and JER. \\

%%%%%%%%%%
\begin{figure}
  \centering
    \subfig{0.48}{figures/ObjectDef/JESUnct2015_pt_144.pdf}{}
    \subfig{0.48}{figures/ObjectDef/JERUnct2015_pt_144.pdf}{}
    \caption{ Measured uncertainty on (a) jet energy scale (JES), and (b) the relative resolution, with the breakdown of the sources \cite{144_JESmeas_2015data}.
      \label{fig::objDef::JERUnct2015_pt_144} }
\end{figure}
%%%%%%%%%%

Systematics associated with flavor tagging are also important since the analysis deeply relies on the classification in terms of $b$-jet multiplicity. 
The uncertainty on the efficiency of $b$-jets and wrongly tagged light-flavor jets is separately evaluated by varying the training samples for each sub-algorithm, as well as the training configuration of the combining algorithm MV2. The resultant uncertainty is typically in a rage of $5\% \sim 10\%$. \\

Other uncertainties are qupted regarding to the angular position determination ($\eta$-calibration uncertainty) or JVT (Jet Vertex Tagger, Sec. \ref{sec::objDef::jets}) modeling. \\


\paragraph{Electrons} \mbox{} \\
Electrons involve three efficiency uncertainties i.e. reconstruction, identification and isolation.
These are mainly evaluated by the observed differences between the efficiencies measured using the $Z\ra ee$ events from data and from the simulation.
The uncertainties on the energy scale and resolution modeling are also taken into account.
They are evaluated based on the discrepancy between simulated and observed response of the EM calorimeter in Run2.  \\

%resolution uncertainty -> morinaga thesis (5.4.2)
%determined by test beam etc.
%extrapolate


\paragraph{Muons} \mbox{} \\
Four efficiency uncertainties and two separated scale uncertainties are associated to muons.
All the uncertainties are derived from the differences between the MC expectation and observed measurement outcome using the $Z\ra\mu\mu$ events. 
The efficiency uncertainties involve the reconstruction, identification, isolation and TTVA (Tracks-To-Vetex-Association), 
while the two scale uncertainties corresponds to the statistical and systematic uncertainty in the measurement. \\

%\figNoH[80]{ObjectDef/muonIDeffUnct.pdf}
%{Muon efficiency uncertainty (reconstruction and identification) \cite{166_muonPerformance2015data}.}
%{fig:Uncertainties::muonIDeffUnct_166}


\paragraph{Missing Transverse Energy} \mbox{} \\
On top of the propagated uncertainties on the scales and resolutions of the reconstructed objects, MET suffers from additional uncertainty regarding to the modeling of the soft term defined in Sec. \ref{sec::objDef::met}.
This is measured using the $Z (\ra\ell\ell)+\mathrm{jets}$ events, by comparing the expected momentum profile of soft terms and the observed ones.  \\

%\fig[120]{ObjectDef/metUnct_177.pdf}
%{}
%{}

\if 0
\subsection{Implementation and the Impact} 
Instrumental uncertainties are implemented by generating the corresponding MC variations in which the scale, resolution or the efficiency for objects are tuned event-by-event.
The expected yields in the CRs and SRs (VRs) by the varied samples are then input into the fit, and the difference with respect to those of nominal sample is taken as the $1\sigma$ deviation by the systematics. \\

%The impact by the scale variation on the background estimation is generally sizable since the tails of kinematics distributions are highly sensitive to it.
%However, due to the normalization applied for the main background
The impact on kinematical extrapolation is relatively sizable, amounting upto $5\% \sim 15\%$ in signal regions, though the normalization applied for the main background in CRs helps a lot.
On the other hand, instrumental uncertainties have tiny impact on the object replacement method. Though they do affect the modeling of lepton efficiencies and the tau response that are fully based on MC, it is confirmed to be negligible by comparing the MC closure between the nominal setup and systematical variations using the $\ttbar$ MC samples.  \\
%The results are shown in Figure\ref{fig::Uncertainties::objRepSys_kineVar1} $\sim$ \ref{fig::Uncertainties::objRepSys_regionYields} where the closure is shown as a function of kinematical variables as well as the yields in SR-like regions.
%No significant effect exceeding a few percent is found, therefore flat $5\%$ uncertainty is assigned on the gross estimation by the object replacement method. \\

The impact on signal modeling is generally marginal compared with the cross-section error and the shape uncertainties as described below.
Therefore, the instrumental uncertainties are not implemented for the non-benchmark models, otherwise the computation cost will skyrocket due to the enormous signal points.
\fi

\clearpage
%%%%%%%%%%%%%%
\begin{figure}[h]
  \centering
  \subfig{0.7}{figures/Uncertainties/objRepSys/All_emu__4j_MET200_MT125/compExpComb_nJet30.pdf}}
  \subfig{0.7}{figures/Uncertainties/objRepSys/All_emu__4j_MET200_MT125/compExpComb_jet1Pt.pdf}}
  \subfig{0.7}{figures/Uncertainties/objRepSys/All_emu__4j_MET200_MT125/compExpComb_meffInc30.pdf}}
  \caption{ Impact by instrumental systematics on the MC closure for (a) jet multiplicity ($p_T>30\gev$) (b) leading-jet pt  (c) $\met$  (d) $\meffInc$. Pre-seelction ($\nJetNoGev \geq 4$, $\met>200$, $\mt>125$) is applied.  \label{fig::Uncertainties::objRepSys_kineVar1} }
\end{figure}
%
\begin{figure}[h]
  \centering
  \subfig{0.7}{figures/Uncertainties/objRepSys/All_emu__4j_MET200_MT125/compExpComb_met.pdf}}
  \subfig{0.7}{figures/Uncertainties/objRepSys/All_emu__4j_MET200_MT125/compExpComb_mt.pdf}}
  %    \subfig{0.7}{figures/Uncertainties/objRepSys/All_emu__4j_MET200_MT125/compExpComb_tagLepPt.pdf}}
  %    \subfig{0.7}{figures/Uncertainties/objRepSys/All_emu__4j_MET200_MT125/compExpComb_metOverMeff.pdf}}
  \subfig{0.7}{figures/Uncertainties/objRepSys/All_emu__4j_MET200_MT125/compExpComb_LepAplanarity.pdf}}
  \caption{ Impact by instrumental systematics on the MC closure for (a) $\mt$ (b) $\lepPt$  (c) $\met$/$\meffInc$  (d) aplanarity. Pre-seelction ($\nJetNoGev \geq 4$, $\met>200$, $\mt>125$) is applied.  \label{fig::Uncertainties::objRepSys_kineVar2} }
\end{figure}
%
%
\begin{figure}[h]
  \centering
  \subfig{0.99}{figures/Uncertainties/objRepSys/All_emu__4j_MET200_MT125/compExpComb_regionYields_btag.pdf}}
  \subfig{0.99}{figures/Uncertainties/objRepSys/All_emu__4j_MET200_MT125/compExpComb_regionYields_bveto.pdf}}
  \caption{ Impact by instrumental systematics on the MC closure under various (a) b-tagged and (b) b-vetoed SR-like selections. (Top) Black dots are the closure with the nominal sample and the error bars are the associated statistical uncertainties. Dashed bands represent the variation caused by the systematic variations. (Bottom) Relative variation in closure with respect to that of the nominal sample.  \label{fig::Uncertainties::objRepSys_regionYields} }
\end{figure}
%
%-------------------------------















%%%%%%%%%%%%%%%%%%%%%%%%% Theo unct. %%%%%%%%%%%%%%%%%%%%%%%%%%%%%%
%\clearpage
\section{Theoretical Uncertainty} \label{sec::Uncertainties::theoUnct}
Uncertainties associated with the generated events in simulation subject to theoretical uncertainty.
This is the main uncertainty for the kinematical extraplation and signal modeling, while the object replacement suffers from negligible impact.
The sources of such uncertainties are as follow: 
%
\begin{description}
\item [Cross-section uncertainty] \mbox{} \\
%is provided by the associated calculation error in the references that the cross-section is quoted by. 
The primary source contributing to it is the missing higher-order terms in the calculation, such as terms beyond NNLO for the NLO calculation, or the absence of soft gluon resummation. 
The other typical sources are from PDF, and measurement precision on standard model parameters, particularly in strong coupling constant and quark masses for higher order QCD correction. 
Since this uncertainty only affecs the global normalization of each physics processes, those of $\wjets$ and top background will valish during the normalization in CRs.

%%%%%%%%%%%%%
\item [Choice of renormalization scales] \mbox{} \\
The renormalization scale ($\mu_{\mathrm{renom.}}$) is a non-physical theory parameter known to have no impact on observables when all the terms in the perturbation series is taken into account. However, it is not the case when considering a fixed-order calculation truncating part of the higher-order terms, leaving a non-physical dependency. This is commonly regarded as the generic precision that the calculation can address. The variation due to the different chioce of $\mu_{\mathrm{renom.}}$ is quoated as systematics in this analysis, which is evaluated by shifting $\mu_{\mathrm{renom.}}$ from the default scale by factor of 2 or 0.5.
There have been claims that the choice of scales appearing in factorization and ressumation can be independent of $\mu_{\mathrm{renom.}}$. The dependencies on those scale are additionally evaluated in the same recipe in case of \sherpa where those scales are separately parametrized.

\item [Parton shower (PS)]  \mbox{} \\
The modeling dependence on parton showering schemes or setup is quoted as systematics.
For top backgrounds, this is done by directly comparing the default scheme using \pythiasix $\mbox{\phantom{k}}$ with the alternative using \herwig. The difference is taken as $1\sigma$ variation. 
For the other backgrounds generated by \sherpa, the dependency on the matching scale in the CKKW matching is quoted for the parton shower systematics. 
While the default matching scale is set to $\mu_{\mathrm{CKKW}} = 20\gev$, the up (down) variation is generated by shifting it to $\mu_{\mathrm{CKKW}} = 15 (30) \gev$. 
For SUSY signals, 5 variations are generated by tuning the internal parameters in \pythiaeight. The uncertainties are evaluated by the variance with respect to the nominal one, and added in quadrature. 


\item [Interference between $WWbb$ diagrams (for top background)] \mbox{} \\
The diagrams of $\ttbar+Wt$ and the other $WWbb$ diagrams do interfere each other since they are in the common final states.
This is actually missed in the MC description as the involvement of top-quarks is explicitly required in the sample generateion of top backgrounds. 
The impact is known to be significant after a high $\mt$ cut is applied where the bulk $\ttbar$ component is suppressed (see Figure \ref{fig::Uncertainties::ttWt_vs_WWbb}).
%In particular, the topness selection is essentially rejecting the $\ttbar$ with the both top quarks being on-shell, in other words, significantly enhancing the contribution from the off-shell tail of top quarks where the interference effect is addressing. 
It is then evaluated by comparing two truth-level MadGraph samples: one with the only diagrams of $\ttbar+Wt$, and the other with inclusive $WWbb$ diagrams.
The difference is taken as $1\sigma$ variation.


\item [Hard process description (for $\ttbar+W/Z/WW$)] \mbox{} \\
As $\ttbar+W/Z/WW$ have not dedicatedly measured in precision using data yet, there is no particular generator favored by data 
i.e. the prediction by \MGMCatNLO $\mbox{\phantom{k}}$ (nominal) and \sherpa (alternative) are equivalent.
An envelope calculation is then done by comparing the modeling of \mgmc and \sherpa, and the difference is quoted as a systematics in terms of the hard process description.
\end{description}

\noindent While the envelope calculations are done all using MC, 
the limited MC statistics becomes an issue in quantifying $5\%\sim10\%$ level difference.
Therefore, some of the cuts are loosened, and different recipes are employed depending on the type of samples, which will be detailed as following sections. \\


\fig[90]{Uncertainties/ttWt_vs_WWbb.pdf}
{Comparison of the $\mt$ shape between $\ttbar+Wt \ra WWbb$ (red) and all $WWbb$ (blue).}
{fig::Uncertainties::ttWt_vs_WWbb}


\clearpage
%%%%%%%%%%%%%%%%%
\subsection{Normalized Backgrounds: $W$+jets and Tops} \label{sec:Uncertainties::normalizedBG}
The uncertainties for the normalized backgrounds ($\wjets$ and top background) are essentially the MC modeling uncertainties on the extrapolation variables ($mt$, $\apl$ and topness etc.).
They are evaluated by computing the variation in the ratio of MC yields between in a CR and a SR (or VR) when the systematical variations are applied.
Some of the cuts are released to suppress the statistical fluctuation in MC to a reasonable level.
$b$-jet requirement is first removed, based on the fact that it is generally orthogonal to kinematics. The $\meffInc$ cut is also removed in addition to it, based on the concept that the shape variation in terms of the extrapolation variables are tested. Therefore, the evaluated systematics are common to all the bins in the same tower eventually. Table \ref{tab::Uncertainties::theoSys_Wjets} (\ref{tab::Uncertainties::theoSys_Tops}) summarizes the evaluated uncertainties for $\wjets$ (top background). Systematics contributing below 5$\%$ or 5 times less than that of the leading uncertainty in the region are ignored (labeled as "-").

\begin{table}[h]
    \caption{Assigned theory uncertainties for (a) $\wjets$ and (b) top background [$\%$]. The uncertainties are all shared by the $\meffInc$-bins in the same tower. The symbols stand for; $\mu_{\mathrm{fact.}}$: systematic variation in factorization scale, $\mu_{\mathrm{resum.}}$: resummation scale, $\mu_{\mathrm{ren.}}$: renormalization scale, ``PS'': Parton shower, ``INTF'': interference between $\ttbar+Wtb$ and other $WWbb$ diagrams.}
    \begin{subtable}{.5\linewidth}
      \centering
        \caption{$\wjets$}
        \begin{tabular}{ c | c c c c }
            \hline
            & $\mu_{\mathrm{fact.}}$ & $\mu_{\mathrm{resum.}}$ & $\mu_{\mathrm{ren.}}$ & PS \\
            \hline
            \hline
            SR 2J  &  - & 7 & 7 & 9 \\
            SR 6J  &  9 & - & 23 & - \\
            SR Low-x  &  - & 11 & - & 6 \\
            SR High-x  &  19 & 7 & - & - \\
            SR 3B  &  36 & 15 & - & 19 \\
            \hline
            VRa 2J  &  - & 8 & 12 & - \\
            VRa 6J  &  13 & 11 & - & - \\
            VRa Low-x  &  9 & - & 8 & 8 \\
            VRa High-x  &  - & 6 & - & 9 \\
            VRa 3B  &  20 & 16 & 7 & 6 \\
            \hline
            VRb 2J  &  - & 4 & 3 & 7 \\
            VRb 6J  &  5 & - & 5 & 6 \\
            VRb Low-x  &  - & 5 & 6 & 5 \\
            VRb High-x  &  5 & - & - & 5 \\
            VRb 3B  &  - & 5 & 5 & 5 \\
            \hline            
        \end{tabular}
        \label{tab::Uncertainties::theoSys_Wjets}
    \end{subtable}%
    \begin{subtable}{.5\linewidth}
      \centering
        \caption{Top background}
        \begin{tabular}{ c | c c c }
            \hline
            & $\mu_{\mathrm{fact.}},\mu_{\mathrm{ren.}}$ & INTF. & PS \\
            \hline
            \hline
            SR 2J  &  22 & 17 & 8 \\
            SR 6J  &  21 & 24 & 25 \\
            SR Low-x  &  15 & 13 & 10 \\
            SR High-x  &  15 & 17 & 28 \\
            SR 3B  &  27 & 25 & 13 \\
            \hline
            VRa 2J  &  12 & 5 & 10 \\
            VRa 6J  &  15 & 7 & 9 \\
            VRa Low-x  &  10 & 12 & 6 \\
            VRa High-x  &  17 & 10 & - \\
            VRa 3B  &  13 & 26 & 8 \\
            \hline
            VRb 2J  &  - & 21 & 10 \\
            VRb 6J  &  - & 19 & 5 \\
            VRb Low-x  &  - & 18 & 5 \\
            VRb High-x  &  - & 23 & 8 \\
            VRb 3B  &  - & 25 & 7 \\
            \hline            
        \end{tabular}
        \label{tab::Uncertainties::theoSys_Tops}
    \end{subtable} 
\end{table}


\clearpage
%%%%%%%%%%%%%%%%%
\subsection{Non-normalized Backgrounds: $Z$+jets, di-bosons and $\ttbar+W/Z/WW$}  
The cross-section uncertainty for $\zjets$, di-bosons and $\ttbar+W/Z/WW$ amounts upto level of $5\%$ \cite{VjetsXsecMeas_ATLAS_Run1}, $6\%$ \cite{ATL-PHYS-PUB-2016-002} and $13\%$ \cite{Alwall:2014hca} respectively. 
The other uncertainties affecting the shape are evaluated in SRs/VRs with the cuts in $\mt$, $\apl$ and topness are removed, as well as the $b$-tagging requirement.
This is because that the impact of systematics is dominantly seen in spectra regarding to jet activity, in particular jet-multiplicity and $\meffInc$. 
The uncertainties derived for each $\meffInc$-bin of SR and VR are summarized in Table \ref{tab::Uncertainties::theoSys_nonnorm}. 
Systematics contributing below 5$\%$ or 5 times less than that of the leading uncertainty in the region are ignored.  \\
%The renormalization scale variation yields the dominant impact for $\zjets$ and di-bosons, due to the fact that it gives the 

\newcommand{\mufac}{$\mu_{\mathrm{fact.}}$}
\newcommand{\muren}{$\mu_{\mathrm{ren.}}$}
\newcommand{\mures}{$\mu_{\mathrm{resum.}}$}

\tab{ c | c c c | c c c | c c } 
{
  \hline
                      & \multicolumn{3}{c|}{$\zjets$} & \multicolumn{3}{|c}{Di-bosons} & \multicolumn{2}{|c}{$\ttbar+W/Z/WW$} \\
                      & \mufac & \muren & PS      & \mufac & \mures & \muren  & \mufac,\muren & Hard proc.  \\
  \hline
  \hline
  2J $\meffIncFirst$  &  -     & 23     & 7       & -      & -      & 16      & -             & 9   \\
  2J $\meffIncSecond$ &  -     & 25     & -       & 21     & -      & 21      & 5             & 10  \\
  2J $\meffIncThird$  &  -     & 25     & -       & -      & -      & 23      & -             & 16  \\
  6J $\meffIncFirst$  &  -     & 35     & -       & 8      & 9      & 19      & -             & 8   \\
  6J $\meffIncSecond$ &  10    & 35     & -       & 8      & 7      & 26      & -             & 17  \\
  6J $\meffIncThird$  &  -     & 39     & 15      & 9      & 11     & 37      & -             & 22  \\
  Low-x               &  -     & 33     & 10      &13      & -      & 22      & -             & 16  \\
  High-x              &  -     & 32     & -       & -      & 12     & 34      & -             & 33  \\
  3B $\meffIncFirst$  &  -     & -      & -       & -      & 7      & 29      & 5             & 5   \\
  3B $\meffIncSecond$ &  -     & -      & -       &13      & -      & 35      & -             & 13  \\
  \hline
}
{Assigned theory uncertainties for $\zjets$, di-bosons and $\ttbar+W/Z/WW$ [$\%$]. 
The symbols stand for; $\mu_{\mathrm{fact.}}$: systematic variation in factorization scale, $\mu_{\mathrm{resum.}}$: resummation scale, $\mu_{\mathrm{ren.}}$: renormalization scale, ``PS'': Parton shower, ``Hard proc.'': systematics assigned based on the generator comparison.}
{tab::Uncertainties::theoSys_nonnorm}

\clearpage
\subsection{SUSY Signals} 
The cross-section uncertainty of gluino pair production amounts up-to $15\% \sim 35\%$, as shown in Figure \ref{fig::Samples::xsec_GG} in Sec. \ref{sec:Samples::generators}.
The other uncertainties affecting the shape are evaluated over the signal points in the $\xhalf$ grid of the model \textbf{QQC1QQC1}, 
and found to be typically marginal compared with the cross-section uncertainty. This is because the jet activity is predominantly sourced by gluino decays rather than the ISRs and FSRs for most of the cases. 
The only exception is found in low $\meffInc$-bins in SR \textbf{2J} where the target signals are with highly compressed mass splitting between gluino and LSP ($\dmg < 50\gev$) that have to rely on the additional radiation to enter the signal regions. In such case, the acceptance can vary upto by $20\%$ by the theoretical variation. Table \ref{tab::Uncertainties::theoSyst_signal} presents the assigned shape uncertainties, which are common to all the signal models and mass points.
\tab{ c | c c}{
  \hline
  &  \mufac, \muren  & PS \\
  \hline
  \hline
  SR 2J $\meffIncFirst$   &  15 & 20    \\
  SR 2J $\meffIncSecond$  &  10 & 10    \\
  SR 2J $\meffIncThird$   &  -  & 5     \\
  The other regions       &  -  & -     \\
  \hline
}
{Shape uncertainties assigned for SUSY signal processes $[\%]$. The uncertainties are common to all the signal models.
The symbols stand for; $\mu_{\mathrm{fact.}}$: systematic variation in factorization scale, $\mu_{\mathrm{ren.}}$: renormalization scale, ``PS'': Parton shower.}
{tab::Uncertainties::theoSyst_signal}


%%%%%%%%%%%%%%%%%%%%%%%%% Syst on the method %%%%%%%%%%%%%%%%%%%%%%%%%%%%%%
\clearpage
\section{Other Uncertainties} 
\subsection{Generaic Uncertainty on the BG Estimation Methods}  \label{sec::Uncertainties::nonClosure}
%The generic errors of the background estimation method need to be quoted as additional uncertainties in the background expectation. 

\paragraph{Kinematical extrapolation method} \mbox{} \\
Though all theoretical uncertainties that are already known are assigned on the extrapolation, one has to notice that none of them can explain the mis-modeling observed in the pre-selection region (Sec. \ref{sec::BGestimation::dataMC}). Therefore, there obviously exists unknown theoretical uncertainties, and in principle it can also affect the extrapolation. \\

It is seemingly impossible to know the impact of ``unknown systematics'' though, remember that we can largely cure the mis-modeling by a ad hoc kinematical reweighting:
\begin{align}                                                                                 
\begin{cases}
y & = 1 - 0.1 \times (\nJetNoGev-2) \mbox{\phantom{MMMMMMMMMMM}} (\wjets) \nn \\
y & = 1.05 \times \left[ 1 - 0.061 \,\times p_T(\ttbar) \right] \mbox{\phantom{MMMMMMk}} (\ttbar, \mbox{@1L,2L}) \nn \\
y & = 1.4 \times \left[ 1 - 0.061 \,\times p_T(\ttbar) \right] \mbox{\phantom{MMMMMMM}} (\ttbar, \mbox{@3B}).
%\label{eq::Uncertainties::rwgt},
\end{cases}
\end{align} 

Reweighted distributions are shown in appendix \ref{sec::Appendix::dataMC_rwgt}. 
The idea is to emulate the ``unknown systematic'' by these reweighting, and quote the variation in the extrapolation as the systematics. 
Although this is not trivial how good the reweighting approximation is, this is the current best thing one could do.
%
Figures in appendix \ref{sec::App::valid_extp} show the extrapolation variation against the magnitude of mis-modeling generated by reweighting the MC events with:
\begin{align}
 y & = 1 - w \times (\nJetNoGev-2), \mbox{\phantom{MMMM}}\,\,\,\,\,\, w \in [0,0.18]  \mbox{\phantom{MMMM}} (\wjets) \nn  \\
 y & = 1 - w \,\times p_T(\ttbar)/100\gev, \,\,\,\,\,\,\,\,           w \in [0,0.09]  \mbox{\phantom{MMMM}} (\ttbar).
\label{eq::BGestimation::injected_MCvariation}
\end{align}
The vertical axis on the top panels in the plots show the amount of variation that a CR and corresponding SR(VR) experience by the MC reweighting as a function of $w$. 
The relative variation in CR (orange) is equivalent to the normalization factor actually obtained via the fit to data, 
while that in SR (blue) to the ideal normalization factor need to fully correct the SR(VR). 
%
The bottom panels display the ratio, namely the resultant extrapolation variation. 
$B$-tagging requirement is removed to maintain sufficient MC statistics, assuming the kinematics are invariant with it. 
For the $\ttbar$ process, component estimated by the object replacement method is excluded from the test. \\

The assigned uncertainty to each SR and VR are decided as Table \ref{tab::Uncertainties::noClosure_kineExtp}, quoting the extrapolation error at $w=0.1$ and $w=0.07$ for $\wjets$ and $\ttbar$ respectively. As single-top is assumed to suffer from the same uncertainty as $\ttbar$, they are collectively noted as ``Tops''.

\clearpage
%%%%%%%%%%% kine extp 
\tab{ c | c c || c | c c || c | c c } 
{
  \hline
                         & $\wjets$ & Tops &                           &   $\wjets$ & Tops &                           & $\wjets$ & Tops \\
  \hline
  SR 2J $\meffIncFirst$  &  15  & 5            &   VRa 2J $\meffIncFirst$  &  -  & 10              &   VRb 2J $\meffIncFirst$  &  10 & 5     \\
  SR 2J $\meffIncSecond$ &  15  & -            &   VRa 2J $\meffIncSecond$ &  5  & 10              &   VRb 2J $\meffIncSecond$ &  5  & 10    \\
  SR 2J $\meffIncThird$  &  15  & 20           &   VRa 2J $\meffIncThird$  &  -  & 20              &   VRb 2J $\meffIncThird$  &  5  & 10    \\
  SR 6J $\meffIncFirst$  &  -   & 5            &   VRa 6J $\meffIncFirst$  &  -  & 5               &   VRb 6J $\meffIncFirst$  &  -  & -     \\
  SR 6J $\meffIncSecond$ &  -   & 10           &   VRa 6J $\meffIncSecond$ &  -  & 5               &   VRb 6J $\meffIncSecond$ &  5  & 5     \\
  SR 6J $\meffIncThird$  &  -   & -            &   VRa 6J $\meffIncThird$  &  -  & 5               &   VRb 6J $\meffIncThird$  &  5  & 10    \\
  SR Low-x               &  10  & -            &   VRa Low-x               &  -  & 5               &   VRb Low-x               & 10  & 5     \\
  SR High-x              &  -   & 10           &   VRa High-x              &  -  & 30              &   VRb High-x              &  5  & 10    \\
  SR 3B $\meffIncFirst$  &  -   & 5            &   VRa 3B $\meffIncFirst$  & 30  & -               &   VRb 3B $\meffIncFirst$  & 20  & 10    \\
  SR 3B $\meffIncSecond$ &  -   & 10           &   VRa 3B $\meffIncSecond$ & 30  & 5               &   VRb 3B $\meffIncSecond$ & 30  & 15    \\
  \hline            
}
{Uncertainty on the kinematical extrapolation by the accounted theoretical uncertainties, for $\wjets$ and top background respectively [$\%$].}
{tab::Uncertainties::noClosure_kineExtp}


%%%% objRep
\paragraph{Object replacement method} \mbox{} \\
For the object replacement method, the observed non-closure error discussed in Sec. \ref{sec::BGestimation::objRep::mcClosure} are included as systematics as listed in Table \ref{tab::Uncertainties::sys_nonClosure}. 
%The uncertainties are commonly assigned to all the SRs and VRs for missing-lepton replacement, while extra uncertainty is quoted for the estimation of the tau replacement in the \textbf{3B} towers, to account for the 

\tab{c|c c}
{
  \hline
                               & BV/BT & 3B \\
  \hline
  \hline
  Tau replacement              & 5 & 20 \\
  Missing electron replacement & \multicolumn{2}{c}{ 15 } \\
  Missing muon replacement     & \multicolumn{2}{c}{ 30 } \\
  \hline
}
{Summary of non-closure errors in the object replacement method [$\%$]. }
{tab::Uncertainties::sys_nonClosure}


%%%%%%%%%%%%%%%%%%%%%%%%%%
\subsection{Control region statistics}
In both of the background estimation methods, reflecting the (semi-)data driven nature, the statistical error in CRs often becomes the primary uncertainty in the estimation.
This typically occurs in case of the high $\meffInc$ bins, for instance the yields in the CRs for the kinematical extrapolation can end up in about 15 events in the worst case, immediately resulting in $20\%-30\%$ of uncertainty. The tendency is more striking concerning to the object replacement method where the uncertainty is solely dominated by the statistical error in terms of seed event statistics that amounts $20\%-60\%$ in SRs depending on the tightness of selection. Furthermore, one has to mind that the statistical error in the object replacement method is not independent between the regions given that the sub-events from a single seed event can fall into different regions. The correlated statistical error between each of the two signal regions is then evaluated by identifying the fraction of common seed events between their estimation. 
Table \ref{sec::BGestimation::objRep::binCor} shows the correlation coefficient in the estimated yields between SR$_i$ and SR$_j$ defined as:
$$\rho := \frac{\sum_e \sqrt{w^{i}_e w^{j}_e}}{\sqrt{\sum_e w^{i}_e} \sqrt{\sum_e w^{j}_e}}$$
where $e$ runs over all seed events, and $w^i_e$ denotes the sum of weighted sub-events falling into SR$_i$ generated by the seed event $e$. 
Correlation is mainly found in adjacent $\meffInc$-bins, between high $\meffInc$ BT and 3B bins, and between high $\meffInc$ hard-lepton and soft-lepton bins. 
The effect of overlapped sub-events is taken into account in the final fitting. 
Though a large inter-bin correlation can potentially spoil the sensitivity of the shape fit, 
the impact on the final result to this analysis is limited, since the signal points rarely lay over multiple bins with equal abundance.  \\

%Seed events overlap between the SRs
\fig[180]{Uncertainties/seed_overlap_objRep/seed_overlap_data.pdf}
{The correlation coefficient in the estimated yields between each of the two signal regions, indicating the level of correlated statistical fluctuation.}
{sec::BGestimation::objRep::binCor}



%%%%%%
\subsection{MC statistics}
Limited MC statistics lead to a non-negligible uncertainty in signal and background yields in regions with tight selection. The largest impact is found in SR 3B $\meffIncSecond$ amounting upto $15\%$, which is still minor compared with the other systematics sources. The statistical behavior is carefully taken into account in the fit, as detailed in the Sec. \ref{sec::Result::statistics}.








