
%%%% objRep
\clearpage
The method is highly robust against the theoretical uncertainties by construction. Figure\ref{fig::Uncertainties::objRep_theoSys_compRad}-\ref{fig::Uncertainties::objRep_theoSys_compHad} are a part of the validation in which consistent level of closures between the nominal $\ttbar$ sample and alternative samples (scaleUp/Down and Powheg+Herwig++ respectivly) are confirmed.

% -------------- syst::compRad
\begin{figure}
  \begin{center}
    \includegraphics[width=140mm]{figures/BGestimation/ObjReplacement/mcClosure/compRad.eps}
    \captionof{figure}{Comparison of MC closure with different radiation configuration. Pre-selection: $p_{T}(\ell_{1})>35\gev$.}
    \label{fig::Uncertainties::objRep_theoSys_compRad}
  \end{center}
\end{figure}
%-------------------------------
% -------------- syst::compHad
\begin{figure}
  \begin{center}
    \includegraphics[width=140mm]{figures/BGestimation/ObjReplacement/mcClosure/compHad.eps}
    \captionof{figure}{Comparison of MC closure with different hadronization scheme. Pre-selection: $p_{T}(\ell_{1})>35\gev$.}
    \label{fig::Uncertainties::objRep_theoSys_compHad}
  \end{center}
\end{figure}
%-------------------------------
